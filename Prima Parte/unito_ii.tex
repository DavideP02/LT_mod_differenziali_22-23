% PACCHETTI FONDAMENTLAI
\usepackage[T1]{fontenc}
\usepackage[utf8]{inputenc}
\usepackage[italian]{babel}
\usepackage[babel]{csquotes}
\usepackage[style=numeric]{biblatex}
\usepackage{microtype}
\usepackage{lmodern}
\usepackage{graphicx} % inserire immagini
\usepackage{subfig} % sottofigure
\usepackage{float}
\usepackage{multicol} % due colonne
\usepackage{ulem} % sottolineare
\usepackage{soul}
\usepackage{lipsum} % lorem ipsum
\usepackage{xcolor} % colori in latex
\usepackage{parskip} % rimuove l'indentazione dei nuovi paragrafi 
\usepackage{centernot}
\usepackage[outline]{contour}\contourlength{3pt}
\usepackage{fancyhdr}
\usepackage{imakeidx}%creare indice in fondo
\usepackage{layout}
\usepackage[intlimits]{empheq} % Riquadri colorati attorno alle equazioni
\usepackage[most]{tcolorbox} % Riquadri colorati
\usepackage{ifthen} % IFTHEN
\usepackage{rotating} %figure ruotate 

% pacchetti matematica
\usepackage[intlimits]{amsmath} 
\usepackage{amssymb}
\usepackage{amsthm}
\usepackage{yhmath}
\usepackage{dsfont}
\usepackage{mathrsfs}
\usepackage{cancel} % semplificare
\usepackage{polynom} %divisione tra polinomi
\usepackage{forest} % grafi ad albero
\usepackage{booktabs} % tabelle
\usepackage{commath} %simboli e differenziali
\usepackage{bm} %bold
\usepackage[fulladjust]{marginnote} %to use marginnote for date notes
\usepackage{arrayjobx}%array
\usepackage{mathtools}
%%%%%%%%%%%%%


%%%% CONTATORE DIMOSTRAZIONI
\newcounter{numerodidimostrazioni}
\newcommand{\mostranumerodimostrazioni}{\cleardoublepage\thispagestyle{empty}\begin{center}\null\vfill \textit{Questo corso comprende {\thenumerodidimostrazioni} dimostrazioni}\vfill\null\end{center}}
%%%%

%%%% QUIVER
\newcommand{\duepunti}{\,\mathchar\numexpr"6000+`:\relax\,}
% A TikZ style for curved arrows of a fixed height, due to AndréC.
\tikzset{curve/.style={settings={#1},to path={(\tikztostart)
    .. controls ($(\tikztostart)!\pv{pos}!(\tikztotarget)!\pv{height}!270:(\tikztotarget)$)
    and ($(\tikztostart)!1-\pv{pos}!(\tikztotarget)!\pv{height}!270:(\tikztotarget)$)
    .. (\tikztotarget)\tikztonodes}},
    settings/.code={\tikzset{quiver/.cd,#1}
        \def\pv##1{\pgfkeysvalueof{/tikz/quiver/##1}}},
    quiver/.cd,pos/.initial=0.35,height/.initial=0}

% TikZ arrowhead/tail styles.
\tikzset{tail reversed/.code={\pgfsetarrowsstart{tikzcd to}}}
\tikzset{2tail/.code={\pgfsetarrowsstart{Implies[reversed]}}}
\tikzset{2tail reversed/.code={\pgfsetarrowsstart{Implies}}}
% TikZ arrow styles.
\tikzset{no body/.style={/tikz/dash pattern=on 0 off 1mm}}
%%%%%%%%%%

%%% HEADER e FOOTER
%% HEAD and FOOT
\newcommand{\datainizion}{testo}
\newcommand{\hdrnew}[1]{
	\fancyhead{} % cancella tutti i campi
	\fancyfoot{}
	\fancyhead[RO,LE]{\datainizion}
	\fancyhead[RE]{#1}
	\fancyhead[LO]{\nouppercase{\leftmark}}
	% \fancyfoot[LE,RO]{\thepage}
	\fancyfoot[C]{\thepage}
	% \fancyfoot[CO,RE]{Per: Dean A. Smith}
	\renewcommand{\headrulewidth}{0.4pt}
	\renewcommand{\footrulewidth}{0pt}
	\pagestyle{fancy}
}
\newcommand{\days}[1]{\renewcommand{\datainizion}{#1}}
%%%%%%%%

%% DEFINIZIONI COMANDI MATEMATICI
\DeclareMathOperator{\epi}{Epi}
\DeclareMathOperator{\cl}{cl}
\DeclareMathOperator{\graph}{graph}
\DeclareMathOperator{\arcsec}{arcsec}
\DeclareMathOperator{\arccot}{arccot}
\DeclareMathOperator{\arccsc}{arccsc}
\DeclareMathOperator{\spettro}{Spettro}
\DeclareMathOperator{\nulls}{nullspace}
\DeclareMathOperator{\dom}{dom}
\DeclareMathOperator{\ar}{ar}
\DeclareMathOperator{\const}{Const}
\DeclareMathOperator{\fun}{Fun}
\DeclareMathOperator{\rel}{Rel}
\DeclareMathOperator{\altezza}{ht}
\let\det\relax %TOGLIE LA DEFINIZIONE SU "\det"
\DeclareMathOperator{\det}{det}
\DeclareMathOperator{\End}{End}
\DeclareMathOperator{\gl}{GL}
\DeclareMathOperator{\Id}{Id}
\DeclareMathOperator{\id}{Id}
\DeclareMathOperator{\I}{\mathds{1}}
\DeclareMathOperator{\II}{II}
\DeclareMathOperator{\rank}{rank}
\DeclareMathOperator{\tr}{tr}
\DeclareMathOperator{\tc}{t.c.}
\DeclareMathOperator{\T}{T}
\newcommand{\R}{\mathds{R}}
\newcommand{\K}{\mathds{K}}
\newcommand{\Q}{\mathds{Q}}
\newcommand{\N}{\mathds{N}}
\newcommand{\C}{\mathds{C}}
\newcommand{\Z}{\mathds{Z}}
\newcommand{\rmn}{\R^{m,n}}
\renewcommand{\tilde}[1]{\widetilde{#1}}
\renewcommand{\parallel}{\mathrel{/\mkern-5mu/}}
\newcommand{\parti}[1]{\wp (#1)}
%tramite i prossimi due comandi posso decidere come scrivere i logaritmi naturali in tutti i documenti: ho infatti eliminato qualsiasi differenza tra "ln" e "log": se si vuole qualcosa di diverso bisogna inserire manualmente il tutto
\let\ln\relax
\DeclareMathOperator{\ln}{log}
\let\log\relax
\DeclareMathOperator{\log}{log}
%%%%%%

%% NUOVI COMANDI
\newcommand{\straniero}[1]{\textit{#1}} %parole straniere
\newcommand{\titolo}[1]{\textsc{#1}} %titoli
\newcommand{\qedd}{\tag*{$\blacksquare$}} %qed per ambienti matemastici
\renewcommand{\qedsymbol}{$\blacksquare$} %modifica colore qed
\newcommand{\ooverline}[1]{\overline{\overline{#1}}}
\newcommand{\circoletto}[1]{\left(#1\right)^\text{o}}
%
\newcommand{\qmatrice}[1]{\begin{pmatrix}
#1_{11} & \cdots & #1_{1n}\\
\vdots & \ddots & \vdots \\
#1_{m1} & \cdots & #1_{mn}
\end{pmatrix}}
%
\newcommand{\parentesi}[2]{%
\underset{#1}{\underbrace{#2}}%
}
%
\newcommand{\norma}[1]{% Norma 
\left\lVert#1\right\rVert%
}
\newcommand{\scalare}[2]{% Scalare
\left\langle #1, #2\right\rangle
}
%%%%%

%% DEFINIZIONE NUMERAZIONE EQUAZIONI
\renewcommand{\theequation}{\thechapter.\arabic{equation}}
\numberwithin{equation}{chapter}
%%%%%%%%%%%%%%%%%

%% ENUMERATE romani
\usepackage{enumitem} %numeri romani come enumerate
\newenvironment{romanen}{\begin{enumerate}[label={\itshape\roman{*}}., ref=(\roman{*})]}{\end{enumerate}}
%%%%%

%% RIFERIMENTI A FINE DOC
\newarray\Riferimentiname
\newarray\Riferimentipoint
\newcounter{Riferimentinum}
\newcommand{\riferimento}[2]{\stepcounter{Riferimentinum}% Questo comando aggiunge alla lista dei riferimenti un elemento (uno al nome e uno alla descrizione): deve essere usato in ogni teorema/enunciato che faccia riferimento a qualcosa nel testo.
\Riferimentiname(\theRiferimentinum)={#1}% 
\Riferimentipoint(\theRiferimentinum)={#2}
}
\newcounter{variabile}
\newcommand{\stampairiferimenti}{%Questo comando va lanciato alla fine, e stampa tutti i riferimenti necessari
	\setcounter{variabile}{0}
	\cleardoublepage
	%\stepcounter{capitoloeccolo}%TODO capire se serve
	\chapter*{Riferimenti in bibliografia}
	\begin{itemize}
		\whiledo{\thevariabile < \theRiferimentinum}{\stepcounter{variabile}\item \textbf{\Riferimentiname(\thevariabile):} \Riferimentipoint(\thevariabile).}
	\end{itemize}
}
%%%%%

%% RESTRIZIONI
\newcommand{\referenze}[2]{
	\phantomsection{}#2\textsuperscript{\textcolor{blue}{\textbf{#1}}}
}
\def\restriction#1#2{\mathchoice
              {\setbox1\hbox{${\displaystyle #1}_{\scriptstyle #2}$}
              \restrictionaux{#1}{#2}}
              {\setbox1\hbox{${\textstyle #1}_{\scriptstyle #2}$}
              \restrictionaux{#1}{#2}}
              {\setbox1\hbox{${\scriptstyle #1}_{\scriptscriptstyle #2}$}
              \restrictionaux{#1}{#2}}
              {\setbox1\hbox{${\scriptscriptstyle #1}_{\scriptscriptstyle #2}$}
              \restrictionaux{#1}{#2}}}
\def\restrictionaux#1#2{{#1\,\smash{\vrule height .8\ht1 depth .85\dp1}}_{\,#2}} 
%%%%%%%%%%%

%% SEZIONE GRAFICA
\usepackage{tikz}
\usetikzlibrary{matrix, patterns, calc, decorations.pathreplacing, hobby, decorations.markings, decorations.pathmorphing, babel}
\usepackage{tikz-3dplot}
\usepackage{mathrsfs} %per geogebra
\usepackage{tikz-cd}
\tikzset
{
  %surface/.style={fill=black!10, shading=ball,fill opacity=0.4},
  plane/.style={black,pattern=north east lines},
  curve/.style={black,line width=0.5mm},
  dritto/.style={decoration={markings,mark=at position 0.5 with {\arrow{Stealth}}}, postaction=decorate},
  rovescio/.style={decoration={markings,mark=at position 0.5 with {\arrow{Stealth[reversed]}}}, postaction=decorate}
}
\usepackage{pgfplots} % stampare le funzioni
	\pgfplotsset{/pgf/number format/use comma,compat=1.15}
	%\pgfplotsset{compat=1.15} %per geogebra
	\usepgfplotslibrary{fillbetween, polar}
%%%%%%

%% NOTE A PIÉ PAGINA
\usepackage[hang, perpage, symbol*, stable, bottom]{footmisc} %per le note a pié pagina
\footnotemargin=0.8em
\DefineFNsymbolsTM{myfnsymbols}{% def. from footmisc.sty "bringhurst" symbols
  \textdagger    \dagger
  \textdaggerdbl \ddagger
  \textsection   \mathsection
  \textbardbl    \|%
  \textparagraph \mathparagraph
  \textdagger\textdagger \dagger\dagger
  \textdaggerdbl\textdaggerdbl \ddagger\ddagger
  \textsection\textsection \mathsection\mathsection
  \textparagraph\textparagraph \mathparagraph\mathparagraph
}%
\setfnsymbol{myfnsymbols}
%comandi per footnotemark consecutivi
	\newcommand{\footnotemarkk}[1]{%
		\hyperref[#1]{\begin{NoHyper}
			\footnotemark
		\end{NoHyper}}}
	\newcommand{\consecfoottext}[3]{
		\addtocounter{footnote}{-#1}
		\footnotetext{#3\label{#2}}
		\addtocounter{footnote}{#1}
	}

	%esempio:
	% \mathscr{B}=\{\underset{\footnotemarkk{e1}}{\underbrace{v_1, \cdots, v_{k_1}}}, \underset{\footnotemarkk{e2}}{\underbrace{v_{k_1+1} , \cdots, v_{k_2+k_1}}}, \cdots, \underset{\footnotemark}{\underbrace{v_{k_{l-1}+kl }}}\}
	% \]
	% \consecfoottext{2}{e1}{autovettori rispetto a $ \lambda_1 $}
	% \consecfoottext{1}{e2}{autovettori rispetto a $ \lambda_2 $}
	% \footnotetext{autovettore rispetto a $ \lambda_{l}  $}
	% \todo{Manca matrice}
%%%%%%

%% CITAZIONI
\usepackage{lineno}

\newcommand{\citazione}[1]{%
  \begin{quotation}
  \begin{linenumbers}
  \modulolinenumbers[5]
  \begingroup
  \setlength{\parindent}{0cm}
  \noindent #1
  \endgroup
  \end{linenumbers}
  \end{quotation}\setcounter{linenumber}{1}
  }
%%%%%%

%% HEADER E FOOTER
% rimuovere header e footer dalle pagine vuote
\usepackage{ifthen}
\makeatletter
\def\cleardoublepage{\clearpage\if@twoside \ifodd\c@page\else
    \hbox{}
    \vspace*{\fill}
    \vspace{\fill}
    \thispagestyle{empty}
    \newpage
    \if@twocolumn\hbox{}\newpage\fi\fi\fi}
\makeatother
%%%%%%

\usepackage{hyperref}
\hypersetup{%
	pdfauthor={Davide Peccioli},
	pdfsubject={Appunti UniTO},
	allcolors=black,
	citecolor=black,
	colorlinks=true, 
	bookmarksopen=true}

%% AMBIENTE DI BASE, PER DEFINIRE TUTTI GLI ALTRI
\newcounter{totale}[chapter]
\newcounter{capitoloeccolo}
\setcounter{capitoloeccolo}{0}
\newcommand{\frammento}[4]{\stepcounter{totale} 
% #1 TITOLO
% #2 RIFERIMENTO A BIBLIOGRAFIA
% #3 CONTENUTO
% #4 STRINGA UNIVOCA
	\ifx&#4&%
		% #4 is empty
		\else%
		\newcounter{frammentoA#4}\addtocounter{frammentoA#4}{\thecapitoloeccolo}
		\newcounter{frammentoB#4}\addtocounter{frammentoB#4}{\thetotale}
		\label{totale#4}% #4 is nonempty {\arabic{frammentoA#4}.\arabic{frammentoB#4}}
	\fi
	\ifx&#1&%
		\paragraph{({\thechapter.\thetotale})} #3% #1 is empty
		\ifx&#2&%
			% #2 is empty
			\else%
			\riferimento{\hyperref[totale#4]{({\thechapter.\thetotale})}}{#2}% #2 is nonempty
			\fi
		\else%
		% #1 nonempty
		\paragraph{#1. ({\thechapter.\thetotale})} #3
		\ifx&#2&%
			% #2 is empty
			\else%
			\riferimento{\hyperref[totale#4]{#1. ({\arabic{frammentoA#4}.\arabic{frammentoB#4}})}}{#2}% #2 is nonempty
		\fi	
	\fi
}
\newcounter{numeroappendice}
\newcounter{appendixever}
\setcounter{appendixever}{0}
\newcommand{\dimframmento}[2]{\stepcounter{numerodidimostrazioni}
	\ifthenelse{\theappendixever = 0}{%
	\paragraph{\hyperref[totale#1]{\textit{Dimostrazione di (\arabic{frammentoA#1}.\arabic{frammentoB#1})}}} #2}{%
	\paragraph{\hyperref[totale#1]{\textit{Dimostrazione di (\Alph{frammentoA#1}.\arabic{frammentoB#1})}}} #2
	}
}
\newcommand{\solframmento}[2]{
	\ifthenelse{\theappendixever = 0}{%
	\paragraph{\hyperref[totale#1]{\textit{Soluzione. (\arabic{frammentoA#1}.\arabic{frammentoB#1})}}} #2}{%
	\paragraph{\hyperref[totale#1]{\textit{Soluzione. (\Alph{frammentoA#1}.\arabic{frammentoB#1})}}} #2
	}
}
\newcommand{\frag}[2][]{%Comando per inserire un frammento sparso, senza fronzoli, solo contenuto ed eventuale titolo
	\frammento{}{}{#2}{#1}
}
\newcommand{\definizionegenerale}[4]{
	\paragrafo{#1}{{\itshape#2}}{#3}{#4}
}
\newcommand{\framref}[1]{%
	\hyperref[totale#1]{(\arabic{frammentoA#1}.\arabic{frammentoB#1})}%
}
%%%%%%%%%%%%%%%%%%%

%%%% UMANIZZAZIONE FRAMMENTO

\newcommand{\paragrafo}[4]{\frammento{#1}{#4}{#2}{#3}}

%-------------------------------------------------------

%% AMBIENTI PER APPUNTI (DA FRAMMENTO)

\newcommand{\esempi}[2][]{
	\frammento{Esempi}{#1}{#2}{}
}
\newcommand{\esempio}[2][]{
	\frammento{Esempio}{#1}{#2}{}
}
\newcommand{\proprieta}[2][]{
	\frammento{Proprietà}{#1}{#2}{}
}
\newcommand{\notazione}[2][]{
	\frammento{Notazione}{#1}{#2}{}
}
\newcommand{\osservazione}[2][]{
	\frammento{Osservazione}{#1}{#2}{}
}
\newcommand{\nota}[1]{
	\frammento{Nota}{#1}{}{}
}
\newcommand{\proposizione}[3][]{
	\frammento{Proposizione}{#1}{#3}{#2}
    % \stepcounter{proposizione}
	% 	\newcounter{prp#2}\addtocounter{prp#2}{\theproposizione}
	% 	\paragraph{Proposizione
	% 		\textit{p.}\roman{proposizione} \label{prp:#2} #1} 
	% 		{\itshape#3}
}
\newcommand{\dimostrazioneprop}[2]{
	\dimframmento{#1}{#2}
	% \paragraph{\textit{dim.} \hyperref[prp:#1]{(\textit{p.}\roman{prp#1})}} #2
}

\newcommand{\lemma}[3][]{
	\frammento{Lemma}{#1}{#3}{#2}
    % \stepcounter{lemma}
	% 	\newcounter{lmm#2}\addtocounter{lmm#2}{\thelemma}
	% 	\paragraph{Lemma
	% 		\textit{l.}\roman{lemma} \label{lmm:#2} #1} 
	% 		#3
}
\newcommand{\dimostrazionelem}[2]{
	\dimframmento{#1}{#2}
	% \paragraph{\textit{dim.} \hyperref[lmm:#1]{(\textit{l.}\roman{lmm#1})}} #2
}

\newcommand{\corollario}[3][]{
    \frammento{Corollario}{#1}{#3}{#2}
	% \stepcounter{corollario}
	% 	\newcounter{crl#2}\addtocounter{crl#2}{\thecorollario}
	% 	\paragraph{Corollario\label{crl:#2} #1} 
	% 	#3
}
\newcommand{\dimostrazionecrl}[2]{
	\dimframmento{#1}{#2}
	% \paragraph{\hyperref[crl:#1]{\textit{dim.} }} #2
}

%----------------------------------------------------

%%% ESERCIZI

\newcommand{\esercizio}[2]{
	\paragrafo{Esercizio}{%
		#1 %Contenuto del paragrafo 
	}{}{}
	\paragraph{\textit{Soluzione ({\thechapter.\thetotale}).}} #2
}
\newcommand{\sesercizio}[1]{
	\paragrafo{Esercizio}{%
		#1 %Contenuto del paragrafo 
	}{}{}
}

%%%%%%

%%%% TEOREMI E DEFINIZIONI

\newcounter{teorema}
\newcommand{\teorema}[3][]{
	\rteorema[#1]{#2}{#3}{}
}
\newcommand{\rteorema}[4][]{
	\stepcounter{teorema}
		\newcounter{thm#2}\addtocounter{thm#2}{\theteorema}
		\begin{tcolorbox}[parbox=false, colback=white,colframe=white!20!black,title=Teorema \Roman{teorema}. \label{thm:#2}, sharpish corners]
		%\ifthenelse{\isundefined{#1}}{\subsection*{#1}{\itshape#3}}{{\itshape#3}}
		\ifx&#1&%
		#3% #1 is empty
		\else%
		\subsection*{#1}#3% #1 is nonempty
		\fi
		%\subsection*{#1}{\itshape#3}
		\end{tcolorbox}
		\ifx&#4&%
		\else%
		\riferimento{\hyperref[thm:#2]{Teorema \Roman{thm#2}}}{#4}% #4 is nonempty
		\fi
		
}
\newcommand{\dimostrazione}[2]{\stepcounter{numerodidimostrazioni}\paragraph{\textit{\hyperref[thm:#1]{Dimostrazione di \Roman{thm#1}.}}} #2}
\newcommand{\teoref}[1]{\hyperref[thm:#1]{\Roman{thm#1}}}
\newcommand{\definizione}[1]{
	% \paragraph{Definizione #1} {\itshape#2}
	\definizionegenerale{Definizione}{#1}{}{}
}


%%%%


\newcommand{\attenzione}[1]{
	\paragrafo{Attenzione}{%
		#1 %Contenuto del paragrafo 
	}{}{}
}

\newcommand{\conseguenza}[1]{
	\paragrafo{Conseguenza}{%
		#1 %Contenuto del paragrafo 
	}{}{}
}

%%%%%%%%%%%%%%%%%%%%%%%%%%%%%%%%%%%%%%%%%%%%%%%%%%%%%%%%%%%%%%%%%% FINE AMBIENTI APPUNTI

%TOdOS e Excursus
%%
\usepackage{tcolorbox}
\tcbuselibrary{breakable}
\newcommand{\todo}[1]{%
\begin{tcolorbox}[parbox=false, colframe=red, colback=white]
	#1
\end{tcolorbox}
}
%Ambiente EXCURSUS
\newenvironment{excursus}[1]%
{	
	\begin{figure}
	\begin{tcolorbox}[colback=white!90!black,colframe=white!90!black]
	\subsubsection*{#1}
}%
{%
	\end{tcolorbox}\end{figure}
}
%%%%%%%%

%%%%%% GEOMETRY SEMPRE PER ULTIMO

\usepackage[% 
textwidth=360pt]{geometry}
\geometry{papersize={16.99cm,24.4cm}, bottom=3cm,%
heightrounded}
