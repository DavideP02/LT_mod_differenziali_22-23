\days{9 marzo 2023}
\newcommand{\settingH}{%
    Sia dato il problema di Cauchy
    \begin{align*}
        \begin{cases}
            \bm{u}'(t)=\bm{f}\left(t,\bm{u}(t)\right)\\ 
            \bm{u}(t_0)= \bm{x}_0
        \end{cases} && (t_0,\bm{x}_0) \in \Omega
    \end{align*}
    con $ \bm{f}: \Omega \subseteq \R\times \R^{n}\to \R^{n}$, continua e localmente lipschitziana nella seconda variabile e uniformemente nella prima.}

\chapter{Teorema di dipendenza continua dai dati iniziali}
\stepcounter{capitoloeccolo}
\section{Il teorema}
\paragrafo{Domanda}{%
    \settingH

    Se prendiamo il sistema di Cauchy sostituendo a $ \bm{x}_0 $ una $ \bm{x} $ vicina ad $ \bm{x}_0 $, cosa succede alla soluzione? 
}{}{}
\esempio{
    Preso il problema di Cauchy \[
        \begin{cases}
            \bm{u}'=\bm{u}\\ 
            \bm{u}(0)=\bm{x}_0
        \end{cases}
    \]si ha che la soluzione $\bm{u}_{(0,\bm{x}_0)}$ è, al variare del dato iniziale $\bm{x}_0$:
     \begin{itemize}
        \item $\bm{u}_{(0,\bm{x}_0)}(t)\equiv 0$, se $ \bm{x}_0=\bm{0} $ 
        \item $\bm{u}_{(0,\bm{x}_0)}(t)=\bm{x}_0\,e^{t} $, $ t \in \R $, se $ \bm{x}_0\neq \bm{0} $
    \end{itemize} 
    
    $\implies$ non è dunque ragionevole pensare che se $ \bm{x}\to \bm{x}_0 $ la soluzione $ \bm{u}_{(t,\bm{x})} $ si mantenga sempre vicina a $ \bm{u}_{(t,\bm{x}_0)}$.
}
\teorema[Teorema di dipendenza continua dai dati iniziali]{fsglkjnsdfglkjnsdflgkjnsdflkgjnsdflkgjjjfjfjfjfjfifroiefdfdskjncdskujhsdfalkjandxvberure}{
    \settingH 

    Sia $ I_{\max}  $ l'intervallo massimale di $ \bm{u}_{(t_0,\bm{x}_0)} $ e $ [a,b] \subset I_{\max}  $. Allora: \begin{enumerate}
        \item esiste un intorno di $ \bm{x}_0 $, $ N $, tale che per ogni $ \bm{x} \in N $ il PdC \[
            \begin{cases}
                \bm{u}'=\bm{f}\left(t,\bm{u}(t)\right)\\ 
                \bm{u}(t_0)=\bm{x}
            \end{cases}
        \]ammette un'unica soluzione il cui intervallo massimale contiene $ [a,b] $;
        \item per ogni $ \overline{\bm{x}}_0 \in N $ e per ogni $ \{\bm{x}_{k} \}_{k \in \N} \subseteq N$ con $ \bm{x}_k \to \overline{\bm{x}}_0 $ in $ \R^{n} $ la soluzione del corrispondente problema di Cauchy \[
            \begin{cases}
                \bm{u}'(t)=\bm{f}\left(t,\bm{u}(t)\right)\\ 
                \bm{u}(t_0)=\bm{x}_k
            \end{cases}
        \]converge uniformemente su $ [a,b] $ alla soluzione di \[
            \begin{cases}
                \bm{u}'(t)=\bm{f}\left(t,\bm{u}(t)\right)\\ 
                \bm{u}(t_0)= \overline{\bm{x}}_0.
            \end{cases}
        \]
    \end{enumerate}
}
\osservazione{
    La richiesta $ \bm{x}_k \to \overline{\bm{x}}_0$ implica\[
        \lim_{k\to \infty}\bm{u}_{(t_0, \bm{x}_k)}(t_0)=\bm{u}_{(t_0, \overline{\bm{x}}_0)}(t_0) = \overline{\bm{x}}_0\\
        %\bm{x}_k = \bm{u}_{(t_0, \bm{x}_k)}(t_0) \longrightarrow \bm{u}_{(t_0, \overline{\bm{x}}_0)}(t_0) = \overline{\bm{x}}_0
    \]ovvero la convergenza \emph{puntuale} della successione $ \left\{\bm{u}_{(t,\bm{x}_k)}\right\}_{k} $ in un punto.
   Il teorema afferma che la successione $ \left\{\bm{u}_{(t,\bm{x}_k)}\right\}_{k} $ converge non solo puntualmente, ma bensì uniformemente su $ [a,b] $. Un risultato più generale è:
}
\teorema[Teorema di Kamke]{dafasdfdasfadsfsadf}{
    \settingH 

    Sia $ I_{\max}  $ l'intervallo massimale di $ \bm{u}_{(t_0,\bm{x}_0)}(t) $. Fissiamo $ [a,b] \subset I_{\max}  $.

    Prendiamo \begin{itemize}
        \item $ \{t_{k} \}_{k \in \N} \subseteq \R $, $ t_{k}\to t_0  $;
        \item $ \{\bm{x}_{k} \}_{k \in \N} $, $ \bm{x}_k \to \bm{x}_0$;
        \item $ \{\bm{f}_{k} \}_{k \in \N} $ funzioni di dominio $ \Omega $ che soddisfano il teorema di esistenza e unicità locale, \[
            \bm{f}_k \to \bm{f}\quad\text{uniformemente sui compatti di }\Omega.
        \]
    \end{itemize}

    Allora definitivamente per ogni $ k $ il problema di Cauchy \[
        \begin{cases}
            \bm{u}'(t) = \bm{f}_k\left(t,\bm{u}(t)\right)\\ 
            \bm{u}(t_{k} ) = \bm{x}_k
        \end{cases}
    \]ammette un'unica soluzione definita su $ [a,b] $ e convergente uniformemente, su $ [a,b] $ stesso, alla soluzione di \[
        \begin{cases}
            \bm{u}'(t)=\bm{f}\left(t,\bm{u}(t)\right)\\ 
            \bm{u}(t_0) = \bm{x}_0.
        \end{cases}
    \]
}
Per la dimostrazione del \hyperref[fsglkjnsdfglkjnsdflgkjnsdflkgjnsdflkgjjjfjfjfjfjfifroiefdfdskjncdskujhsdfalkjandxvberure]{Teorema V} ci serviremo del
\paragrafo{Lemma di Gronwall}{%
    Sia $ \phi:[a,b]\to \R $ continua, e supponiamo che $ \exists\,A \in \R $, $ \exists\, B\ge 0 $ tali che \[
        \phi(t) \le A + B\,\int_{a}^{t}\phi(s)\,ds,\quad \forall\, t \in [a,b] 
    \]Allora \[
        \phi(t)\le A\,e^{B\,(t-a)},\quad \forall\, t \in [a,b]
    \]
}{dafkjasdlkfjnadslkfjnalskfdjnjdjsidiujhfsiosdi}{}
\dimframmento{dafkjasdlkfjnadslkfjnalskfdjnjdjsidiujhfsiosdi}{
    Sia $ \displaystyle w(t)\coloneqq A + B\,\int_{a}^{t}\phi(s)\,ds $. Per ipotesi \begin{itemize}
        \item $ \phi(t)\le w(t) $ su $ [a,b] $;
        \item $ w $ è derivabile (perché $ \phi $ è continua) e $w'(t) = B\,\phi(t)$
        
    \end{itemize}

    Prendiamo \begin{align*}
        \frac{d}{dt}\left[w(t)\,e^{-B\,(t-a)}\right] &= \left[w'(t)-B\,w(t)\right]\,e^{-B\,(t-a)}\\ 
        &= \underbrace{B}_{\ge 0}\,\underbrace{\left(\phi(t)-w(t)\right)}_{\leq 0}\,\underbrace{e^{-B\,(t-a)}}_{\ge 0}\leq  0
    \end{align*} 
    
    $\implies$ la funzione $ \displaystyle w(t)\,e^{-B\,(t-a)} $ è decrescente su $ [a,b] $ 
    
    $\implies$ è massima in $ t=a $, ovvero \[
       A \coloneq w(a)\,e^{0}\ge w(t)\,e^{-B\,(t-a)}\ge \phi(t)\,e^{-B\,(t-a)}
    \] 
    
    $\implies$ $ \displaystyle \phi(t)\le A\,e^{B\,(t-a)} $\qed
}
\dimostrazione{fsglkjnsdfglkjnsdflgkjnsdflkgjnsdflkgjjjfjfjfjfjfifroiefdfdskjncdskujhsdfalkjandxvberure}{
    \begin{enumerate}
        \item Dimostriamo per assurdo. Supponiamo che $\forall\: \varepsilon>0 $, $ \exists\,\bm{x}_{\epsilon} \in B_{ \varepsilon}(\bm{x}_0)  $ tale per cui la soluzione massimale $ \bm{u}_{ \varepsilon}(t)\coloneqq \bm{u}_{(t_0,\bm{x}_{ \varepsilon})}(t) $ di \[
            \begin{cases}
                \bm{u}'=\bm{f}\left(t,\bm{u}(t)\right)\\ 
                \bm{u}(t_0)=\bm{x}_{ \varepsilon}
            \end{cases}
        \]non sia definita su tutto $ [a,b] $. 

        Per semplicità prendo $ t_0=a $ e ``lavoro a destra''. 

        Prendiamo $ \delta>0 $ sufficientemente piccolo e \[
            k_{\delta} = \left\{(t,\bm{x}) \in \Omega: t \in [a,b], \norma{\bm{u}_0(t)-\bm{x}}<\delta\right\} 
        \]

        Sia $ [a,b_{ \varepsilon})  $ con $ b_{ \varepsilon}< b  $ l'intervallo massimale destro di $ \bm{u}_{ \varepsilon} $. 

        Necessariamente $ \bm{u}_{ \varepsilon} $ deve uscire dal compatto $ k_{\delta} $ prima di $ b_{ \varepsilon} $: \[
            \forall\, \varepsilon>0\quad \exists\, t_{ \varepsilon} \in (a,b_{ \varepsilon} ): \begin{aligned}
                \norma{ \bm{u}_{ \varepsilon}(t_{ \varepsilon} )- \bm{u}_0(t_{ \varepsilon} ) } &= \delta \\ 
                \norma{ \bm{u}_{ \varepsilon}(t )- \bm{u}_0(t) } &< \delta \quad \forall\, t \in [a,t_{ \varepsilon} )
            \end{aligned}
        \]

        Sfruttiamo il fatto che $ \bm{u}_0 $ e $ \bm{u}_{ \varepsilon} $ siano le soluzioni di problemi di Cauchy e usiamo le loro equazioni di Volterra. \[
            \bm{u}_{ \varepsilon} (t) =\bm{u}_{ \varepsilon} (a) + \int_{a}^{t} \bm{f}\left(s, \bm{u}_{ \varepsilon}(s)\right) \,ds
        \]Definiamo ora, per ogni $ t \in [a,t_{ \varepsilon} ] $, la funzione \begin{align*}
            \phi(t) &\coloneq \norma{\bm{u}_{ \varepsilon}(t)-\bm{u}_{ 0}(t)}\\ &= \norma{%
            \bm{u}_{ \varepsilon}(a) + \int_{a}^{t} \bm{f}\left(s,\bm{u}_{ \varepsilon}(s)\right) \,ds - \bm{u}_0(a) - \int_{a}^{t} \bm{f}\left(s, \bm{u}_0(s)\right)\,ds 
            }\\ 
            &\le \norma{\bm{u}_{ \varepsilon}(a)-\bm{u}_0(a)} + \int_{a}^{t}\norma{\bm{f}\left(s, \bm{u}_{ \varepsilon}(s)\right) - \bm{f}\left(s, \bm{u}_0(s)\right)}\,ds \\ 
            &\underset{\footnotemark}{\le} \parentesi{A_{ \varepsilon} }{%
                \norma{\bm{x}_{ \varepsilon} -\bm{x}_0} 
            } + \underbrace{L}_{>0}\,\int_{0}^{t}\norma{\bm{u}_{ \varepsilon}(s) - \bm{u}_0(s)}\,ds .
        \end{align*}\footnotetext{dove $ L $ è la costante di Lipschitz di $ \bm{f} $ su $ k_{\delta}  $} 

        Dunque, per il lemma di Gronwall, $ \displaystyle \phi(t)\le A_{ \varepsilon}\, e^{L(t-a)}  $. Inoltre, avendo $ \phi(t_{ \varepsilon} ) = \delta $, si ha che \[
            0< \delta \le\underbrace{A_{ \varepsilon}}_{\to 0 } \, \underbrace{e^{L(t_{ \varepsilon} -a)}}_{>0} \xlongrightarrow[]{\bm{x}_{\varepsilon}\to \bm{x}_0} 0 
        \]che è assurdo.
        \item Consideriamo $ \overline{\bm{x}}_0 \in N $ e $ \{\bm{x}_{k} \}_{k \in \N} \subseteq N$ tale che $ \bm{x}_k\longrightarrow \overline{\bm{x}}_0 $. Definiamo inoltre $ \overline{\bm{u}}(t)\coloneqq \bm{u}_{(t_0,\overline{\bm{x}}_0)}(t) $ e 
         $ \bm{u}_k (t)\coloneqq \bm{u}_{(t_0,\bm{x}_k)}(t)$.
         \begin{itemize}
            \item Sia $ \overline{K}_{\delta} $ il $ \delta $-intorno compatto di $ \overline{\bm{u}} $;
            \item per $ k $ sufficientemente grandi, il grafico di $ \bm{u}_k $ rimane in $ \overline{K}_{\delta} $ (ragionando come il punto precedente);
            \item uso il lemma di Gronwall sulla \[
                \phi(t)\coloneqq\norma{\bm{u}_k(t)-\overline{\bm{u}}(t)}
            \]e ottengo, sempre utilizzando l'equazione di Volterra \[
                \norma{\bm{u}_k(t)-\overline{\bm{u}}(t)} \le \parentesi{\to 0}{\norma{\bm{x}_k-\overline{\bm{x}}_0}}\,e^{L(b-a)}
            \]dove $ L $ è la costante di Lipschitz di $ \bm{f} $ su $ \overline{K}_{\delta}$. Così facendo: 
            \begin{align*}
                \displaystyle \norma{\bm{u}_k-\overline{\bm{u}}}_{\infty} = \max_{t \in [a,b]} \norma{\bm{u}_k(t)-\overline{\bm{u}}(t)}\le \norma{\bm{x}_k-\overline{\bm{x}}_0}\,e^{L(b-a)}\to 0  \\
                &&\qed
            \end{align*}
        \end{itemize}
    \end{enumerate}
}