\documentclass[11pt, titlepage, twoside, a4paper]{book}

% PACCHETTI FONDAMENTLAI
\usepackage[T1]{fontenc}
\usepackage[utf8]{inputenc}
\usepackage[italian]{babel}
\usepackage[babel]{csquotes}
\usepackage[style=numeric]{biblatex}
\usepackage{microtype}
\usepackage{lmodern}
\usepackage{graphicx} % inserire immagini
\usepackage{subfig} % sottofigure
\usepackage{float}
\usepackage{multicol} % due colonne
\usepackage{ulem} % sottolineare
\usepackage{soul}
\usepackage{lipsum} % lorem ipsum
\usepackage{xcolor} % colori in latex
\usepackage{parskip} % rimuove l'indentazione dei nuovi paragrafi 
\usepackage{centernot}
\usepackage[outline]{contour}\contourlength{3pt}
\usepackage{fancyhdr}
\usepackage{imakeidx}%creare indice in fondo
\usepackage{layout}
\usepackage[intlimits]{empheq} % Riquadri colorati attorno alle equazioni
\usepackage[most]{tcolorbox} % Riquadri colorati
\usepackage{ifthen} % IFTHEN
\usepackage{rotating} %figure ruotate 

% pacchetti matematica
\usepackage[intlimits]{amsmath} 
\usepackage{amssymb}
\usepackage{amsthm}
\usepackage{yhmath}
\usepackage{dsfont}
\usepackage{mathrsfs}
\usepackage{cancel} % semplificare
\usepackage{polynom} %divisione tra polinomi
\usepackage{forest} % grafi ad albero
\usepackage{booktabs} % tabelle
\usepackage{commath} %simboli e differenziali
\usepackage{bm} %bold
\usepackage[fulladjust]{marginnote} %to use marginnote for date notes
\usepackage{arrayjobx}%array
\usepackage{mathtools}
%%%%%%%%%%%%%


%%%% CONTATORE DIMOSTRAZIONI
\newcounter{numerodidimostrazioni}
\newcommand{\mostranumerodimostrazioni}{\cleardoublepage\thispagestyle{empty}\begin{center}\null\vfill \textit{Questo corso comprende {\thenumerodidimostrazioni} dimostrazioni}\vfill\null\end{center}}
%%%%

%%%% QUIVER
\newcommand{\duepunti}{\,\mathchar\numexpr"6000+`:\relax\,}
% A TikZ style for curved arrows of a fixed height, due to AndréC.
\tikzset{curve/.style={settings={#1},to path={(\tikztostart)
    .. controls ($(\tikztostart)!\pv{pos}!(\tikztotarget)!\pv{height}!270:(\tikztotarget)$)
    and ($(\tikztostart)!1-\pv{pos}!(\tikztotarget)!\pv{height}!270:(\tikztotarget)$)
    .. (\tikztotarget)\tikztonodes}},
    settings/.code={\tikzset{quiver/.cd,#1}
        \def\pv##1{\pgfkeysvalueof{/tikz/quiver/##1}}},
    quiver/.cd,pos/.initial=0.35,height/.initial=0}

% TikZ arrowhead/tail styles.
\tikzset{tail reversed/.code={\pgfsetarrowsstart{tikzcd to}}}
\tikzset{2tail/.code={\pgfsetarrowsstart{Implies[reversed]}}}
\tikzset{2tail reversed/.code={\pgfsetarrowsstart{Implies}}}
% TikZ arrow styles.
\tikzset{no body/.style={/tikz/dash pattern=on 0 off 1mm}}
%%%%%%%%%%

%%% HEADER e FOOTER
%% HEAD and FOOT
\newcommand{\datainizion}{testo}
\newcommand{\hdrnew}[1]{
	\fancyhead{} % cancella tutti i campi
	\fancyfoot{}
	\fancyhead[RO,LE]{\datainizion}
	\fancyhead[RE]{#1}
	\fancyhead[LO]{\nouppercase{\leftmark}}
	% \fancyfoot[LE,RO]{\thepage}
	\fancyfoot[C]{\thepage}
	% \fancyfoot[CO,RE]{Per: Dean A. Smith}
	\renewcommand{\headrulewidth}{0.4pt}
	\renewcommand{\footrulewidth}{0pt}
	\pagestyle{fancy}
}
\newcommand{\days}[1]{\renewcommand{\datainizion}{#1}}
%%%%%%%%

%% DEFINIZIONI COMANDI MATEMATICI
\DeclareMathOperator{\epi}{Epi}
\DeclareMathOperator{\cl}{cl}
\DeclareMathOperator{\graph}{graph}
\DeclareMathOperator{\arcsec}{arcsec}
\DeclareMathOperator{\arccot}{arccot}
\DeclareMathOperator{\arccsc}{arccsc}
\DeclareMathOperator{\spettro}{Spettro}
\DeclareMathOperator{\nulls}{nullspace}
\DeclareMathOperator{\dom}{dom}
\DeclareMathOperator{\ar}{ar}
\DeclareMathOperator{\const}{Const}
\DeclareMathOperator{\fun}{Fun}
\DeclareMathOperator{\rel}{Rel}
\DeclareMathOperator{\altezza}{ht}
\let\det\relax %TOGLIE LA DEFINIZIONE SU "\det"
\DeclareMathOperator{\det}{det}
\DeclareMathOperator{\End}{End}
\DeclareMathOperator{\gl}{GL}
\DeclareMathOperator{\Id}{Id}
\DeclareMathOperator{\id}{Id}
\DeclareMathOperator{\I}{\mathds{1}}
\DeclareMathOperator{\II}{II}
\DeclareMathOperator{\rank}{rank}
\DeclareMathOperator{\tr}{tr}
\DeclareMathOperator{\tc}{t.c.}
\DeclareMathOperator{\T}{T}
\newcommand{\R}{\mathds{R}}
\newcommand{\K}{\mathds{K}}
\newcommand{\Q}{\mathds{Q}}
\newcommand{\N}{\mathds{N}}
\newcommand{\C}{\mathds{C}}
\newcommand{\Z}{\mathds{Z}}
\newcommand{\rmn}{\R^{m,n}}
\renewcommand{\tilde}[1]{\widetilde{#1}}
\renewcommand{\parallel}{\mathrel{/\mkern-5mu/}}
\newcommand{\parti}[1]{\wp (#1)}
%tramite i prossimi due comandi posso decidere come scrivere i logaritmi naturali in tutti i documenti: ho infatti eliminato qualsiasi differenza tra "ln" e "log": se si vuole qualcosa di diverso bisogna inserire manualmente il tutto
\let\ln\relax
\DeclareMathOperator{\ln}{log}
\let\log\relax
\DeclareMathOperator{\log}{log}
%%%%%%

%% NUOVI COMANDI
\newcommand{\straniero}[1]{\textit{#1}} %parole straniere
\newcommand{\titolo}[1]{\textsc{#1}} %titoli
\newcommand{\qedd}{\tag*{$\blacksquare$}} %qed per ambienti matemastici
\renewcommand{\qedsymbol}{$\blacksquare$} %modifica colore qed
\newcommand{\ooverline}[1]{\overline{\overline{#1}}}
\newcommand{\circoletto}[1]{\left(#1\right)^\text{o}}
%
\newcommand{\qmatrice}[1]{\begin{pmatrix}
#1_{11} & \cdots & #1_{1n}\\
\vdots & \ddots & \vdots \\
#1_{m1} & \cdots & #1_{mn}
\end{pmatrix}}
%
\newcommand{\parentesi}[2]{%
\underset{#1}{\underbrace{#2}}%
}
%
\newcommand{\norma}[1]{% Norma 
\left\lVert#1\right\rVert%
}
\newcommand{\scalare}[2]{% Scalare
\left\langle #1, #2\right\rangle
}
%%%%%

%% DEFINIZIONE NUMERAZIONE EQUAZIONI
\renewcommand{\theequation}{\thechapter.\arabic{equation}}
\numberwithin{equation}{chapter}
%%%%%%%%%%%%%%%%%

%% ENUMERATE romani
\usepackage{enumitem} %numeri romani come enumerate
\newenvironment{romanen}{\begin{enumerate}[label={\itshape\roman{*}}., ref=(\roman{*})]}{\end{enumerate}}
%%%%%

%% RIFERIMENTI A FINE DOC
\newarray\Riferimentiname
\newarray\Riferimentipoint
\newcounter{Riferimentinum}
\newcommand{\riferimento}[2]{\stepcounter{Riferimentinum}% Questo comando aggiunge alla lista dei riferimenti un elemento (uno al nome e uno alla descrizione): deve essere usato in ogni teorema/enunciato che faccia riferimento a qualcosa nel testo.
\Riferimentiname(\theRiferimentinum)={#1}% 
\Riferimentipoint(\theRiferimentinum)={#2}
}
\newcounter{variabile}
\newcommand{\stampairiferimenti}{%Questo comando va lanciato alla fine, e stampa tutti i riferimenti necessari
	\setcounter{variabile}{0}
	\cleardoublepage
	%\stepcounter{capitoloeccolo}%TODO capire se serve
	\chapter*{Riferimenti in bibliografia}
	\begin{itemize}
		\whiledo{\thevariabile < \theRiferimentinum}{\stepcounter{variabile}\item \textbf{\Riferimentiname(\thevariabile):} \Riferimentipoint(\thevariabile).}
	\end{itemize}
}
%%%%%

%% RESTRIZIONI
\newcommand{\referenze}[2]{
	\phantomsection{}#2\textsuperscript{\textcolor{blue}{\textbf{#1}}}
}
\def\restriction#1#2{\mathchoice
              {\setbox1\hbox{${\displaystyle #1}_{\scriptstyle #2}$}
              \restrictionaux{#1}{#2}}
              {\setbox1\hbox{${\textstyle #1}_{\scriptstyle #2}$}
              \restrictionaux{#1}{#2}}
              {\setbox1\hbox{${\scriptstyle #1}_{\scriptscriptstyle #2}$}
              \restrictionaux{#1}{#2}}
              {\setbox1\hbox{${\scriptscriptstyle #1}_{\scriptscriptstyle #2}$}
              \restrictionaux{#1}{#2}}}
\def\restrictionaux#1#2{{#1\,\smash{\vrule height .8\ht1 depth .85\dp1}}_{\,#2}} 
%%%%%%%%%%%

%% SEZIONE GRAFICA
\usepackage{tikz}
\usetikzlibrary{matrix, patterns, calc, decorations.pathreplacing, hobby, decorations.markings, decorations.pathmorphing, babel}
\usepackage{tikz-3dplot}
\usepackage{mathrsfs} %per geogebra
\usepackage{tikz-cd}
\tikzset
{
  %surface/.style={fill=black!10, shading=ball,fill opacity=0.4},
  plane/.style={black,pattern=north east lines},
  curve/.style={black,line width=0.5mm},
  dritto/.style={decoration={markings,mark=at position 0.5 with {\arrow{Stealth}}}, postaction=decorate},
  rovescio/.style={decoration={markings,mark=at position 0.5 with {\arrow{Stealth[reversed]}}}, postaction=decorate}
}
\usepackage{pgfplots} % stampare le funzioni
	\pgfplotsset{/pgf/number format/use comma,compat=1.15}
	%\pgfplotsset{compat=1.15} %per geogebra
	\usepgfplotslibrary{fillbetween, polar}
%%%%%%

%% NOTE A PIÉ PAGINA
\usepackage[hang, perpage, symbol*, stable, bottom]{footmisc} %per le note a pié pagina
\footnotemargin=0.8em
\DefineFNsymbolsTM{myfnsymbols}{% def. from footmisc.sty "bringhurst" symbols
  \textdagger    \dagger
  \textdaggerdbl \ddagger
  \textsection   \mathsection
  \textbardbl    \|%
  \textparagraph \mathparagraph
  \textdagger\textdagger \dagger\dagger
  \textdaggerdbl\textdaggerdbl \ddagger\ddagger
  \textsection\textsection \mathsection\mathsection
  \textparagraph\textparagraph \mathparagraph\mathparagraph
}%
\setfnsymbol{myfnsymbols}
%comandi per footnotemark consecutivi
	\newcommand{\footnotemarkk}[1]{%
		\hyperref[#1]{\begin{NoHyper}
			\footnotemark
		\end{NoHyper}}}
	\newcommand{\consecfoottext}[3]{
		\addtocounter{footnote}{-#1}
		\footnotetext{#3\label{#2}}
		\addtocounter{footnote}{#1}
	}

	%esempio:
	% \mathscr{B}=\{\underset{\footnotemarkk{e1}}{\underbrace{v_1, \cdots, v_{k_1}}}, \underset{\footnotemarkk{e2}}{\underbrace{v_{k_1+1} , \cdots, v_{k_2+k_1}}}, \cdots, \underset{\footnotemark}{\underbrace{v_{k_{l-1}+kl }}}\}
	% \]
	% \consecfoottext{2}{e1}{autovettori rispetto a $ \lambda_1 $}
	% \consecfoottext{1}{e2}{autovettori rispetto a $ \lambda_2 $}
	% \footnotetext{autovettore rispetto a $ \lambda_{l}  $}
	% \todo{Manca matrice}
%%%%%%

%% CITAZIONI
\usepackage{lineno}

\newcommand{\citazione}[1]{%
  \begin{quotation}
  \begin{linenumbers}
  \modulolinenumbers[5]
  \begingroup
  \setlength{\parindent}{0cm}
  \noindent #1
  \endgroup
  \end{linenumbers}
  \end{quotation}\setcounter{linenumber}{1}
  }
%%%%%%

%% HEADER E FOOTER
% rimuovere header e footer dalle pagine vuote
\usepackage{ifthen}
\makeatletter
\def\cleardoublepage{\clearpage\if@twoside \ifodd\c@page\else
    \hbox{}
    \vspace*{\fill}
    \vspace{\fill}
    \thispagestyle{empty}
    \newpage
    \if@twocolumn\hbox{}\newpage\fi\fi\fi}
\makeatother
%%%%%%

\usepackage{hyperref}
\hypersetup{%
	pdfauthor={Davide Peccioli},
	pdfsubject={Appunti UniTO},
	allcolors=black,
	citecolor=black,
	colorlinks=true, 
	bookmarksopen=true}

%% AMBIENTE DI BASE, PER DEFINIRE TUTTI GLI ALTRI
\newcounter{totale}[chapter]
\newcounter{capitoloeccolo}
\setcounter{capitoloeccolo}{0}
\newcommand{\frammento}[4]{\stepcounter{totale} 
% #1 TITOLO
% #2 RIFERIMENTO A BIBLIOGRAFIA
% #3 CONTENUTO
% #4 STRINGA UNIVOCA
	\ifx&#4&%
		% #4 is empty
		\else%
		\newcounter{frammentoA#4}\addtocounter{frammentoA#4}{\thecapitoloeccolo}
		\newcounter{frammentoB#4}\addtocounter{frammentoB#4}{\thetotale}
		\label{totale#4}% #4 is nonempty {\arabic{frammentoA#4}.\arabic{frammentoB#4}}
	\fi
	\ifx&#1&%
		\paragraph{({\thechapter.\thetotale})} #3% #1 is empty
		\ifx&#2&%
			% #2 is empty
			\else%
			\riferimento{\hyperref[totale#4]{({\thechapter.\thetotale})}}{#2}% #2 is nonempty
			\fi
		\else%
		% #1 nonempty
		\paragraph{#1. ({\thechapter.\thetotale})} #3
		\ifx&#2&%
			% #2 is empty
			\else%
			\riferimento{\hyperref[totale#4]{#1. ({\arabic{frammentoA#4}.\arabic{frammentoB#4}})}}{#2}% #2 is nonempty
		\fi	
	\fi
}
\newcounter{numeroappendice}
\newcounter{appendixever}
\setcounter{appendixever}{0}
\newcommand{\dimframmento}[2]{\stepcounter{numerodidimostrazioni}
	\ifthenelse{\theappendixever = 0}{%
	\paragraph{\hyperref[totale#1]{\textit{Dimostrazione di (\arabic{frammentoA#1}.\arabic{frammentoB#1})}}} #2}{%
	\paragraph{\hyperref[totale#1]{\textit{Dimostrazione di (\Alph{frammentoA#1}.\arabic{frammentoB#1})}}} #2
	}
}
\newcommand{\solframmento}[2]{
	\ifthenelse{\theappendixever = 0}{%
	\paragraph{\hyperref[totale#1]{\textit{Soluzione. (\arabic{frammentoA#1}.\arabic{frammentoB#1})}}} #2}{%
	\paragraph{\hyperref[totale#1]{\textit{Soluzione. (\Alph{frammentoA#1}.\arabic{frammentoB#1})}}} #2
	}
}
\newcommand{\frag}[2][]{%Comando per inserire un frammento sparso, senza fronzoli, solo contenuto ed eventuale titolo
	\frammento{}{}{#2}{#1}
}
\newcommand{\definizionegenerale}[4]{
	\paragrafo{#1}{{\itshape#2}}{#3}{#4}
}
\newcommand{\framref}[1]{%
	\hyperref[totale#1]{(\arabic{frammentoA#1}.\arabic{frammentoB#1})}%
}
%%%%%%%%%%%%%%%%%%%

%%%% UMANIZZAZIONE FRAMMENTO

\newcommand{\paragrafo}[4]{\frammento{#1}{#4}{#2}{#3}}

%-------------------------------------------------------

%% AMBIENTI PER APPUNTI (DA FRAMMENTO)

\newcommand{\esempi}[2][]{
	\frammento{Esempi}{#1}{#2}{}
}
\newcommand{\esempio}[2][]{
	\frammento{Esempio}{#1}{#2}{}
}
\newcommand{\proprieta}[2][]{
	\frammento{Proprietà}{#1}{#2}{}
}
\newcommand{\notazione}[2][]{
	\frammento{Notazione}{#1}{#2}{}
}
\newcommand{\osservazione}[2][]{
	\frammento{Osservazione}{#1}{#2}{}
}
\newcommand{\nota}[1]{
	\frammento{Nota}{#1}{}{}
}
\newcommand{\proposizione}[3][]{
	\frammento{Proposizione}{#1}{#3}{#2}
    % \stepcounter{proposizione}
	% 	\newcounter{prp#2}\addtocounter{prp#2}{\theproposizione}
	% 	\paragraph{Proposizione
	% 		\textit{p.}\roman{proposizione} \label{prp:#2} #1} 
	% 		{\itshape#3}
}
\newcommand{\dimostrazioneprop}[2]{
	\dimframmento{#1}{#2}
	% \paragraph{\textit{dim.} \hyperref[prp:#1]{(\textit{p.}\roman{prp#1})}} #2
}

\newcommand{\lemma}[3][]{
	\frammento{Lemma}{#1}{#3}{#2}
    % \stepcounter{lemma}
	% 	\newcounter{lmm#2}\addtocounter{lmm#2}{\thelemma}
	% 	\paragraph{Lemma
	% 		\textit{l.}\roman{lemma} \label{lmm:#2} #1} 
	% 		#3
}
\newcommand{\dimostrazionelem}[2]{
	\dimframmento{#1}{#2}
	% \paragraph{\textit{dim.} \hyperref[lmm:#1]{(\textit{l.}\roman{lmm#1})}} #2
}

\newcommand{\corollario}[3][]{
    \frammento{Corollario}{#1}{#3}{#2}
	% \stepcounter{corollario}
	% 	\newcounter{crl#2}\addtocounter{crl#2}{\thecorollario}
	% 	\paragraph{Corollario\label{crl:#2} #1} 
	% 	#3
}
\newcommand{\dimostrazionecrl}[2]{
	\dimframmento{#1}{#2}
	% \paragraph{\hyperref[crl:#1]{\textit{dim.} }} #2
}

%----------------------------------------------------

%%% ESERCIZI

\newcommand{\esercizio}[2]{
	\paragrafo{Esercizio}{%
		#1 %Contenuto del paragrafo 
	}{}{}
	\paragraph{\textit{Soluzione ({\thechapter.\thetotale}).}} #2
}
\newcommand{\sesercizio}[1]{
	\paragrafo{Esercizio}{%
		#1 %Contenuto del paragrafo 
	}{}{}
}

%%%%%%

%%%% TEOREMI E DEFINIZIONI

\newcounter{teorema}
\newcommand{\teorema}[3][]{
	\rteorema[#1]{#2}{#3}{}
}
\newcommand{\rteorema}[4][]{
	\stepcounter{teorema}
		\newcounter{thm#2}\addtocounter{thm#2}{\theteorema}
		\begin{tcolorbox}[parbox=false, colback=white,colframe=white!20!black,title=Teorema \Roman{teorema}. \label{thm:#2}, sharpish corners]
		%\ifthenelse{\isundefined{#1}}{\subsection*{#1}{\itshape#3}}{{\itshape#3}}
		\ifx&#1&%
		#3% #1 is empty
		\else%
		\subsection*{#1}#3% #1 is nonempty
		\fi
		%\subsection*{#1}{\itshape#3}
		\end{tcolorbox}
		\ifx&#4&%
		\else%
		\riferimento{\hyperref[thm:#2]{Teorema \Roman{thm#2}}}{#4}% #4 is nonempty
		\fi
		
}
\newcommand{\dimostrazione}[2]{\stepcounter{numerodidimostrazioni}\paragraph{\textit{\hyperref[thm:#1]{Dimostrazione di \Roman{thm#1}.}}} #2}
\newcommand{\teoref}[1]{\hyperref[thm:#1]{\Roman{thm#1}}}
\newcommand{\definizione}[1]{
	% \paragraph{Definizione #1} {\itshape#2}
	\definizionegenerale{Definizione}{#1}{}{}
}


%%%%


\newcommand{\attenzione}[1]{
	\paragrafo{Attenzione}{%
		#1 %Contenuto del paragrafo 
	}{}{}
}

\newcommand{\conseguenza}[1]{
	\paragrafo{Conseguenza}{%
		#1 %Contenuto del paragrafo 
	}{}{}
}

%%%%%%%%%%%%%%%%%%%%%%%%%%%%%%%%%%%%%%%%%%%%%%%%%%%%%%%%%%%%%%%%%% FINE AMBIENTI APPUNTI

%TOdOS e Excursus
%%
\usepackage{tcolorbox}
\tcbuselibrary{breakable}
\newcommand{\todo}[1]{%
\begin{tcolorbox}[parbox=false, colframe=red, colback=white]
	#1
\end{tcolorbox}
}
%Ambiente EXCURSUS
\newenvironment{excursus}[1]%
{	
	\begin{figure}
	\begin{tcolorbox}[colback=white!90!black,colframe=white!90!black]
	\subsubsection*{#1}
}%
{%
	\end{tcolorbox}\end{figure}
}
%%%%%%%%

%%%%%% GEOMETRY SEMPRE PER ULTIMO

\usepackage[% 
textwidth=360pt]{geometry}
\geometry{papersize={16.99cm,24.4cm}, bottom=3cm,%
heightrounded}


\hypersetup{
    pdfauthor={Davide Peccioli},
    pdftitle={Modelli Differenziali},
}

% \geometry{showframe}
\usetikzlibrary{external}
\tikzexternalize[prefix=figures/] 

\bibliography{mod_diff_biblio}

\begin{document}

\frontmatter
% \layout

\begin{titlepage}
\null
\vfill
\begin{center}
{\Huge \textbf{Modelli Differenziali}}\\
\vspace{1em}
{\large Davide Peccioli}\\
\vspace{0.6em}
{\large Anno accademico 2022-2023}
\vfill
Università degli studi di Torino
\end{center}
\end{titlepage}
{\pagestyle{empty}
\null\cleardoublepage}


\fancyhead{} % cancella tutti i campi
\fancyfoot{}
\fancyhead[RE]{Modelli Differenziali}
\fancyhead[LO]{Indice}
\fancyfoot[C]{\thepage}
\renewcommand{\headrulewidth}{0.4pt}
\renewcommand{\footrulewidth}{0pt}
\pagestyle{fancy}

\tableofcontents\cleardoublepage

\hdrnew{Modelli Differenziali}

\mainmatter

% \part{Complementi di Teoria delle E.D.O.}

\stepcounter{capitoloeccolo}\chapter{Ripasso E.D.O.}
\days{23 febbraio}
%% BEGIN Problema di Cauchy
\paragrafo{Problema di Cauchy}{%
    Data una funzione a valori in $ \R^{n} $, $ \bm{f}=\bm{f}(t,\bm{x}) $, con \begin{itemize}
        \item $ r \in \R  $, $ \bm{x} \in \R^{n} $ o, più precisamente
        \item $ (t,\bm{x}) \in \Omega \subseteq \R\times \R^{n}$, $ \Omega $ aperto
    \end{itemize}ci si chiede sotto quali condizioni su $ \bm{f} $ il Problema di Cauchy \[
        \begin{cases}
            \bm{u}'(t)= \bm{f}\left(t, \bm{u}(t)\right)\\ 
            \bm{u}(t_0) = \bm{x}_0
        \end{cases}
    \]ammette \emph{almeno} una soluzione o ammette \emph{esattamente} una soluzione, al variare della condizione iniziale $ (t_0, \bm{x}_0) \in \Omega $.
}{}{}
%% END
%% BEGIN Teorema di Peano
\teorema[Teorema di Peano]{ljkndasflkjnasdlfkjndsalkfjndafs}{
    Se $ \bm{f} $ è continua su $ \Omega $, allora per ogni punto $ (t_0,\bm{x}_0) \in \Omega$ esiste un intorno di $ t_0 $ nel quale è definita \emph{almeno} una soluzione del Problema di Cauchy \[
        \begin{cases}
            \bm{u}'(t)= \bm{f}\left(t, \bm{u}(t)\right)\\ 
            \bm{u}(t_0) = \bm{x}_0
        \end{cases}
    \]
}
%% END
%% BEGIN Pennello di Peano
\paragrafo{Pennello di Peano}{%
    Se si dimostra che il problema di Cauchy ammette due soluzioni distinte, allora in realtà ne ha infinite.
}{}{}
\esempio{
    Consideriamo il problema di Cauchy \[
        \begin{cases}
            u'(t)=\sqrt[3]{u(t)}\\ 
            u(0)=0
        \end{cases}
    \]Questo problema ammette certamente la soluzione $ u \equiv 0 $, ma anche le soluzioni \[
        u_{0}^{\pm} =\begin{cases}
            \pm \displaystyle \left(\frac{2}{3}\, t\right)^{\frac{3}{2}} & t\ge 0\\ 
            0 & t<0
        \end{cases} 
    \]
}
%% END
%% BEGIN Teorema di Cauchy-Lipschitz
\teorema[Teorema di Cauchy-Lipschitz]{ldksnaflkjnadflkjadnsflajknkljndasf}{
    Se $ \bm{f} $ \begin{itemize}
        \item è continua,
        \item è localmente lipschitziana rispetto alla seconda variabile e uniformemente nella prima\footnote{Ovvero \[
            \forall\, K \subset \Omega,\quad \exists\,L>0:\quad \norma{\bm{f}(t,\bm{x})-f(t,\bm{y})}\le L\,\norma{\bm{x}-\bm{y}}\qquad \forall\,(t,\bm{x}), (t,\bm{y}) \in K.
        \]}
    \end{itemize}allora per ogni punto $ (t_0,\bm{x}_0) \in \Omega $ esiste un intorno di $ t_0 $, $ [t_0-\delta,t_0+\delta] $ nel quale è definita un'unica soluzione del problema di Cauchy \[
        \begin{cases}
            \bm{u}'(t)= \bm{f}\left(t, \bm{u}(t)\right)\\ 
            \bm{u}(t_0) = \bm{x}_0
        \end{cases}
    \]
}
\dimostrazione{ldksnaflkjnadflkjadnsflajknkljndasf}{
    La dimostrazione si articola nei seguenti passaggi: \begin{itemize}
        \item si considera l'\emph{equazione di Volterra} \[
            \bm{u}(t)= \bm{u}_0 + \int_{t_0}^{t} \bm{f}\left(s,\bm{u}(s)\right)\,ds
        \]e si mostra che quest’ultima ammette unica soluzione continua in un intorno di $ t_0 $;
        \item esistenza e unicità dell’equazione di Volterra si dimostrano applicando il Teorema delle contrazioni di Banach-Cacioppoli;
        \item la seguente successione \[
            \bm{u}_0(t) \equiv \bm{u}_0,\qquad \bm{u}_n(t) = \bm{u}_0 + \int_{t_0}^{t} \bm{f}\left(s,\bm{u}_{n-1}(s)\right)\,ds,\qquad \forall\, n \in \N
        \]risulta convergere uniformemente alla soluzione dell’equazione di Volterra e dunque all’unica soluzione del Problema di Cauchy.
    \end{itemize}
}
\osservazione{
    L'intervallo di definizione della soluzione del problema di Cauchy è certamente più ampio di $ [t_0-\delta,t_0+\delta] $: possiamo infatti applicare lo stesso teorema di esistenza ed unicità locale ai problemi di Cauchy con condizioni iniziali \[
        \left(t_0\pm\delta, \bm{u}(t_0 \pm \delta)\right) \in \Omega
    \]ed iterare questo procedimento. 

    In generale, quindi, esiste un intervallo $ (T_{\min}, T_{\max}  ) $ per la soluzione $ \bm{u}(t;t_0,\bm{u}_0) $, che per costruzione non può che essere aperto e connesso.
}
%% END
%% BEGIN Teorema di Esistenza Globale
\teorema[Teorema di Esistenza Globale]{sasdfkjnasdflkjnasdlkfjnfasd}{
    Sia $ f $ tale che le ipotesi del teorema di Cauchy-Lipschitz siano soddisfatte. 

    Sia inoltre $ S=(a,b)\times \R^{n} $ una striscia tale che $ \overline{S} \subseteq \Omega $. Se esiste una coppia di costanti positive $ k_1,k_2 $ tali per cui \[
        \norma{\bm{f}(t,\bm{x})} \le k_1+k_2\,\norma{\bm{x}},\quad \forall\, (t,\bm{x}) \in \overline{S}
    \]allora, per ogni $ (t_0,\bm{u}_0) \in S $, l'intervallo massimale della soluzione $ \bm{u}(t;t_0,u_0) $ contiene l'intervallo $ [a,b] $.
}
\paragrafo{}{%
    Questo teorema è ciò che garantisce l'esistenza globale per i sistemi lineari del tipo \[
        \bm{x}'(t)= A(t)\, \bm{x}(t)+ \bm{b}(t)
    \]con $ A(t) $ matrice $ n\times n $
}{}{}
%% END
%% BEGIN Generalizzazione esplosione tempo finito
\teorema{fasdkjnfalskjnfalkjdnflaj}{
    Sia $ \bm{f} $ tale che le ipotesi del teorema di Cauchy-Lipschitz siano soddisfatte. Sia $ K \subset \subset \Omega $\footnote{Ovvero $ K $ contenuto in $ \Omega $ e $ K $ compatto.} $ (t_0,\bm{u}_0) \in K $ e $ (T_{\min}, T_{\max}  ) $ l'intervallo massimale di definizione di $ \bm{u}(t;t_0,\bm{u}(0)) $.

    Allora il grafico di $ \bm{u} $ esce definitivamente da $ K $ quando $ t \to T_{\min}^{+}  $ o $ t \to T_{\max}^{-}  $
}
%% END
%% BEGIN Esplosione in tempo finito
\paragrafo{Corollario - Esplosione in tempo finito}{%
    Sia $ \bm{f} $ tale che soddisfi le condizioni del teorema di Cauchy-Lipschitz, e sia $ (T_{\min},T_{\max})$ l'intervallo massimale di $ \bm{u}(t;t_0,u_0) $.

    Se $ T_{\max} < + \infty   $ allora \[
        \lim_{t\to T_{\max}^{-} } \norma{\bm{u}(t;t_0,u_0)} = + \infty
    \]se tale limite esiste. Analogamente se $ T_{\min}> - \infty  $.
}{}{}
%% END
\paragrafo{Attenzione}{%
    Per i prossimi risultati si consideri $ \Omega=\R\times \R^{n} $
}{}{}
%% BEGIN Limitatezza a priori
\paragrafo{Corollario - Limitatezza a priori}{%
    Sia $ \bm{f} $ tale che le ipotesi del teorema di Cauchy-Lipschitz siano soddisfatte. Sia $ (T_{\min},T_{\max}) $ l'intervallo massimale di $ \bm{u}(t;t_0, \bm{u}_0).$

    Se esiste $ C>0 $ tale per cui \[
        \norma{\bm{u}(t;t_0,\bm{u}_0)}\le C,\qquad \forall\, t \in [t_0,T_{\max})
    \]allora $ T_{\max} = + \infty  $
}{dakfjhbaksjdhfbaksjdhfbakjhbkjhb}{}
%% END % Lezione 1 - Ripasso EDO
\days{9 marzo 2023}
\newcommand{\settingH}{%
    Sia dato il problema di Cauchy \[
        \begin{cases}
            \bm{u}'(t)=\bm{f}\left(t,\bm{u}(t)\right)\\ 
            \bm{u}(t_0)= \bm{x}_0
        \end{cases}
    \]con $ \bm{f}: \Omega \subseteq \R\times \R^{n}\to \R^{n}$, $ (t_0,\bm{x}_0) \in \Omega $ continua e localmente lipschitziana nella seconda variabile e uniformemente nella prima.}

\chapter{Teorema di dipendenza continua dai dati iniziali}
\stepcounter{capitoloeccolo}
\section{Il teorema}
\paragrafo{Domanda}{%
    \settingH

    Se prendiamo il sistema di Cauchy sostituendo a $ \bm{x}_0 $ una $ \bm{x} $ vicina ad $ \bm{x}_0 $, cosa succede alla soluzione? 
}{}{}
\esempio{
    Preso il problema di Cauchy \[
        \begin{cases}
            \bm{u}'=\bm{u}\\ 
            \bm{u}(0)=\bm{x}
        \end{cases}
    \]si ha che la soluzione è $ \bm{u}(t,\bm{x})=\bm{x}\,e^{t} $, $ t \in \R $. \begin{itemize}
        \item Se $ \bm{x}=\bm{0} $ $ \leadsto $ $ \bm{u}(t,\bm{0})\equiv 0 $;
        \item se $ \bm{x}\neq \bm{0} $ è una esponenziale
    \end{itemize} 
    
    $\implies$ non è ragionevole pensare che se $ \bm{x}\to \bm{x}_0 $ la soluzione $ \bm{u}(t,\bm{x}) $ si mantenga sempre vicina a $ \bm{u}(t,\bm{x}_0)$.
}
\teorema[Teorema di dipendenza continua dai dati iniziali]{fsglkjnsdfglkjnsdflgkjnsdflkgjnsdflkgjjjfjfjfjfjfifroiefdfdskjncdskujhsdfalkjandxvberure}{
    \settingH 

    Sia $ I_{\max}  $ l'intervallo massimale di $ \bm{u}(t,\bm{x}_0) $. Fissiamo $ [a,b] \subset I_{\max}  $. Allora: \begin{enumerate}
        \item esiste un intorno di $ \bm{x}_0 $, $ N $, tale che per ogni $ \bm{x} \in N $ la soluzione di \[
            \begin{cases}
                \bm{u}'=\bm{f}\left(t,\bm{u}(t)\right)\\ 
                \bm{u}(t_0)=\bm{x}
            \end{cases}
        \]ammette un'unica soluzione il cui intervallo massimale contiene $ [a,b] $;
        \item per ogni $ \overline{\bm{x}}_0 \in N $ e per ogni $ \{\bm{x}_{k} \}_{k \in \N} \subseteq N$ con $ \bm{x}_k \to \overline{\bm{x}}_0 $ in $ \R^{n} $ la soluzione del corrispondente problema di Cauchy \[
            \begin{cases}
                \bm{u}'(t)=\bm{f}\left(t,\bm{u}(t)\right)\\ 
                \bm{u}(t_0)=\bm{x}_k
            \end{cases}
        \]converge uniformemente su $ [a,b] $ alla soluzione di \[
            \begin{cases}
                \bm{u}'(t)=\bm{f}\left(t,\bm{u}(t)\right)\\ 
                \bm{u}(t_0)= \bm{x}_0.
            \end{cases}
        \]
    \end{enumerate}
}
\osservazione{
    La richiesta $ \bm{x}_k \to \overline{\bm{x}}_0$ implica\[
        \bm{x}_k = \bm{u}(t_0, \bm{x}_k) \longrightarrow \bm{u}(t_0, \overline{\bm{x}}_0) = \overline{\bm{x}}_0
    \]ovvero la convergenza \emph{puntuale} della successione $ \left\{\bm{u}(t,\bm{x}_k)\right\}_{k} $ in un punto;
    
    $\implies$ la successione $ \left\{\bm{u}(t,\bm{x}_k)\right\}_{k} $ converge uniformemente su $ [a,b] $.
}
\teorema[Teorema di Kamke]{dafasdfdasfadsfsadf}{
    \settingH 

    Sia $ I_{\max}  $ l'intervallo massimale di $ \bm{u}(t,\bm{x}_0) $. Fissiamo $ [a,b] \subset I_{\max}  $.

    Prendiamo \begin{itemize}
        \item $ \{t_{k} \}_{k \in \N} \subseteq \R $, $ t_{k}\to t_0  $;
        \item $ \{\bm{x}_{k} \}_{k \in \N} $, $ \bm{x}_k \to \bm{x}_0$;
        \item $ \{\bm{f}_{k} \}_{k \in \N} $ successioni di dominio $ \Omega $ che soddisfano il teorema di esistenza e unicità locale, \[
            \bm{f}_k \to \bm{f}\quad\text{uniformemente sui compatti di }\Omega.
        \]
    \end{itemize}

    Allora definitivamente per ogni $ k $ il problema di Cauchy \[
        \begin{cases}
            \bm{u}'(t) = \bm{f}_k\left(t,\bm{u}(t)\right)\\ 
            \bm{u}'(t_{k} ) = \bm{x}_k
        \end{cases}
    \]ammette un'unica soluzione definita su $ [a,b] $ e convergente uniformemente su $ [a,b] $ alla soluzione di \[
        \begin{cases}
            \bm{u}'(t)=\bm{f}\left(t,\bm{u}(t)\right)\\ 
            \bm{u}(t_0) = \bm{x}_0.
        \end{cases}
    \]
}
\paragrafo{Lemma di Gronwall}{%
    Sia $ \phi:[a,b]\to \R $ continua, e supponiamo che $ \exists\,A \in \R $, $ \exists\, B\ge 0 $ tali che \[
        \phi(t) \le A + B\,\int_{a}^{t}\phi(s)\,ds,\quad \forall\, t \in [a,b] 
    \]Allora \[
        \phi(t)\le A\,e^{B\,(t-a)},\quad \forall\, t \in [a,b]
    \]
}{dafkjasdlkfjnadslkfjnalskfdjnjdjsidiujhfsiosdi}{}
\dimframmento{dafkjasdlkfjnadslkfjnalskfdjnjdjsidiujhfsiosdi}{
    Sia $ \displaystyle w(t)\coloneqq A + B\,\int_{a}^{t}\phi(s)\,ds $. Per ipotesi \begin{itemize}
        \item $ \phi(t)\le w(t) $ su $ [a,b] $;
        \item $ w $ è derivabile (perché $ \phi $ è continua) e \[
            w'(t) = B\,\phi(t)
        \]
    \end{itemize}

    Prendiamo \begin{align*}
        \frac{d}{dt}\left[w(t)\,e^{-B\,(t-a)}\right] &= \left[w'(t)-B\,w(t)\right]\,e^{-B\,(t-a)}\\ 
        &= B\,\left(\phi(t)-w(t)\right)\,e^{-B\,(t-a)}\ge 0
    \end{align*} 
    
    $\implies$ la funzione $ \displaystyle w(t)\,e^{-B\,(t-a)} $ è decrescente su $ [a,b] $ 
    
    $\implies$ è massima in $ t=a $, ovvero \[
       A = w(a)\,e^{0}\ge w(t)\,e^{-B\,(t-a)}\ge \phi(t)\,e^{-B\,(t-a)}
    \] 
    
    $\implies$ $ \displaystyle \varphi(t)\le A\,e^{B\,(t-a)} $\qed
}
\dimostrazione{fsglkjnsdfglkjnsdflgkjnsdflkgjnsdflkgjjjfjfjfjfjfifroiefdfdskjncdskujhsdfalkjandxvberure}{
    \begin{enumerate}
        \item Dimostriamo per assurdo. Suppongo che per ogni $ \varepsilon>0 $, $ \exists\,\bm{x}_{\epsilon} \in B_{ \varepsilon}(\bm{x}_0)  $, la soluzione massimale $ \bm{u}_{ \varepsilon}\coloneqq \bm{u}( \cdot , \bm{x}_{ \varepsilon}) $ di \[
            \begin{cases}
                \bm{u}'=\bm{f}\left(t,\bm{u}(t)\right)\\ 
                \bm{u}(t_0)=\bm{x}_{ \varepsilon}
            \end{cases}
        \]non è definita su tutto $ [a,b] $. 

        Per semplicità prendo $ t_0=a $ e ``lavoro a destra''. 

        Prendiamo $ \delta>0 $ sufficientemente piccolo e \[
            k_{\delta} = \left\{(t,\bm{x}) \in \Omega: t \in [a,b], \norma{\bm{u}_{ \varepsilon}(t)-\bm{x}}<\delta\right\} 
        \]

        Sia $ [a,b_{ \varepsilon})  $ con $ b_{ \varepsilon}< b  $ l'intervallo massimale detro di $ \bm{u}_{ \varepsilon} $. 

        Necessariamente $ \bm{u}_{ \varepsilon} $ deve uscire dal compatto $ k_{\delta} $ prima di $ b_{ \varepsilon} $: \[
            \forall\, \varepsilon>0\quad \exists\, t_{ \varepsilon} \in (a,b_{ \varepsilon} ): \begin{aligned}
                \norma{ \bm{u}_{ \varepsilon}(t_{ \varepsilon} )- \bm{u}_0(t_{ \varepsilon} ) } &= \delta \\ 
                \norma{ \bm{u}_{ \varepsilon}(t )- \bm{u}_0(t) } &< \delta \quad \forall\, t \in [a,t_{ \varepsilon} )
            \end{aligned}
        \]

        Sfruttiamo il fatto che $ \bm{u}_0 $ e $ \bm{u}_{ \varepsilon} $ siano le soluzioni di problemi di Cauchy e usiamo le loro equazioni di Volterra. \[
            \bm{u}_{ \varepsilon} (t) =\bm{u}_{ \varepsilon} (a) + \int_{a}^{t} \bm{f}\left(s, \bm{u}_{ \varepsilon}(s)\right) \,ds
        \]Definiamo ora, per ogni $ t \in [a,t_{ \varepsilon} ] $, la funzione \begin{align*}
            \phi(t) &= \norma{\bm{u}_{ \varepsilon}(t)-\bm{u}_{ 0}(t)}\\ &= \norma{%
            \bm{u}_{ \varepsilon}(a) + \int_{a}^{t} \bm{f}\left(s,\bm{u}_{ \varepsilon}(s)\right) \,ds - \bm{u}_0 - \int_{a}^{t} \bm{f}\left(s, \bm{u}_0(s)\right)\,ds 
            }\\ 
            &\le \norma{\bm{u}_{ \varepsilon}(a)-\bm{u}_0(a)} + \int_{a}^{t}\norma{\bm{f}\left(s, \bm{u}_{ \varepsilon}(s)\right) - \bm{f}\left(s, \bm{u}_0(s)\right)}\,ds \\ 
            &\underset{\footnotemark}{\le} \parentesi{A_{ \varepsilon} }{%
                \norma{\bm{x}_{ \varepsilon} -\bm{x}_0} 
            } + L\,\int_{0}^{t}\norma{\bm{u}_{ \varepsilon}(s) - \bm{u}_0(s)}\,ds .
        \end{align*}\footnotetext{dove $ L $ è la costante di lipschitzianità di $ f $ su $ k_{\delta}  $} 

        Dunque, per il lemma di Gronwall, $ \displaystyle \phi(t)\le A_{ \varepsilon}\, e^{L(t-a)}  $. Inoltre, $ \phi(t_{ \varepsilon} ) = \delta $, e si ha che \[
            0< \delta \le A_{ \varepsilon} \, e^{L(t_{ \varepsilon} -a)} \longrightarrow 0 
        \]che è assurdo.
        \item Sia $ \overline{\bm{x}}_0 \in N $, e sia $ \overline{\bm{u}}(t)\coloneqq \bm{u}(t,\overline{\bm{x}}_0) $; 
        
        sia $ \{\bm{x}_{k} \}_{k \in \N} \subseteq N$ tale che $ \bm{x}_k\longrightarrow \overline{\bm{x}}_0 $, e sia $ \bm{u}_k (t)\coloneqq \bm{u}(t,\bm{x}_k)$.\begin{itemize}
            \item Sia $ \overline{K}_{\delta} $ il $ \delta $-intorno compatto di $ \overline{\bm{u}} $;
            \item per $ k $ sufficientemente grandi, il grafico di $ \bm{u}_k $ rimane in $ \overline{K}_{\delta} $ (ragionando come il punto precedente);
            \item uso il lemma di Gronwall sulla \[
                \phi(t)\coloneqq\norma{\bm{u}_k(t)-\overline{\bm{u}}(t)}
            \]e ottengo, sempre utilizzando l'equazione di volterra \[
                \norma{\bm{u}_k(t)-\overline{\bm{u}}(t)} \le \parentesi{\to 0}{\norma{\bm{x}_k-\overline{\bm{x}}_0}}\,e^{L(b-a)}
            \]dove $ L $ è la costante di lipschitzianità di $ \bm{f} $ su $ \overline{K}_{\delta} $ 
            
            $\implies$ $\displaystyle \norma{\bm{u}_k-\overline{\bm{u}}}_{\infty} = \max_{t \in [a,b]} \norma{\bm{u}_k(t)-\overline{\bm{u}}(t)}\le \norma{\bm{x}_k-\overline{\bm{x}}_0}\,e^{L(b-a)}\to 0  $\qed
        \end{itemize}
    \end{enumerate}
} % Lezione 5 - Teo di dipendenza continua
\days{14 marzo 2023}

\osservazione{
    Questo teorema si chiama di ``dipendenza continua'' perché la funzione:\begin{align*}
    N &\longrightarrow \mathscr{C}\left([a,b]; \norma{ \cdot }_{\infty}\right) \\
    \bm{x} &\longmapsto \bm{u}(t,\bm{x})
    \end{align*}è continua; infatti, per $ \bm{x}_k\to \overline{\bm{x}}_0 \in N $ le soluzioni corrispondenti convergono uniformemente.
}
\section{Equazione alle variazioni}
\paragrafo{Domanda}{%
    Ci chiediamo ora se la funzione $ \bm{u}(t,\bm{x}) $ ha regolarità maggiore rispetto a $ \bm{x} $? Se sì, come si comporta la funzione $ \bm{u}_{\bm{x}} (t,\bm{x}) $?
}{}{}
%% BEGIN Equazione alle variazioni
\paragrafo{Equazione alle variazioni}{%
\[
    \begin{cases}
        {u}'=t\,{u}^{2}\\ 
        {u}(0)={x}
    \end{cases}\qquad u(t,x) = \frac{2x}{2-t^{2}x}, \quad t \in \left(-\sqrt{\frac{2}{x}},\sqrt{\frac{2}{x}}\right)
\]$u(t,x)$ la interpretiamo come funzione di due variabili $ t $ e $ x $. \begin{align*}
    u_{t}(t,x)&=\frac{\partial}{\partial t} u(t,x) = \cdots = t\,u^{2}(t,x)\\[3ex]
    u_{x}(t,x)&=\frac{\partial}{\partial x} u(t,x) = \frac{\partial}{\partial x}\left(\frac{2x}{2-t^{2}x}\right)\\[2ex] 
    &= \frac{4-\cancel{2t^{2}x}-\cancel{2x(-t^{2})}}{(2-t^{2}x)^{2}}= \frac{4}{(2-t^{2}x)^{2}}
\end{align*}Noto che la soluzione è $ C^{2} $: calcolo del derivate parziali seconde: \begin{align*}
    \frac{\partial}{\partial t} u_{x}(t,x) &= \frac{\partial}{\partial x} u_{t}(t,x)\\[2ex]
    &= t \cdot 2u(t,x) \cdot u_{x}(t,x) 
\end{align*}Valutiamo tutto in $ x = x_0 $, poniamo $ v(t)\coloneqq u_{x}(t,x_0)$. Si ha che \[
    \frac{d}{dt} v(t)= 2t\,u(t,x_0)\,v(t)
\]ovvero \[
    v'(t)=g(t)\,v(t),\qquad g(t) = 2t\,u(t,x_0)
\]La funzione $ v(t) $ risolve un'equazione \emph{lineare} che si chiama \emph{equazione alle variazioni} dove \[
    g(t)=\frac{\partial}{\partial u}\left[f(t,u)\right]_{u=u(t,x_0)}.
\]
}{daoijfdaoijcdiiididiidiidii}{}
\paragrafo{Generalizzazione}{%
    \begin{itemize}
        \item Sia $ \Omega \subseteq \R^{2} $, $ f:\Omega \to \R $, $ f \in C^{1}(\Omega) $.
        \item Sia $ (t_0,x_0) \in \Omega $, $ u(t,x_0) $ soluzione di \[
            \begin{cases}
                u'(t) = f\left(t,u(t)\right)\\ 
                u(t_0)=x_0
            \end{cases}
        \]e sia $ [a,b] \subseteq I_{\max}  $, dove $ I_{\max}  $ è l'intervallo massimale di $ u(t,x_0) $
        \item Se $ N $ è intorno di $ x_0 $: $ \forall\, x \in N $, l'unica soluzione $ u(t,x) $ di \[
            \begin{cases}
                u'(t) = f\left(t,u(t)\right)\\ 
                u(t_0)=x
            \end{cases}
        \]è definita su tutto $ [a,b] $
    \end{itemize}Sono nella situazione in cui vale: \begin{align}
            u_{t}(t,x) &= f\left(t,u(t,x)\right) &\text{perché è soluzione}\label{eq:1dd}\\ 
            u(t_0,x)&=x &\text{è il dato iniziale}\label{eq:2dd}
    \end{align}
    \begin{itemize}
        \item[\eqref{eq:1dd}:] Supponiamo che $ u(t,x) $ sia derivabile in $ x $ (e questa cosa \emph{non} è stata dimostrata). Allora il secondo membro di \eqref{eq:1dd} è derivabile in $ x $, e dunque anche il primo. Derivando a sinistra e a destra in $ x $ otteniamo: \begin{equation}
            \label{eq:asterisco} u_{tx}(t,x)=\frac{\partial}{\partial x} f\left(t,u(t,x)\right) = \frac{\partial\,f}{\partial u}\left(t,u(t,x)\right) \cdot u_{x}(t,x) 
        \end{equation}
        \item[\eqref{eq:2dd}:] Derivando ambo i membri rispetto a $ x $, ottengo \begin{equation}
            \label{eq:puntino} u_{x}(t_0,x) = 1 
        \end{equation}
    \end{itemize}

    Sia $ v(t)=u_{x}(t,x_0)  $, allora da \eqref{eq:asterisco} \begin{align*}
        v'(t) =\frac{d}{dt} v(t) &= \frac{\partial \,f}{\partial u}\left(t, u(t,x_0)\right) \cdot u_{x}(t,x_0)\\ 
        &= \parentesi{g(t)\coloneqq}{\frac{\partial\,f}{\partial u}\left(t,u(t,x_0)\right)} \,v(t),
    \end{align*}mentre da \eqref{eq:puntino} ottengo $ v(t_0)=1 $. 

    Abbiamo trovato che $ v $ risolve il problema di Cauchy lineare: \[
        \begin{cases}
            v'(t)=g(t)\,v(t)\\ 
            v(t_0)=1
        \end{cases}
    \]
}{}{}
%% END
\section{Flusso associato ad una equazione differenziale}
%% BEGIN Flusso associato ad una equazione differenziale
\paragrafo{Definizione del flusso}{%
    Sia $ \bm{f}:\Omega \subseteq \R\times \R^{n}\to \R^{n} $ di classe $ C^{1}(\Omega) $. 

    Sia $ (t_0,\bm{x}_0) \in \Omega$, e sia \[
        \Omega_{0} = \{\bm{x} \in \R^{n}: (t_0,\bm{x}) \in \Omega\}
    \]e sia $ I(\bm{x}) $ l'intervallo massimale della soluzione $ \bm{u}(t,\bm{x}) $ del problema di Cauchy \[
        \begin{cases}
            \bm{u}'=\bm{f}(t,\bm{u})\\ 
            \bm{u}(t_0)= \bm{x}
        \end{cases}
    \]

    Considero ora l'insieme \[
        E=\{(t,x) \in\Omega : \bm{x} \in \Omega_{0}, t \in I(\bm{x}) \}. 
    \]

    La funzione $ \bm{\Psi} $ definita sotto si chiama \emph{flusso}, e indica dove si trova al tempo $ t $ la soluzione con dato iniziale $ \bm{u}(t_0)=\bm{x} $. \[
        \begin{aligned}
            \bm{\Psi}:E &\longrightarrow \R^{n} \\
    (t,\bm{x}) &\longmapsto \bm{u}(t,\bm{x})
        \end{aligned}\qquad \bm{\Psi}_{\bm{x}}(t) \coloneqq \bm{\Psi}(t,\bm{x}) 
    \]
}{}{}
\teorema{}{
    Se $ \bm{f} \in C^{1}(\Omega)$, allora la funzione $ \bm{\Psi} $ è di classe $ C^{1}(E) $.
}
%% END
\paragrafo{Sistemi autonomi}{%
    Cosa succede nei sistemi autonomi? Considero \[
        \begin{cases}
            \bm{u}'=\bm{f}(\bm{u})\\ 
            \bm{u}(0)= \bm{x} \in \Omega'
        \end{cases}
    \]con $ \displaystyle f: \Omega = \R\times \Omega' \subseteq \R\times \R^{n}\longrightarrow \R^{n} $: \begin{itemize}
        \item $ \Omega $ è una ``striscia'' di $ \R^{n+1} $, 
        \item mentre $ \Omega' $ è il dominio di $ \bm{f} $.
    \end{itemize} 
    
    $\implies$ $ \Omega_0 $ non dipende da $ t_0 $ 
    
    $\implies$ fissiamo $ t_0=0 $.

    Se $ I(\bm{x}) $ è l'intervallo massimale per la soluzione con $ \bm{u}(0)=\bm{x} $, e $ E $ è l'insieme: \[
        E=\{(t,\bm{x}): t \in I(\bm{x}), \bm{x} \in \Omega'\}
    \]allora definisco la funzione \begin{align*}
    \bm{\Psi}: E &\longrightarrow \R^{n} \\
    (t,\bm{x}) &\longmapsto \bm{u}(t,\bm{x})
    \end{align*}e $\bm{\Psi}_{\bm{x}}(t)\coloneqq \bm{\Psi}(t,\bm{x}) $ è soluzione con $ \bm{u}(0)=\bm{x} $.
}{}{}
\esempio{
    \[
        \begin{cases}
            u'=u(1-x)\\ 
            u(0)=x
        \end{cases}
    \]Si ha che\begin{itemize}
        \item se $ x>1 $: $ I(x) = (\alpha_{x}, + \infty ) $ con $ \alpha_{x}>-\infty  $;
        \item se $ x \in [0,1:] $ $ I(x) = \R $;
        \item se $ x<0 $: $ I(x)=(- \infty, \omega_{x} ) $ con $ \omega_{x}<+ \infty  $
    \end{itemize}
} % Lezione 6 - Teo di dipendenza continua
\stepcounter{capitoloeccolo}\chapter{Equazioni autonome}
\section{Equazioni autonome in una dimensione} 
\subsection{Equazione logistica}
%% BEGIN Equazione logistica
\paragrafo{Studio delle soluzioni}{%
    Sia $ p'(t)= \left(k-h\,p(t)\right) \cdot p(t)$, dove $ p(t) $ è il numero di individui in un popolazione al tempo $ t $, con $ k,h>0 $ e $ p(t)\ge 0 $. 

    Supponiamo che $ p(0)=p_0\ge 0 $, ci chiediamo l'evoluzione di $ p(t) $ per $ t>0 $. \begin{itemize}
        \item Cerco le soluzioni costanti, ovvero $ f(p)=0 $: $ p=0 $ e $ p=(k/h) $. Queste due sono soluzioni costanti per ogni tempo $ t $.
        \item Studio la monotonia delle soluzioni: $ p'(t)\ge 0 $ \[
            \left(k-h\,p(t)\right) \cdot \parentesi{\ge 0}{p(t)}\ge 0.
        \]Dunque $ p'(t)\ge 0 $ $ \iff $ $ p(t)\le k/h $ 
        
        $\implies$ se $ p(t) \in (0, k/h) $ la soluzione cresce, mentre se $ p(t)> k/h $ la soluzione decresce.
    \end{itemize}

    Se il dato iniziale $ p_0 \in (0,k/h) $ allora la soluzione è crescente, e si troverà sempre nella striscia $ [0,+ \infty)\times (0,k/h) $ (per esistenza e unicità locale) 
        
        $\implies$ la soluzione si mantiene limitata, e in particolare \[
            \norma{p}\le k/h.
        \]È soddisfatto il corollario \framref{dakfjhbaksjdhfbaksjdhfbakjhbkjhb}, e quindi $ T_{\max}= + \infty  $

    Se il dato iniziale $ p_0>k/h $, la soluzione è sempre monotona decrescente, e si troverà sempre nel semispazio $ p>k/h $. Anche in questo caso si applica il corollario \framref{dakfjhbaksjdhfbaksjdhfbakjhbkjhb} 
    
    $\implies$ $ T_{\max}= + \infty  $. 

    Osservo infine che il teorema dell'asintoto ci garantisce che tutte le soluzioni non costanti abbiano come limite a $ + \infty $ la soluzione constante $ k/h $.
}{}{}
\paragrafo{Diagramma di fase}{%
    Poiché $ p(t) \in \R $, si dice che $ \R $ è lo \emph{spazio delle fasi} o degli stati.

    \begin{figure}
        \begin{center}
            \subfloat[Grafico delle funzioni]{%
            \begin{tikzpicture}
                \draw [-Stealth] (-0.5, 0) -- (3,0);
                \draw [-Stealth] (0, -0.5) -- (0,3);
                \foreach \y in {0,0.2,...,3}{
                    \draw (-0.5, \y-0.5) -- (-0.1, \y-0.1);
                };
                \foreach \x in {0.2,0.4,...,3}{
                    \draw (\x-0.5, -0.5) -- (\x-0.1, -0.1);
                };
                \node at (3.1,-0.2) {$t$};
                \node at (3.4,1.2) {$k/h$};
                \draw [ultra thick] (0,0) -- (1.7,0);
                \draw [ultra thick, dashed] (1.8,0) -- (2.7,0);
                \draw [ultra thick] (0,1.2) -- (1.7,1.2);
                \draw [ultra thick, dashed] (1.8,1.2) -- (3,1.2);
                \draw [smooth, domain=0:2.2, thick] plot (\x,{0.7 * ln(\x + 1.4)});
                \draw [smooth, domain=2.2:2.9, dashed, thick] plot (\x,{0.7 * ln(\x + 1.4)});
                \draw [smooth, domain=0:2.2, thick] plot (\x,{-0.7 * ln(\x + 1.4)+2.5});
                \draw [smooth, domain=2.2:2.9, dashed, thick] plot (\x,{-0.7 * ln(\x + 1.4)+2.5});
            \end{tikzpicture}
        }\qquad\subfloat[Diagramma di fase]{%
            \begin{tikzpicture}
                \draw [white] (-3.5/2, 0) -- (3.5/2,0);
                \draw (0, 0) -- (0,2.6);
                \draw [dashed] (0, 3) -- (0,2.6);
                \draw [white] (0, -0.5) -- (0,0);
                \fill (0,0) circle (0.05);
                \fill (0,1.2) circle (0.05);
                \draw [decoration={markings,mark=at position 0.6 with {\arrow{Stealth}}}, postaction=decorate] (0,0) -- (0,1.2);
                \draw [decoration={markings,mark=at position 0.6 with {\arrow{Stealth}}}, postaction=decorate] (0,2.6) -- (0,1.2);
            \end{tikzpicture}%TODO finire il disegno
        }
        \end{center}

        \caption{Equazione Logistica (\textbf{da finire})}\label{fig:equazionelogistica}
    \end{figure}

    Osservando la figura \ref{fig:equazionelogistica}, il grafico a destra prende il nome di \emph{diagramma di fase}.
}{}{}
%% END
\osservazione{
    Tra due zeri consecutivi di $ f $, la monotonia della soluzione (nel caso autonomo) non cambia.
}
\subsection{Diagramma di fase}
%% BEGIN Diagramma di fase
\definizione{Se $ u $ è una soluzione massimale\footnote{Ovvero il suo dominio di definizione è massimale.} di $ y'=f(y) $, l'insieme \[
    \gamma_{u}=\left\{u(t): t \in \left(T_{\min},\ T_{\max}\right)\right\} 
\]è detta \emph{orbita di $ u $}.}
\paragrafo{Nota}{%
    Le soluzioni stazionarie di $ y'=f(y) $ hanno come orbita un singolo punto.
}{}{}
\definizione{L'insieme delle orbite con il loro verso di percorrenza costituisce il \emph{ritratto di fase} di $ y'=f(y) $.}
\esempio{
    L'equazione logistica ha 5 orbite: due semirette, due punti e un segmento.
}
\paragrafo{}{%
    Dimostreremo in $ \R^{n} $ che per ogni punto dello spazio delle fasi passa una ed una sola orbita. Nel caso 1-dimensionale, si può notare che se $ u_{x_0}  $ è soluzione di \[
        \begin{cases}
            x'=f(x)\\ 
        x(0)=x_0
        \end{cases}
    \]allora la funzione $ w(t)\coloneqq u_{x_0}(t+\tau)  $ risolve \[
        \begin{cases}
            x'=f(x)\\ 
        x(0)=u_{x_0}(\tau) 
        \end{cases}
    \] 
    
    $\implies$ la soluzione di \[
        \begin{cases}
            x'=f(x)\\ 
        x(\tau)=x_0
        \end{cases}
    \]è $ u_{x_0}(t-\tau)  $
}{}{}
%% END
\sesercizio{Fare il diagramma di fase dell'equazione differenziale \[
    y'=y(2-y)\,e^{\sin y}
\]} % Lezione 1 - Equazioni autonome
\days{28 febbraio}
%TODO leggere teorema dell'asintoto e scrivere una nota a riguardo

\section{Risultati e definizioni sui sistemi autonomi}

\definizione{Si dice \emph{sistema autonomo} una equazione differenziale nella forma \[
    \bm{x}'=\bm{f}(\bm{x}),\qquad \bm{f}:\Omega \subseteq \R^{n}\to \R^{n}
\]}
\paragrafo{Ipotesi}{%
    Dato il problema di Cauchy\[
        \begin{cases}
            \bm{x}'=\bm{f}(\bm{x})\\ 
            \bm{x}(t_0)=\bm{x}_0 \in \Omega
        \end{cases}
    \]
    $ \bm{f}:\Omega \subseteq \R^{n}\to \R^{n} $, $ n\ge 1 $ localemente lipschitziana ($\impliedby C^{1}$), vale il teorema di esistenza e unicità locale della soluzione, $ \forall\, t_0 \in\R  $.
}{daflkjnasdlkfjnasdlkfjnasdkfjnaskdjnfkjnkj}{}
\definizione{
    $ \Omega $ si dice \emph{spazio delle fasi} o \emph{degli stati}.
}
\definizione{
    Se $ \bm{u} $ è una soluzione massimale di $ \bm{x}'=\bm{f}(\bm{x}) $, l'insieme \[
        \gamma\coloneqq\left\{u(t): t \in (T_{\min}, T_{\max}  )\right\} 
    \]è detta \emph{orbita} per $ \bm{x}'=\bm{f}(\bm{x}) $ e la soluzione $ \bm{u} $ è una parametrizzazione di $ \gamma  $.
}
\osservazione{
    \[
        \left\{\begin{aligned}
            (T_{\min}, T_{\max}  ) &\to \R^{n}\\ 
            t \mapsto \bm{u}(t)
        \end{aligned}\right.
    \]è una curva in $ \R^{n} $ con sostegno $ \gamma  $.
}
\definizione{%
        Lo spazio delle fasi in cui vengono disegnate le orbite con il loro verso di percorrenza (indotto dalle soluzioni) si dice \emph{ritratto di fase}.
}
\definizione{%
    I punti $ \bm{p} \in \Omega $ tali che $ \bm{f}(\bm{p})=\bm{0} $ si chiamano \emph{equilibri} o \emph{punti singolari}
}
\paragrafo{Esempio}{%
\[
    \begin{cases}
        x'=-y^{2}\\ 
        y'=x^{2}
    \end{cases}\qquad \begin{aligned}
        \bm{f}(x,y) &= (-y^{2},x^{2})\\ 
        \bm{f}:\R^{2} &\to \R^{2}
    \end{aligned}
\]Chi sono gli equilibri? Cerco $ (x_0,y_0): \bm{f}(x_0,y_0)=\bm{0} $ \[
    (-y^{2},x^{2})=\bm{0}\,\iff\, (x,y)=\bm{0}
\]Dunque l'unico punto di equilibrio è l'origine, come mostrato in figura \ref{fig:puntoequilibrioA}.
\begin{figure}
    \begin{center}
        \begin{tikzpicture}
            \draw [-Stealth] (-2,0) -- (2,0);
            \draw [-Stealth] (0,-1.4) -- (0,2);
            \node at (2,-0.3) {$x$};
            \node at (.3,1.9) {$y$};
            \fill (0,0) circle (0.1);
        \end{tikzpicture}
    \end{center}    
    \caption{Punti di equilibrio per l'esempio \framref{daflkjansdflkjnasdflkjadsnchiduaskjbfcgukjbh}}
\end{figure}
}{daflkjansdflkjnasdflkjadsnchiduaskjbfcgukjbh}{}
\paragrafo{Esempio}{%
\[
    \begin{cases}
        x'=-y^{2}\,x\\ 
        y'=x^{2}
    \end{cases}\qquad
        \bm{f}(x,y) = (-y^{2}\,x,x^{2})
\]Gli equilibri sono tutti punti nella forma $ (0,y_0) $, come mostrato in figura \ref{fig:puntoequilibrioB}.
\begin{figure}
    \begin{center}
        \begin{tikzpicture}
            \draw [-Stealth] (-2,0) -- (2,0);
            \draw [-Stealth] (0,-1.5) -- (0,2.1);
            \node at (2,-0.3) {$x$};
            \node at (.3,1.9) {$y$};
            \draw [ultra thick] (0,-0.8) -- (0,1.3);
            \draw [ultra thick, dashed] (0,1.3) -- (0,1.8);
            \draw [ultra thick, dashed] (0,-0.8) -- (0,-1.3);
        \end{tikzpicture}
    \end{center}
    \caption{Punti di equilibrio per l'esempio \framref{jjdjjduudjjdekkjnsdikjnskjnsdikjsnbfik}}
\end{figure}
}{jjdjjduudjjdekkjnsdikjnskjnsdikjsnbfik}{}
\teorema{daflkjndasflkjndaslkjnfdalkjnaslkjfnlkjn}{
    Sotto le ipotesi di \framref{daflkjnasdlkfjnasdlkfjnasdkfjnaskdjnfkjnkj}, per ogni punto dello spazio delle fasi, $ \Omega $, passa una e una sola orbita.
}
\dimostrazione{daflkjndasflkjndaslkjnfdalkjnaslkjfnlkjn}{
    Sia $ \bm{p} \in \Omega $, e consideriamo il Problema di Cauchy \[
        \begin{cases}
            \bm{x}'=\bm{f}(\bm{x})\\ 
            \bm{x}(t_0)=\bm{p} 
        \end{cases}\tag{$PC_{\bm{p}}$}
    \]
    \begin{itemize}
        \item[($\exists$)] Il problema di Cauchy ammette un'unica soluzione $ \bm{u} $ (per il teorema di Cauchy-Lipschitz) e l'orbita associata a $ \bm{u} $ passa per $ \bm{p} $: \[
            \bm{p} \in \gamma=\left\{\bm{u}(t): t \in (T_{\min}, T_{\max}  )\right\} 
        \]
        \item[($!$)] Intuitivamente, l'unicità è giustificata dal fatto che il sistema è autonomo, e quindi dal fatto che le traslate in $ t $ delle soluzioni sono ancora soluzioni.
        
        Sappiamo che $ \bm{p} \in \gamma  $. Supponiamo che esista un'altra orbita $ \tilde{\gamma } $ tale che $ \bm{p} \in \tilde{\gamma } $. Allora $ \tilde{\gamma} $ è orbita di $ \tilde{\bm{u}} $, soluzione di \[
            \begin{cases}
                \bm{x}'=\bm{f}(\bm{x})\\ 
                \bm{x}(\tilde{t_0})= \bm{p}
            \end{cases}
        \]Siano $ I $ e $ \tilde{I} $ gli intervalli massimali, rispettivamente, di $ \bm{u} $ e $ \tilde{\bm{u}} $. Siano\begin{align*}
            T&\coloneqq \tilde{t_0}-{t_0}\\
            \bm{v}(t) &\coloneqq \tilde{\bm{u}}(t+\tilde{t_0}-t_0)=\tilde{\bm{u}}(t+T).
        \end{align*}Si ha che \begin{itemize}
            \item l'orbita di $ \bm{v} $ è $ \tilde{\gamma} $;
            \item $ \bm{v} $ è massimale, in quanto lo è $ \tilde{\bm{u}} $;
            \item $ \bm{v} $ soddisfa \[
                \begin{cases}
                    \bm{v}'(t)= \tilde{\bm{u}}'(t+T)= \bm{f}\left(\tilde{\bm{u}}(t+T)\right)=\bm{f}(\bm{v}(t))\\ 
                   \bm{v}(t_0)=\tilde{\bm{u}}(t_0+T)=\tilde{\bm{u}}(\tilde{t_0})=\bm{p}
                \end{cases}
            \]e quindi $ \bm{v} $ è soluzione massimale di \[
                \begin{cases}
                    \bm{v}'=\bm{f}(\bm{v})\\ 
                    \bm{v}(t_0)=\bm{p}
                \end{cases}
            \] 
            
            $\implies$ $ \bm{u} $ e $ \bm{v} $ sono la stessa soluzione (massimale) di \[
                \begin{cases}
                    \bm{x}'=\bm{f}(\bm{x})\\ 
                    \bm{x}(t_0)=\bm{p}
                \end{cases}
            \] 
            
            $\implies$ $ I $ e $ \tilde{I} $ sono uno traslato dell'altro, e $ \gamma $ e $ \tilde{\gamma} $ coincidono.\qed
        \end{itemize}
    \end{itemize}
}
\osservazione{
    La mappa: \[
        \parbox{5.5em}{\centering soluzione di $ \bm{x}'=\bm{f}(\bm{x}) $}\quad\longmapsto\quad\text{orbita}
    \]è ben definita, ma non è iniettiva (è suriettiva!).

    Infatti, se $ \gamma $ è orbita \[
        \gamma=\left\{u(t):t \in (a,b)\right\}\underset{\footnotemark}{=}\left\{u_{\tau}(t)=u(t+\tau): t \in (a-\tau, b-\tau) \right\}.
    \]\footnotetext{$\forall \tau \in \R$}Dunque ogni orbita di $ \bm{x}'=\bm{f}(\bm{x}) $ ha infinite parametrizzazioni.
}
\teorema{lkjndalkjndfalkjndasklfjnasdkljfnasdkljfnasdjkndaslkjnfaldskjnaslfkjnsaldkjn}{%
    Sotto le ipotesi di \framref{daflkjnasdlkfjnasdlkfjnasdkfjnaskdjnfkjnkj}, sia $ \gamma^{\star} $ un'orbita di $ \bm{x}'=\bm{f}(\bm{x}) $. Allora: \[
        \gamma^{\star} =\{\bm{p}\}\,\iff\, \bm{f}(\bm{p})=\bm{0}.
    \]
}
\dimostrazione{lkjndalkjndfalkjndasklfjnasdkljfnasdkljfnasdjkndaslkjnfaldskjnaslfkjnsaldkjn}{%
    \begin{itemize}
        \item[($\impliedby$)] Se $ \bm{f}(\bm{p})=\bm{0} $, allora $ \bm{u}(t)=\bm{p} $ $ \forall\, t \in \R $ è soluzione, e \[
            \gamma^{\star} = \left\{\bm{u}(t): t \in \R\right\}=\{\bm{p}\}.
        \]
        \item[($\implies$)] Se $ \gamma^{\star} =\{\bm{p}\} $ allora esiste una soluzione $ \bm{u} $ di $ \bm{x}'=\bm{f}(\bm{x}) $ tale che $ \bm{u}(t)=\bm{p} $ per ogni $ t \in (T_{\min},T_{\max}  ) $. Allora $ \bm{u}(t) $ è costante, e dunque \[
            \bm{0}=\bm{u}'(t)=\bm{f}\left(\bm{u}(t)\right)=\bm{f}(\bm{p}).\qedd
        \]
    \end{itemize}
}
\definizione{%
    Una soluzione $ \bm{u} $ di $ \bm{x}'=\bm{f}(\bm{x}) $ si dice \emph{periodica} di periodo $ T>0 $ se \begin{itemize}
        \item $ \bm{u} $ è definita su $ \R $;
        \item $ \bm{u}(t+T)=\bm{u}(t) $, $ \forall\, t \in \R $
        \item $ \displaystyle T=\inf\left\{\tau >0: u(t+\tau)=u(t)\vspace{1em}\forall\,t \in \R\right\} $
    \end{itemize}
}
\definizione{
    L'orbita corrispondente ad una soluzione periodica si chiama \emph{orbita periodica} e i suoi punti si chiamano \emph{punti periodici}.
}
\teorema{daflkjnadfkljansdfklasjdnfcaskdjncdas}{
    Sotto le ipotesi di \framref{daflkjnasdlkfjnasdlkfjnasdkfjnaskdjnfkjnkj}, se $ \bm{u} $ è una soluzione \emph{non} costante di $ \bm{x}'=\bm{f}(\bm{x}) $ con intervallo massimale $ J $, e se \[
        \exists\, t_1,t_2 \in J:\quad t_1\neq t_2,\quad \bm{u}(t_1)=\bm{u}(t_2)
    \]allora $ \bm{u} $ è una soluzione \emph{periodica}.
}
\osservazione{
    Le orbite periodiche si chiamano anche \emph{orbite chiuse}. Infatti, il teorema \teoref{daflkjnadfkljansdfklasjdnfcaskdjncdas} ci dice che se un'orbita si autointerseca, allora è periodica.
}
\corollario{adkjnaldfkjnadslfkjnadsflkjnadslkjn}{
    Le orbite di $ \bm{x}'=\bm{f}(\bm{x}) $ possono essere: \begin{enumerate}
        \item punti di equilibrio, $ \{\bm{p}\} $;
        \item periodiche/chiuse;
        \item orbite senza autointersezioni e contenenti più di un solo punto.
    \end{enumerate}
}
\paragrafo{Caso particolare}{%
    Per $ n=1 $ non esistono orbite periodiche non costanti, perché dovrebbero cambiare la monotonia e non è possibile perché $ x'=f(x) $, e la cambierebbero su punti di equilibrio.
}{}{}
\dimostrazione{daflkjnadfkljansdfklasjdnfcaskdjncdas}{
    È analoga al teorema \teoref{daflkjndasflkjndaslkjnfdalkjnaslkjfnlkjn}. 

    Sia $ \bm{p}\coloneqq \bm{u}(t_1)= \bm{u}(t_2) $. Allora $ \bm{u} $ risolve due problemi di Cauchy: \[
        \begin{cases}
            \bm{x}'=\bm{f}(\bm{x})\\ 
            \bm{x}(t_1)=\bm{p}
        \end{cases}\qquad \begin{cases}
            \bm{x}'=\bm{f}(\bm{x})\\ 
            \bm{x}(t_2)=\bm{p}
        \end{cases}
    \]entrambi risolti su $ J $.

    Gli intervalli massimali di questi due problemi di Cauchy devono essere uno traslato dell'altro, dunque $ J=\R $. 

    Inoltre, supponendo $ t_2>t_1 $, si ha che \[
        \bm{u}(t),\qquad \bm{u}\left(t+(t_2-t_1)\right)
    \]risolvono entrambe \[
        \begin{cases}
            \bm{x}'=\bm{f}(\bm{x})\\ 
            \bm{x}(t_1)=\bm{p}
        \end{cases}
    \]e per esistenza e unicità della soluzione si ha che \[
        \bm{u}(t)=\bm{u}(t+T),\quad \forall t \in\R, T=t_2-t_1\qedd
    \]
}
%% BEGIN Orbita di una equazione differenziale come curva tangente al campo
\teorema{dakfjnasldkfjnlasdkjfnadslkfjnaslkj}{
    Sia $ \gamma $ un'orbita non singolare\footnote{Ovvero $ \gamma \neq \{\bm{p}\} $.} di $ \bm{x}'=\bm{f}(\bm{x}) $, $ \bm{x} \in \Omega \subseteq \R^{n} $ aperto, $ \bm{f}:\Omega \to \R^{n} $ localmente lipschitziana. 

    Allora $ \gamma $ è una curva orientata in $ \Omega $ tangente in ogni punto al campo $ \bm{f} $.
}
\dimostrazione{dakfjnasldkfjnlasdkjfnadslkfjnaslkj}{
Sia $ \bm{u} $ soluzione di $ \bm{x}'=\bm{f}(\bm{x}) $ tale che $ \gamma= \gamma_{\bm{u}}  $. Sia $ \bm{p}_0  \in \gamma$ e $ t_0 $ tale che $ \bm{u}(t_0)=\bm{p}_0 $. 

$ \gamma $ non è singolare, quindi $ \bm{f}(\bm{p}_0) \neq \bm{0}$ (altrimenti $ \gamma=\{\bm{p}_0\} $ per il teorema precedente.)

La funzione $ t\mapsto \bm{u}(t) $ parametrizza $ \gamma $. Quindi il vettore: \[
    \bm{v}\coloneqq \lim_{h\to 0} \frac{\bm{u}(t_0+h)-u(t_0)}{h} =
\]è il vettore tangente a $\gamma$ nel punto $ \bm{p}_0 $, ammesso che esista. 

La funzione $ \bm{u} $ è di classe $ C^{1} $, quindi il limite esiste ed è $ \bm{u}'(t_0) $. Essendo \[
    \bm{u}'(t)=\bm{f}\left(\bm{u}(t)\right)
\]si ha che \[
    \bm{v}=\bm{u}'(t_0)=\bm{f}\left(\bm{u}(t_0)\right)= \bm{f}(\bm{p}_0)\qedd
\]}
%% END
%% BEGIN Orbita singolare e regolare
\definizione{
    Un’orbita si dice \emph{singolare} se si riduce ad un punto solo. Altrimenti si dice \emph{regolare}.
}
%% END
\section{Stabilità dei punti di equilibrio}
\paragrafo{Obiettivo}{%
    Vogliamo classificare i punti di equilibrio in base a come le altre soluzioni si comportano in loro prossimità.
}{}{}
\definizione{
    Sotto le ipotesi di \framref{daflkjnasdlkfjnasdlkfjnasdkfjnaskdjnfkjnkj}, sia $ \bm{p} $ tale che $ \bm{f}(\bm{p})=\bm{0} $. $ \bm{p} $ si dice \emph{stabile} se $ \forall\,\varepsilon>0 $, $ \exists\, \delta>0 $: $\hat{\bm{x}} \in \Omega $, $ \norma{\hat{\bm{x}}-\bm{p}}<\delta $ 
    
    $\implies$ la soluzione di \[
        \begin{cases}
            \bm{x}'=\bm{f}(\bm{x})\\ 
            \bm{x}(t_0)=\hat{\bm{x}}
        \end{cases}
    \]è definita su $ (t_0,+ \infty) $ e dista da $ \bm{p} $ al più $ \varepsilon $.
}
\paragrafo{}{%
    La soluzione quindi rimane vicina all'equilibrio (quanto vogliamo) pur di partire sufficientemente vicino.
}{}{}
\definizione{
    Sotto le ipotesi di \framref{daflkjnasdlkfjnasdlkfjnasdkfjnaskdjnfkjnkj}, sia $ \bm{p} $ tale che $ \bm{f}(\bm{p})=\bm{0} $. $ \bm{p} $ si dice \emph{instabile} se non è stabile. 
}
\paragrafo{}{%
    Essere instabile significa che $ \exists\, \varepsilon $ e una successione $ \{\bm{x}_{n} \}_{n \in \N} \subseteq \Omega $, $ \bm{x}_{n} \to \bm{p} $ tale che \[
        \forall\, n,\quad \exists\, t_{n}: \quad \bm{u}_{\bm{x}_n}(t_{n} )> \varepsilon 
    \]dove $ \bm{u}_{\bm{x}_n} $ è soluzione di \[
        \begin{cases}
            \bm{x}'=\bm{f}(\bm{x})\\ 
            \bm{x}(t_0)=\bm{x}_n
        \end{cases}
    \]
}{}{}
\definizione{
    Sotto le ipotesi di \framref{daflkjnasdlkfjnasdlkfjnasdkfjnaskdjnfkjnkj}, sia $ \bm{p} $ tale che $ \bm{f}(\bm{p})=\bm{0} $. $ \bm{p} $ si dice \emph{asintoticamente stabile} se:
    \begin{itemize}
        \item $ \bm{p} $ è stabile;
        \item $ \exists\,\delta>0 $ tale che $ \forall\, \overline{\bm{x}} \in B_{\delta}(\bm{p}) $, $ \overline{\bm{x}} \in \Omega $ si ha che \[
            \lim_{t\to \infty} \bm{u}_{\overline{\bm{x}}} (t) = \bm{p}
        \]
    \end{itemize}
} % Lezione 2
\days{2 marzo 2023}
\section{Equazioni autonome in due dimensioni}
%% BEGIN Riduzione di un problema di Cauchy autonomo in due dimensioni 
\paragrafo{Orbite non singolari}{%
    Consideriamo l'equazione \[
        \begin{cases}
            x'=f_{1}(x,y)\\ 
            y' = f_{2}(x,y)  
        \end{cases}
    \]Supponiamo $ \bm{p}:\bm{f}(\bm{p})\neq \bm{0} $, $ \bm{p}=(p_1,p_2) $. Com è fatta l'orbita per $ \bm{p} $?

    Se $ \bm{f}(\bm{p})\neq \bm{0} \neq \bm{0} $, allora almeno una delle sue componenti è non nulla. Supponiamo che $ f_{1}(\bm{p})\neq 0  $. 
    
    $\implies$ il vettore tangente all'orbita per $ \bm{p} $ non è verticale. 
    
    Consideriamo allora il sistema: \[
        \begin{cases}
            x'=f_{1}(x,y)\\ 
            y' = f_{2}(x,y)\\ 
            \left(x(0),y(0)\right)=\bm{p}
        \end{cases}
    \]e sia $ \bm{u} $ una soluzione massimale, tale che $ \bm{u}(0)=\bm{p} $. $ \bm{u}'(0) $ non è un vettore verticale 
    
    $\implies$ localmente posso esprimere la seconda componente dell'orbita per $ \bm{p} $ in funzione della prima. \[
        \exists!\, \left\{ \begin{aligned}
        \varphi: I_{p_1}  &\longrightarrow I_{p_2}  \\
        x &\longmapsto \varphi(x)
        \end{aligned}\right.
    \]e questa funzione descrive l'orbita \[
        \begin{cases}
            x=u_1(t)\\ 
            \begin{aligned}
                y=\varphi(x)&= u_2(t)\\ 
                & = \varphi\left(u_1(t)\right)
            \end{aligned}
        \end{cases}
    \]Si ha che \[
        f_2\left(\bm{u}(t)\right) = u_2'(t) = \varphi'\left(u_1(t)\right) \cdot u_1'(t)
    \]da cui $ \displaystyle f_2(x,y)=\varphi'(x)\,f_1(x,y) $ \[
        \varphi'(x)=\frac{f_2(x,y)}{f_1(x,y)} = \frac{f_2\left(x,\varphi(x)\right)}{f_1\left(x,\varphi(x)\right)}.
    \]

    Stiamo dicendo che se $ f_1(p)\neq 0 $ l'orbita che passa per $ \bm{p} $ la posso esprimere mediante una funzione $ y=\varphi(x) $ che soddisfa \[
        \begin{cases}
            \displaystyle\varphi'(x)=\frac{f_2\left(x,\varphi(x)\right)}{f_1\left(x,\varphi(x)\right)}\\[2ex]
            \varphi(p_1)=p_2
        \end{cases}
    \]ovvero un problema di Cauchy monodimensionale \emph{non autonomo}.
}{dafkjbnadlfjknasdlkjncdajklscnadlskjn}{}
\teorema{dlkjanbflkasjdbfnlakjnflkasjdnlkasdjnflaksdfjnlkjnlkjn}{
    Dato il problema di Cauchy \[
        \begin{cases}
            x'=f_1(x,y)\\ 
            y'=f_2(x,y)\\
            \left(x(0),y(0)\right)=\bm{p}
        \end{cases}\tag*{(PC)}
    \]con $ \bm{f}:\Omega \subseteq \R^{2}\to \R^{2}$ di classe $ C^{1} $. Supponiamo che $ f_1(\bm{p})\neq 0 $ (campo non verticale in $ \bm{p} $). Allora per degli opportuni intorni $ I_{p_1}, I_{p_2}$: 

    $ \varphi: I_{p_1}\longrightarrow I_{p_2}   $ è una rappresentazione locale dell'orbita della soluzione di (PC)

    $ \iff $ $ \varphi $ è soluzione di \[
        \begin{cases}
            \displaystyle\varphi'(x)=\frac{f_2\left(x,\varphi(x)\right)}{f_1\left(x,\varphi(x)\right)}\\[2ex]
            \varphi(p_1)=p_2
        \end{cases}\tag*{(PC)\textsubscript{$\varphi$}}
    \]
}
\dimostrazione{dlkjanbflkasjdbfnlakjnflkasjdnlkasdjnflaksdfjnlkjnlkjn}{
    \begin{itemize}
        \item[($\implies$)] Vedi: \framref{dafkjbnadlfjknasdlkjncdajklscnadlskjn}
        \item[($\impliedby$)] $ \varphi $ è soluzione di (PC)\textsubscript{$\varphi$}. Per ipotesi sappiamo che $ f_1\left(x,\varphi(x)\right) \neq 0$ in un opportuno intorno di $ p_1 $, $ I_{p_1}  $ (poiché $ f $ è $ C^{1} $). 
        
        Allora, $ \forall\, x \in I_{p_1}  $ sia\[
            \omega(x)=\int_{p_1}^{x}\frac{d\,\xi}{f_1\left(\xi, \varphi(\xi)\right)}.
        \]Poiché $ f_1 $ è continua e non nulla 
        
        $\implies$ $ \omega $ è derivabile e \[
            \begin{cases}
                \omega'(x)=\displaystyle\frac{1}{f_1\left(x,\varphi(x)\right)}\\[2ex]
                \omega(p_1)=0
            \end{cases}
        \]Poiché $ \omega' $ ha segno costante 
        
        $\implies$ $ \omega $ è strettamente monotona 
        
        $\implies$ $ \omega $ è invertibile. \[
             \begin{aligned}
                \omega: I_{p_1}\to I \ni 0 \quad\leadsto\quad\omega^{-1}=: v:I&\longrightarrow I_{p_1}\\ 
                v(0) &= p_1\\ 
                v'(t) &=\displaystyle \frac{1}{\omega'\left(v(t)\right)}  
            \end{aligned}
        \] 
        
        $\implies$ $ \displaystyle v'(t) = f_1\left(v(t), \varphi\left(v(t)\right)\right) $. 

        Se chiamo \[
            \bm{u}(t)\coloneqq\begin{pmatrix}
                v(t)\\\varphi\left(v(t)\right)
            \end{pmatrix}
        \]ho che \[
            \begin{cases}
                u_1'(t)=v'(t)=  f_1\left(v(t), \varphi\left(v(t)\right)\right) = f_1\left(u_1(t),u_2(t)\right)\\[2ex]
                u_2'(t)=\varphi'\left(v(t)\right)\,v'(t) = \displaystyle\frac{f_2}{f_1} f_1 = f_2
            \end{cases}
        \]e inoltre $ \bm{u}(0)=\bm{p}. $\qed
    \end{itemize}
}
%% END
\esempio{
    \[
        \begin{cases}
            x'=-y^{2}\\ 
            y'=x^{2}
        \end{cases}
    \]
    \begin{enumerate}
        \item \emph{Equilibri}. \[
            \bm{f}(x,y)=(-y^{2},x^{2}) = (0,0)
        \] 
        
        $\implies$ ho un solo punto di equilibrio, $ \bm{0} $. 
        \item \emph{Punti con tangente orizzontale o verticale}.
        
        Il vettore tangente ad una soluzione in $ (x,y) $ è $ \bm{f}(x,y) $, dunque: \begin{itemize}
            \item tangente verticale: $ x'=0 $ $ \iff $ $ y^{2} =0$ 
            
            $\implies$ l'asse $ x $ viene intersecato verticalmente dalle orbite;
            \item tangente verticale: $ y'=0 $ $ \iff $ $ x^{2}=0 $ 
            
            $\implies$ l'asse $ y $ viene intersecato orizzontalmente dalle orbite.
        \end{itemize}

        In $ (x,0) $ il vettore tangente alla soluzione che passa per quel punto è $ (0,x^{2}) $, dunque tutte le orbite che passano per l'asse delle $ x $ sono percorse verso l'alto.

        In $ (0,y) $ il vettore tangente alla soluzione che passa per quel punto è $ (-y^{2}, 0 ) $, dunque tutte le orbite che passano per l'asse delle $ y $ sono percorse verso sinistra.
        \item \emph{Orbite non singolari}
        
        Consideriamo i punti in cui $ f_1(x,y)\neq 0 $, ovvero i punti con $ y\neq 0 $. Cerchiamo l'equazione delle orbite dei punti che non sono sull'asse $ x $. \[
            \begin{cases}
                \varphi'(x)=-\frac{x^{2}}{\varphi^{2}(x)}\\
                \varphi(p_1)=p_2\tag*{(PC)\textsubscript{$\varphi$}}
            \end{cases}
        \]che è a variabili separabili: \begin{align*}
            \varphi^{2}(x)\,\varphi'(x) & = -x^{2}\\[2ex] 
            \int \varphi^{2}(x)\,\varphi'(x)\,dx & = \int -x^{2}\,dx\\[2ex] 
            \int \varphi\,d\varphi & = -\int x^{2}\,dx\\[2ex]
            \frac{1}{3} \varphi^{3}(x) &= -\frac{1}{3}x^{3} + c
        \end{align*}

        Imponendo il passaggio per $ \bm{p} $, otteniamo che $ \displaystyle c= \frac{1}{3}\, p_2^{3}$. 
        
        Dunque \[
            \varphi(x)=\sqrt[3]{p_2^{3}-x^{3}}
        \]e le orbite sono quelle mostrate in figura \ref{fig:orbiteradiceterza}
        \begin{figure}
            \begin{center}
                % This file was created by matlab2tikz.
%
%The latest updates can be retrieved from
%  http://www.mathworks.com/matlabcentral/fileexchange/22022-matlab2tikz-matlab2tikz
%where you can also make suggestions and rate matlab2tikz.
%
\begin{tikzpicture}

    \begin{axis}[%
      width=0.8*\textwidth,
      axis equal,
      axis lines=middle,
      xmin=-5,
      xmax=5,
      ymin=-5,
      ymax=5,
      axis background/.style={fill=white}
      ]
    \addplot [black] {-x};
    \fill [black] (0,0) circle (0.1);
    \draw [-{Latex[length=2.5mm]}] (1,-1) -- (0.15,-0.15);
    \draw [-{Latex[length=2.5mm]}] (0,0) -- (-0.45,0.45);
    \addplot [color=black, forget plot]
      table[row sep=crcr]{%
    -5	2.43333124830413\\
    -4.95	2.20325791411787\\
    -4.9	1.91810939473754\\
    -4.85	1.51715490996357\\
    -4.8	0\\
    };
    \addplot [color=black, forget plot]
      table[row sep=crcr]{%
    -4.8	-0\\
    -4.75	-1.50665561007714\\
    -4.7	-1.89165313589352\\
    -4.65	-2.15783154291702\\
    -4.6	-2.36666887081282\\
    -4.55	-2.54044224221422\\
    -4.5	-2.69008710671805\\
    -4.45	-2.82188944180945\\
    -4.4	-2.939838638621\\
    -4.35	-3.04664595911424\\
    -4.3	-3.14424918656787\\
    -4.25	-3.23408790150043\\
    -4.2	-3.31726488820732\\
    -4.15	-3.39464627442907\\
    -4.1	-3.46692655020797\\
    -4.05	-3.53467239404558\\
    -4	-3.59835315622033\\
    -3.95	-3.65836263567708\\
    -3.9	-3.71503499958271\\
    -3.85	-3.76865665681006\\
    -3.8	-3.8194752712541\\
    -3.75	-3.86770671173796\\
    -3.7	-3.91354048616405\\
    -3.65	-3.95714404406568\\
    -3.6	-3.99866622197514\\
    -3.55	-4.0382400308764\\
    -3.5	-4.07598493260902\\
    -3.45	-4.11200871494441\\
    -3.4	-4.14640904832589\\
    -3.35	-4.17927478776402\\
    -3.3	-4.21068706897219\\
    -3.25	-4.24072023705863\\
    -3.2	-4.26944263795511\\
    -3.15	-4.29691729655335\\
    -3.1	-4.32320250073683\\
    -3.05	-4.34835230677976\\
    -3	-4.3724169786731\\
    -2.95	-4.39544337163964\\
    -2.9	-4.41747526827306\\
    -2.85	-4.43855367427352\\
    -2.8	-4.45871707957417\\
    -2.75	-4.47800168969859\\
    -2.7	-4.49644163141121\\
    -2.65	-4.51406913608526\\
    -2.6	-4.5309147036877\\
    -2.55	-4.54700724984663\\
    -2.5	-4.56237423810478\\
    -2.45	-4.57704179916166\\
    -2.4	-4.59103483865373\\
    -2.35	-4.60437713480893\\
    -2.3	-4.61709142713204\\
    -2.25	-4.62919949712464\\
    -2.2	-4.64072224191325\\
    -2.15	-4.65167974154844\\
    -2.1	-4.66209132064226\\
    -2.05	-4.67197560492966\\
    -2	-4.68135057326899\\
    -1.95	-4.69023360553561\\
    -1.9	-4.69864152680957\\
    -1.85	-4.70659064821255\\
    -1.8	-4.7140968047088\\
    -1.75	-4.72117539014982\\
    -1.7	-4.72784138981183\\
    -1.65	-4.73410941064774\\
    -1.6	-4.73999370945179\\
    -1.55	-4.7455082191138\\
    -1.5	-4.7506665731213\\
    -1.45	-4.75548212845142\\
    -1.4	-4.75996798697958\\
    -1.35	-4.76413701551886\\
    -1.3	-4.76800186459246\\
    -1.25	-4.77157498603078\\
    -1.2	-4.77486864947567\\
    -1.15	-4.77789495786555\\
    -1.1	-4.78066586196769\\
    -1.05	-4.78319317401706\\
    -1	-4.78548858051474\\
    -0.95	-4.78756365423345\\
    -0.899999999999999	-4.78942986547258\\
    -0.85	-4.79109859260026\\
    -0.8	-4.79258113191607\\
    -0.75	-4.79388870686385\\
    -0.699999999999999	-4.79503247662074\\
    -0.649999999999999	-4.79602354408525\\
    -0.6	-4.79687296328433\\
    -0.55	-4.7975917462165\\
    -0.5	-4.79819086914608\\
    -0.449999999999999	-4.79868127836094\\
    -0.399999999999999	-4.79907389540437\\
    -0.35	-4.79937962178998\\
    -0.3	-4.79960934320654\\
    -0.25	-4.79977393321878\\
    -0.199999999999999	-4.79988425646833\\
    -0.149999999999999	-4.79995117137829\\
    -0.0999999999999996	-4.7999855323638\\
    -0.0499999999999998	-4.79999819155024\\
    8.88178419700125e-16	-4.8\\
    0.0500000000000007	-4.80000180844839\\
    0.1	-4.80001446754899\\
    0.149999999999999	-4.8000488276283\\
    0.199999999999999	-4.80011573795004\\
    0.25	-4.80022604548897\\
    0.3	-4.80039059321517\\
    0.35	-4.80062021788932\\
    0.399999999999999	-4.80092574737107\\
    0.45	-4.80131799744245\\
    0.5	-4.80180776814992\\
    0.55	-4.80240583966935\\
    0.6	-4.8031229676995\\
    0.649999999999999	-4.8039698783909\\
    0.7	-4.80495726281873\\
    0.75	-4.80609577100958\\
    0.8	-4.80739600553389\\
    0.85	-4.80886851467779\\
    0.899999999999999	-4.81052378520972\\
    0.95	-4.8123722347595\\
    1	-4.81442420382937\\
    1.05	-4.81668994745888\\
    1.1	-4.81917962656736\\
    1.15	-4.82190329900017\\
    1.2	-4.82487091030687\\
    1.25	-4.82809228428169\\
    1.3	-4.83157711329864\\
    1.35	-4.8353349484757\\
    1.4	-4.8393751897043\\
    1.45	-4.84370707558215\\
    1.5	-4.84833967328888\\
    1.55	-4.8532818684457\\
    1.6	-4.85854235500106\\
    1.65	-4.86412962518571\\
    1.7	-4.87005195958094\\
    1.75	-4.87631741734448\\
    1.8	-4.88293382663862\\
    1.85	-4.88990877530508\\
    1.9	-4.89724960183065\\
    1.95	-4.90496338664699\\
    2	-4.91305694380694\\
    2.05	-4.92153681307814\\
    2.1	-4.93040925249353\\
    2.15	-4.93968023139562\\
    2.2	-4.94935542401002\\
    2.25	-4.95944020358025\\
    2.3	-4.96993963709383\\
    2.35	-4.98085848062601\\
    2.4	-4.99220117532457\\
    2.45	-5.00397184405552\\
    2.5	-5.01617428872595\\
    2.55	-5.02881198829654\\
    2.6	-5.04188809749251\\
    2.65	-5.05540544621769\\
    2.7	-5.06936653967282\\
    2.75	-5.08377355917515\\
    2.8	-5.09862836367258\\
    2.85	-5.11393249194192\\
    2.9	-5.12968716545745\\
    2.95	-5.14589329191211\\
    3	-5.16255146937099\\
    3.05	-5.17966199103314\\
    3.1	-5.19722485057551\\
    3.15	-5.2152397480499\\
    3.2	-5.23370609630174\\
    3.25	-5.25262302787746\\
    3.3	-5.27198940238559\\
    3.35	-5.29180381427528\\
    3.4	-5.31206460099498\\
    3.45	-5.33276985149294\\
    3.5	-5.35391741502115\\
    3.55	-5.37550491020363\\
    3.6	-5.3975297343305\\
    3.65	-5.41998907283918\\
    3.7	-5.44287990894502\\
    3.75	-5.46619903338409\\
    3.8	-5.48994305423219\\
    3.85	-5.51410840676505\\
    3.9	-5.53869136332627\\
    3.95	-5.56368804317094\\
    4	-5.58909442225448\\
    4.05	-5.61490634293814\\
    4.1	-5.64111952358409\\
    4.15	-5.6677295680154\\
    4.2	-5.6947319748176\\
    4.25	-5.72212214646078\\
    4.3	-5.74989539822316\\
    4.35	-5.77804696689878\\
    4.4	-5.80657201927403\\
    4.45	-5.83546566035971\\
    4.5	-5.86472294136691\\
    4.55	-5.89433886741701\\
    4.6	-5.92430840497767\\
    4.65	-5.9546264890184\\
    4.7	-5.98528802988077\\
    4.75	-6.01628791985994\\
    4.8	-6.04762103949539\\
    4.85	-6.07928226357011\\
    4.9	-6.11126646681883\\
    4.95	-6.14356852934671\\
    5	-6.17618334176118\\
    };
    
    \addplot[area legend, draw=black, fill=black, forget plot]
    table[row sep=crcr] {%
    x	y\\
    -3.72083693095564	-3.89868811849171\\
    -3.70976252472368	-3.90976252472368\\
    -3.70976252472368	-3.90976252472368\\
    -3.70976252472368	-3.90976252472368\\
    -3.70976252472368	-3.90976252472368\\
    -3.70976252472368	-3.90976252472368\\
    -3.73191133718761	-3.88761371225975\\
    -3.7829093445653	-3.93861171963744\\
    -3.90976252472368	-3.70976252472368\\
    -3.68091332980991	-3.83661570488206\\
    -3.73191133718761	-3.88761371225975\\
    }--cycle;
    \addplot [color=black, forget plot]
      table[row sep=crcr]{%
    -5	3.41469990581837\\
    -4.95	3.30508475380455\\
    -4.9	3.19010616025606\\
    -4.85	3.06878578458167\\
    -4.8	2.939838638621\\
    -4.75	2.80152446746804\\
    -4.7	2.65139359475723\\
    -4.65	2.48581971993629\\
    -4.6	2.2990544317303\\
    -4.55	2.08103675817557\\
    -4.5	1.81114484514709\\
    -4.45	1.43210263327108\\
    -4.4	0\\
    };
    \addplot [color=black, forget plot]
      table[row sep=crcr]{%
    -4.4	-0\\
    -4.35	-1.42129436301186\\
    -4.3	-1.7839101015784\\
    -4.25	-2.03427387882488\\
    -4.2	-2.23043112043839\\
    -4.15	-2.39342092816138\\
    -4.1	-2.53357376911577\\
    -4.05	-2.65682942377063\\
    -4	-2.76695856664369\\
    -3.95	-2.86652450319574\\
    -3.9	-2.95735977133685\\
    -3.85	-3.04082614752338\\
    -3.8	-3.11796711754744\\
    -3.75	-3.18960248319805\\
    -3.7	-3.25638979571086\\
    -3.65	-3.3188657699515\\
    -3.6	-3.37747509342703\\
    -3.55	-3.4325910094841\\
    -3.5	-3.48453036602697\\
    -3.45	-3.53356484084942\\
    -3.4	-3.57992946398274\\
    -3.35	-3.62382918986462\\
    -3.3	-3.66544403681055\\
    -3.25	-3.70493315680415\\
    -3.2	-3.74243809494084\\
    -3.15	-3.77808542685497\\
    -3.1	-3.81198891294557\\
    -3.05	-3.84425127311386\\
    -3	-3.87496566046572\\
    -2.95	-3.90421689400247\\
    -2.9	-3.93208249670732\\
    -2.85	-3.95863357525606\\
    -2.8	-3.98393556988968\\
    -2.75	-4.00804889711701\\
    -2.7	-4.03102950339355\\
    -2.65	-4.05292934440889\\
    -2.6	-4.07379680186249\\
    -2.55	-4.09367704743449\\
    -2.5	-4.11261236193034\\
    -2.45	-4.1306424161952\\
    -2.4	-4.1478045192799\\
    -2.35	-4.16413383843723\\
    -2.3	-4.17966359479151\\
    -2.25	-4.19442523792143\\
    -2.2	-4.20844860209926\\
    -2.15	-4.22176204651851\\
    -2.1	-4.23439258150042\\
    -2.05	-4.24636598238375\\
    -2	-4.2577068925634\\
    -1.95	-4.2684389169411\\
    -1.9	-4.27858470688149\\
    -1.85	-4.28816603762166\\
    -1.8	-4.29720387895949\\
    -1.75	-4.30571845994028\\
    -1.7	-4.31372932817138\\
    -1.65	-4.3212554043164\\
    -1.6	-4.32831503225368\\
    -1.55	-4.33492602532579\\
    -1.5	-4.34110570905614\\
    -1.45	-4.34687096066514\\
    -1.4	-4.35223824567986\\
    -1.35	-4.35722365189759\\
    -1.3	-4.3618429209344\\
    -1.25	-4.36611147756342\\
    -1.2	-4.37004445702512\\
    -1.15	-4.37365673047123\\
    -1.1	-4.37696292868603\\
    -1.05	-4.3799774642127\\
    -0.999999999999999	-4.3827145519982\\
    -0.949999999999999	-4.38518822865711\\
    -0.899999999999999	-4.38741237044377\\
    -0.849999999999999	-4.38940071001136\\
    -0.799999999999999	-4.39116685202757\\
    -0.749999999999999	-4.39272428770812\\
    -0.699999999999999	-4.39408640832168\\
    -0.649999999999999	-4.39526651771311\\
    -0.599999999999999	-4.39627784388566\\
    -0.549999999999999	-4.3971335496773\\
    -0.499999999999999	-4.39784674256114\\
    -0.449999999999999	-4.39843048359551\\
    -0.399999999999999	-4.39889779554507\\
    -0.35	-4.39926167019068\\
    -0.299999999999999	-4.39953507484244\\
    -0.249999999999999	-4.39973095806743\\
    -0.199999999999999	-4.39986225464099\\
    -0.149999999999999	-4.39994188972841\\
    -0.0999999999999996	-4.39998278230177\\
    -0.0499999999999989	-4.39999784779509\\
    8.88178419700125e-16	-4.4\\
    0.0500000000000007	-4.4000021522028\\
    0.100000000000001	-4.40001721756348\\
    0.15	-4.40005810873671\\
    0.200000000000001	-4.4001377367351\\
    0.250000000000001	-4.40026900903497\\
    0.3	-4.40046482692586\\
    0.35	-4.40073808210567\\
    0.399999999999999	-4.40110165252475\\
    0.45	-4.40156839748363\\
    0.5	-4.40215115199106\\
    0.55	-4.40286272039121\\
    0.6	-4.40371586927097\\
    0.649999999999999	-4.40472331966154\\
    0.7	-4.40589773855084\\
    0.75	-4.40725172972692\\
    0.8	-4.40879782397563\\
    0.85	-4.41054846865957\\
    0.899999999999999	-4.41251601670897\\
    0.95	-4.41471271505911\\
    1	-4.41715069257253\\
    1.05	-4.41984194748881\\
    1.1	-4.42279833444796\\
    1.15	-4.426031551138\\
    1.2	-4.42955312462073\\
    1.25	-4.43337439739352\\
    1.3	-4.43750651324839\\
    1.35	-4.44196040299268\\
    1.4	-4.44674677009875\\
    1.45	-4.45187607635236\\
    1.5	-4.45735852757163\\
    1.55	-4.4632040594701\\
    1.6	-4.46942232373823\\
    1.65	-4.47602267441873\\
    1.7	-4.4830141546505\\
    1.75	-4.49040548385564\\
    1.8	-4.49820504544276\\
    1.85	-4.50642087509766\\
    1.9	-4.51506064972999\\
    1.95	-4.52413167714154\\
    2	-4.53364088647765\\
    2.05	-4.54359481951908\\
    2.1	-4.55399962286679\\
    2.15	-4.56486104106651\\
    2.2	-4.57618441071419\\
    2.25	-4.58797465557711\\
    2.3	-4.60023628275896\\
    2.35	-4.61297337993022\\
    2.4	-4.62618961363815\\
    2.45	-4.63988822870376\\
    2.5	-4.65407204870588\\
    2.55	-4.66874347754531\\
    2.6	-4.68390450207526\\
    2.65	-4.69955669577749\\
    2.7	-4.71570122345718\\
    2.75	-4.73233884692341\\
    2.8	-4.74946993161649\\
    2.85	-4.76709445413835\\
    2.9	-4.78521201063719\\
    2.95	-4.80382182599364\\
    3	-4.82292276375218\\
    3.05	-4.84251333673835\\
    3.1	-4.86259171830001\\
    3.15	-4.8831557541092\\
    3.2	-4.90420297445978\\
    3.25	-4.92573060699547\\
    3.3	-4.94773558980296\\
    3.35	-4.97021458480507\\
    3.4	-4.99316399138997\\
    3.45	-5.01657996021387\\
    3.5	-5.04045840711649\\
    3.55	-5.06479502709083\\
    3.6	-5.0895853082511\\
    3.65	-5.11482454574594\\
    3.7	-5.14050785556663\\
    3.75	-5.16663018820354\\
    3.8	-5.19318634210752\\
    3.85	-5.22017097691613\\
    3.9	-5.24757862640856\\
    3.95	-5.27540371115633\\
    4	-5.30364055084068\\
    4.05	-5.332283376211\\
    4.1	-5.36132634066212\\
    4.15	-5.39076353141155\\
    4.2	-5.42058898026123\\
    4.25	-5.45079667393119\\
    4.3	-5.48138056395562\\
    4.35	-5.51233457613462\\
    4.4	-5.54365261953744\\
    4.45	-5.57532859505547\\
    4.5	-5.6073564035055\\
    4.55	-5.63972995328585\\
    4.6	-5.67244316758968\\
    4.65	-5.70548999118167\\
    4.7	-5.73886439674557\\
    4.75	-5.77256039081165\\
    4.8	-5.80657201927403\\
    4.85	-5.84089337250893\\
    4.9	-5.87551859010587\\
    4.95	-5.91044186522429\\
    5	-5.94565744858888\\
    };
    
    \addplot[area legend, draw=black, fill=black, forget plot]
    table[row sep=crcr] {%
    x	y\\
    -3.39322040106781	-3.59134422759226\\
    -3.39228231433004	-3.59228231433004\\
    -3.39228231433004	-3.59228231433004\\
    -3.39228231433004	-3.59228231433004\\
    -3.39228231433004	-3.59228231433004\\
    -3.39228231433004	-3.59228231433004\\
    -3.39415848780559	-3.59040614085449\\
    -3.45096958087035	-3.64721723391925\\
    -3.59228231433004	-3.39228231433004\\
    -3.33734739474082	-3.53359504778972\\
    -3.39415848780559	-3.59040614085449\\
    }--cycle;
    \addplot [color=black, forget plot]
      table[row sep=crcr]{%
    -5	3.93649718310217\\
    -4.95	3.85495791238075\\
    -4.9	3.77155585461804\\
    -4.85	3.68609646863451\\
    -4.8	3.59835315622033\\
    -4.75	3.50805965759495\\
    -4.7	3.41490000543241\\
    -4.65	3.31849501754868\\
    -4.6	3.2183837676756\\
    -4.55	3.11399757505469\\
    -4.5	3.00462250345868\\
    -4.45	2.88934356885749\\
    -4.4	2.76695856664369\\
    -4.35	2.63583878230465\\
    -4.3	2.49369075744464\\
    -4.25	2.3371182901163\\
    -4.2	2.16073591750959\\
    -4.15	1.95511477752707\\
    -4.1	1.70092222164965\\
    -4.05	1.34444447606544\\
    -4	0\\
    };
    \addplot [color=black, forget plot]
      table[row sep=crcr]{%
    -4	-0\\
    -3.95	-1.33328732480132\\
    -3.9	-1.67280845040534\\
    -3.85	-1.90684282377084\\
    -3.8	-2.08989857798625\\
    -3.75	-2.24173925559838\\
    -3.7	-2.37207210759166\\
    -3.65	-2.48648035278273\\
    -3.6	-2.58850945078415\\
    -3.55	-2.68057039355508\\
    -3.5	-2.76438740683944\\
    -3.45	-2.84124222420702\\
    -3.4	-2.91211735227267\\
    -3.35	-2.97778497771623\\
    -3.3	-3.03886470765518\\
    -3.25	-3.09586249965226\\
    -3.2	-3.14919774648174\\
    -3.15	-3.19922263018076\\
    -3.1	-3.24623627419026\\
    -3.05	-3.2904953014888\\
    -3	-3.33222185164595\\
    -2.95	-3.37160976432446\\
    -2.9	-3.40882941563296\\
    -2.85	-3.4440315485722\\
    -2.8	-3.47735034137574\\
    -2.75	-3.50890589081016\\
    -2.7	-3.53880624095771\\
    -2.65	-3.56714905500759\\
    -2.6	-3.59402300383506\\
    -2.55	-3.61950892782091\\
    -2.5	-3.64368081556092\\
    -2.45	-3.66660663354419\\
    -2.4	-3.68834903364676\\
    -2.35	-3.70896595976667\\
    -2.3	-3.72851117067382\\
    -2.25	-3.74703469284268\\
    -2.2	-3.7645832144467\\
    -2.15	-3.78120042964864\\
    -2.1	-3.79692734069548\\
    -2.05	-3.81180252402532\\
    -2	-3.82586236554478\\
    -1.95	-3.839141269386\\
    -1.9	-3.85167184375929\\
    -1.85	-3.8634850669495\\
    -1.8	-3.87461043603686\\
    -1.75	-3.88507610053522\\
    -1.7	-3.89490898281869\\
    -1.65	-3.90413488693867\\
    -1.6	-3.912778597207\\
    -1.55	-3.92086396773085\\
    -1.5	-3.928414003924\\
    -1.45	-3.93545093688214\\
    -1.4	-3.94199629139342\\
    -1.35	-3.94807094825566\\
    -1.3	-3.95369520148593\\
    -1.25	-3.95888881093441\\
    -1.2	-3.96367105075071\\
    -1.15	-3.96806075409521\\
    -1.1	-3.97207635444015\\
    -1.05	-3.97573592376282\\
    -1	-3.97905720789639\\
    -0.95	-3.98205765927177\\
    -0.9	-3.98475446725496\\
    -0.85	-3.9871645862595\\
    -0.8	-3.98930476179092\\
    -0.75	-3.99119155456048\\
    -0.7	-3.99284136278774\\
    -0.65	-3.99427044279541\\
    -0.6	-3.9954949279864\\
    -0.55	-3.9965308462796\\
    -0.5	-3.99739413607028\\
    -0.45	-3.99810066077037\\
    -0.4	-3.99866622197514\\
    -0.35	-3.99910657129448\\
    -0.3	-3.99943742087989\\
    -0.25	-3.99967445267212\\
    -0.2	-3.99983332638841\\
    -0.15	-3.999929686264\\
    -0.0999999999999996	-3.99997916655816\\
    -0.0499999999999998	-3.99999739583164\\
    0	-4\\
    0.0499999999999998	-4.00000260416497\\
    0.100000000000001	-4.00002083322483\\
    0.15	-4.00007031126407\\
    0.2	-4.0001666597227\\
    0.25	-4.00032549434597\\
    0.3	-4.00056242091697\\
    0.350000000000001	-4.00089302977628\\
    0.4	-4.00133288913564\\
    0.45	-4.00189753719581\\
    0.5	-4.00260247308292\\
    0.55	-4.00346314662189\\
    0.6	-4.00449494697088\\
    0.649999999999999	-4.00571319014622\\
    0.7	-4.00713310547364\\
    0.75	-4.0087698210081\\
    0.8	-4.01063834797167\\
    0.85	-4.01275356426607\\
    0.899999999999999	-4.01513019712441\\
    0.95	-4.01778280497382\\
    1	-4.02072575858906\\
    1.05	-4.02397322162419\\
    1.1	-4.02753913061748\\
    1.15	-4.0314371745714\\
    1.2	-4.03568077421651\\
    1.25	-4.04028306107407\\
    1.3	-4.04525685643787\\
    1.35	-4.05061465039998\\
    1.4	-4.05636858104921\\
    1.45	-4.06253041397349\\
    1.5	-4.06911152219885\\
    1.55	-4.07612286669812\\
    1.6	-4.08357497760117\\
    1.65	-4.09147793623635\\
    1.7	-4.09984135812903\\
    1.75	-4.10867437707794\\
    1.8	-4.11798563042365\\
    1.85	-4.12778324561581\\
    1.9	-4.13807482817678\\
    1.95	-4.14886745114917\\
    2	-4.16016764610381\\
    2.05	-4.17198139577253\\
    2.1	-4.18431412835758\\
    2.15	-4.19717071355611\\
    2.2	-4.21055546032458\\
    2.25	-4.22447211639401\\
    2.3	-4.23892386953327\\
    2.35	-4.25391335054368\\
    2.4	-4.26944263795511\\
    2.45	-4.28551326438077\\
    2.5	-4.30212622447582\\
    2.55	-4.31928198443375\\
    2.6	-4.33698049294394\\
    2.65	-4.3552211935248\\
    2.7	-4.3740030381387\\
    2.75	-4.39332450198799\\
    2.8	-4.41318359938584\\
    2.85	-4.43357790059152\\
    2.9	-4.45450454949628\\
    2.95	-4.47596028204469\\
    3	-4.49794144527541\\
    3.05	-4.52044401686613\\
    3.1	-4.54346362506894\\
    3.15	-4.5669955689254\\
    3.2	-4.59103483865373\\
    3.25	-4.61557613610522\\
    3.3	-4.64061389519198\\
    3.35	-4.66614230219372\\
    3.4	-4.69215531585751\\
    3.45	-4.71864668721076\\
    3.5	-4.74560997901466\\
    3.55	-4.77303858479212\\
    3.6	-4.80092574737107\\
    3.65	-4.82926457689136\\
    3.7	-4.85804806822989\\
    3.75	-4.88726911780576\\
    3.8	-4.9169205397334\\
    3.85	-4.94699508129806\\
    3.9	-4.97748543773394\\
    3.95	-5.00838426629069\\
    4	-5.03968419957949\\
    4.05	-5.07137785819438\\
    4.1	-5.1034578626093\\
    4.15	-5.13591684435504\\
    4.2	-5.16874745648424\\
    4.25	-5.20194238333536\\
    4.3	-5.23549434960994\\
    4.35	-5.26939612877947\\
    4.4	-5.30364055084068\\
    4.45	-5.33822050943947\\
    4.5	-5.37312896838572\\
    4.55	-5.4083589675817\\
    4.6	-5.44390362838836\\
    4.65	-5.47975615845403\\
    4.7	-5.51590985603062\\
    4.75	-5.55235811380267\\
    4.8	-5.58909442225448\\
    4.85	-5.6261123726007\\
    4.9	-5.66340565930516\\
    4.95	-5.70096808221258\\
    5	-5.73879354831717\\
    };
    
    \addplot[area legend, draw=black, fill=black, forget plot]
    table[row sep=crcr] {%
    x	y\\
    -3.06589924167803	-3.28370496619477\\
    -3.06589924167803	-3.28370496619477\\
    -3.06589924167803	-3.28370496619477\\
    -3.06589924167803	-3.28370496619477\\
    -3.06589924167803	-3.28370496619477\\
    -3.06589924167803	-3.28370496619477\\
    -3.06589924167803	-3.28370496619477\\
    -3.12580117350242	-3.34360689801916\\
    -3.2748021039364	-3.0748021039364\\
    -3.00599730985363	-3.22380303437037\\
    -3.06589924167803	-3.28370496619477\\
    }--cycle;
    \addplot [color=black, forget plot]
      table[row sep=crcr]{%
    -5	4.27893064384067\\
    -4.95	4.21024285680844\\
    -4.9	4.14068166173797\\
    -4.85	4.07018073559454\\
    -4.8	3.99866622197514\\
    -4.75	3.92605553987667\\
    -4.7	3.85225594439255\\
    -4.65	3.77716277394065\\
    -4.6	3.70065729739648\\
    -4.55	3.62260404495532\\
    -4.5	3.54284746479196\\
    -4.45	3.46120768760821\\
    -4.4	3.37747509342703\\
    -4.35	3.29140324406587\\
    -4.3	3.20269954491915\\
    -4.25	3.11101268703176\\
    -4.2	3.01591541718809\\
    -4.15	2.91688034715677\\
    -4.1	2.81324507009292\\
    -4.05	2.70416025311282\\
    -4	2.58850945078415\\
    -3.95	2.46477946918048\\
    -3.9	2.33083860350175\\
    -3.85	2.18352884962524\\
    -3.8	2.01784038710825\\
    -3.75	1.8249984362281\\
    -3.7	1.58700410245128\\
    -3.65	1.25382826354938\\
    -3.6	0\\
    };
    \addplot [color=black, forget plot]
      table[row sep=crcr]{%
    -3.6	-0\\
    -3.55	-1.24227232487739\\
    -3.5	-1.5578855860124\\
    -3.45	-1.77500165310962\\
    -3.4	-1.94447262241856\\
    -3.35	-2.08474393357267\\
    -3.3	-2.20487897896575\\
    -3.25	-2.31009269819094\\
    -3.2	-2.40369800276695\\
    -3.15	-2.48794866615549\\
    -3.1	-2.56445716036143\\
    -3.05	-2.63442267416152\\
    -3	-2.69876486716525\\
    -2.95	-2.75820688858728\\
    -2.9	-2.81332930277019\\
    -2.85	-2.86460645344058\\
    -2.8	-2.91243176788213\\
    -2.75	-2.95713584190443\\
    -2.7	-2.99899966648136\\
    -2.65	-3.03826449734771\\
    -2.6	-3.07513935076953\\
    -2.55	-3.10980678624441\\
    -2.5	-3.14242743042701\\
    -2.45	-3.17314356103136\\
    -2.4	-3.20208197846633\\
    -2.35	-3.22935633063355\\
    -2.3	-3.25506901284616\\
    -2.25	-3.27931273400483\\
    -2.2	-3.30217181798291\\
    -2.15	-3.32372329298502\\
    -2.1	-3.34403780968063\\
    -2.05	-3.36318041997095\\
    -2	-3.38121124148806\\
    -1.95	-3.39818602776578\\
    -1.9	-3.41415666004636\\
    -1.85	-3.42917157359721\\
    -1.8	-3.4432761289903\\
    -1.75	-3.45651293688558\\
    -1.7	-3.46892214333853\\
    -1.65	-3.48054168143494\\
    -1.6	-3.49140749407442\\
    -1.55	-3.50155373192867\\
    -1.5	-3.51101292995174\\
    -1.45	-3.51981616528734\\
    -1.4	-3.52799319897976\\
    -1.35	-3.5355726035316\\
    -1.3	-3.54258187804877\\
    -1.25	-3.54904755246047\\
    -1.2	-3.55499528208884\\
    -1.15	-3.56044993366385\\
    -1.1	-3.56543566372656\\
    -1.05	-3.56997599023468\\
    -1	-3.57409385807344\\
    -0.949999999999999	-3.57781169907984\\
    -0.899999999999999	-3.58115148710675\\
    -0.85	-3.5841347885821\\
    -0.799999999999999	-3.58678280895741\\
    -0.75	-3.58911643538605\\
    -0.699999999999999	-3.59115627592492\\
    -0.649999999999999	-3.5929226955121\\
    -0.6	-3.59443584893705\\
    -0.549999999999999	-3.5957157109879\\
    -0.5	-3.59678210393225\\
    -0.449999999999999	-3.59765472246325\\
    -0.399999999999999	-3.59835315622033\\
    -0.35	-3.5988969099748\\
    -0.299999999999999	-3.59930542155328\\
    -0.25	-3.59959807755697\\
    -0.199999999999999	-3.59979422692153\\
    -0.149999999999999	-3.59991319235125\\
    -0.0999999999999996	-3.59997427965163\\
    -0.0499999999999994	-3.59999678497655\\
    4.44089209850063e-16	-3.6\\
    0.0500000000000007	-3.6000032150177\\
    0.100000000000001	-3.60002571998085\\
    0.15	-3.60008680346253\\
    0.200000000000001	-3.60020574955751\\
    0.25	-3.6004018327177\\
    0.300000000000001	-3.6006943105283\\
    0.350000000000001	-3.60110241443493\\
    0.4	-3.60164533843748\\
    0.45	-3.60234222577462\\
    0.500000000000001	-3.60321215363176\\
    0.550000000000001	-3.60427411591559\\
    0.600000000000001	-3.60554700415042\\
    0.65	-3.60704958656418\\
    0.7	-3.60880048544596\\
    0.75	-3.61081815287203\\
    0.8	-3.61312084491233\\
    0.85	-3.61572659444601\\
    0.899999999999999	-3.61865318273015\\
    0.95	-3.62191810988224\\
    1	-3.62553856445265\\
    1.05	-3.62953139227823\\
    1.1	-3.63391306482233\\
    1.15	-3.63869964721907\\
    1.2	-3.64390676625079\\
    1.25	-3.64954957849654\\
    1.3	-3.65564273889617\\
    1.35	-3.66220036997897\\
    1.4	-3.66923603200685\\
    1.45	-3.67676269428092\\
    1.5	-3.6847927078552\\
    1.55	-3.69333777989389\\
    1.6	-3.70240894989728\\
    1.65	-3.71201656800751\\
    1.7	-3.72217027558834\\
    1.75	-3.73287898825314\\
    1.8	-3.74415088149343\\
    1.85	-3.75599337903529\\
    1.9	-3.76841314402527\\
    1.95	-3.78141607311938\\
    2	-3.79500729352051\\
    2.05	-3.80919116298053\\
    2.1	-3.82397127275443\\
    2.15	-3.8393504534655\\
    2.2	-3.85533078381304\\
    2.25	-3.87191360202824\\
    2.3	-3.88909951995964\\
    2.35	-3.90688843964739\\
    2.4	-3.92527957222631\\
    2.45	-3.9442714589805\\
    2.5	-3.96386199435843\\
    2.55	-3.98404845074623\\
    2.6	-4.00482750478888\\
    2.65	-4.02619526504359\\
    2.7	-4.04814730074787\\
    2.75	-4.07067867148454\\
    2.8	-4.09378395752964\\
    2.85	-4.11745729067414\\
    2.9	-4.14169238531793\\
    2.95	-4.16648256964389\\
    3	-4.19182081669086\\
    3.05	-4.21769977515622\\
    3.1	-4.24411179977208\\
    3.15	-4.2710489811132\\
    3.2	-4.29850317470858\\
    3.25	-4.32646602934382\\
    3.3	-4.35492901445552\\
    3.35	-4.38388344653348\\
    3.4	-4.41332051446052\\
    3.45	-4.44323130373325\\
    3.5	-4.47360681951958\\
    3.55	-4.50443800852139\\
    3.6	-4.53571577962154\\
    3.65	-4.56743102330508\\
    3.7	-4.59957462985363\\
    3.75	-4.63213750632089\\
    3.8	-4.66511059230433\\
    3.85	-4.69848487453516\\
    3.9	-4.73225140031419\\
    3.95	-4.76640128982643\\
    4	-4.80092574737107\\
    4.05	-4.83581607154715\\
    4.1	-4.87106366443767\\
    4.15	-4.9066600398371\\
    4.2	-4.94259683056856\\
    4.25	-4.97886579493782\\
    4.3	-5.01545882237184\\
    4.35	-5.05236793828941\\
    4.4	-5.0895853082511\\
    4.45	-5.12710324143526\\
    4.5	-5.16491419348545\\
    4.55	-5.20301076877394\\
    4.6	-5.24138572212416\\
    4.65	-5.28003196003382\\
    4.7	-5.31894254143834\\
    4.75	-5.35811067805309\\
    4.8	-5.3975297343305\\
    4.85	-5.43719322706679\\
    4.9	-5.47709482469102\\
    4.95	-5.51722834626726\\
    5	-5.557587760239\\
    };
    
    \addplot[area legend, draw=black, fill=black, forget plot]
    table[row sep=crcr] {%
    x	y\\
    -2.74134520650515	-2.97329858058037\\
    -2.74134520650515	-2.97329858058037\\
    -2.74134520650515	-2.97329858058037\\
    -2.74134520650515	-2.97329858058037\\
    -2.74134520650515	-2.97329858058037\\
    -2.74134520650515	-2.97329858058037\\
    -2.74134520650515	-2.97329858058037\\
    -2.80327552494466	-3.03522889901988\\
    -2.95732189354276	-2.75732189354276\\
    -2.67941488806564	-2.91136826214086\\
    -2.74134520650515	-2.97329858058037\\
    }--cycle;
    \addplot [color=black, forget plot]
      table[row sep=crcr]{%
    -5	4.51814892747109\\
    -4.95	4.45669362552509\\
    -4.9	4.39477686003393\\
    -4.85	4.33237018370575\\
    -4.8	4.26944263795511\\
    -4.75	4.20596045253427\\
    -4.7	4.14188669869105\\
    -4.65	4.07718088688333\\
    -4.6	4.01179849797794\\
    -4.55	3.94569043417403\\
    -4.5	3.87880237242552\\
    -4.45	3.81107399863058\\
    -4.4	3.74243809494084\\
    -4.35	3.67281944469957\\
    -4.3	3.60213350900703\\
    -4.25	3.53028481465819\\
    -4.2	3.45716497362754\\
    -4.15	3.38265022702969\\
    -4.1	3.30659836798173\\
    -4.05	3.2288448424795\\
    -4	3.14919774648174\\
    -3.95	3.0674313166234\\
    -3.9	2.983277327649\\
    -3.85	2.89641352118544\\
    -3.8	2.80644772607596\\
    -3.75	2.71289555841655\\
    -3.7	2.61514825740898\\
    -3.65	2.51242481396752\\
    -3.6	2.40369800276695\\
    -3.55	2.28757477335282\\
    -3.5	2.16209159538837\\
    -3.45	2.02433804508015\\
    -3.4	1.86969463209304\\
    -3.35	1.69006605486246\\
    -3.3	1.46883860941191\\
    -3.25	1.15980897567894\\
    -3.2	0\\
    };
    \addplot [color=black, forget plot]
      table[row sep=crcr]{%
    -3.2	-0\\
    -3.15	-1.14779033794501\\
    -3.1	-1.43855436194468\\
    -3.05	-1.63806819432795\\
    -3	-1.7933914044787\\
    -2.95	-1.92160248521068\\
    -2.9	-2.03109730607616\\
    -2.85	-2.12670912096187\\
    -2.8	-2.21150992547155\\
    -2.75	-2.28759069782819\\
    -2.7	-2.35644826269705\\
    -2.65	-2.41919645247475\\
    -2.6	-2.47668999972181\\
    -2.55	-2.52960145159073\\
    -2.5	-2.57847114115864\\
    -2.45	-2.62374089275194\\
    -2.4	-2.66577748131676\\
    -2.35	-2.70488940294455\\
    -2.3	-2.74133914329627\\
    -2.25	-2.77535233466951\\
    -2.2	-2.80712471264813\\
    -2.15	-2.83682748460024\\
    -2.1	-2.86461153103556\\
    -2.05	-2.89061073525953\\
    -2	-2.91494465244874\\
    -1.95	-2.93772067151641\\
    -1.9	-2.95903578284901\\
    -1.85	-2.97897803642651\\
    -1.8	-2.99762775427414\\
    -1.75	-3.01505854618548\\
    -1.7	-3.03133816656442\\
    -1.65	-3.04652924193967\\
    -1.6	-3.06068989243582\\
    -1.55	-3.07387426569862\\
    -1.5	-3.08613299808309\\
    -1.45	-3.09751361504498\\
    -1.4	-3.10806088042817\\
    -1.35	-3.11781710256464\\
    -1.3	-3.12682240369041\\
    -1.25	-3.13511495804882\\
    -1.2	-3.1427312031392\\
    -1.15	-3.14970602782766\\
    -1.1	-3.15607294043183\\
    -1.05	-3.16186421939409\\
    -1	-3.16711104874753\\
    -0.95	-3.17184364023773\\
    -0.899999999999999	-3.17609134367924\\
    -0.85	-3.17988274688643\\
    -0.799999999999999	-3.18324576631711\\
    -0.75	-3.18620772939668\\
    -0.7	-3.18879544934471\\
    -0.649999999999999	-3.19103529320198\\
    -0.6	-3.19295324364839\\
    -0.549999999999999	-3.19457495511033\\
    -0.5	-3.1959258045759\\
    -0.45	-3.19703093746689\\
    -0.399999999999999	-3.19791530885622\\
    -0.35	-3.1986037202666\\
    -0.299999999999999	-3.19912085224062\\
    -0.25	-3.19949129283252\\
    -0.2	-3.1997395621377\\
    -0.149999999999999	-3.19989013294668\\
    -0.0999999999999996	-3.19996744758552\\
    -0.0499999999999994	-3.19999593098441\\
    4.44089209850063e-16	-3.2\\
    0.0500000000000003	-3.20000406900524\\
    0.100000000000001	-3.2000325517522\\
    0.15	-3.20010985950961\\
    0.200000000000001	-3.20026039547678\\
    0.25	-3.20050854547952\\
    0.3	-3.20087866496163\\
    0.350000000000001	-3.20139506229741\\
    0.4	-3.20208197846633\\
    0.450000000000001	-3.20296356315068\\
    0.5	-3.20406384733972\\
    0.55	-3.20540671255072\\
    0.600000000000001	-3.20701585680648\\
    0.65	-3.20891475754098\\
    0.700000000000001	-3.21112663163925\\
    0.75	-3.21367439285311\\
    0.8	-3.21658060687125\\
    0.85	-3.21986744435871\\
    0.9	-3.22355663231713\\
    0.95	-3.22766940415125\\
    1	-3.23222644885926\\
    1.05	-3.23724785979289\\
    1.1	-3.24275308345714\\
    1.15	-3.24876086883799\\
    1.2	-3.25528921775908\\
    1.25	-3.2623553367739\\
    1.3	-3.269975591098\\
    1.35	-3.27816546107627\\
    1.4	-3.28693950166235\\
    1.45	-3.29631130536136\\
    1.5	-3.3062934690535\\
    1.55	-3.31689756507478\\
    1.6	-3.32813411688305\\
    1.65	-3.34001257958378\\
    1.7	-3.35254132553103\\
    1.75	-3.36572763515658\\
    1.8	-3.37957769311521\\
    1.85	-3.3940965897682\\
    1.9	-3.40928832796128\\
    1.95	-3.42515583498954\\
    2	-3.44170097958066\\
    2.05	-3.45892459367131\\
    2.1	-3.47682649869993\\
    2.15	-3.49540553609387\\
    2.2	-3.51465960159039\\
    2.25	-3.53458568299978\\
    2.3	-3.55517990099529\\
    2.35	-3.5764375524985\\
    2.4	-3.59835315622033\\
    2.45	-3.62092049991666\\
    2.5	-3.64413268892273\\
    2.55	-3.66798219554225\\
    2.6	-3.69246090888418\\
    2.65	-3.71756018476147\\
    2.7	-3.74327089529209\\
    2.75	-3.76958347787087\\
    2.8	-3.79648798321173\\
    2.85	-3.82397412219231\\
    2.9	-3.85203131126585\\
    2.95	-3.88064871623874\\
    3	-3.90981529424461\\
    3.05	-3.93951983377774\\
    3.1	-3.96975099267893\\
    3.15	-4.00049733399549\\
    3.2	-4.03174735966359\\
    3.25	-4.0634895419857\\
    3.3	-4.09571235289781\\
    3.35	-4.12840429104089\\
    3.4	-4.16155390666829\\
    3.45	-4.19514982443573\\
    3.5	-4.22918076413343\\
    3.55	-4.26363555943029\\
    3.6	-4.29850317470858\\
    3.65	-4.33377272007456\\
    3.7	-4.36943346463502\\
    3.75	-4.40547484813357\\
    3.8	-4.44188649104214\\
    3.85	-4.47865820320454\\
    3.9	-4.51577999112813\\
    3.95	-4.55324206401897\\
    4	-4.59103483865373\\
    4.05	-4.6291489431793\\
    4.1	-4.6675752199277\\
    4.15	-4.70630472733087\\
    4.2	-4.74532874101596\\
    4.25	-4.78463875415766\\
    4.3	-4.82422647716039\\
    4.35	-4.86408383673865\\
    4.4	-4.90420297445978\\
    4.45	-4.94457624480922\\
    4.5	-4.98519621283419\\
    4.55	-5.0260556514178\\
    4.6	-5.06714753823154\\
    4.65	-5.10846505241062\\
    4.7	-5.1500015709927\\
    4.75	-5.1917506651576\\
    4.8	-5.23370609630174\\
    4.85	-5.27586181197868\\
    4.9	-5.31821194173362\\
    4.95	-5.36075079285757\\
    5	-5.40347284608396\\
    };
    
    \addplot[area legend, draw=black, fill=black, forget plot]
    table[row sep=crcr] {%
    x	y\\
    -2.41892334053047	-2.66076002576777\\
    -2.41892334053047	-2.66076002576777\\
    -2.41892334053047	-2.66076002576777\\
    -2.41892334053047	-2.66076002576777\\
    -2.41892334053047	-2.66076002576777\\
    -2.41892334053047	-2.66076002576777\\
    -2.41892334053047	-2.66076002576777\\
    -2.48227065590585	-2.72410734114315\\
    -2.63984168314912	-2.43984168314912\\
    -2.35557602515509	-2.59741271039238\\
    -2.41892334053047	-2.66076002576777\\
    }--cycle;
    \addplot [color=black, forget plot]
      table[row sep=crcr]{%
    -5	4.68827619738926\\
    -4.95	4.63128291461227\\
    -4.9	4.5740345539613\\
    -4.85	4.51651748681601\\
    -4.8	4.45871707957417\\
    -4.75	4.40061759581542\\
    -4.7	4.34220208634965\\
    -4.65	4.28345226529961\\
    -4.6	4.22434837002808\\
    -4.55	4.164869002306\\
    -4.5	4.1049909476128\\
    -4.45	4.04468896883863\\
    -4.4	3.98393556988968\\
    -4.35	3.92270072374216\\
    -4.3	3.86095155829406\\
    -4.25	3.79865199185627\\
    -4.2	3.73576230821072\\
    -4.15	3.67223865871609\\
    -4.1	3.60803247578532\\
    -4.05	3.54308977795475\\
    -4	3.47735034137574\\
    -3.95	3.41074670541341\\
    -3.9	3.34320297045916\\
    -3.85	3.27463333307323\\
    -3.8	3.20494028573448\\
    -3.75	3.13401238363513\\
    -3.7	3.06172144585091\\
    -3.65	2.98791900777284\\
    -3.6	2.91243176788213\\
    -3.55	2.83505566177586\\
    -3.5	2.75554802817152\\
    -3.45	2.6736170683825\\
    -3.4	2.58890737691659\\
    -3.35	2.50097961609079\\
    -3.3	2.40928119148166\\
    -3.25	2.31310259436156\\
    -3.2	2.21150992547155\\
    -3.15	2.10323575242108\\
    -3.1	1.98649230986882\\
    -3.05	1.85862782751761\\
    -3	1.71543045234635\\
    -2.95	1.54951424251114\\
    -2.9	1.34571112600068\\
    -2.85	1.06180923959634\\
    -2.8	0\\
    };
    \addplot [color=black, forget plot]
      table[row sep=crcr]{%
    -2.8	-0\\
    -2.75	-1.04924359733213\\
    -2.7	-1.31404881408406\\
    -2.65	-1.49515100841119\\
    -2.6	-1.63565775765116\\
    -2.55	-1.75122363400783\\
    -2.5	-1.84954943982631\\
    -2.45	-1.93507118478761\\
    -2.4	-2.01061027812539\\
    -2.35	-2.07808849597626\\
    -2.3	-2.13888257964319\\
    -2.25	-2.19401783917641\\
    -2.2	-2.2442817761255\\
    -2.15	-2.29029464245921\\
    -2.1	-2.33255529615217\\
    -2.05	-2.37147213991386\\
    -2	-2.40738466192258\\
    -1.95	-2.44057884009659\\
    -1.9	-2.47129841528428\\
    -1.85	-2.49975330899155\\
    -1.8	-2.52612602132067\\
    -1.75	-2.55057657089264\\
    -1.7	-2.57324636310784\\
    -1.65	-2.59426125790634\\
    -1.6	-2.61373403083503\\
    -1.55	-2.63176636823053\\
    -1.5	-2.64845050035241\\
    -1.45	-2.66387055007701\\
    -1.4	-2.67810365588134\\
    -1.35	-2.69122091406451\\
    -1.3	-2.70328817496487\\
    -1.25	-2.71436672031028\\
    -1.2	-2.72451384307563\\
    -1.15	-2.7337833468196\\
    -1.1	-2.74222597807767\\
    -1.05	-2.7498898027468\\
    -1	-2.75682053532422\\
    -0.95	-2.76306182822313\\
    -0.9	-2.7686555270807\\
    -0.85	-2.77364189692434\\
    -0.8	-2.77805982321302\\
    -0.75	-2.78194699107888\\
    -0.7	-2.78534004552747\\
    -0.65	-2.78827473488684\\
    -0.6	-2.79078603940697\\
    -0.55	-2.79290828658616\\
    -0.5	-2.7946752545277\\
    -0.45	-2.79612026439937\\
    -0.399999999999999	-2.79727626287235\\
    -0.35	-2.79817589524919\\
    -0.3	-2.79885156984793\\
    -0.25	-2.79933551408778\\
    -0.2	-2.79965982261845\\
    -0.149999999999999	-2.79985649774756\\
    -0.0999999999999996	-2.79995748234757\\
    -0.0499999999999998	-2.79999468536406\\
    4.44089209850063e-16	-2.8\\
    0.0500000000000003	-2.80000531461576\\
    0.100000000000001	-2.80004251636121\\
    0.15	-2.80014348754473\\
    0.2	-2.80034009474402\\
    0.25	-2.80066417067502\\
    0.3	-2.80114748885848\\
    0.350000000000001	-2.80182173115804\\
    0.4	-2.80271844830938\\
    0.45	-2.80386901361591\\
    0.5	-2.80530457005292\\
    0.55	-2.80705597109652\\
    0.600000000000001	-2.80915371567669\\
    0.65	-2.81162787774153\\
    0.7	-2.81450803101234\\
    0.75	-2.81782316960242\\
    0.8	-2.82160162526491\\
    0.850000000000001	-2.82587098212319\\
    0.9	-2.83065798981818\\
    0.95	-2.83598847607753\\
    1	-2.84188725976877\\
    1.05	-2.8483780655392\\
    1.1	-2.85548344116701\\
    1.15	-2.86322467874888\\
    1.2	-2.87162174082733\\
    1.25	-2.88069319251583\\
    1.3	-2.89045614061086\\
    1.35	-2.90092618058876\\
    1.4	-2.91211735227267\\
    1.45	-2.9240421048243\\
    1.5	-2.93671127156856\\
    1.55	-2.95013405500169\\
    1.6	-2.96431802216825\\
    1.65	-2.97926911042469\\
    1.7	-2.99499164344107\\
    1.75	-3.01148835713308\\
    1.8	-3.02876043506725\\
    1.85	-3.04680755274759\\
    1.9	-3.06562793007403\\
    1.95	-3.08521839116532\\
    2	-3.10557443066252\\
    2.05	-3.12669028557426\\
    2.1	-3.14855901169279\\
    2.15	-3.17117256359859\\
    2.2	-3.19452187728006\\
    2.25	-3.21859695442227\\
    2.3	-3.24338694746137\\
    2.35	-3.26888024455833\\
    2.4	-3.29506455371237\\
    2.45	-3.32192698531026\\
    2.5	-3.34945413248832\\
    2.55	-3.37763214876757\\
    2.6	-3.40644682250726\\
    2.65	-3.43588364780516\\
    2.7	-3.46592789155308\\
    2.75	-3.49656465643253\\
    2.8	-3.52777893970564\\
    2.85	-3.55955568772126\\
    2.9	-3.59187984611387\\
    2.95	-3.62473640572403\\
    3	-3.65811044431324\\
    3.05	-3.69198716418307\\
    3.1	-3.7263519258398\\
    3.15	-3.76119027787001\\
    3.2	-3.79648798321173\\
    3.25	-3.83223104202003\\
    3.3	-3.86840571133459\\
    3.35	-3.90499852176285\\
    3.4	-3.94199629139342\\
    3.45	-3.97938613715373\\
    3.5	-4.01715548382202\\
    3.55	-4.05529207089804\\
    3.6	-4.09378395752964\\
    3.65	-4.13261952568365\\
    3.7	-4.17178748174028\\
    3.75	-4.21127685667977\\
    3.8	-4.25107700501994\\
    3.85	-4.29117760265231\\
    3.9	-4.33156864371396\\
    3.95	-4.37224043662185\\
    4	-4.41318359938584\\
    4.05	-4.45438905430713\\
    4.1	-4.49584802215894\\
    4.15	-4.53755201593769\\
    4.2	-4.57949283426403\\
    4.25	-4.62166255450535\\
    4.3	-4.66405352568401\\
    4.35	-4.7066583612282\\
    4.4	-4.74946993161649\\
    4.45	-4.79248135696076\\
    4.5	-4.83568599956711\\
    4.55	-4.87907745650921\\
    4.6	-4.92264955224409\\
    4.65	-4.96639633129637\\
    4.7	-5.01031205103314\\
    4.75	-5.05439117454833\\
    4.8	-5.09862836367258\\
    4.85	-5.14301847212166\\
    4.9	-5.18755653879443\\
    4.95	-5.23223778122896\\
    5	-5.27705758922345\\
    };
    
    \addplot[area legend, draw=black, fill=black, forget plot]
    table[row sep=crcr] {%
    x	y\\
    -2.09792872329418	-2.34679422221677\\
    -2.09792872329418	-2.34679422221677\\
    -2.09792872329418	-2.34679422221677\\
    -2.09792872329418	-2.34679422221677\\
    -2.09792872329418	-2.34679422221677\\
    -2.09792872329418	-2.34679422221677\\
    -2.09792872329418	-2.34679422221677\\
    -2.16228377861537	-2.41114927753796\\
    -2.32236147275548	-2.12236147275548\\
    -2.03357366797299	-2.28243916689559\\
    -2.09792872329418	-2.34679422221677\\
    }--cycle;
    \addplot [color=black, forget plot]
      table[row sep=crcr]{%
    -5	4.80843424529093\\
    -4.95	4.75430265724247\\
    -4.9	4.70003017937465\\
    -4.85	4.64560999326295\\
    -4.8	4.59103483865373\\
    -4.75	4.53629697659133\\
    -4.7	4.48138814870033\\
    -4.65	4.42629953213446\\
    -4.6	4.37102168962962\\
    -4.55	4.3155445140118\\
    -4.5	4.25985716640718\\
    -4.45	4.20394800728021\\
    -4.4	4.1478045192799\\
    -4.35	4.09141322070095\\
    -4.3	4.0347595681581\\
    -4.25	3.97782784682101\\
    -4.2	3.92060104625254\\
    -4.15	3.86306071952359\\
    -4.1	3.80518682282467\\
    -4.05	3.74695753223821\\
    -4	3.68834903364676\\
    -3.95	3.62933528089639\\
    -3.9	3.56988771626229\\
    -3.85	3.50997494591197\\
    -3.8	3.44956236134631\\
    -3.75	3.38861169560332\\
    -3.7	3.32708050017967\\
    -3.65	3.26492152494216\\
    -3.6	3.20208197846633\\
    -3.55	3.13850263982626\\
    -3.5	3.07411678426132\\
    -3.45	3.00884887348345\\
    -3.4	2.94261294536828\\
    -3.35	2.87531061546316\\
    -3.3	2.80682857120563\\
    -3.25	2.73703539441857\\
    -3.2	2.66577748131676\\
    -3.15	2.59287373022066\\
    -3.1	2.51810851596043\\
    -3.05	2.44122223322373\\
    -3	2.3618983098613\\
    -2.95	2.27974495710834\\
    -2.9	2.19426882937724\\
    -2.85	2.10483579455697\\
    -2.8	2.01061027812539\\
    -2.75	1.9104571148998\\
    -2.7	1.80277350207521\\
    -2.65	1.68517971004287\\
    -2.6	1.5538924023345\\
    -2.55	1.40227097406978\\
    -2.5	1.21666562415206\\
    -2.45	0.959054697368768\\
    -2.4	0\\
    };
    \addplot [color=black, forget plot]
      table[row sep=crcr]{%
    -2.4	-0\\
    -2.35	-0.94582656794676\\
    -2.3	-1.18333443540641\\
    -2.25	-1.34504355335903\\
    -2.2	-1.4699193193105\\
    -2.15	-1.57212459328393\\
    -2.1	-1.65863244410366\\
    -2.05	-1.73346327510399\\
    -2	-1.79917657811017\\
    -1.95	-1.85751749979135\\
    -1.9	-1.90973763562705\\
    -1.85	-1.95677024308203\\
    -1.8	-1.99933311098757\\
    -1.75	-2.03799246630451\\
    -1.7	-2.07320452416294\\
    -1.65	-2.1053435344861\\
    -1.6	-2.13472131897756\\
    -1.55	-2.16160125036841\\
    -1.5	-2.18620848933655\\
    -1.45	-2.20873763413653\\
    -1.4	-2.22935853978708\\
    -1.35	-2.24822081570561\\
    -1.3	-2.26545735184385\\
    -1.25	-2.28118711905239\\
    -1.2	-2.29551741932687\\
    -1.15	-2.30854571356602\\
    -1.1	-2.32036112095662\\
    -1.05	-2.33104566032113\\
    -0.999999999999999	-2.3406752866345\\
    -0.949999999999999	-2.34932076340479\\
    -0.899999999999999	-2.3570484023544\\
    -0.849999999999999	-2.36392069490592\\
    -0.799999999999999	-2.3699968547259\\
    -0.749999999999999	-2.37533328656065\\
    -0.699999999999999	-2.37998399348979\\
    -0.649999999999999	-2.38400093229623\\
    -0.599999999999999	-2.38743432473783\\
    -0.549999999999999	-2.39033293098385\\
    -0.499999999999999	-2.39274429025737\\
    -0.449999999999999	-2.39471493273629\\
    -0.399999999999999	-2.39629056595804\\
    -0.349999999999999	-2.39751623831037\\
    -0.299999999999999	-2.39843648164217\\
    -0.25	-2.39909543457304\\
    -0.199999999999999	-2.39953694770219\\
    -0.149999999999999	-2.39980467160327\\
    -0.0999999999999992	-2.39994212823416\\
    -0.0499999999999994	-2.3999927661819\\
    8.88178419700125e-16	-2.4\\
    0.0500000000000007	-2.40000723377449\\
    0.100000000000001	-2.40005786897502\\
    0.150000000000001	-2.40019529660758\\
    0.200000000000001	-2.40046287368554\\
    0.250000000000001	-2.40090388407496\\
    0.300000000000001	-2.40156148384975\\
    0.350000000000001	-2.40247863140937\\
    0.400000000000001	-2.40369800276695\\
    0.450000000000001	-2.40526189260486\\
    0.500000000000001	-2.40721210191468\\
    0.550000000000001	-2.40958981328368\\
    0.600000000000001	-2.41243545515343\\
    0.650000000000001	-2.41578855664932\\
    0.700000000000001	-2.41968759485215\\
    0.750000000000001	-2.42416983664444\\
    0.800000000000001	-2.42927117750053\\
    0.850000000000001	-2.43502597979047\\
    0.900000000000001	-2.44146691331931\\
    0.950000000000001	-2.44862480091532\\
    1	-2.45652847190347\\
    1.05	-2.46520462624676\\
    1.1	-2.47467771200501\\
    1.15	-2.48496981854691\\
    1.2	-2.49610058766228\\
    1.25	-2.50808714436297\\
    1.3	-2.52094404874626\\
    1.35	-2.53468326983641\\
    1.4	-2.54931418183629\\
    1.45	-2.56484358272872\\
    1.5	-2.58127573468549\\
    1.55	-2.59861242528775\\
    1.6	-2.61685304815087\\
    1.65	-2.63599470119279\\
    1.7	-2.65603230049749\\
    1.75	-2.67695870751057\\
    1.8	-2.69876486716525\\
    1.85	-2.72143995447251\\
    1.9	-2.74497152711609\\
    1.95	-2.76934568166313\\
    2	-2.79454721112724\\
    2.05	-2.82055976179205\\
    2.1	-2.8473659874088\\
    2.15	-2.87494769911158\\
    2.2	-2.90328600963702\\
    2.25	-2.93236147068346\\
    2.3	-2.96215420248884\\
    2.35	-2.99264401494038\\
    2.4	-3.02381051974769\\
    2.45	-3.05563323340941\\
    2.5	-3.08809167088059\\
    2.55	-3.12116543000122\\
    2.6	-3.15483426687613\\
    2.65	-3.18907816250322\\
    2.7	-3.22387738103143\\
    2.75	-3.25921252009453\\
    2.8	-3.29506455371237\\
    2.85	-3.3314148682816\\
    2.9	-3.36824529219294\\
    2.95	-3.40553811961646\\
    3	-3.4432761289903\\
    3.05	-3.48144259673413\\
    3.1	-3.52002130668921\\
    3.15	-3.55899655576197\\
    3.2	-3.59835315622033\\
    3.25	-3.63807643506226\\
    3.3	-3.67815223084484\\
    3.35	-3.71856688833088\\
    3.4	-3.75930725127913\\
    3.45	-3.80036065367366\\
    3.5	-3.84171490965925\\
    3.55	-3.88335830242172\\
    3.6	-3.92527957222631\\
    3.65	-3.9674679038028\\
    3.7	-4.00991291324399\\
    3.75	-4.05260463456297\\
    3.8	-4.09553350603637\\
    3.85	-4.1386903564432\\
    3.9	-4.18206639129361\\
    3.95	-4.22565317912812\\
    4	-4.26944263795511\\
    4.05	-4.31342702188364\\
    4.1	-4.35759890799854\\
    4.15	-4.40195118351628\\
    4.2	-4.44647703325237\\
    4.25	-4.49116992742435\\
    4.3	-4.53602360980881\\
    4.35	-4.58103208626559\\
    4.4	-4.62618961363815\\
    4.45	-4.6714906890354\\
    4.5	-4.7169300394969\\
    4.55	-4.76250261204083\\
    4.6	-4.80820356409192\\
    4.65	-4.85402825428448\\
    4.7	-4.89997223363413\\
    4.75	-4.94603123707086\\
    4.8	-4.99220117532457\\
    4.85	-5.03847812715392\\
    4.9	-5.08485833190836\\
    4.95	-5.13133818241302\\
    5	-5.17791421816565\\
    };
    
    \addplot[area legend, draw=black, fill=black, forget plot]
    table[row sep=crcr] {%
    x	y\\
    -1.77796634607363	-2.03179617865005\\
    -1.77796634607363	-2.03179617865005\\
    -1.77796634607363	-2.03179617865005\\
    -1.77796634607363	-2.03179617865005\\
    -1.77796634607363	-2.03179617865005\\
    -1.77796634607363	-2.03179617865005\\
    -1.77796634607363	-2.03179617865005\\
    -1.84303315127912	-2.09686298385554\\
    -2.00488126236184	-1.80488126236184\\
    -1.71289954086813	-1.96672937344456\\
    -1.77796634607363	-2.03179617865005\\
    }--cycle;
    \addplot [color=black, forget plot]
      table[row sep=crcr]{%
    -5	4.89097324650875\\
    -4.95	4.83868300516958\\
    -4.9	4.78631809839936\\
    -4.85	4.73387514737148\\
    -4.8	4.68135057326899\\
    -4.75	4.62874058236986\\
    -4.7	4.5760411497677\\
    -4.65	4.52324800157807\\
    -4.6	4.47035659546091\\
    -4.55	4.4173620992673\\
    -4.5	4.36425936759302\\
    -4.45	4.31104291599141\\
    -4.4	4.2577068925634\\
    -4.35	4.20424504660246\\
    -4.3	4.15065069392485\\
    -4.25	4.09691667846105\\
    -4.2	4.04303532961917\\
    -4.15	3.98899841485557\\
    -4.1	3.93479708679752\\
    -4.05	3.88042182415671\\
    -4	3.82586236554478\\
    -3.95	3.77110763515079\\
    -3.9	3.71614565905798\\
    -3.85	3.66096347075747\\
    -3.8	3.60554700415042\\
    -3.75	3.54988097200598\\
    -3.7	3.49394872744598\\
    -3.65	3.43773210553963\\
    -3.6	3.38121124148806\\
    -3.55	3.32436436112808\\
    -3.5	3.26716753854379\\
    -3.45	3.20959441438957\\
    -3.4	3.15161586702219\\
    -3.35	3.09319962661313\\
    -3.3	3.03430981992715\\
    -3.25	2.97490643021857\\
    -3.2	2.91494465244874\\
    -3.15	2.85437411839193\\
    -3.1	2.79313795863858\\
    -3.05	2.73117165824929\\
    -3	2.66840164872194\\
    -2.95	2.60474355930451\\
    -2.9	2.54010002292785\\
    -2.85	2.47435789212902\\
    -2.8	2.40738466192258\\
    -2.75	2.33902380933802\\
    -2.7	2.26908862610406\\
    -2.65	2.1973539123126\\
    -2.6	2.12354456278193\\
    -2.55	2.04731951866994\\
    -2.5	1.96824859155109\\
    -2.45	1.88577792730902\\
    -2.4	1.79917657811017\\
    -2.35	1.70745000271621\\
    -2.3	1.6091918838378\\
    -2.25	1.50231125172934\\
    -2.2	1.38347928332185\\
    -2.15	1.24684537872232\\
    -2.1	1.08036795875479\\
    -2.05	0.850461110824827\\
    -2	0\\
    };
    \addplot [color=black, forget plot]
      table[row sep=crcr]{%
    -2	-0\\
    -1.95	-0.836404225202668\\
    -1.9	-1.04494928899312\\
    -1.85	-1.18603605379583\\
    -1.8	-1.29425472539207\\
    -1.75	-1.38219370341972\\
    -1.7	-1.45605867613633\\
    -1.65	-1.51943235382759\\
    -1.6	-1.57459887324087\\
    -1.55	-1.62311813709513\\
    -1.5	-1.66611092582298\\
    -1.45	-1.70441470781648\\
    -1.4	-1.73867517068787\\
    -1.35	-1.76940312047886\\
    -1.3	-1.79701150191753\\
    -1.25	-1.82184040778046\\
    -1.2	-1.84417451682338\\
    -1.15	-1.86425558533691\\
    -1.1	-1.88229160722335\\
    -1.05	-1.89846367034774\\
    -1	-1.91293118277239\\
    -0.95	-1.92583592187964\\
    -0.899999999999999	-1.93730521801843\\
    -0.849999999999999	-1.94745449140934\\
    -0.799999999999999	-1.9563892986035\\
    -0.75	-1.964207001962\\
    -0.7	-1.97099814569671\\
    -0.649999999999999	-1.97684760074296\\
    -0.599999999999999	-1.98183552537535\\
    -0.549999999999999	-1.98603817722008\\
    -0.5	-1.9895286039482\\
    -0.45	-1.99237723362748\\
    -0.399999999999999	-1.99465238089546\\
    -0.349999999999999	-1.99642068139387\\
    -0.299999999999999	-1.9977474639932\\
    -0.25	-1.99869706803514\\
    -0.2	-1.99933311098757\\
    -0.149999999999999	-1.99971871043995\\
    -0.0999999999999994	-1.9999166631942\\
    -0.0499999999999994	-1.99998958327908\\
    4.44089209850063e-16	-2\\
    0.0500000000000007	-2.00001041661241\\
    0.100000000000001	-2.00008332986135\\
    0.15	-2.00028121045849\\
    0.200000000000001	-2.00066644456782\\
    0.25	-2.00130123654146\\
    0.300000000000001	-2.00224747348544\\
    0.350000000000001	-2.00356655273682\\
    0.400000000000001	-2.00531917398583\\
    0.450000000000001	-2.0075650985622\\
    0.5	-2.01036287929453\\
    0.550000000000001	-2.01376956530874\\
    0.600000000000001	-2.01784038710825\\
    0.650000000000001	-2.02262842821894\\
    0.700000000000001	-2.02818429052461\\
    0.75	-2.03455576109942\\
    0.800000000000001	-2.04178748880059\\
    0.850000000000001	-2.04992067906451\\
    0.900000000000001	-2.05899281521183\\
    0.950000000000001	-2.06903741408839\\
    1	-2.0800838230519\\
    1.05	-2.09215706417879\\
    1.1	-2.10527773016229\\
    1.15	-2.11946193476664\\
    1.2	-2.13472131897756\\
    1.25	-2.15106311223791\\
    1.3	-2.16849024647197\\
    1.35	-2.18700151906935\\
    1.4	-2.20659179969292\\
    1.45	-2.22725227474814\\
    1.5	-2.24897072263771\\
    1.55	-2.27173181253447\\
    1.6	-2.29551741932687\\
    1.65	-2.32030694759599\\
    1.7	-2.34607765792876\\
    1.75	-2.37280498950733\\
    1.8	-2.40046287368554\\
    1.85	-2.42902403411495\\
    1.9	-2.4584602698667\\
    1.95	-2.48874271886697\\
    2	-2.51984209978975\\
    2.05	-2.55172893130465\\
    2.1	-2.58437372824212\\
    2.15	-2.61774717480497\\
    2.2	-2.65182027542034\\
    2.25	-2.68656448419286\\
    2.3	-2.72195181419418\\
    2.35	-2.75795492801531\\
    2.4	-2.79454721112724\\
    2.45	-2.83170282965258\\
    2.5	-2.86939677415858\\
    2.55	-2.90760489104936\\
    2.6	-2.94630390307269\\
    2.65	-2.98547142037234\\
    2.7	-3.02508594341803\\
    2.75	-3.06512685903736\\
    2.8	-3.10557443066252\\
    2.85	-3.14640978379311\\
    2.9	-3.18761488756736\\
    2.95	-3.22917253323018\\
    3	-3.27106631018859\\
    3.05	-3.31328058025453\\
    3.1	-3.35580045059213\\
    3.15	-3.39861174581109\\
    3.2	-3.44170097958066\\
    3.25	-3.48505532607801\\
    3.3	-3.52866259153212\\
    3.35	-3.5725111860773\\
    3.4	-3.61659009608984\\
    3.45	-3.66088885714617\\
    3.5	-3.70539752771031\\
    3.55	-3.75010666363274\\
    3.6	-3.79500729352051\\
    3.65	-3.84009089502003\\
    3.7	-3.88534937203808\\
    3.75	-3.93077503291408\\
    3.8	-3.97636056954548\\
    3.85	-4.02209903745978\\
    3.9	-4.06798383681954\\
    3.95	-4.11400869434114\\
    4	-4.16016764610381\\
    4.05	-4.20645502122203\\
    4.1	-4.25286542635209\\
    4.15	-4.29939373100183\\
    4.2	-4.34603505361155\\
    4.25	-4.39278474837337\\
    4.3	-4.43963839275638\\
    4.35	-4.48659177570492\\
    4.4	-4.53364088647765\\
    4.45	-4.58078190409608\\
    4.5	-4.62801118737159\\
    4.55	-4.67532526548129\\
    4.6	-4.72272082906374\\
    4.65	-4.77019472180698\\
    4.7	-4.81774393250215\\
    4.75	-4.86536558753726\\
    4.8	-4.91305694380694\\
    4.85	-4.96081538201475\\
    4.9	-5.00863840034637\\
    4.95	-5.05652360849244\\
    5	-5.10446872200146\\
    };
    
    \addplot[area legend, draw=black, fill=black, forget plot]
    table[row sep=crcr] {%
    x	y\\
    -1.45878108492119	-1.7160210190152\\
    -1.45878108492119	-1.7160210190152\\
    -1.45878108492119	-1.7160210190152\\
    -1.45878108492119	-1.7160210190152\\
    -1.45878108492119	-1.7160210190152\\
    -1.45878108492119	-1.7160210190152\\
    -1.45878108492119	-1.7160210190152\\
    -1.52433680556425	-1.78157673965827\\
    -1.6874010519682	-1.4874010519682\\
    -1.39322536427813	-1.65046529837215\\
    -1.45878108492119	-1.7160210190152\\
    }--cycle;
    \addplot [color=black, forget plot]
      table[row sep=crcr]{%
    -5	4.94477904098059\\
    -4.95	4.89363848904296\\
    -4.9	4.84246180797421\\
    -4.85	4.79124743234037\\
    -4.8	4.73999370945179\\
    -4.75	4.68869889334021\\
    -4.7	4.63736113823424\\
    -4.65	4.5859784914838\\
    -4.6	4.53454888587835\\
    -4.55	4.48307013129735\\
    -4.5	4.4315399056241\\
    -4.45	4.3799557448459\\
    -4.4	4.32831503225368\\
    -4.35	4.27661498664397\\
    -4.3	4.22485264941332\\
    -4.25	4.17302487042138\\
    -4.2	4.12112829248242\\
    -4.15	4.06915933432656\\
    -4.1	4.01711417185019\\
    -4.05	3.96498871745015\\
    -4	3.912778597207\\
    -3.95	3.86047912564922\\
    -3.9	3.80808527779036\\
    -3.85	3.75559165808512\\
    -3.8	3.70299246589589\\
    -3.75	3.65028145699731\\
    -3.7	3.5974519005707\\
    -3.65	3.54449653105004\\
    -3.6	3.49140749407442\\
    -3.55	3.4381762856732\\
    -3.5	3.38479368365655\\
    -3.45	3.33124966999775\\
    -3.4	3.27753334276884\\
    -3.35	3.22363281591635\\
    -3.3	3.16953510482831\\
    -3.25	3.11522599523022\\
    -3.2	3.06068989243582\\
    -3.15	3.00590964734172\\
    -3.1	2.95086635475631\\
    -3.05	2.89553911864629\\
    -3	2.83990477760478\\
    -2.95	2.78393758220642\\
    -2.9	2.72760881380063\\
    -2.85	2.67088633154152\\
    -2.8	2.61373403083503\\
    -2.75	2.55611119158266\\
    -2.7	2.49797168815881\\
    -2.65	2.43926302431389\\
    -2.6	2.37992514417486\\
    -2.55	2.31988895376138\\
    -2.5	2.25907446373555\\
    -2.45	2.19738843001697\\
    -2.4	2.13472131897756\\
    -2.35	2.07094334934552\\
    -2.3	2.00589924898897\\
    -2.25	1.93940118621276\\
    -2.2	1.87121904747042\\
    -2.15	1.80106675450351\\
    -2.1	1.72858248681377\\
    -2.05	1.65329918399086\\
    -2	1.57459887324087\\
    -1.95	1.4916386638245\\
    -1.9	1.40322386303798\\
    -1.85	1.30757412870449\\
    -1.8	1.20184900138347\\
    -1.75	1.08104579769418\\
    -1.7	0.93484731604652\\
    -1.65	0.734419304705956\\
    -1.6	0\\
    };
    \addplot [color=black, forget plot]
      table[row sep=crcr]{%
    -1.6	-0\\
    -1.55	-0.719277180972339\\
    -1.5	-0.896695702239352\\
    -1.45	-1.01554865303808\\
    -1.4	-1.10575496273577\\
    -1.35	-1.17822413134853\\
    -1.3	-1.2383449998609\\
    -1.25	-1.28923557057932\\
    -1.2	-1.33288874065838\\
    -1.15	-1.37066957164814\\
    -1.1	-1.40356235632406\\
    -1.05	-1.43230576551778\\
    -1	-1.45747232622437\\
    -0.95	-1.47951789142451\\
    -0.9	-1.49881387713707\\
    -0.85	-1.51566908328221\\
    -0.8	-1.53034494621791\\
    -0.75	-1.54306649904155\\
    -0.7	-1.55403044021409\\
    -0.65	-1.5634112018452\\
    -0.6	-1.5713656015696\\
    -0.55	-1.57803647021591\\
    -0.5	-1.58355552437377\\
    -0.45	-1.58804567183962\\
    -0.399999999999999	-1.59162288315856\\
    -0.35	-1.59439772467235\\
    -0.3	-1.59647662182419\\
    -0.25	-1.59796290228795\\
    -0.2	-1.59895765442811\\
    -0.149999999999999	-1.59956042612031\\
    -0.0999999999999996	-1.59986978106885\\
    -0.0499999999999996	-1.59998372379276\\
    4.44089209850063e-16	-1.6\\
    0.0500000000000005	-1.6000162758761\\
    0.100000000000001	-1.60013019773839\\
    0.15	-1.60043933248082\\
    0.2	-1.60104098923317\\
    0.25	-1.60203192366986\\
    0.3	-1.60350792840324\\
    0.350000000000001	-1.60556331581963\\
    0.4	-1.60829030343562\\
    0.450000000000001	-1.61177831615857\\
    0.5	-1.61611322442963\\
    0.55	-1.62137654172857\\
    0.600000000000001	-1.62764460887954\\
    0.65	-1.634987795549\\
    0.700000000000001	-1.64346975083118\\
    0.75	-1.65314673452675\\
    0.800000000000001	-1.66406705844152\\
    0.850000000000001	-1.67627066276551\\
    0.9	-1.6897888465576\\
    0.950000000000001	-1.70464416398064\\
    1	-1.72085048979033\\
    1.05	-1.73841324934997\\
    1.1	-1.7573298007952\\
    1.15	-1.77758995049765\\
    1.2	-1.79917657811017\\
    1.25	-1.82206634446136\\
    1.3	-1.84623045444209\\
    1.35	-1.87163544764605\\
    1.4	-1.89824399160587\\
    1.45	-1.92601565563293\\
    1.5	-1.95490764712231\\
    1.55	-1.98487549633947\\
    1.6	-2.0158736798318\\
    1.65	-2.0478561764489\\
    1.7	-2.08077695333414\\
    1.75	-2.11459038206672\\
    1.8	-2.14925158735429\\
    1.85	-2.18471673231751\\
    1.9	-2.22094324552107\\
    1.95	-2.25788999556407\\
    2	-2.29551741932687\\
    2.05	-2.33378760996385\\
    2.1	-2.37266437050798\\
    2.15	-2.41211323858019\\
    2.2	-2.45210148722989\\
    2.25	-2.4925981064171\\
    2.3	-2.53357376911577\\
    2.35	-2.57500078549635\\
    2.4	-2.61685304815087\\
    2.45	-2.65910597086681\\
    2.5	-2.70173642304198\\
    2.55	-2.74472266146485\\
    2.6	-2.78804426086241\\
    2.65	-2.83168204433867\\
    2.7	-2.87561801458909\\
    2.75	-2.91983528657429\\
    2.8	-2.96431802216825\\
    2.85	-3.00905136715629\\
    2.9	-3.05402139084372\\
    2.95	-3.09921502844282\\
    3	-3.14462002633127\\
    3.05	-3.19022489021596\\
    3.1	-3.23601883618965\\
    3.15	-3.28199174463239\\
    3.2	-3.32813411688305\\
    3.25	-3.37443703458653\\
    3.3	-3.42089212160868\\
    3.35	-3.46749150840182\\
    3.4	-3.5142277986987\\
    3.45	-3.56109403841015\\
    3.5	-3.60808368660189\\
    3.55	-3.65519058842745\\
    3.6	-3.70240894989728\\
    3.65	-3.74973331436812\\
    3.7	-3.7971585406413\\
    3.75	-3.84467978256357\\
    3.8	-3.89229247002981\\
    3.85	-3.93999229129192\\
    3.9	-3.98777517648406\\
    3.95	-4.03563728227971\\
    4	-4.08357497760117\\
    4.05	-4.1315848303075\\
    4.1	-4.17966359479151\\
    4.15	-4.2278082004213\\
    4.2	-4.27601574076625\\
    4.25	-4.32428346355148\\
    4.3	-4.37260876128874\\
    4.35	-4.42098916253565\\
    4.4	-4.46942232373823\\
    4.45	-4.51790602161539\\
    4.5	-4.56643814604669\\
    4.55	-4.61501669342777\\
    4.6	-4.66363976046024\\
    4.65	-4.71230553834548\\
    4.7	-4.76101230735385\\
    4.75	-4.80975843174301\\
    4.8	-4.85854235500106\\
    4.85	-4.90736259539177\\
    4.9	-4.9562177417811\\
    4.95	-5.00510644972551\\
    5	-5.05402743780422\\
    };
    
    \addplot[area legend, draw=black, fill=black, forget plot]
    table[row sep=crcr] {%
    x	y\\
    -1.14018938045399	-1.39965230269513\\
    -1.14018938045399	-1.39965230269513\\
    -1.14018938045399	-1.39965230269513\\
    -1.14018938045399	-1.39965230269513\\
    -1.14018938045399	-1.39965230269513\\
    -1.14018938045399	-1.39965230269513\\
    -1.14018938045399	-1.39965230269513\\
    -1.20606381689394	-1.46552673913508\\
    -1.36992084157456	-1.16992084157456\\
    -1.07431494401403	-1.33377786625518\\
    -1.14018938045399	-1.39965230269513\\
    }--cycle;
    \addplot [color=black, forget plot]
      table[row sep=crcr]{%
    -5	4.97685300871582\\
    -4.95	4.92637966277395\\
    -4.9	4.87589157482422\\
    -4.85	4.82538812237561\\
    -4.8	4.77486864947567\\
    -4.75	4.72433246450907\\
    -4.7	4.67377883782354\\
    -4.65	4.62320699916736\\
    -4.6	4.572616134921\\
    -4.55	4.52200538510347\\
    -4.5	4.471373840132\\
    -4.45	4.42072053731118\\
    -4.4	4.37004445702512\\
    -4.35	4.31934451860321\\
    -4.3	4.26861957582661\\
    -4.25	4.21786841203885\\
    -4.2	4.16708973481953\\
    -4.15	4.11628217017533\\
    -4.1	4.06544425619686\\
    -4.05	4.0145744361234\\
    -4	3.96367105075071\\
    -3.95	3.91273233010832\\
    -3.9	3.86175638432359\\
    -3.85	3.81074119357857\\
    -3.8	3.75968459705344\\
    -3.75	3.70858428073544\\
    -3.7	3.65743776395566\\
    -3.65	3.60624238449633\\
    -3.6	3.55499528208884\\
    -3.55	3.50369338009591\\
    -3.5	3.45233336514064\\
    -3.45	3.40091166440876\\
    -3.4	3.3494244203076\\
    -3.35	3.29786746211483\\
    -3.3	3.24623627419026\\
    -3.25	3.19452596025271\\
    -3.2	3.1427312031392\\
    -3.15	3.09084621936181\\
    -3.1	3.03886470765518\\
    -3.05	2.9867797905596\\
    -3	2.93458394790525\\
    -2.95	2.88226894084361\\
    -2.9	2.829825724804\\
    -2.85	2.77724434942192\\
    -2.8	2.72451384307563\\
    -2.75	2.67162207915641\\
    -2.7	2.61855562055581\\
    -2.65	2.56529953804358\\
    -2.6	2.51183719717883\\
    -2.55	2.45815000707663\\
    -2.5	2.40421712264547\\
    -2.45	2.35001508968732\\
    -2.4	2.29551741932687\\
    -2.35	2.24069407435017\\
    -2.3	2.18551084481481\\
    -2.25	2.12992858320359\\
    -2.2	2.07390225963995\\
    -2.15	2.01737978407905\\
    -2.1	1.96030052312627\\
    -2.05	1.90259341141233\\
    -2	1.84417451682338\\
    -1.95	1.78494385813114\\
    -1.9	1.72478118067315\\
    -1.85	1.66354025008984\\
    -1.8	1.60104098923317\\
    -1.75	1.53705839213066\\
    -1.7	1.47130647268414\\
    -1.65	1.40341428560281\\
    -1.6	1.33288874065838\\
    -1.55	1.25905425798022\\
    -1.5	1.18094915493065\\
    -1.45	1.09713441468862\\
    -1.4	1.0053051390627\\
    -1.35	0.901386751037605\\
    -1.3	0.776946201167251\\
    -1.25	0.608332812076032\\
    -1.2	0\\
    };
    \addplot [color=black, forget plot]
      table[row sep=crcr]{%
    -1.2	-0\\
    -1.15	-0.591667217703205\\
    -1.1	-0.734959659655251\\
    -1.05	-0.829316222051831\\
    -1	-0.899588289055083\\
    -0.95	-0.954868817813526\\
    -0.9	-0.999666555493786\\
    -0.85	-1.03660226208147\\
    -0.8	-1.06736065948878\\
    -0.75	-1.09310424466828\\
    -0.7	-1.11467926989354\\
    -0.65	-1.13272867592193\\
    -0.6	-1.14775870966343\\
    -0.55	-1.16018056047831\\
    -0.5	-1.17033764331725\\
    -0.45	-1.1785242011772\\
    -0.4	-1.18499842736295\\
    -0.35	-1.18999199674489\\
    -0.3	-1.19371716236892\\
    -0.25	-1.19637214512868\\
    -0.2	-1.19814528297902\\
    -0.15	-1.19921824082108\\
    -0.0999999999999996	-1.19976847385109\\
    -0.0499999999999996	-1.19997106411708\\
    4.44089209850063e-16	-1.2\\
    0.0500000000000003	-1.20002893448751\\
    0.1	-1.20023143684277\\
    0.15	-1.20078074192487\\
    0.2	-1.20184900138347\\
    0.25	-1.20360605095734\\
    0.3	-1.20621772757672\\
    0.35	-1.20984379742608\\
    0.4	-1.21463558875026\\
    0.45	-1.22073345665966\\
    0.5	-1.22826423595173\\
    0.55	-1.2373388560025\\
    0.6	-1.24805029383114\\
    0.65	-1.26047202437313\\
    0.7	-1.27465709091814\\
    0.75	-1.29063786734275\\
    0.8	-1.30842652407544\\
    0.850000000000001	-1.32801615024874\\
    0.9	-1.34938243358262\\
    0.95	-1.37248576355805\\
    1	-1.39727360556362\\
    1.05	-1.4236829937044\\
    1.1	-1.45164300481851\\
    1.15	-1.48107710124442\\
    1.2	-1.51190525987385\\
    1.25	-1.5440458354403\\
    1.3	-1.57741713343806\\
    1.35	-1.61193869051572\\
    1.4	-1.64753227685619\\
    1.45	-1.68412264609647\\
    1.5	-1.72163806449515\\
    1.55	-1.76001065334461\\
    1.6	-1.79917657811017\\
    1.65	-1.83907611542242\\
    1.7	-1.87965362563957\\
    1.75	-1.92085745482962\\
    1.8	-1.96263978611315\\
    1.85	-2.00495645662199\\
    1.9	-2.04776675301819\\
    1.95	-2.0910331956468\\
    2	-2.13472131897756\\
    2.05	-2.17879945399927\\
    2.1	-2.22323851662619\\
    2.15	-2.26801180490441\\
    2.2	-2.31309480681907\\
    2.25	-2.35846501974845\\
    2.3	-2.40410178204596\\
    2.35	-2.44998611681707\\
    2.4	-2.49610058766228\\
    2.45	-2.54242916595418\\
    2.5	-2.58895710908282\\
    2.55	-2.63567084902402\\
    2.6	-2.68255789054354\\
    2.65	-2.72960671833826\\
    2.7	-2.77680671242296\\
    2.75	-2.82414807109337\\
    2.8	-2.87162174082733\\
    2.85	-2.91921935252236\\
    2.9	-2.96693316350804\\
    2.95	-3.01475600481223\\
    3	-3.06268123320088\\
    3.05	-3.11070268755082\\
    3.1	-3.1588146491524\\
    3.15	-3.20701180557469\\
    3.2	-3.25528921775908\\
    3.25	-3.30364229003798\\
    3.3	-3.35206674280367\\
    3.35	-3.40055858757846\\
    3.4	-3.44911410426097\\
    3.45	-3.49772982034518\\
    3.5	-3.54640249192816\\
    3.55	-3.59512908634073\\
    3.6	-3.64390676625079\\
    3.65	-3.69273287510416\\
    3.7	-3.74160492378036\\
    3.75	-3.79052057835317\\
    3.8	-3.83947764885575\\
    3.85	-3.88847407896054\\
    3.9	-3.93750793649194\\
    3.95	-3.98657740469835\\
    4	-4.03568077421651\\
    4.05	-4.08481643566787\\
    4.1	-4.13398287283209\\
    4.15	-4.18317865634805\\
    4.2	-4.23240243789736\\
    4.25	-4.2816529448295\\
    4.3	-4.33092897519133\\
    4.35	-4.38022939312749\\
    4.4	-4.42955312462073\\
    4.45	-4.47889915354446\\
    4.5	-4.52826651800205\\
    4.55	-4.57765430692961\\
    4.6	-4.6270616569413\\
    4.65	-4.67648774939779\\
    4.7	-4.72593180768029\\
    4.75	-4.77539309465408\\
    4.8	-4.82487091030687\\
    4.85	-4.87436458954849\\
    4.9	-4.92387350015961\\
    4.95	-4.97339704087823\\
    5	-5.02293463961359\\
    };
    
    \addplot[area legend, draw=black, fill=black, forget plot]
    table[row sep=crcr] {%
    x	y\\
    -0.822046601550703	-1.08283466081114\\
    -0.822046601550703	-1.08283466081114\\
    -0.822046601550703	-1.08283466081114\\
    -0.822046601550703	-1.08283466081114\\
    -0.822046601550703	-1.08283466081114\\
    -0.822046601550703	-1.08283466081114\\
    -0.822046601550703	-1.08283466081114\\
    -0.888111026453545	-1.14889908571398\\
    -1.05244063118092	-0.85244063118092\\
    -0.75598217664786	-1.01677023590829\\
    -0.822046601550703	-1.08283466081114\\
    }--cycle;
    \addplot [color=black, forget plot]
      table[row sep=crcr]{%
    -5	4.99316399138997\\
    -4.95	4.94302490027429\\
    -4.9	4.89288151422978\\
    -4.85	4.84273365425806\\
    -4.8	4.79258113191607\\
    -4.75	4.74242374870999\\
    -4.7	4.69226129544322\\
    -4.65	4.64209355151434\\
    -4.6	4.59192028416048\\
    -4.55	4.54174124764133\\
    -4.5	4.4915561823583\\
    -4.45	4.44136481390288\\
    -4.4	4.39116685202757\\
    -4.35	4.34096198953208\\
    -4.3	4.29074990105671\\
    -4.25	4.24053024177388\\
    -4.2	4.19030264596785\\
    -4.15	4.1400667254915\\
    -4.1	4.08982206808783\\
    -4.05	4.03956823556236\\
    -4	3.98930476179092\\
    -3.95	3.93903115054581\\
    -3.9	3.88874687312076\\
    -3.85	3.83845136573318\\
    -3.8	3.78814402667926\\
    -3.75	3.73782421321457\\
    -3.7	3.68749123812914\\
    -3.65	3.63714436598227\\
    -3.6	3.58678280895741\\
    -3.55	3.53640572229232\\
    -3.5	3.48601219923388\\
    -3.45	3.43560126545931\\
    -3.4	3.38517187289809\\
    -3.35	3.33472289287901\\
    -3.3	3.28425310851596\\
    -3.25	3.23376120623338\\
    -3.2	3.18324576631711\\
    -3.15	3.13270525235931\\
    -3.1	3.08213799944491\\
    -3.05	3.0315422009035\\
    -3	2.98091589342134\\
    -2.95	2.93025694027429\\
    -2.9	2.87956301240214\\
    -2.85	2.82883156699606\\
    -2.8	2.77805982321302\\
    -2.75	2.72724473456117\\
    -2.7	2.6763829574156\\
    -2.65	2.62547081502136\\
    -2.6	2.57450425621572\\
    -2.55	2.52347880794844\\
    -2.5	2.4723895204903\\
    -2.45	2.42123090398711\\
    -2.4	2.3699968547259\\
    -2.35	2.31868056911712\\
    -2.3	2.26727444293918\\
    -2.25	2.21576995281205\\
    -2.2	2.16415751612684\\
    -2.15	2.11242632470666\\
    -2.1	2.06056414624121\\
    -2.05	2.0085570859251\\
    -2	1.9563892986035\\
    -1.95	1.90404263889518\\
    -1.9	1.85149623294794\\
    -1.85	1.79872595028535\\
    -1.8	1.74570374703721\\
    -1.75	1.69239684182827\\
    -1.7	1.63876667138442\\
    -1.65	1.58476755241416\\
    -1.6	1.53034494621791\\
    -1.55	1.47543317737815\\
    -1.5	1.41995238880239\\
    -1.45	1.36380440690032\\
    -1.4	1.30686701541751\\
    -1.35	1.2489858440794\\
    -1.3	1.18996257208743\\
    -1.25	1.12953723186777\\
    -1.2	1.06736065948878\\
    -1.15	1.00294962449448\\
    -1.1	0.935609523735209\\
    -1.05	0.864291243406886\\
    -1	0.787299436620434\\
    -0.95	0.70161193151899\\
    -0.9	0.600924500691737\\
    -0.85	0.46742365802326\\
    -0.8	0\\
    };
    \addplot [color=black, forget plot]
      table[row sep=crcr]{%
    -0.8	-0\\
    -0.75	-0.448347851119676\\
    -0.7	-0.552877481367887\\
    -0.65	-0.619172499930452\\
    -0.6	-0.66644437032919\\
    -0.55	-0.701781178162032\\
    -0.5	-0.728736163112184\\
    -0.45	-0.749406938568536\\
    -0.4	-0.765172473108956\\
    -0.35	-0.777015220107044\\
    -0.3	-0.7856828007848\\
    -0.25	-0.791777762186883\\
    -0.2	-0.795811441579278\\
    -0.15	-0.798238310912097\\
    -0.0999999999999998	-0.799478827214055\\
    -0.0499999999999998	-0.799934890534424\\
    2.22044604925031e-16	-0.8\\
    0.0500000000000003	-0.800065098869194\\
    0.1	-0.800520494616583\\
    0.15	-0.801753964201621\\
    0.2	-0.804145151717811\\
    0.25	-0.808056612214814\\
    0.3	-0.81382230443977\\
    0.35	-0.821734875415588\\
    0.4	-0.832033529220762\\
    0.45	-0.844894423278802\\
    0.5	-0.860425244895165\\
    0.55	-0.878664900397598\\
    0.6	-0.899588289055083\\
    0.65	-0.923115227221045\\
    0.7	-0.949121995802933\\
    0.75	-0.977453823561153\\
    0.8	-1.0079368399159\\
    0.85	-1.04038847666707\\
    0.9	-1.07462579367714\\
    0.95	-1.11047162276053\\
    1	-1.14775870966343\\
    1.05	-1.18633218525399\\
    1.1	-1.22605074361495\\
    1.15	-1.26678688455789\\
    1.2	-1.30842652407544\\
    1.25	-1.35086821152099\\
    1.3	-1.3940221304312\\
    1.35	-1.43780900729455\\
    1.4	-1.48215901108412\\
    1.45	-1.52701069542186\\
    1.5	-1.57231001316563\\
    1.55	-1.61800941809483\\
    1.6	-1.66406705844152\\
    1.65	-1.71044606080434\\
    1.7	-1.75711389934935\\
    1.75	-1.80404184330095\\
    1.8	-1.85120447494864\\
    1.85	-1.89857927032065\\
    1.9	-1.94614623501491\\
    1.95	-1.99388758824203\\
    2	-2.04178748880059\\
    2.05	-2.08983179739575\\
    2.1	-2.13800787038313\\
    2.15	-2.18630438064437\\
    2.2	-2.23471116186911\\
    2.25	-2.28321907302334\\
    2.3	-2.33181988023012\\
    2.35	-2.38050615367692\\
    2.4	-2.42927117750053\\
    2.45	-2.47810887089055\\
    2.5	-2.52701371890211\\
    2.55	-2.57598071168189\\
    2.6	-2.62500529099463\\
    2.65	-2.67408330309355\\
    2.7	-2.72321095711191\\
    2.75	-2.77238478826693\\
    2.8	-2.82160162526491\\
    2.85	-2.87085856138022\\
    2.9	-2.92015292875166\\
    2.95	-2.96948227550153\\
    3	-3.0188443453347\\
    3.05	-3.06823705932049\\
    3.1	-3.11765849959853\\
    3.15	-3.16710689478337\\
    3.2	-3.21658060687125\\
    3.25	-3.26607811947718\\
    3.3	-3.31559802725215\\
    3.35	-3.36513902634858\\
    3.4	-3.41469990581837\\
    3.45	-3.46427953984192\\
    3.5	-3.51387688069854\\
    3.55	-3.56349095239924\\
    3.6	-3.61312084491233\\
    3.65	-3.66276570891993\\
    3.7	-3.71242475105105\\
    3.75	-3.76209722954255\\
    3.8	-3.81178245028514\\
    3.85	-3.86147976321609\\
    3.9	-3.9111885590246\\
    3.95	-3.96090826613942\\
    4	-4.01063834797167\\
    4.05	-4.0603783003885\\
    4.1	-4.11012764939604\\
    4.15	-4.159885949012\\
    4.2	-4.20965277931068\\
    4.25	-4.25942774462447\\
    4.3	-4.30921047188801\\
    4.35	-4.35900060911202\\
    4.4	-4.40879782397564\\
    4.45	-4.45860180252669\\
    4.5	-4.50841224798078\\
    4.55	-4.55822887961054\\
    4.6	-4.60805143171761\\
    4.65	-4.65787965268028\\
    4.7	-4.70771330407054\\
    4.75	-4.75755215983494\\
    4.8	-4.80739600553389\\
    4.85	-4.85724463763487\\
    4.9	-4.9070978628551\\
    4.95	-4.95695549754991\\
    5	-5.00681736714308\\
    };
    
    \addplot[area legend, draw=black, fill=black, forget plot]
    table[row sep=crcr] {%
    x	y\\
    -0.504229446767442	-0.765691394807118\\
    -0.504229446767441	-0.765691394807118\\
    -0.504229446767441	-0.765691394807118\\
    -0.504229446767441	-0.765691394807118\\
    -0.504229446767441	-0.765691394807118\\
    -0.504229446767441	-0.765691394807118\\
    -0.504229446767442	-0.765691394807118\\
    -0.570390488919266	-0.831852436958942\\
    -0.73496042078728	-0.53496042078728\\
    -0.438068404615619	-0.699530352655294\\
    -0.504229446767442	-0.765691394807118\\
    }--cycle;
    \addplot [color=black, forget plot]
      table[row sep=crcr]{%
    -5	4.99914652098967\\
    -4.95	4.94912918735758\\
    -4.9	4.89911132016312\\
    -4.85	4.84909289727554\\
    -4.8	4.79907389540437\\
    -4.75	4.74905429002565\\
    -4.7	4.69903405530267\\
    -4.65	4.64901316400076\\
    -4.6	4.59899158739541\\
    -4.55	4.54896929517344\\
    -4.5	4.49894625532624\\
    -4.45	4.44892243403476\\
    -4.4	4.39889779554507\\
    -4.35	4.34887230203398\\
    -4.3	4.29884591346357\\
    -4.25	4.24881858742365\\
    -4.2	4.19879027896099\\
    -4.15	4.148760940394\\
    -4.1	4.0987305211115\\
    -4.05	4.04869896735392\\
    -4	3.99866622197514\\
    -3.95	3.94863222418314\\
    -3.9	3.89859690925702\\
    -3.85	3.84856020823817\\
    -3.8	3.79852204759267\\
    -3.75	3.74848234884181\\
    -3.7	3.6984410281574\\
    -3.65	3.64839799591779\\
    -3.6	3.59835315622033\\
    -3.55	3.54830640634515\\
    -3.5	3.4982576361648\\
    -3.45	3.44820672749324\\
    -3.4	3.39815355336708\\
    -3.35	3.34809797725079\\
    -3.3	3.29803985215652\\
    -3.25	3.247979019668\\
    -3.2	3.19791530885622\\
    -3.15	3.14784853507302\\
    -3.1	3.09777849860655\\
    -3.05	3.04770498318025\\
    -3	2.99762775427414\\
    -2.95	2.94754655724383\\
    -2.9	2.89746111520897\\
    -2.85	2.84737112667825\\
    -2.8	2.79727626287235\\
    -2.75	2.74717616470037\\
    -2.7	2.69707043933719\\
    -2.65	2.64695865634035\\
    -2.6	2.59684034323389\\
    -2.55	2.54671498047403\\
    -2.5	2.49658199569499\\
    -2.45	2.44644075711489\\
    -2.4	2.39629056595804\\
    -2.35	2.34613064772161\\
    -2.3	2.29596014208024\\
    -2.25	2.24577809117915\\
    -2.2	2.19558342601378\\
    -2.15	2.14537495052835\\
    -2.1	2.09515132298392\\
    -2.05	2.04491103404392\\
    -2	1.99465238089546\\
    -1.95	1.94437343656038\\
    -1.9	1.89407201333963\\
    -1.85	1.84374561906457\\
    -1.8	1.7933914044787\\
    -1.75	1.74300609961694\\
    -1.7	1.69258593644905\\
    -1.65	1.64212655425798\\
    -1.6	1.59162288315856\\
    -1.55	1.54106899972245\\
    -1.5	1.49045794671067\\
    -1.45	1.43978150620107\\
    -1.4	1.38902991160651\\
    -1.35	1.3381914787078\\
    -1.3	1.28725212810786\\
    -1.25	1.23619476024515\\
    -1.2	1.18499842736295\\
    -1.15	1.13363722146959\\
    -1.1	1.08207875806342\\
    -1.05	1.0302820731206\\
    -1	0.978194649301749\\
    -0.95	0.925748116473972\\
    -0.899999999999999	0.872851873518604\\
    -0.849999999999999	0.819383335692209\\
    -0.799999999999999	0.765172473108955\\
    -0.75	0.709976194401196\\
    -0.7	0.653433507708757\\
    -0.649999999999999	0.594981286043714\\
    -0.599999999999999	0.533680329744389\\
    -0.549999999999999	0.467804761867604\\
    -0.499999999999999	0.393649718310217\\
    -0.449999999999999	0.300462250345868\\
    -0.399999999999999	0\\
    };
    \addplot [color=black, forget plot]
      table[row sep=crcr]{%
    -0.399999999999999	-0\\
    -0.349999999999999	-0.276438740683943\\
    -0.299999999999999	-0.333222185164595\\
    -0.249999999999999	-0.364368081556092\\
    -0.199999999999999	-0.382586236554477\\
    -0.149999999999999	-0.392841400392399\\
    -0.0999999999999994	-0.397905720789639\\
    -0.0499999999999994	-0.399739413607027\\
    5.55111512312578e-16	-0.399999999999999\\
    0.0500000000000005	-0.400260247308291\\
    0.100000000000001	-0.402072575858905\\
    0.150000000000001	-0.406911152219885\\
    0.200000000000001	-0.416016764610381\\
    0.250000000000001	-0.430212622447582\\
    0.300000000000001	-0.449794144527541\\
    0.350000000000001	-0.474560997901466\\
    0.400000000000001	-0.503968419957949\\
    0.450000000000001	-0.537312896838572\\
    0.500000000000001	-0.573879354831717\\
    0.550000000000001	-0.613025371807473\\
    0.600000000000001	-0.654213262037718\\
    0.650000000000001	-0.697011065215602\\
    0.700000000000001	-0.741079505542062\\
    0.750000000000001	-0.786155006582817\\
    0.800000000000001	-0.832033529220762\\
    0.850000000000001	-0.878556949674674\\
    0.900000000000001	-0.925602237474319\\
    0.950000000000001	-0.973073117507453\\
    1	-1.02089374440029\\
    1.05	-1.06900393519156\\
    1.1	-1.11735558093456\\
    1.15	-1.16590994011506\\
    1.2	-1.21463558875027\\
    1.25	-1.26350685945106\\
    1.3	-1.31250264549731\\
    1.35	-1.36160547855596\\
    1.4	-1.41080081263245\\
    1.45	-1.46007646437583\\
    1.5	-1.50942217266735\\
    1.55	-1.55882924979926\\
    1.6	-1.60829030343562\\
    1.65	-1.65779901362608\\
    1.7	-1.70734995290919\\
    1.75	-1.75693844034927\\
    1.8	-1.80656042245616\\
    1.85	-1.85621237552553\\
    1.9	-1.90589122514257\\
    1.95	-1.9555942795123\\
    2	-2.00531917398583\\
    2.05	-2.05506382469802\\
    2.1	-2.10482638965534\\
    2.15	-2.15460523594401\\
    2.2	-2.20439891198782\\
    2.25	-2.25420612399039\\
    2.3	-2.30402571585881\\
    2.35	-2.35385665203527\\
    2.4	-2.40369800276695\\
    2.45	-2.45354893142755\\
    2.5	-2.50340868357154\\
    2.55	-2.55327657745637\\
    2.6	-2.60315199581259\\
    2.65	-2.65303437867816\\
    2.7	-2.7029232171428\\
    2.75	-2.75281804787314\\
    2.8	-2.80271844830938\\
    2.85	-2.85262403244123\\
    2.9	-2.90253444708476\\
    2.95	-2.95244936859361\\
    3	-3.0023684999476\\
    3.05	-3.05229156817026\\
    3.1	-3.10221832203344\\
    3.15	-3.15214853001325\\
    3.2	-3.20208197846633\\
    3.25	-3.25201846999985\\
    3.3	-3.301957822012\\
    3.35	-3.35189986538303\\
    3.4	-3.40184444329931\\
    3.45	-3.45179141019516\\
    3.5	-3.50174063079926\\
    3.55	-3.55169197927399\\
    3.6	-3.60164533843748\\
    3.65	-3.65160059905935\\
    3.7	-3.70155765922244\\
    3.75	-3.75151642374346\\
    3.8	-3.80147680364636\\
    3.85	-3.85143871568324\\
    3.9	-3.90140208189766\\
    3.95	-3.95136682922639\\
    4	-4.00133288913564\\
    4.05	-4.0513001972884\\
    4.1	-4.10126869323999\\
    4.15	-4.15123832015895\\
    4.2	-4.20120902457112\\
    4.25	-4.25118075612447\\
    4.3	-4.30115346737305\\
    4.35	-4.35112711357808\\
    4.4	-4.40110165252475\\
    4.45	-4.45107704435337\\
    4.5	-4.50105325140348\\
    4.55	-4.5510302380699\\
    4.6	-4.60100797066964\\
    4.65	-4.65098641731874\\
    4.7	-4.70096554781825\\
    4.75	-4.7509453335485\\
    4.8	-4.80092574737107\\
    4.85	-4.85090676353777\\
    4.9	-4.90088835760606\\
    4.95	-4.95087050636041\\
    5	-5.00085318773919\\
    };
    
    \addplot[area legend, draw=black, fill=black, forget plot]
    table[row sep=crcr] {%
    x	y\\
    -0.186625652171389	-0.44833476861589\\
    -0.186625652171388	-0.44833476861589\\
    -0.186625652171388	-0.44833476861589\\
    -0.186625652171388	-0.44833476861589\\
    -0.186625652171388	-0.44833476861589\\
    -0.186625652171388	-0.44833476861589\\
    -0.186625652171389	-0.44833476861589\\
    -0.252822131523008	-0.514531247967509\\
    -0.41748021039364	-0.217480210393639\\
    -0.120429172819772	-0.382138289264271\\
    -0.186625652171389	-0.44833476861589\\
    }--cycle;
    \addplot [color=black, forget plot]
      table[row sep=crcr]{%
    -5	5.00085318773919\\
    -4.95	4.95087050636041\\
    -4.9	4.90088835760606\\
    -4.85	4.85090676353777\\
    -4.8	4.80092574737107\\
    -4.75	4.7509453335485\\
    -4.7	4.70096554781825\\
    -4.65	4.65098641731874\\
    -4.6	4.60100797066964\\
    -4.55	4.5510302380699\\
    -4.5	4.50105325140348\\
    -4.45	4.45107704435337\\
    -4.4	4.40110165252475\\
    -4.35	4.35112711357808\\
    -4.3	4.30115346737305\\
    -4.25	4.25118075612447\\
    -4.2	4.20120902457112\\
    -4.15	4.15123832015895\\
    -4.1	4.10126869323999\\
    -4.05	4.0513001972884\\
    -4	4.00133288913564\\
    -3.95	3.95136682922639\\
    -3.9	3.90140208189766\\
    -3.85	3.85143871568324\\
    -3.8	3.80147680364636\\
    -3.75	3.75151642374346\\
    -3.7	3.70155765922244\\
    -3.65	3.65160059905935\\
    -3.6	3.60164533843748\\
    -3.55	3.55169197927399\\
    -3.5	3.50174063079926\\
    -3.45	3.45179141019516\\
    -3.4	3.40184444329931\\
    -3.35	3.35189986538303\\
    -3.3	3.301957822012\\
    -3.25	3.25201846999985\\
    -3.2	3.20208197846633\\
    -3.15	3.15214853001325\\
    -3.1	3.10221832203344\\
    -3.05	3.05229156817026\\
    -3	3.0023684999476\\
    -2.95	2.95244936859361\\
    -2.9	2.90253444708476\\
    -2.85	2.85262403244123\\
    -2.8	2.80271844830938\\
    -2.75	2.75281804787314\\
    -2.7	2.7029232171428\\
    -2.65	2.65303437867816\\
    -2.6	2.60315199581259\\
    -2.55	2.55327657745637\\
    -2.5	2.50340868357154\\
    -2.45	2.45354893142755\\
    -2.4	2.40369800276695\\
    -2.35	2.35385665203527\\
    -2.3	2.30402571585881\\
    -2.25	2.25420612399039\\
    -2.2	2.20439891198782\\
    -2.15	2.15460523594401\\
    -2.1	2.10482638965534\\
    -2.05	2.05506382469802\\
    -2	2.00531917398583\\
    -1.95	1.9555942795123\\
    -1.9	1.90589122514257\\
    -1.85	1.85621237552553\\
    -1.8	1.80656042245616\\
    -1.75	1.75693844034927\\
    -1.7	1.70734995290919\\
    -1.65	1.65779901362608\\
    -1.6	1.60829030343562\\
    -1.55	1.55882924979926\\
    -1.5	1.50942217266735\\
    -1.45	1.46007646437583\\
    -1.4	1.41080081263245\\
    -1.35	1.36160547855596\\
    -1.3	1.31250264549731\\
    -1.25	1.26350685945106\\
    -1.2	1.21463558875027\\
    -1.15	1.16590994011506\\
    -1.1	1.11735558093456\\
    -1.05	1.06900393519156\\
    -1	1.02089374440029\\
    -0.95	0.973073117507453\\
    -0.9	0.925602237474319\\
    -0.85	0.878556949674674\\
    -0.8	0.832033529220762\\
    -0.75	0.786155006582817\\
    -0.7	0.741079505542062\\
    -0.65	0.697011065215602\\
    -0.6	0.654213262037718\\
    -0.55	0.613025371807473\\
    -0.5	0.573879354831717\\
    -0.45	0.537312896838572\\
    -0.4	0.503968419957949\\
    -0.35	0.474560997901467\\
    -0.3	0.449794144527542\\
    -0.25	0.430212622447582\\
    -0.2	0.416016764610381\\
    -0.15	0.406911152219885\\
    -0.1	0.402072575858906\\
    -0.05	0.400260247308292\\
    0	0.4\\
    0.05	0.399739413607028\\
    0.1	0.397905720789639\\
    0.15	0.3928414003924\\
    0.2	0.382586236554478\\
    0.25	0.364368081556092\\
    0.3	0.333222185164595\\
    0.35	0.276438740683944\\
    0.4	0\\
    };
    \addplot [color=black, forget plot]
      table[row sep=crcr]{%
    0.4	-0\\
    0.45	-0.300462250345868\\
    0.5	-0.393649718310217\\
    0.55	-0.467804761867605\\
    0.6	-0.533680329744389\\
    0.65	-0.594981286043714\\
    0.7	-0.653433507708757\\
    0.75	-0.709976194401196\\
    0.8	-0.765172473108956\\
    0.85	-0.81938333569221\\
    0.9	-0.872851873518604\\
    0.95	-0.925748116473972\\
    1	-0.97819464930175\\
    1.05	-1.0302820731206\\
    1.1	-1.08207875806342\\
    1.15	-1.13363722146959\\
    1.2	-1.18499842736295\\
    1.25	-1.23619476024515\\
    1.3	-1.28725212810786\\
    1.35	-1.3381914787078\\
    1.4	-1.38902991160651\\
    1.45	-1.43978150620107\\
    1.5	-1.49045794671067\\
    1.55	-1.54106899972245\\
    1.6	-1.59162288315856\\
    1.65	-1.64212655425798\\
    1.7	-1.69258593644905\\
    1.75	-1.74300609961694\\
    1.8	-1.7933914044787\\
    1.85	-1.84374561906457\\
    1.9	-1.89407201333963\\
    1.95	-1.94437343656038\\
    2	-1.99465238089546\\
    2.05	-2.04491103404392\\
    2.1	-2.09515132298392\\
    2.15	-2.14537495052835\\
    2.2	-2.19558342601378\\
    2.25	-2.24577809117915\\
    2.3	-2.29596014208024\\
    2.35	-2.34613064772161\\
    2.4	-2.39629056595804\\
    2.45	-2.44644075711489\\
    2.5	-2.49658199569499\\
    2.55	-2.54671498047403\\
    2.6	-2.59684034323389\\
    2.65	-2.64695865634035\\
    2.7	-2.69707043933719\\
    2.75	-2.74717616470037\\
    2.8	-2.79727626287235\\
    2.85	-2.84737112667825\\
    2.9	-2.89746111520897\\
    2.95	-2.94754655724383\\
    3	-2.99762775427414\\
    3.05	-3.04770498318025\\
    3.1	-3.09777849860655\\
    3.15	-3.14784853507302\\
    3.2	-3.19791530885622\\
    3.25	-3.247979019668\\
    3.3	-3.29803985215652\\
    3.35	-3.34809797725079\\
    3.4	-3.39815355336708\\
    3.45	-3.44820672749324\\
    3.5	-3.4982576361648\\
    3.55	-3.54830640634515\\
    3.6	-3.59835315622033\\
    3.65	-3.64839799591779\\
    3.7	-3.6984410281574\\
    3.75	-3.74848234884181\\
    3.8	-3.79852204759267\\
    3.85	-3.84856020823817\\
    3.9	-3.89859690925702\\
    3.95	-3.94863222418314\\
    4	-3.99866622197514\\
    4.05	-4.04869896735392\\
    4.1	-4.0987305211115\\
    4.15	-4.148760940394\\
    4.2	-4.19879027896099\\
    4.25	-4.24881858742365\\
    4.3	-4.29884591346357\\
    4.35	-4.34887230203398\\
    4.4	-4.39889779554507\\
    4.45	-4.44892243403476\\
    4.5	-4.49894625532624\\
    4.55	-4.54896929517344\\
    4.6	-4.59899158739541\\
    4.65	-4.64901316400076\\
    4.7	-4.69903405530267\\
    4.75	-4.74905429002565\\
    4.8	-4.79907389540437\\
    4.85	-4.84909289727554\\
    4.9	-4.89911132016312\\
    4.95	-4.94912918735758\\
    5	-4.99914652098967\\
    };
    
    \addplot[area legend, draw=black, fill=black, forget plot]
    table[row sep=crcr] {%
    x	y\\
    0.448370024112	0.186590396675279\\
    0.448370024111999	0.186590396675279\\
    0.448370024111999	0.186590396675279\\
    0.448370024111999	0.186590396675279\\
    0.448370024111999	0.186590396675279\\
    0.448370024111999	0.186590396675279\\
    0.448370024112001	0.186590396675279\\
    0.382163435409551	0.120383807972829\\
    0.21748021039364	0.41748021039364\\
    0.51457661281445	0.252796985377729\\
    0.448370024112001	0.186590396675279\\
    }--cycle;
    \addplot [color=black, forget plot]
      table[row sep=crcr]{%
    -5	5.00681736714308\\
    -4.95	4.95695549754991\\
    -4.9	4.9070978628551\\
    -4.85	4.85724463763487\\
    -4.8	4.80739600553389\\
    -4.75	4.75755215983494\\
    -4.7	4.70771330407054\\
    -4.65	4.65787965268028\\
    -4.6	4.60805143171761\\
    -4.55	4.55822887961054\\
    -4.5	4.50841224798078\\
    -4.45	4.45860180252669\\
    -4.4	4.40879782397564\\
    -4.35	4.35900060911202\\
    -4.3	4.30921047188801\\
    -4.25	4.25942774462447\\
    -4.2	4.20965277931068\\
    -4.15	4.159885949012\\
    -4.1	4.11012764939604\\
    -4.05	4.0603783003885\\
    -4	4.01063834797167\\
    -3.95	3.96090826613942\\
    -3.9	3.9111885590246\\
    -3.85	3.86147976321609\\
    -3.8	3.81178245028514\\
    -3.75	3.76209722954255\\
    -3.7	3.71242475105105\\
    -3.65	3.66276570891993\\
    -3.6	3.61312084491233\\
    -3.55	3.56349095239924\\
    -3.5	3.51387688069854\\
    -3.45	3.46427953984192\\
    -3.4	3.41469990581837\\
    -3.35	3.36513902634858\\
    -3.3	3.31559802725215\\
    -3.25	3.26607811947718\\
    -3.2	3.21658060687125\\
    -3.15	3.16710689478337\\
    -3.1	3.11765849959853\\
    -3.05	3.06823705932049\\
    -3	3.0188443453347\\
    -2.95	2.96948227550153\\
    -2.9	2.92015292875166\\
    -2.85	2.87085856138022\\
    -2.8	2.82160162526491\\
    -2.75	2.77238478826693\\
    -2.7	2.72321095711191\\
    -2.65	2.67408330309355\\
    -2.6	2.62500529099463\\
    -2.55	2.57598071168189\\
    -2.5	2.52701371890211\\
    -2.45	2.47810887089055\\
    -2.4	2.42927117750053\\
    -2.35	2.38050615367692\\
    -2.3	2.33181988023012\\
    -2.25	2.28321907302334\\
    -2.2	2.23471116186911\\
    -2.15	2.18630438064437\\
    -2.1	2.13800787038313\\
    -2.05	2.08983179739575\\
    -2	2.04178748880059\\
    -1.95	1.99388758824203\\
    -1.9	1.94614623501491\\
    -1.85	1.89857927032065\\
    -1.8	1.85120447494864\\
    -1.75	1.80404184330094\\
    -1.7	1.75711389934935\\
    -1.65	1.71044606080434\\
    -1.6	1.66406705844152\\
    -1.55	1.61800941809483\\
    -1.5	1.57231001316563\\
    -1.45	1.52701069542186\\
    -1.4	1.48215901108412\\
    -1.35	1.43780900729455\\
    -1.3	1.3940221304312\\
    -1.25	1.35086821152099\\
    -1.2	1.30842652407544\\
    -1.15	1.26678688455789\\
    -1.1	1.22605074361495\\
    -1.05	1.18633218525399\\
    -1	1.14775870966343\\
    -0.95	1.11047162276053\\
    -0.9	1.07462579367714\\
    -0.85	1.04038847666707\\
    -0.8	1.0079368399159\\
    -0.75	0.977453823561153\\
    -0.7	0.949121995802933\\
    -0.65	0.923115227221045\\
    -0.6	0.899588289055083\\
    -0.55	0.878664900397598\\
    -0.5	0.860425244895165\\
    -0.45	0.844894423278803\\
    -0.4	0.832033529220762\\
    -0.35	0.821734875415588\\
    -0.3	0.81382230443977\\
    -0.25	0.808056612214814\\
    -0.2	0.804145151717812\\
    -0.15	0.801753964201621\\
    -0.1	0.800520494616584\\
    -0.05	0.800065098869195\\
    0	0.8\\
    0.05	0.799934890534424\\
    0.1	0.799478827214056\\
    0.15	0.798238310912097\\
    0.2	0.795811441579278\\
    0.25	0.791777762186883\\
    0.3	0.7856828007848\\
    0.35	0.777015220107044\\
    0.4	0.765172473108956\\
    0.45	0.749406938568536\\
    0.5	0.728736163112184\\
    0.55	0.701781178162032\\
    0.6	0.666444370329191\\
    0.65	0.619172499930452\\
    0.7	0.552877481367887\\
    0.75	0.448347851119676\\
    0.8	0\\
    };
    \addplot [color=black, forget plot]
      table[row sep=crcr]{%
    0.8	-0\\
    0.85	-0.46742365802326\\
    0.9	-0.600924500691737\\
    0.95	-0.70161193151899\\
    1	-0.787299436620435\\
    1.05	-0.864291243406886\\
    1.1	-0.935609523735209\\
    1.15	-1.00294962449448\\
    1.2	-1.06736065948878\\
    1.25	-1.12953723186777\\
    1.3	-1.18996257208743\\
    1.35	-1.2489858440794\\
    1.4	-1.30686701541751\\
    1.45	-1.36380440690032\\
    1.5	-1.41995238880239\\
    1.55	-1.47543317737815\\
    1.6	-1.53034494621791\\
    1.65	-1.58476755241416\\
    1.7	-1.63876667138442\\
    1.75	-1.69239684182827\\
    1.8	-1.74570374703721\\
    1.85	-1.79872595028535\\
    1.9	-1.85149623294794\\
    1.95	-1.90404263889518\\
    2	-1.9563892986035\\
    2.05	-2.0085570859251\\
    2.1	-2.06056414624121\\
    2.15	-2.11242632470666\\
    2.2	-2.16415751612684\\
    2.25	-2.21576995281205\\
    2.3	-2.26727444293918\\
    2.35	-2.31868056911712\\
    2.4	-2.3699968547259\\
    2.45	-2.42123090398711\\
    2.5	-2.4723895204903\\
    2.55	-2.52347880794844\\
    2.6	-2.57450425621572\\
    2.65	-2.62547081502136\\
    2.7	-2.6763829574156\\
    2.75	-2.72724473456117\\
    2.8	-2.77805982321302\\
    2.85	-2.82883156699606\\
    2.9	-2.87956301240214\\
    2.95	-2.93025694027429\\
    3	-2.98091589342134\\
    3.05	-3.0315422009035\\
    3.1	-3.08213799944491\\
    3.15	-3.13270525235931\\
    3.2	-3.18324576631711\\
    3.25	-3.23376120623338\\
    3.3	-3.28425310851596\\
    3.35	-3.33472289287901\\
    3.4	-3.38517187289809\\
    3.45	-3.43560126545931\\
    3.5	-3.48601219923388\\
    3.55	-3.53640572229232\\
    3.6	-3.58678280895741\\
    3.65	-3.63714436598227\\
    3.7	-3.68749123812914\\
    3.75	-3.73782421321457\\
    3.8	-3.78814402667926\\
    3.85	-3.83845136573318\\
    3.9	-3.88874687312076\\
    3.95	-3.93903115054581\\
    4	-3.98930476179092\\
    4.05	-4.03956823556236\\
    4.1	-4.08982206808783\\
    4.15	-4.1400667254915\\
    4.2	-4.19030264596785\\
    4.25	-4.24053024177388\\
    4.3	-4.29074990105671\\
    4.35	-4.34096198953208\\
    4.4	-4.39116685202757\\
    4.45	-4.44136481390288\\
    4.5	-4.4915561823583\\
    4.55	-4.54174124764133\\
    4.6	-4.59192028416048\\
    4.65	-4.64209355151434\\
    4.7	-4.69226129544322\\
    4.75	-4.74242374870999\\
    4.8	-4.79258113191607\\
    4.85	-4.84273365425806\\
    4.9	-4.89288151422978\\
    4.95	-4.94302490027429\\
    5	-4.99316399138997\\
    };
    
    \addplot[area legend, draw=black, fill=black, forget plot]
    table[row sep=crcr] {%
    x	y\\
    0.765973439638583	0.503947401935975\\
    0.765973439638583	0.503947401935975\\
    0.765973439638583	0.503947401935975\\
    0.765973439638583	0.503947401935975\\
    0.765973439638583	0.503947401935975\\
    0.765973439638583	0.503947401935975\\
    0.765973439638583	0.503947401935974\\
    0.69973152243276	0.43770548473015\\
    0.53496042078728	0.734960420787281\\
    0.832215356844406	0.570189319141798\\
    0.765973439638583	0.503947401935974\\
    }--cycle;
    \addplot [color=black, forget plot]
      table[row sep=crcr]{%
    -5	5.02293463961359\\
    -4.95	4.97339704087823\\
    -4.9	4.92387350015961\\
    -4.85	4.87436458954849\\
    -4.8	4.82487091030687\\
    -4.75	4.77539309465408\\
    -4.7	4.72593180768029\\
    -4.65	4.67648774939779\\
    -4.6	4.6270616569413\\
    -4.55	4.57765430692961\\
    -4.5	4.52826651800205\\
    -4.45	4.47889915354446\\
    -4.4	4.42955312462073\\
    -4.35	4.38022939312749\\
    -4.3	4.33092897519133\\
    -4.25	4.2816529448295\\
    -4.2	4.23240243789736\\
    -4.15	4.18317865634805\\
    -4.1	4.13398287283209\\
    -4.05	4.08481643566787\\
    -4	4.03568077421651\\
    -3.95	3.98657740469835\\
    -3.9	3.93750793649194\\
    -3.85	3.88847407896054\\
    -3.8	3.83947764885575\\
    -3.75	3.79052057835317\\
    -3.7	3.74160492378036\\
    -3.65	3.69273287510416\\
    -3.6	3.64390676625079\\
    -3.55	3.59512908634073\\
    -3.5	3.54640249192816\\
    -3.45	3.49772982034518\\
    -3.4	3.44911410426097\\
    -3.35	3.40055858757846\\
    -3.3	3.35206674280367\\
    -3.25	3.30364229003798\\
    -3.2	3.25528921775908\\
    -3.15	3.20701180557469\\
    -3.1	3.1588146491524\\
    -3.05	3.11070268755082\\
    -3	3.06268123320088\\
    -2.95	3.01475600481223\\
    -2.9	2.96693316350804\\
    -2.85	2.91921935252236\\
    -2.8	2.87162174082733\\
    -2.75	2.82414807109337\\
    -2.7	2.77680671242296\\
    -2.65	2.72960671833826\\
    -2.6	2.68255789054354\\
    -2.55	2.63567084902402\\
    -2.5	2.58895710908282\\
    -2.45	2.54242916595418\\
    -2.4	2.49610058766228\\
    -2.35	2.44998611681707\\
    -2.3	2.40410178204596\\
    -2.25	2.35846501974845\\
    -2.2	2.31309480681907\\
    -2.15	2.26801180490441\\
    -2.1	2.22323851662619\\
    -2.05	2.17879945399927\\
    -2	2.13472131897756\\
    -1.95	2.0910331956468\\
    -1.9	2.04776675301819\\
    -1.85	2.00495645662199\\
    -1.8	1.96263978611315\\
    -1.75	1.92085745482962\\
    -1.7	1.87965362563957\\
    -1.65	1.83907611542242\\
    -1.6	1.79917657811017\\
    -1.55	1.76001065334461\\
    -1.5	1.72163806449515\\
    -1.45	1.68412264609647\\
    -1.4	1.64753227685619\\
    -1.35	1.61193869051572\\
    -1.3	1.57741713343806\\
    -1.25	1.5440458354403\\
    -1.2	1.51190525987385\\
    -1.15	1.48107710124442\\
    -1.1	1.45164300481851\\
    -1.05	1.4236829937044\\
    -1	1.39727360556362\\
    -0.95	1.37248576355805\\
    -0.9	1.34938243358262\\
    -0.85	1.32801615024874\\
    -0.8	1.30842652407544\\
    -0.75	1.29063786734275\\
    -0.7	1.27465709091815\\
    -0.65	1.26047202437313\\
    -0.6	1.24805029383114\\
    -0.55	1.2373388560025\\
    -0.5	1.22826423595173\\
    -0.45	1.22073345665966\\
    -0.4	1.21463558875027\\
    -0.35	1.20984379742608\\
    -0.3	1.20621772757672\\
    -0.25	1.20360605095734\\
    -0.2	1.20184900138347\\
    -0.15	1.20078074192488\\
    -0.0999999999999999	1.20023143684277\\
    -0.0499999999999998	1.20002893448751\\
    0	1.2\\
    0.05	1.19997106411708\\
    0.1	1.19976847385109\\
    0.15	1.19921824082108\\
    0.2	1.19814528297902\\
    0.25	1.19637214512868\\
    0.3	1.19371716236892\\
    0.35	1.18999199674489\\
    0.4	1.18499842736295\\
    0.45	1.1785242011772\\
    0.5	1.17033764331725\\
    0.55	1.16018056047831\\
    0.6	1.14775870966343\\
    0.65	1.13272867592193\\
    0.7	1.11467926989354\\
    0.75	1.09310424466828\\
    0.8	1.06736065948878\\
    0.85	1.03660226208147\\
    0.9	0.999666555493786\\
    0.95	0.954868817813526\\
    1	0.899588289055083\\
    1.05	0.829316222051831\\
    1.1	0.734959659655251\\
    1.15	0.591667217703205\\
    1.2	0\\
    };
    \addplot [color=black, forget plot]
      table[row sep=crcr]{%
    1.2	-0\\
    1.25	-0.608332812076033\\
    1.3	-0.776946201167251\\
    1.35	-0.901386751037605\\
    1.4	-1.0053051390627\\
    1.45	-1.09713441468862\\
    1.5	-1.18094915493065\\
    1.55	-1.25905425798022\\
    1.6	-1.33288874065838\\
    1.65	-1.40341428560281\\
    1.7	-1.47130647268414\\
    1.75	-1.53705839213066\\
    1.8	-1.60104098923317\\
    1.85	-1.66354025008984\\
    1.9	-1.72478118067315\\
    1.95	-1.78494385813114\\
    2	-1.84417451682338\\
    2.05	-1.90259341141233\\
    2.1	-1.96030052312627\\
    2.15	-2.01737978407905\\
    2.2	-2.07390225963995\\
    2.25	-2.12992858320359\\
    2.3	-2.18551084481481\\
    2.35	-2.24069407435017\\
    2.4	-2.29551741932687\\
    2.45	-2.35001508968732\\
    2.5	-2.40421712264546\\
    2.55	-2.45815000707663\\
    2.6	-2.51183719717883\\
    2.65	-2.56529953804358\\
    2.7	-2.61855562055581\\
    2.75	-2.67162207915641\\
    2.8	-2.72451384307563\\
    2.85	-2.77724434942192\\
    2.9	-2.829825724804\\
    2.95	-2.88226894084361\\
    3	-2.93458394790525\\
    3.05	-2.9867797905596\\
    3.1	-3.03886470765518\\
    3.15	-3.09084621936181\\
    3.2	-3.1427312031392\\
    3.25	-3.19452596025271\\
    3.3	-3.24623627419026\\
    3.35	-3.29786746211483\\
    3.4	-3.3494244203076\\
    3.45	-3.40091166440876\\
    3.5	-3.45233336514064\\
    3.55	-3.50369338009591\\
    3.6	-3.55499528208884\\
    3.65	-3.60624238449633\\
    3.7	-3.65743776395566\\
    3.75	-3.70858428073544\\
    3.8	-3.75968459705344\\
    3.85	-3.81074119357857\\
    3.9	-3.86175638432359\\
    3.95	-3.91273233010832\\
    4	-3.96367105075071\\
    4.05	-4.0145744361234\\
    4.1	-4.06544425619686\\
    4.15	-4.11628217017533\\
    4.2	-4.16708973481953\\
    4.25	-4.21786841203885\\
    4.3	-4.26861957582661\\
    4.35	-4.31934451860321\\
    4.4	-4.37004445702512\\
    4.45	-4.42072053731118\\
    4.5	-4.471373840132\\
    4.55	-4.52200538510347\\
    4.6	-4.572616134921\\
    4.65	-4.62320699916736\\
    4.7	-4.67377883782354\\
    4.75	-4.72433246450907\\
    4.8	-4.77486864947567\\
    4.85	-4.82538812237561\\
    4.9	-4.87589157482422\\
    4.95	-4.92637966277395\\
    5	-4.97685300871582\\
    };
    
    \addplot[area legend, draw=black, fill=black, forget plot]
    table[row sep=crcr] {%
    x	y\\
    1.08378659285017	0.821094669511674\\
    1.08378659285017	0.821094669511674\\
    1.08378659285017	0.821094669511674\\
    1.08378659285017	0.821094669511674\\
    1.08378659285017	0.821094669511674\\
    1.08378659285017	0.821094669511674\\
    1.08378659285017	0.821094669511674\\
    1.01744920582758	0.754757282489084\\
    0.85244063118092	1.05244063118092\\
    1.15012397987276	0.887432056534264\\
    1.08378659285017	0.821094669511674\\
    }--cycle;
    \addplot [color=black, forget plot]
      table[row sep=crcr]{%
    -5	5.05402743780422\\
    -4.95	5.00510644972551\\
    -4.9	4.9562177417811\\
    -4.85	4.90736259539177\\
    -4.8	4.85854235500106\\
    -4.75	4.80975843174301\\
    -4.7	4.76101230735385\\
    -4.65	4.71230553834548\\
    -4.6	4.66363976046024\\
    -4.55	4.61501669342777\\
    -4.5	4.56643814604669\\
    -4.45	4.51790602161539\\
    -4.4	4.46942232373823\\
    -4.35	4.42098916253565\\
    -4.3	4.37260876128874\\
    -4.25	4.32428346355148\\
    -4.2	4.27601574076625\\
    -4.15	4.2278082004213\\
    -4.1	4.17966359479151\\
    -4.05	4.1315848303075\\
    -4	4.08357497760117\\
    -3.95	4.03563728227971\\
    -3.9	3.98777517648406\\
    -3.85	3.93999229129192\\
    -3.8	3.89229247002981\\
    -3.75	3.84467978256357\\
    -3.7	3.7971585406413\\
    -3.65	3.74973331436812\\
    -3.6	3.70240894989728\\
    -3.55	3.65519058842745\\
    -3.5	3.60808368660189\\
    -3.45	3.56109403841015\\
    -3.4	3.5142277986987\\
    -3.35	3.46749150840182\\
    -3.3	3.42089212160868\\
    -3.25	3.37443703458653\\
    -3.2	3.32813411688305\\
    -3.15	3.28199174463239\\
    -3.1	3.23601883618965\\
    -3.05	3.19022489021596\\
    -3	3.14462002633127\\
    -2.95	3.09921502844282\\
    -2.9	3.05402139084372\\
    -2.85	3.00905136715629\\
    -2.8	2.96431802216825\\
    -2.75	2.91983528657429\\
    -2.7	2.87561801458909\\
    -2.65	2.83168204433867\\
    -2.6	2.78804426086241\\
    -2.55	2.74472266146485\\
    -2.5	2.70173642304198\\
    -2.45	2.65910597086681\\
    -2.4	2.61685304815087\\
    -2.35	2.57500078549635\\
    -2.3	2.53357376911577\\
    -2.25	2.4925981064171\\
    -2.2	2.45210148722989\\
    -2.15	2.41211323858019\\
    -2.1	2.37266437050798\\
    -2.05	2.33378760996385\\
    -2	2.29551741932687\\
    -1.95	2.25788999556407\\
    -1.9	2.22094324552107\\
    -1.85	2.18471673231751\\
    -1.8	2.14925158735429\\
    -1.75	2.11459038206672\\
    -1.7	2.08077695333414\\
    -1.65	2.0478561764489\\
    -1.6	2.0158736798318\\
    -1.55	1.98487549633947\\
    -1.5	1.95490764712231\\
    -1.45	1.92601565563293\\
    -1.4	1.89824399160587\\
    -1.35	1.87163544764605\\
    -1.3	1.84623045444209\\
    -1.25	1.82206634446136\\
    -1.2	1.79917657811017\\
    -1.15	1.77758995049765\\
    -1.1	1.7573298007952\\
    -1.05	1.73841324934997\\
    -1	1.72085048979033\\
    -0.95	1.70464416398064\\
    -0.9	1.68978884655761\\
    -0.85	1.67627066276552\\
    -0.8	1.66406705844152\\
    -0.75	1.65314673452675\\
    -0.7	1.64346975083118\\
    -0.65	1.634987795549\\
    -0.6	1.62764460887954\\
    -0.55	1.62137654172857\\
    -0.5	1.61611322442963\\
    -0.45	1.61177831615857\\
    -0.4	1.60829030343562\\
    -0.35	1.60556331581963\\
    -0.3	1.60350792840324\\
    -0.25	1.60203192366986\\
    -0.2	1.60104098923317\\
    -0.15	1.60043933248082\\
    -0.1	1.60013019773839\\
    -0.05	1.6000162758761\\
    0	1.6\\
    0.05	1.59998372379276\\
    0.1	1.59986978106885\\
    0.15	1.59956042612031\\
    0.2	1.59895765442811\\
    0.25	1.59796290228795\\
    0.3	1.59647662182419\\
    0.35	1.59439772467235\\
    0.4	1.59162288315856\\
    0.45	1.58804567183962\\
    0.5	1.58355552437377\\
    0.55	1.57803647021591\\
    0.6	1.5713656015696\\
    0.65	1.5634112018452\\
    0.7	1.55403044021409\\
    0.75	1.54306649904155\\
    0.8	1.53034494621791\\
    0.85	1.51566908328221\\
    0.9	1.49881387713707\\
    0.95	1.47951789142451\\
    1	1.45747232622437\\
    1.05	1.43230576551778\\
    1.1	1.40356235632406\\
    1.15	1.37066957164814\\
    1.2	1.33288874065838\\
    1.25	1.28923557057932\\
    1.3	1.2383449998609\\
    1.35	1.17822413134853\\
    1.4	1.10575496273577\\
    1.45	1.01554865303808\\
    1.5	0.896695702239352\\
    1.55	0.719277180972339\\
    1.6	0\\
    };
    \addplot [color=black, forget plot]
      table[row sep=crcr]{%
    1.6	-0\\
    1.65	-0.734419304705956\\
    1.7	-0.93484731604652\\
    1.75	-1.08104579769418\\
    1.8	-1.20184900138347\\
    1.85	-1.30757412870449\\
    1.9	-1.40322386303798\\
    1.95	-1.4916386638245\\
    2	-1.57459887324087\\
    2.05	-1.65329918399086\\
    2.1	-1.72858248681377\\
    2.15	-1.80106675450351\\
    2.2	-1.87121904747042\\
    2.25	-1.93940118621276\\
    2.3	-2.00589924898897\\
    2.35	-2.07094334934552\\
    2.4	-2.13472131897756\\
    2.45	-2.19738843001697\\
    2.5	-2.25907446373555\\
    2.55	-2.31988895376138\\
    2.6	-2.37992514417486\\
    2.65	-2.43926302431389\\
    2.7	-2.49797168815881\\
    2.75	-2.55611119158266\\
    2.8	-2.61373403083503\\
    2.85	-2.67088633154152\\
    2.9	-2.72760881380063\\
    2.95	-2.78393758220642\\
    3	-2.83990477760478\\
    3.05	-2.89553911864629\\
    3.1	-2.95086635475631\\
    3.15	-3.00590964734172\\
    3.2	-3.06068989243582\\
    3.25	-3.11522599523022\\
    3.3	-3.16953510482831\\
    3.35	-3.22363281591635\\
    3.4	-3.27753334276884\\
    3.45	-3.33124966999775\\
    3.5	-3.38479368365655\\
    3.55	-3.4381762856732\\
    3.6	-3.49140749407442\\
    3.65	-3.54449653105004\\
    3.7	-3.5974519005707\\
    3.75	-3.65028145699731\\
    3.8	-3.70299246589589\\
    3.85	-3.75559165808512\\
    3.9	-3.80808527779036\\
    3.95	-3.86047912564921\\
    4	-3.912778597207\\
    4.05	-3.96498871745015\\
    4.1	-4.01711417185019\\
    4.15	-4.06915933432656\\
    4.2	-4.12112829248242\\
    4.25	-4.17302487042138\\
    4.3	-4.22485264941332\\
    4.35	-4.27661498664397\\
    4.4	-4.32831503225368\\
    4.45	-4.3799557448459\\
    4.5	-4.4315399056241\\
    4.55	-4.48307013129735\\
    4.6	-4.53454888587835\\
    4.65	-4.5859784914838\\
    4.7	-4.63736113823424\\
    4.75	-4.68869889334021\\
    4.8	-4.73999370945179\\
    4.85	-4.79124743234037\\
    4.9	-4.84246180797421\\
    4.95	-4.89363848904296\\
    5	-4.94477904098059\\
    };
    
    \addplot[area legend, draw=black, fill=black, forget plot]
    table[row sep=crcr] {%
    x	y\\
    1.40190910322764	1.13793257992148\\
    1.40190910322764	1.13793257992148\\
    1.40190910322764	1.13793257992148\\
    1.40190910322764	1.13793257992148\\
    1.40190910322764	1.13793257992148\\
    1.40190910322764	1.13793257992148\\
    1.40190910322764	1.13793257992148\\
    1.33538753964841	1.07141101634225\\
    1.16992084157456	1.36992084157456\\
    1.46843066680687	1.20445414350071\\
    1.40190910322764	1.13793257992148\\
    }--cycle;
    \addplot [color=black, forget plot]
      table[row sep=crcr]{%
    -5	5.10446872200146\\
    -4.95	5.05652360849244\\
    -4.9	5.00863840034637\\
    -4.85	4.96081538201475\\
    -4.8	4.91305694380694\\
    -4.75	4.86536558753726\\
    -4.7	4.81774393250215\\
    -4.65	4.77019472180698\\
    -4.6	4.72272082906374\\
    -4.55	4.67532526548129\\
    -4.5	4.62801118737159\\
    -4.45	4.58078190409608\\
    -4.4	4.53364088647765\\
    -4.35	4.48659177570492\\
    -4.3	4.43963839275638\\
    -4.25	4.39278474837337\\
    -4.2	4.34603505361155\\
    -4.15	4.29939373100183\\
    -4.1	4.25286542635209\\
    -4.05	4.20645502122203\\
    -4	4.16016764610381\\
    -3.95	4.11400869434114\\
    -3.9	4.06798383681954\\
    -3.85	4.02209903745978\\
    -3.8	3.97636056954548\\
    -3.75	3.93077503291408\\
    -3.7	3.88534937203808\\
    -3.65	3.84009089502003\\
    -3.6	3.79500729352051\\
    -3.55	3.75010666363274\\
    -3.5	3.70539752771031\\
    -3.45	3.66088885714617\\
    -3.4	3.61659009608984\\
    -3.35	3.5725111860773\\
    -3.3	3.52866259153212\\
    -3.25	3.48505532607801\\
    -3.2	3.44170097958066\\
    -3.15	3.39861174581109\\
    -3.1	3.35580045059213\\
    -3.05	3.31328058025453\\
    -3	3.27106631018859\\
    -2.95	3.22917253323018\\
    -2.9	3.18761488756736\\
    -2.85	3.14640978379311\\
    -2.8	3.10557443066252\\
    -2.75	3.06512685903736\\
    -2.7	3.02508594341803\\
    -2.65	2.98547142037234\\
    -2.6	2.94630390307269\\
    -2.55	2.90760489104936\\
    -2.5	2.86939677415858\\
    -2.45	2.83170282965258\\
    -2.4	2.79454721112724\\
    -2.35	2.75795492801531\\
    -2.3	2.72195181419418\\
    -2.25	2.68656448419286\\
    -2.2	2.65182027542034\\
    -2.15	2.61774717480497\\
    -2.1	2.58437372824212\\
    -2.05	2.55172893130465\\
    -2	2.51984209978975\\
    -1.95	2.48874271886697\\
    -1.9	2.4584602698667\\
    -1.85	2.42902403411495\\
    -1.8	2.40046287368554\\
    -1.75	2.37280498950733\\
    -1.7	2.34607765792876\\
    -1.65	2.32030694759599\\
    -1.6	2.29551741932687\\
    -1.55	2.27173181253447\\
    -1.5	2.24897072263771\\
    -1.45	2.22725227474814\\
    -1.4	2.20659179969292\\
    -1.35	2.18700151906935\\
    -1.3	2.16849024647197\\
    -1.25	2.15106311223791\\
    -1.2	2.13472131897756\\
    -1.15	2.11946193476664\\
    -1.1	2.10527773016229\\
    -1.05	2.09215706417879\\
    -1	2.0800838230519\\
    -0.95	2.06903741408839\\
    -0.9	2.05899281521183\\
    -0.85	2.04992067906451\\
    -0.8	2.04178748880059\\
    -0.75	2.03455576109943\\
    -0.7	2.02818429052461\\
    -0.65	2.02262842821894\\
    -0.6	2.01784038710825\\
    -0.55	2.01376956530874\\
    -0.5	2.01036287929453\\
    -0.45	2.0075650985622\\
    -0.4	2.00531917398583\\
    -0.35	2.00356655273682\\
    -0.3	2.00224747348544\\
    -0.25	2.00130123654146\\
    -0.2	2.00066644456782\\
    -0.15	2.00028121045849\\
    -0.1	2.00008332986135\\
    -0.0500000000000003	2.00001041661241\\
    0	2\\
    0.0499999999999998	1.99998958327908\\
    0.0999999999999999	1.9999166631942\\
    0.15	1.99971871043995\\
    0.2	1.99933311098757\\
    0.25	1.99869706803514\\
    0.3	1.9977474639932\\
    0.35	1.99642068139387\\
    0.4	1.99465238089546\\
    0.45	1.99237723362748\\
    0.5	1.9895286039482\\
    0.55	1.98603817722008\\
    0.6	1.98183552537535\\
    0.65	1.97684760074296\\
    0.7	1.97099814569671\\
    0.75	1.964207001962\\
    0.8	1.9563892986035\\
    0.85	1.94745449140934\\
    0.9	1.93730521801843\\
    0.95	1.92583592187964\\
    1	1.91293118277239\\
    1.05	1.89846367034774\\
    1.1	1.88229160722335\\
    1.15	1.86425558533691\\
    1.2	1.84417451682338\\
    1.25	1.82184040778046\\
    1.3	1.79701150191753\\
    1.35	1.76940312047886\\
    1.4	1.73867517068787\\
    1.45	1.70441470781648\\
    1.5	1.66611092582298\\
    1.55	1.62311813709513\\
    1.6	1.57459887324087\\
    1.65	1.51943235382759\\
    1.7	1.45605867613633\\
    1.75	1.38219370341972\\
    1.8	1.29425472539207\\
    1.85	1.18603605379583\\
    1.9	1.04494928899312\\
    1.95	0.836404225202668\\
    2	0\\
    };
    \addplot [color=black, forget plot]
      table[row sep=crcr]{%
    2	-0\\
    2.05	-0.850461110824827\\
    2.1	-1.08036795875479\\
    2.15	-1.24684537872232\\
    2.2	-1.38347928332185\\
    2.25	-1.50231125172934\\
    2.3	-1.6091918838378\\
    2.35	-1.70745000271621\\
    2.4	-1.79917657811017\\
    2.45	-1.88577792730902\\
    2.5	-1.96824859155109\\
    2.55	-2.04731951866994\\
    2.6	-2.12354456278193\\
    2.65	-2.1973539123126\\
    2.7	-2.26908862610406\\
    2.75	-2.33902380933802\\
    2.8	-2.40738466192258\\
    2.85	-2.47435789212902\\
    2.9	-2.54010002292785\\
    2.95	-2.60474355930451\\
    3	-2.66840164872194\\
    3.05	-2.73117165824929\\
    3.1	-2.79313795863858\\
    3.15	-2.85437411839193\\
    3.2	-2.91494465244874\\
    3.25	-2.97490643021857\\
    3.3	-3.03430981992715\\
    3.35	-3.09319962661313\\
    3.4	-3.15161586702219\\
    3.45	-3.20959441438957\\
    3.5	-3.26716753854379\\
    3.55	-3.32436436112808\\
    3.6	-3.38121124148806\\
    3.65	-3.43773210553963\\
    3.7	-3.49394872744598\\
    3.75	-3.54988097200598\\
    3.8	-3.60554700415042\\
    3.85	-3.66096347075747\\
    3.9	-3.71614565905798\\
    3.95	-3.77110763515079\\
    4	-3.82586236554478\\
    4.05	-3.88042182415671\\
    4.1	-3.93479708679752\\
    4.15	-3.98899841485557\\
    4.2	-4.04303532961917\\
    4.25	-4.09691667846105\\
    4.3	-4.15065069392485\\
    4.35	-4.20424504660246\\
    4.4	-4.2577068925634\\
    4.45	-4.31104291599141\\
    4.5	-4.36425936759302\\
    4.55	-4.4173620992673\\
    4.6	-4.47035659546091\\
    4.65	-4.52324800157807\\
    4.7	-4.5760411497677\\
    4.75	-4.62874058236986\\
    4.8	-4.68135057326899\\
    4.85	-4.73387514737148\\
    4.9	-4.78631809839936\\
    4.95	-4.83868300516958\\
    5	-4.89097324650875\\
    };
    
    \addplot[area legend, draw=black, fill=black, forget plot]
    table[row sep=crcr] {%
    x	y\\
    1.72043130526918	1.45437079866722\\
    1.72043130526918	1.45437079866722\\
    1.72043130526918	1.45437079866722\\
    1.72043130526918	1.45437079866722\\
    1.72043130526918	1.45437079866722\\
    1.72043130526918	1.45437079866722\\
    1.72043130526918	1.45437079866722\\
    1.65361095539292	1.38755044879096\\
    1.4874010519682	1.6874010519682\\
    1.78725165514544	1.52119114854348\\
    1.72043130526918	1.45437079866722\\
    }--cycle;
    \addplot [color=black, forget plot]
      table[row sep=crcr]{%
    -5	5.17791421816565\\
    -4.95	5.13133818241302\\
    -4.9	5.08485833190836\\
    -4.85	5.03847812715392\\
    -4.8	4.99220117532457\\
    -4.75	4.94603123707086\\
    -4.7	4.89997223363413\\
    -4.65	4.85402825428448\\
    -4.6	4.80820356409192\\
    -4.55	4.76250261204083\\
    -4.5	4.7169300394969\\
    -4.45	4.6714906890354\\
    -4.4	4.62618961363815\\
    -4.35	4.58103208626559\\
    -4.3	4.53602360980881\\
    -4.25	4.49116992742435\\
    -4.2	4.44647703325237\\
    -4.15	4.40195118351628\\
    -4.1	4.35759890799854\\
    -4.05	4.31342702188364\\
    -4	4.26944263795511\\
    -3.95	4.22565317912812\\
    -3.9	4.18206639129361\\
    -3.85	4.1386903564432\\
    -3.8	4.09553350603637\\
    -3.75	4.05260463456297\\
    -3.7	4.00991291324399\\
    -3.65	3.9674679038028\\
    -3.6	3.92527957222631\\
    -3.55	3.88335830242172\\
    -3.5	3.84171490965925\\
    -3.45	3.80036065367366\\
    -3.4	3.75930725127913\\
    -3.35	3.71856688833088\\
    -3.3	3.67815223084484\\
    -3.25	3.63807643506226\\
    -3.2	3.59835315622033\\
    -3.15	3.55899655576197\\
    -3.1	3.52002130668921\\
    -3.05	3.48144259673413\\
    -3	3.4432761289903\\
    -2.95	3.40553811961646\\
    -2.9	3.36824529219294\\
    -2.85	3.3314148682816\\
    -2.8	3.29506455371237\\
    -2.75	3.25921252009453\\
    -2.7	3.22387738103143\\
    -2.65	3.18907816250322\\
    -2.6	3.15483426687613\\
    -2.55	3.12116543000122\\
    -2.5	3.08809167088059\\
    -2.45	3.05563323340941\\
    -2.4	3.0238105197477\\
    -2.35	2.99264401494038\\
    -2.3	2.96215420248884\\
    -2.25	2.93236147068346\\
    -2.2	2.90328600963702\\
    -2.15	2.87494769911158\\
    -2.1	2.8473659874088\\
    -2.05	2.82055976179205\\
    -2	2.79454721112724\\
    -1.95	2.76934568166313\\
    -1.9	2.74497152711609\\
    -1.85	2.72143995447251\\
    -1.8	2.69876486716525\\
    -1.75	2.67695870751057\\
    -1.7	2.65603230049749\\
    -1.65	2.63599470119279\\
    -1.6	2.61685304815087\\
    -1.55	2.59861242528775\\
    -1.5	2.58127573468549\\
    -1.45	2.56484358272872\\
    -1.4	2.54931418183629\\
    -1.35	2.53468326983641\\
    -1.3	2.52094404874626\\
    -1.25	2.50808714436297\\
    -1.2	2.49610058766228\\
    -1.15	2.48496981854691\\
    -1.1	2.47467771200501\\
    -1.05	2.46520462624676\\
    -1	2.45652847190347\\
    -0.95	2.44862480091532\\
    -0.9	2.44146691331931\\
    -0.85	2.43502597979047\\
    -0.8	2.42927117750053\\
    -0.75	2.42416983664444\\
    -0.7	2.41968759485215\\
    -0.65	2.41578855664932\\
    -0.6	2.41243545515343\\
    -0.55	2.40958981328368\\
    -0.5	2.40721210191468\\
    -0.45	2.40526189260486\\
    -0.4	2.40369800276695\\
    -0.35	2.40247863140937\\
    -0.3	2.40156148384975\\
    -0.25	2.40090388407496\\
    -0.2	2.40046287368554\\
    -0.15	2.40019529660758\\
    -0.1	2.40005786897502\\
    -0.0500000000000003	2.40000723377449\\
    -4.44089209850063e-16	2.4\\
    0.0499999999999998	2.3999927661819\\
    0.0999999999999996	2.39994212823416\\
    0.15	2.39980467160327\\
    0.2	2.39953694770219\\
    0.25	2.39909543457304\\
    0.3	2.39843648164217\\
    0.35	2.39751623831037\\
    0.4	2.39629056595804\\
    0.45	2.39471493273629\\
    0.5	2.39274429025737\\
    0.55	2.39033293098385\\
    0.6	2.38743432473783\\
    0.65	2.38400093229623\\
    0.7	2.37998399348979\\
    0.75	2.37533328656065\\
    0.8	2.3699968547259\\
    0.85	2.36392069490592\\
    0.9	2.3570484023544\\
    0.95	2.34932076340479\\
    1	2.3406752866345\\
    1.05	2.33104566032113\\
    1.1	2.32036112095663\\
    1.15	2.30854571356602\\
    1.2	2.29551741932687\\
    1.25	2.28118711905239\\
    1.3	2.26545735184385\\
    1.35	2.24822081570561\\
    1.4	2.22935853978708\\
    1.45	2.20873763413653\\
    1.5	2.18620848933655\\
    1.55	2.16160125036841\\
    1.6	2.13472131897756\\
    1.65	2.1053435344861\\
    1.7	2.07320452416294\\
    1.75	2.03799246630451\\
    1.8	1.99933311098757\\
    1.85	1.95677024308203\\
    1.9	1.90973763562705\\
    1.95	1.85751749979135\\
    2	1.79917657811017\\
    2.05	1.73346327510399\\
    2.1	1.65863244410366\\
    2.15	1.57212459328393\\
    2.2	1.4699193193105\\
    2.25	1.34504355335903\\
    2.3	1.18333443540641\\
    2.35	0.94582656794676\\
    2.4	0\\
    };
    \addplot [color=black, forget plot]
      table[row sep=crcr]{%
    2.4	-0\\
    2.45	-0.959054697368768\\
    2.5	-1.21666562415206\\
    2.55	-1.40227097406978\\
    2.6	-1.5538924023345\\
    2.65	-1.68517971004287\\
    2.7	-1.80277350207521\\
    2.75	-1.9104571148998\\
    2.8	-2.01061027812539\\
    2.85	-2.10483579455697\\
    2.9	-2.19426882937724\\
    2.95	-2.27974495710834\\
    3	-2.3618983098613\\
    3.05	-2.44122223322374\\
    3.1	-2.51810851596044\\
    3.15	-2.59287373022066\\
    3.2	-2.66577748131676\\
    3.25	-2.73703539441857\\
    3.3	-2.80682857120563\\
    3.35	-2.87531061546316\\
    3.4	-2.94261294536828\\
    3.45	-3.00884887348345\\
    3.5	-3.07411678426132\\
    3.55	-3.13850263982626\\
    3.6	-3.20208197846633\\
    3.65	-3.26492152494216\\
    3.7	-3.32708050017967\\
    3.75	-3.38861169560332\\
    3.8	-3.44956236134631\\
    3.85	-3.50997494591197\\
    3.9	-3.56988771626229\\
    3.95	-3.62933528089639\\
    4	-3.68834903364676\\
    4.05	-3.74695753223821\\
    4.1	-3.80518682282467\\
    4.15	-3.86306071952359\\
    4.2	-3.92060104625254\\
    4.25	-3.97782784682101\\
    4.3	-4.0347595681581\\
    4.35	-4.09141322070095\\
    4.4	-4.1478045192799\\
    4.45	-4.20394800728021\\
    4.5	-4.25985716640718\\
    4.55	-4.3155445140118\\
    4.6	-4.37102168962962\\
    4.65	-4.42629953213446\\
    4.7	-4.48138814870033\\
    4.75	-4.53629697659133\\
    4.8	-4.59103483865373\\
    4.85	-4.64560999326295\\
    4.9	-4.70003017937465\\
    4.95	-4.75430265724247\\
    5	-4.80843424529093\\
    };
    
    \addplot[area legend, draw=black, fill=black, forget plot]
    table[row sep=crcr] {%
    x	y\\
    2.03942871719321	1.77033380753047\\
    2.03942871719321	1.77033380753047\\
    2.03942871719321	1.77033380753047\\
    2.03942871719321	1.77033380753047\\
    2.03942871719321	1.77033380753047\\
    2.03942871719321	1.77033380753047\\
    2.03942871719321	1.77033380753046\\
    1.97217331677884	1.7030784071161\\
    1.80488126236184	2.00488126236184\\
    2.10668411760758	1.83758920794483\\
    2.03942871719321	1.77033380753046\\
    }--cycle;
    \addplot [color=black, forget plot]
      table[row sep=crcr]{%
    -5	5.27705758922345\\
    -4.95	5.23223778122896\\
    -4.9	5.18755653879443\\
    -4.85	5.14301847212166\\
    -4.8	5.09862836367258\\
    -4.75	5.05439117454833\\
    -4.7	5.01031205103314\\
    -4.65	4.96639633129637\\
    -4.6	4.92264955224409\\
    -4.55	4.87907745650921\\
    -4.5	4.83568599956711\\
    -4.45	4.79248135696076\\
    -4.4	4.74946993161649\\
    -4.35	4.7066583612282\\
    -4.3	4.66405352568401\\
    -4.25	4.62166255450535\\
    -4.2	4.57949283426403\\
    -4.15	4.53755201593769\\
    -4.1	4.49584802215894\\
    -4.05	4.45438905430713\\
    -4	4.41318359938584\\
    -3.95	4.37224043662185\\
    -3.9	4.33156864371396\\
    -3.85	4.29117760265231\\
    -3.8	4.25107700501994\\
    -3.75	4.21127685667977\\
    -3.7	4.17178748174028\\
    -3.65	4.13261952568365\\
    -3.6	4.09378395752964\\
    -3.55	4.05529207089804\\
    -3.5	4.01715548382202\\
    -3.45	3.97938613715373\\
    -3.4	3.94199629139342\\
    -3.35	3.90499852176285\\
    -3.3	3.86840571133459\\
    -3.25	3.83223104202003\\
    -3.2	3.79648798321173\\
    -3.15	3.76119027787001\\
    -3.1	3.7263519258398\\
    -3.05	3.69198716418307\\
    -3	3.65811044431324\\
    -2.95	3.62473640572403\\
    -2.9	3.59187984611387\\
    -2.85	3.55955568772126\\
    -2.8	3.52777893970564\\
    -2.75	3.49656465643253\\
    -2.7	3.46592789155308\\
    -2.65	3.43588364780516\\
    -2.6	3.40644682250726\\
    -2.55	3.37763214876757\\
    -2.5	3.34945413248832\\
    -2.45	3.32192698531027\\
    -2.4	3.29506455371237\\
    -2.35	3.26888024455833\\
    -2.3	3.24338694746137\\
    -2.25	3.21859695442227\\
    -2.2	3.19452187728006\\
    -2.15	3.17117256359859\\
    -2.1	3.14855901169279\\
    -2.05	3.12669028557426\\
    -2	3.10557443066252\\
    -1.95	3.08521839116532\\
    -1.9	3.06562793007403\\
    -1.85	3.04680755274759\\
    -1.8	3.02876043506725\\
    -1.75	3.01148835713308\\
    -1.7	2.99499164344107\\
    -1.65	2.97926911042469\\
    -1.6	2.96431802216825\\
    -1.55	2.95013405500169\\
    -1.5	2.93671127156856\\
    -1.45	2.9240421048243\\
    -1.4	2.91211735227267\\
    -1.35	2.90092618058876\\
    -1.3	2.89045614061086\\
    -1.25	2.88069319251583\\
    -1.2	2.87162174082733\\
    -1.15	2.86322467874888\\
    -1.1	2.85548344116701\\
    -1.05	2.8483780655392\\
    -1	2.84188725976877\\
    -0.95	2.83598847607753\\
    -0.9	2.83065798981818\\
    -0.85	2.82587098212319\\
    -0.8	2.82160162526491\\
    -0.75	2.81782316960242\\
    -0.7	2.81450803101234\\
    -0.65	2.81162787774153\\
    -0.6	2.80915371567669\\
    -0.55	2.80705597109652\\
    -0.5	2.80530457005292\\
    -0.45	2.80386901361591\\
    -0.4	2.80271844830938\\
    -0.35	2.80182173115804\\
    -0.3	2.80114748885848\\
    -0.25	2.80066417067502\\
    -0.2	2.80034009474402\\
    -0.15	2.80014348754474\\
    -0.1	2.80004251636121\\
    -0.0499999999999998	2.80000531461576\\
    0	2.8\\
    0.0500000000000003	2.79999468536406\\
    0.1	2.79995748234758\\
    0.15	2.79985649774756\\
    0.2	2.79965982261845\\
    0.25	2.79933551408778\\
    0.3	2.79885156984793\\
    0.35	2.79817589524919\\
    0.4	2.79727626287235\\
    0.45	2.79612026439937\\
    0.5	2.7946752545277\\
    0.55	2.79290828658616\\
    0.6	2.79078603940697\\
    0.65	2.78827473488684\\
    0.7	2.78534004552747\\
    0.75	2.78194699107888\\
    0.8	2.77805982321302\\
    0.85	2.77364189692434\\
    0.9	2.7686555270807\\
    0.95	2.76306182822314\\
    1	2.75682053532422\\
    1.05	2.7498898027468\\
    1.1	2.74222597807768\\
    1.15	2.7337833468196\\
    1.2	2.72451384307563\\
    1.25	2.71436672031028\\
    1.3	2.70328817496487\\
    1.35	2.69122091406451\\
    1.4	2.67810365588134\\
    1.45	2.66387055007701\\
    1.5	2.64845050035241\\
    1.55	2.63176636823053\\
    1.6	2.61373403083503\\
    1.65	2.59426125790634\\
    1.7	2.57324636310784\\
    1.75	2.55057657089265\\
    1.8	2.52612602132067\\
    1.85	2.49975330899155\\
    1.9	2.47129841528428\\
    1.95	2.44057884009659\\
    2	2.40738466192258\\
    2.05	2.37147213991386\\
    2.1	2.33255529615217\\
    2.15	2.29029464245921\\
    2.2	2.2442817761255\\
    2.25	2.19401783917641\\
    2.3	2.13888257964319\\
    2.35	2.07808849597626\\
    2.4	2.01061027812539\\
    2.45	1.93507118478761\\
    2.5	1.84954943982631\\
    2.55	1.75122363400783\\
    2.6	1.63565775765116\\
    2.65	1.49515100841119\\
    2.7	1.31404881408406\\
    2.75	1.04924359733213\\
    2.8	0\\
    };
    \addplot [color=black, forget plot]
      table[row sep=crcr]{%
    2.8	-0\\
    2.85	-1.06180923959634\\
    2.9	-1.34571112600068\\
    2.95	-1.54951424251114\\
    3	-1.71543045234635\\
    3.05	-1.85862782751761\\
    3.1	-1.98649230986883\\
    3.15	-2.10323575242108\\
    3.2	-2.21150992547155\\
    3.25	-2.31310259436156\\
    3.3	-2.40928119148166\\
    3.35	-2.50097961609079\\
    3.4	-2.58890737691659\\
    3.45	-2.6736170683825\\
    3.5	-2.75554802817152\\
    3.55	-2.83505566177586\\
    3.6	-2.91243176788214\\
    3.65	-2.98791900777284\\
    3.7	-3.06172144585091\\
    3.75	-3.13401238363513\\
    3.8	-3.20494028573448\\
    3.85	-3.27463333307323\\
    3.9	-3.34320297045916\\
    3.95	-3.41074670541341\\
    4	-3.47735034137574\\
    4.05	-3.54308977795475\\
    4.1	-3.60803247578532\\
    4.15	-3.67223865871609\\
    4.2	-3.73576230821072\\
    4.25	-3.79865199185627\\
    4.3	-3.86095155829406\\
    4.35	-3.92270072374216\\
    4.4	-3.98393556988968\\
    4.45	-4.04468896883863\\
    4.5	-4.1049909476128\\
    4.55	-4.164869002306\\
    4.6	-4.22434837002808\\
    4.65	-4.28345226529961\\
    4.7	-4.34220208634965\\
    4.75	-4.40061759581542\\
    4.8	-4.45871707957417\\
    4.85	-4.51651748681601\\
    4.9	-4.5740345539613\\
    4.95	-4.63128291461227\\
    5	-4.68827619738926\\
    };
    
    \addplot[area legend, draw=black, fill=black, forget plot]
    table[row sep=crcr] {%
    x	y\\
    2.35895698338541	2.08576596212555\\
    2.35895698338541	2.08576596212555\\
    2.35895698338541	2.08576596212555\\
    2.35895698338541	2.08576596212555\\
    2.35895698338541	2.08576596212555\\
    2.35895698338541	2.08576596212555\\
    2.35895698338541	2.08576596212555\\
    2.29111431242103	2.01792329116116\\
    2.12236147275548	2.32236147275548\\
    2.42679965434979	2.15360863308993\\
    2.35895698338541	2.08576596212555\\
    }--cycle;
    \addplot [color=black, forget plot]
      table[row sep=crcr]{%
    -5	5.40347284608396\\
    -4.95	5.36075079285757\\
    -4.9	5.31821194173362\\
    -4.85	5.27586181197868\\
    -4.8	5.23370609630174\\
    -4.75	5.1917506651576\\
    -4.7	5.1500015709927\\
    -4.65	5.10846505241062\\
    -4.6	5.06714753823154\\
    -4.55	5.0260556514178\\
    -4.5	4.98519621283419\\
    -4.45	4.94457624480922\\
    -4.4	4.90420297445978\\
    -4.35	4.86408383673865\\
    -4.3	4.82422647716039\\
    -4.25	4.78463875415766\\
    -4.2	4.74532874101596\\
    -4.15	4.70630472733087\\
    -4.1	4.6675752199277\\
    -4.05	4.6291489431793\\
    -4	4.59103483865373\\
    -3.95	4.55324206401897\\
    -3.9	4.51577999112813\\
    -3.85	4.47865820320454\\
    -3.8	4.44188649104214\\
    -3.75	4.40547484813357\\
    -3.7	4.36943346463502\\
    -3.65	4.33377272007456\\
    -3.6	4.29850317470858\\
    -3.55	4.26363555943029\\
    -3.5	4.22918076413343\\
    -3.45	4.19514982443573\\
    -3.4	4.16155390666829\\
    -3.35	4.12840429104089\\
    -3.3	4.09571235289781\\
    -3.25	4.0634895419857\\
    -3.2	4.03174735966359\\
    -3.15	4.00049733399549\\
    -3.1	3.96975099267893\\
    -3.05	3.93951983377774\\
    -3	3.90981529424461\\
    -2.95	3.88064871623874\\
    -2.9	3.85203131126585\\
    -2.85	3.82397412219231\\
    -2.8	3.79648798321173\\
    -2.75	3.76958347787087\\
    -2.7	3.74327089529209\\
    -2.65	3.71756018476147\\
    -2.6	3.69246090888418\\
    -2.55	3.66798219554225\\
    -2.5	3.64413268892273\\
    -2.45	3.62092049991666\\
    -2.4	3.59835315622033\\
    -2.35	3.5764375524985\\
    -2.3	3.55517990099529\\
    -2.25	3.53458568299978\\
    -2.2	3.51465960159039\\
    -2.15	3.49540553609387\\
    -2.1	3.47682649869993\\
    -2.05	3.45892459367131\\
    -2	3.44170097958066\\
    -1.95	3.42515583498954\\
    -1.9	3.40928832796128\\
    -1.85	3.3940965897682\\
    -1.8	3.37957769311521\\
    -1.75	3.36572763515658\\
    -1.7	3.35254132553103\\
    -1.65	3.34001257958378\\
    -1.6	3.32813411688305\\
    -1.55	3.31689756507478\\
    -1.5	3.3062934690535\\
    -1.45	3.29631130536136\\
    -1.4	3.28693950166235\\
    -1.35	3.27816546107627\\
    -1.3	3.269975591098\\
    -1.25	3.2623553367739\\
    -1.2	3.25528921775908\\
    -1.15	3.24876086883799\\
    -1.1	3.24275308345714\\
    -1.05	3.23724785979289\\
    -1	3.23222644885926\\
    -0.95	3.22766940415125\\
    -0.9	3.22355663231713\\
    -0.85	3.21986744435871\\
    -0.8	3.21658060687125\\
    -0.75	3.21367439285311\\
    -0.7	3.21112663163925\\
    -0.65	3.20891475754098\\
    -0.6	3.20701585680648\\
    -0.55	3.20540671255072\\
    -0.5	3.20406384733972\\
    -0.45	3.20296356315068\\
    -0.4	3.20208197846633\\
    -0.35	3.20139506229741\\
    -0.3	3.20087866496163\\
    -0.25	3.20050854547952\\
    -0.2	3.20026039547678\\
    -0.15	3.20010985950961\\
    -0.1	3.2000325517522\\
    -0.0499999999999998	3.20000406900524\\
    0	3.2\\
    0.0499999999999998	3.19999593098441\\
    0.1	3.19996744758552\\
    0.15	3.19989013294668\\
    0.2	3.1997395621377\\
    0.25	3.19949129283252\\
    0.3	3.19912085224062\\
    0.35	3.1986037202666\\
    0.4	3.19791530885622\\
    0.45	3.19703093746689\\
    0.5	3.1959258045759\\
    0.55	3.19457495511033\\
    0.6	3.19295324364839\\
    0.65	3.19103529320198\\
    0.7	3.18879544934471\\
    0.75	3.18620772939668\\
    0.8	3.18324576631711\\
    0.85	3.17988274688643\\
    0.9	3.17609134367924\\
    0.95	3.17184364023773\\
    1	3.16711104874753\\
    1.05	3.16186421939409\\
    1.1	3.15607294043183\\
    1.15	3.14970602782766\\
    1.2	3.1427312031392\\
    1.25	3.13511495804882\\
    1.3	3.12682240369041\\
    1.35	3.11781710256465\\
    1.4	3.10806088042818\\
    1.45	3.09751361504498\\
    1.5	3.08613299808309\\
    1.55	3.07387426569862\\
    1.6	3.06068989243582\\
    1.65	3.04652924193967\\
    1.7	3.03133816656442\\
    1.75	3.01505854618548\\
    1.8	2.99762775427414\\
    1.85	2.97897803642651\\
    1.9	2.95903578284902\\
    1.95	2.93772067151641\\
    2	2.91494465244874\\
    2.05	2.89061073525953\\
    2.1	2.86461153103556\\
    2.15	2.83682748460024\\
    2.2	2.80712471264813\\
    2.25	2.77535233466951\\
    2.3	2.74133914329627\\
    2.35	2.70488940294455\\
    2.4	2.66577748131676\\
    2.45	2.62374089275194\\
    2.5	2.57847114115864\\
    2.55	2.52960145159073\\
    2.6	2.47668999972181\\
    2.65	2.41919645247475\\
    2.7	2.35644826269705\\
    2.75	2.28759069782819\\
    2.8	2.21150992547155\\
    2.85	2.12670912096187\\
    2.9	2.03109730607616\\
    2.95	1.92160248521068\\
    3	1.7933914044787\\
    3.05	1.63806819432795\\
    3.1	1.43855436194468\\
    3.15	1.14779033794501\\
    3.2	0\\
    };
    \addplot [color=black, forget plot]
      table[row sep=crcr]{%
    3.2	-0\\
    3.25	-1.15980897567894\\
    3.3	-1.46883860941191\\
    3.35	-1.69006605486246\\
    3.4	-1.86969463209304\\
    3.45	-2.02433804508015\\
    3.5	-2.16209159538837\\
    3.55	-2.28757477335282\\
    3.6	-2.40369800276695\\
    3.65	-2.51242481396752\\
    3.7	-2.61514825740898\\
    3.75	-2.71289555841655\\
    3.8	-2.80644772607596\\
    3.85	-2.89641352118544\\
    3.9	-2.983277327649\\
    3.95	-3.0674313166234\\
    4	-3.14919774648174\\
    4.05	-3.2288448424795\\
    4.1	-3.30659836798173\\
    4.15	-3.38265022702969\\
    4.2	-3.45716497362754\\
    4.25	-3.53028481465819\\
    4.3	-3.60213350900703\\
    4.35	-3.67281944469957\\
    4.4	-3.74243809494084\\
    4.45	-3.81107399863058\\
    4.5	-3.87880237242552\\
    4.55	-3.94569043417403\\
    4.6	-4.01179849797794\\
    4.65	-4.07718088688333\\
    4.7	-4.14188669869105\\
    4.75	-4.20596045253427\\
    4.8	-4.26944263795511\\
    4.85	-4.33237018370575\\
    4.9	-4.39477686003393\\
    4.95	-4.45669362552509\\
    5	-4.51814892747109\\
    };
    
    \addplot[area legend, draw=black, fill=black, forget plot]
    table[row sep=crcr] {%
    x	y\\
    2.67904861900171	2.40063474729653\\
    2.67904861900171	2.40063474729653\\
    2.67904861900171	2.40063474729653\\
    2.67904861900171	2.40063474729653\\
    2.67904861900171	2.40063474729653\\
    2.67904861900171	2.40063474729653\\
    2.67904861900171	2.40063474729653\\
    2.61045713390448	2.3320432621993\\
    2.43984168314912	2.63984168314912\\
    2.74764010409894	2.46922623239376\\
    2.67904861900171	2.40063474729653\\
    }--cycle;
    \addplot [color=black, forget plot]
      table[row sep=crcr]{%
    -5	5.557587760239\\
    -4.95	5.51722834626726\\
    -4.9	5.47709482469102\\
    -4.85	5.43719322706679\\
    -4.8	5.3975297343305\\
    -4.75	5.35811067805309\\
    -4.7	5.31894254143834\\
    -4.65	5.28003196003382\\
    -4.6	5.24138572212416\\
    -4.55	5.20301076877394\\
    -4.5	5.16491419348545\\
    -4.45	5.12710324143526\\
    -4.4	5.0895853082511\\
    -4.35	5.05236793828941\\
    -4.3	5.01545882237184\\
    -4.25	4.97886579493782\\
    -4.2	4.94259683056856\\
    -4.15	4.9066600398371\\
    -4.1	4.87106366443767\\
    -4.05	4.83581607154715\\
    -4	4.80092574737107\\
    -3.95	4.76640128982643\\
    -3.9	4.73225140031419\\
    -3.85	4.69848487453516\\
    -3.8	4.66511059230433\\
    -3.75	4.63213750632089\\
    -3.7	4.59957462985363\\
    -3.65	4.56743102330508\\
    -3.6	4.53571577962154\\
    -3.55	4.50443800852139\\
    -3.5	4.47360681951958\\
    -3.45	4.44323130373325\\
    -3.4	4.41332051446052\\
    -3.35	4.38388344653348\\
    -3.3	4.35492901445552\\
    -3.25	4.32646602934382\\
    -3.2	4.29850317470858\\
    -3.15	4.2710489811132\\
    -3.1	4.24411179977208\\
    -3.05	4.21769977515622\\
    -3	4.19182081669086\\
    -2.95	4.16648256964389\\
    -2.9	4.14169238531793\\
    -2.85	4.11745729067414\\
    -2.8	4.09378395752964\\
    -2.75	4.07067867148454\\
    -2.7	4.04814730074787\\
    -2.65	4.0261952650436\\
    -2.6	4.00482750478888\\
    -2.55	3.98404845074623\\
    -2.5	3.96386199435843\\
    -2.45	3.9442714589805\\
    -2.4	3.92527957222631\\
    -2.35	3.90688843964739\\
    -2.3	3.88909951995964\\
    -2.25	3.87191360202824\\
    -2.2	3.85533078381304\\
    -2.15	3.8393504534655\\
    -2.1	3.82397127275444\\
    -2.05	3.80919116298053\\
    -2	3.79500729352051\\
    -1.95	3.78141607311939\\
    -1.9	3.76841314402527\\
    -1.85	3.75599337903529\\
    -1.8	3.74415088149343\\
    -1.75	3.73287898825314\\
    -1.7	3.72217027558834\\
    -1.65	3.71201656800751\\
    -1.6	3.70240894989728\\
    -1.55	3.69333777989389\\
    -1.5	3.6847927078552\\
    -1.45	3.67676269428093\\
    -1.4	3.66923603200685\\
    -1.35	3.66220036997897\\
    -1.3	3.65564273889617\\
    -1.25	3.64954957849654\\
    -1.2	3.6439067662508\\
    -1.15	3.63869964721907\\
    -1.1	3.63391306482233\\
    -1.05	3.62953139227823\\
    -1	3.62553856445265\\
    -0.95	3.62191810988224\\
    -0.899999999999999	3.61865318273015\\
    -0.85	3.61572659444601\\
    -0.8	3.61312084491233\\
    -0.75	3.61081815287203\\
    -0.7	3.60880048544596\\
    -0.65	3.60704958656418\\
    -0.6	3.60554700415042\\
    -0.55	3.60427411591559\\
    -0.5	3.60321215363176\\
    -0.45	3.60234222577463\\
    -0.4	3.60164533843748\\
    -0.35	3.60110241443494\\
    -0.3	3.60069431052831\\
    -0.25	3.6004018327177\\
    -0.2	3.60020574955752\\
    -0.15	3.60008680346253\\
    -0.1	3.60002571998085\\
    -0.0500000000000003	3.6000032150177\\
    0	3.6\\
    0.0499999999999998	3.59999678497655\\
    0.1	3.59997427965163\\
    0.15	3.59991319235125\\
    0.2	3.59979422692153\\
    0.25	3.59959807755697\\
    0.3	3.59930542155328\\
    0.35	3.5988969099748\\
    0.4	3.59835315622033\\
    0.45	3.59765472246325\\
    0.5	3.59678210393225\\
    0.55	3.5957157109879\\
    0.6	3.59443584893706\\
    0.65	3.5929226955121\\
    0.7	3.59115627592492\\
    0.75	3.58911643538605\\
    0.8	3.58678280895741\\
    0.85	3.5841347885821\\
    0.9	3.58115148710675\\
    0.95	3.57781169907985\\
    1	3.57409385807344\\
    1.05	3.56997599023468\\
    1.1	3.56543566372656\\
    1.15	3.56044993366385\\
    1.2	3.55499528208884\\
    1.25	3.54904755246047\\
    1.3	3.54258187804877\\
    1.35	3.5355726035316\\
    1.4	3.52799319897976\\
    1.45	3.51981616528734\\
    1.5	3.51101292995174\\
    1.55	3.50155373192867\\
    1.6	3.49140749407442\\
    1.65	3.48054168143494\\
    1.7	3.46892214333853\\
    1.75	3.45651293688558\\
    1.8	3.4432761289903\\
    1.85	3.42917157359721\\
    1.9	3.41415666004636\\
    1.95	3.39818602776578\\
    2	3.38121124148806\\
    2.05	3.36318041997095\\
    2.1	3.34403780968063\\
    2.15	3.32372329298502\\
    2.2	3.30217181798291\\
    2.25	3.27931273400483\\
    2.3	3.25506901284616\\
    2.35	3.22935633063355\\
    2.4	3.20208197846633\\
    2.45	3.17314356103136\\
    2.5	3.14242743042701\\
    2.55	3.10980678624441\\
    2.6	3.07513935076953\\
    2.65	3.03826449734771\\
    2.7	2.99899966648136\\
    2.75	2.95713584190443\\
    2.8	2.91243176788214\\
    2.85	2.86460645344058\\
    2.9	2.81332930277019\\
    2.95	2.75820688858728\\
    3	2.69876486716525\\
    3.05	2.63442267416152\\
    3.1	2.56445716036143\\
    3.15	2.48794866615549\\
    3.2	2.40369800276695\\
    3.25	2.31009269819094\\
    3.3	2.20487897896575\\
    3.35	2.08474393357267\\
    3.4	1.94447262241857\\
    3.45	1.77500165310962\\
    3.5	1.5578855860124\\
    3.55	1.24227232487739\\
    3.6	0\\
    };
    \addplot [color=black, forget plot]
      table[row sep=crcr]{%
    3.6	-0\\
    3.65	-1.25382826354938\\
    3.7	-1.58700410245128\\
    3.75	-1.8249984362281\\
    3.8	-2.01784038710825\\
    3.85	-2.18352884962524\\
    3.9	-2.33083860350175\\
    3.95	-2.46477946918048\\
    4	-2.58850945078415\\
    4.05	-2.70416025311281\\
    4.1	-2.81324507009292\\
    4.15	-2.91688034715677\\
    4.2	-3.01591541718809\\
    4.25	-3.11101268703176\\
    4.3	-3.20269954491915\\
    4.35	-3.29140324406586\\
    4.4	-3.37747509342703\\
    4.45	-3.46120768760821\\
    4.5	-3.54284746479196\\
    4.55	-3.62260404495532\\
    4.6	-3.70065729739648\\
    4.65	-3.77716277394065\\
    4.7	-3.85225594439255\\
    4.75	-3.92605553987667\\
    4.8	-3.99866622197514\\
    4.85	-4.07018073559454\\
    4.9	-4.14068166173797\\
    4.95	-4.21024285680844\\
    5	-4.27893064384067\\
    };
    
    \addplot[area legend, draw=black, fill=black, forget plot]
    table[row sep=crcr] {%
    x	y\\
    2.99971246073864	2.71493132634688\\
    2.99971246073864	2.71493132634688\\
    2.99971246073864	2.71493132634688\\
    2.99971246073864	2.71493132634688\\
    2.99971246073864	2.71493132634688\\
    2.99971246073864	2.71493132634688\\
    2.99971246073864	2.71493132634688\\
    2.93020808404376	2.645426949652\\
    2.75732189354276	2.95732189354276\\
    3.06921683743352	2.78443570304176\\
    2.99971246073864	2.71493132634688\\
    }--cycle;
    \addplot [color=black, forget plot]
      table[row sep=crcr]{%
    -5	5.73879354831717\\
    -4.95	5.70096808221258\\
    -4.9	5.66340565930516\\
    -4.85	5.6261123726007\\
    -4.8	5.58909442225448\\
    -4.75	5.55235811380267\\
    -4.7	5.51590985603062\\
    -4.65	5.47975615845403\\
    -4.6	5.44390362838836\\
    -4.55	5.4083589675817\\
    -4.5	5.37312896838572\\
    -4.45	5.33822050943947\\
    -4.4	5.30364055084068\\
    -4.35	5.26939612877947\\
    -4.3	5.23549434960994\\
    -4.25	5.20194238333536\\
    -4.2	5.16874745648424\\
    -4.15	5.13591684435504\\
    -4.1	5.1034578626093\\
    -4.05	5.07137785819438\\
    -4	5.03968419957949\\
    -3.95	5.00838426629069\\
    -3.9	4.97748543773394\\
    -3.85	4.94699508129806\\
    -3.8	4.9169205397334\\
    -3.75	4.88726911780576\\
    -3.7	4.85804806822989\\
    -3.65	4.82926457689136\\
    -3.6	4.80092574737107\\
    -3.55	4.77303858479212\\
    -3.5	4.74560997901466\\
    -3.45	4.71864668721076\\
    -3.4	4.69215531585751\\
    -3.35	4.66614230219372\\
    -3.3	4.64061389519198\\
    -3.25	4.61557613610522\\
    -3.2	4.59103483865373\\
    -3.15	4.5669955689254\\
    -3.1	4.54346362506894\\
    -3.05	4.52044401686613\\
    -3	4.49794144527541\\
    -2.95	4.47596028204469\\
    -2.9	4.45450454949628\\
    -2.85	4.43357790059152\\
    -2.8	4.41318359938584\\
    -2.75	4.39332450198799\\
    -2.7	4.3740030381387\\
    -2.65	4.3552211935248\\
    -2.6	4.33698049294394\\
    -2.55	4.31928198443375\\
    -2.5	4.30212622447582\\
    -2.45	4.28551326438077\\
    -2.4	4.26944263795511\\
    -2.35	4.25391335054368\\
    -2.3	4.23892386953327\\
    -2.25	4.22447211639401\\
    -2.2	4.21055546032458\\
    -2.15	4.19717071355611\\
    -2.1	4.18431412835758\\
    -2.05	4.17198139577253\\
    -2	4.16016764610381\\
    -1.95	4.14886745114917\\
    -1.9	4.13807482817678\\
    -1.85	4.12778324561581\\
    -1.8	4.11798563042365\\
    -1.75	4.10867437707794\\
    -1.7	4.09984135812903\\
    -1.65	4.09147793623635\\
    -1.6	4.08357497760117\\
    -1.55	4.07612286669812\\
    -1.5	4.06911152219885\\
    -1.45	4.06253041397349\\
    -1.4	4.05636858104921\\
    -1.35	4.05061465039998\\
    -1.3	4.04525685643787\\
    -1.25	4.04028306107407\\
    -1.2	4.03568077421651\\
    -1.15	4.0314371745714\\
    -1.1	4.02753913061748\\
    -1.05	4.02397322162419\\
    -1	4.02072575858906\\
    -0.95	4.01778280497382\\
    -0.899999999999999	4.01513019712441\\
    -0.85	4.01275356426607\\
    -0.8	4.01063834797167\\
    -0.75	4.0087698210081\\
    -0.7	4.00713310547364\\
    -0.649999999999999	4.00571319014622\\
    -0.6	4.00449494697088\\
    -0.55	4.00346314662189\\
    -0.5	4.00260247308292\\
    -0.45	4.00189753719581\\
    -0.4	4.00133288913564\\
    -0.350000000000001	4.00089302977628\\
    -0.3	4.00056242091697\\
    -0.25	4.00032549434597\\
    -0.2	4.0001666597227\\
    -0.15	4.00007031126407\\
    -0.100000000000001	4.00002083322483\\
    -0.0499999999999998	4.00000260416497\\
    0	4\\
    0.0499999999999998	3.99999739583164\\
    0.0999999999999996	3.99997916655816\\
    0.15	3.999929686264\\
    0.2	3.99983332638841\\
    0.25	3.99967445267212\\
    0.3	3.99943742087989\\
    0.35	3.99910657129448\\
    0.4	3.99866622197514\\
    0.45	3.99810066077037\\
    0.5	3.99739413607028\\
    0.55	3.9965308462796\\
    0.6	3.9954949279864\\
    0.65	3.99427044279541\\
    0.7	3.99284136278774\\
    0.75	3.99119155456048\\
    0.8	3.98930476179092\\
    0.85	3.9871645862595\\
    0.9	3.98475446725496\\
    0.95	3.98205765927177\\
    1	3.97905720789639\\
    1.05	3.97573592376282\\
    1.1	3.97207635444015\\
    1.15	3.96806075409521\\
    1.2	3.96367105075071\\
    1.25	3.95888881093441\\
    1.3	3.95369520148593\\
    1.35	3.94807094825566\\
    1.4	3.94199629139342\\
    1.45	3.93545093688214\\
    1.5	3.928414003924\\
    1.55	3.92086396773085\\
    1.6	3.912778597207\\
    1.65	3.90413488693867\\
    1.7	3.89490898281869\\
    1.75	3.88507610053522\\
    1.8	3.87461043603686\\
    1.85	3.8634850669495\\
    1.9	3.85167184375929\\
    1.95	3.839141269386\\
    2	3.82586236554478\\
    2.05	3.81180252402532\\
    2.1	3.79692734069548\\
    2.15	3.78120042964864\\
    2.2	3.7645832144467\\
    2.25	3.74703469284268\\
    2.3	3.72851117067382\\
    2.35	3.70896595976667\\
    2.4	3.68834903364676\\
    2.45	3.66660663354419\\
    2.5	3.64368081556092\\
    2.55	3.61950892782091\\
    2.6	3.59402300383506\\
    2.65	3.56714905500759\\
    2.7	3.53880624095771\\
    2.75	3.50890589081016\\
    2.8	3.47735034137574\\
    2.85	3.4440315485722\\
    2.9	3.40882941563296\\
    2.95	3.37160976432446\\
    3	3.33222185164595\\
    3.05	3.2904953014888\\
    3.1	3.24623627419026\\
    3.15	3.19922263018076\\
    3.2	3.14919774648174\\
    3.25	3.09586249965226\\
    3.3	3.03886470765518\\
    3.35	2.97778497771623\\
    3.4	2.91211735227267\\
    3.45	2.84124222420702\\
    3.5	2.76438740683944\\
    3.55	2.68057039355508\\
    3.6	2.58850945078415\\
    3.65	2.48648035278273\\
    3.7	2.37207210759166\\
    3.75	2.24173925559838\\
    3.8	2.08989857798625\\
    3.85	1.90684282377084\\
    3.9	1.67280845040534\\
    3.95	1.33328732480132\\
    4	0\\
    };
    \addplot [color=black, forget plot]
      table[row sep=crcr]{%
    4	-0\\
    4.05	-1.34444447606544\\
    4.1	-1.70092222164965\\
    4.15	-1.95511477752707\\
    4.2	-2.16073591750959\\
    4.25	-2.3371182901163\\
    4.3	-2.49369075744464\\
    4.35	-2.63583878230465\\
    4.4	-2.76695856664369\\
    4.45	-2.88934356885749\\
    4.5	-3.00462250345868\\
    4.55	-3.11399757505469\\
    4.6	-3.2183837676756\\
    4.65	-3.31849501754868\\
    4.7	-3.41490000543241\\
    4.75	-3.50805965759495\\
    4.8	-3.59835315622033\\
    4.85	-3.68609646863451\\
    4.9	-3.77155585461804\\
    4.95	-3.85495791238075\\
    5	-3.93649718310217\\
    };
    
    \addplot[area legend, draw=black, fill=black, forget plot]
    table[row sep=crcr] {%
    x	y\\
    3.32093593264383	3.02866827522897\\
    3.32093593264383	3.02866827522896\\
    3.32093593264383	3.02866827522896\\
    3.32093593264383	3.02866827522896\\
    3.32093593264383	3.02866827522896\\
    3.32093593264383	3.02866827522896\\
    3.32093593264383	3.02866827522897\\
    3.25035819298282	2.95809053556796\\
    3.0748021039364	3.2748021039364\\
    3.39151367230484	3.09924601488997\\
    3.32093593264383	3.02866827522897\\
    }--cycle;
    \addplot [color=black, forget plot]
      table[row sep=crcr]{%
    -5	5.94565744858888\\
    -4.95	5.91044186522429\\
    -4.9	5.87551859010587\\
    -4.85	5.84089337250893\\
    -4.8	5.80657201927403\\
    -4.75	5.77256039081165\\
    -4.7	5.73886439674557\\
    -4.65	5.70548999118167\\
    -4.6	5.67244316758968\\
    -4.55	5.63972995328585\\
    -4.5	5.6073564035055\\
    -4.45	5.57532859505547\\
    -4.4	5.54365261953744\\
    -4.35	5.51233457613462\\
    -4.3	5.48138056395562\\
    -4.25	5.45079667393119\\
    -4.2	5.42058898026123\\
    -4.15	5.39076353141155\\
    -4.1	5.36132634066212\\
    -4.05	5.332283376211\\
    -4	5.30364055084068\\
    -3.95	5.27540371115633\\
    -3.9	5.24757862640856\\
    -3.85	5.22017097691613\\
    -3.8	5.19318634210752\\
    -3.75	5.16663018820354\\
    -3.7	5.14050785556663\\
    -3.65	5.11482454574594\\
    -3.6	5.0895853082511\\
    -3.55	5.06479502709083\\
    -3.5	5.04045840711649\\
    -3.45	5.01657996021387\\
    -3.4	4.99316399138997\\
    -3.35	4.97021458480507\\
    -3.3	4.94773558980296\\
    -3.25	4.92573060699547\\
    -3.2	4.90420297445978\\
    -3.15	4.88315575410921\\
    -3.1	4.86259171830001\\
    -3.05	4.84251333673835\\
    -3	4.82292276375219\\
    -2.95	4.80382182599364\\
    -2.9	4.78521201063719\\
    -2.85	4.76709445413835\\
    -2.8	4.74946993161649\\
    -2.75	4.73233884692341\\
    -2.7	4.71570122345719\\
    -2.65	4.69955669577749\\
    -2.6	4.68390450207526\\
    -2.55	4.66874347754531\\
    -2.5	4.65407204870588\\
    -2.45	4.63988822870376\\
    -2.4	4.62618961363815\\
    -2.35	4.61297337993023\\
    -2.3	4.60023628275897\\
    -2.25	4.58797465557711\\
    -2.2	4.57618441071419\\
    -2.15	4.56486104106651\\
    -2.1	4.55399962286679\\
    -2.05	4.54359481951908\\
    -2	4.53364088647765\\
    -1.95	4.52413167714154\\
    -1.9	4.51506064972999\\
    -1.85	4.50642087509766\\
    -1.8	4.49820504544276\\
    -1.75	4.49040548385564\\
    -1.7	4.4830141546505\\
    -1.65	4.47602267441874\\
    -1.6	4.46942232373823\\
    -1.55	4.4632040594701\\
    -1.5	4.45735852757163\\
    -1.45	4.45187607635236\\
    -1.4	4.44674677009875\\
    -1.35	4.44196040299268\\
    -1.3	4.43750651324839\\
    -1.25	4.43337439739352\\
    -1.2	4.42955312462073\\
    -1.15	4.426031551138\\
    -1.1	4.42279833444796\\
    -1.05	4.41984194748881\\
    -1	4.41715069257253\\
    -0.95	4.41471271505911\\
    -0.899999999999999	4.41251601670898\\
    -0.85	4.41054846865957\\
    -0.8	4.40879782397564\\
    -0.75	4.40725172972693\\
    -0.7	4.40589773855084\\
    -0.649999999999999	4.40472331966154\\
    -0.6	4.40371586927097\\
    -0.55	4.40286272039121\\
    -0.5	4.40215115199107\\
    -0.45	4.40156839748363\\
    -0.399999999999999	4.40110165252475\\
    -0.35	4.40073808210567\\
    -0.3	4.40046482692586\\
    -0.25	4.40026900903497\\
    -0.2	4.4001377367351\\
    -0.149999999999999	4.40005810873671\\
    -0.0999999999999996	4.40001721756348\\
    -0.0499999999999998	4.4000021522028\\
    0	4.4\\
    0.0499999999999998	4.39999784779509\\
    0.100000000000001	4.39998278230177\\
    0.15	4.39994188972842\\
    0.2	4.39986225464099\\
    0.25	4.39973095806743\\
    0.3	4.39953507484244\\
    0.350000000000001	4.39926167019068\\
    0.4	4.39889779554507\\
    0.45	4.39843048359551\\
    0.5	4.39784674256115\\
    0.55	4.39713354967731\\
    0.6	4.39627784388566\\
    0.65	4.39526651771311\\
    0.7	4.39408640832168\\
    0.75	4.39272428770812\\
    0.8	4.39116685202757\\
    0.85	4.38940071001136\\
    0.9	4.38741237044377\\
    0.95	4.38518822865711\\
    1	4.3827145519982\\
    1.05	4.37997746421271\\
    1.1	4.37696292868603\\
    1.15	4.37365673047123\\
    1.2	4.37004445702512\\
    1.25	4.36611147756342\\
    1.3	4.3618429209344\\
    1.35	4.35722365189759\\
    1.4	4.35223824567986\\
    1.45	4.34687096066515\\
    1.5	4.34110570905614\\
    1.55	4.33492602532579\\
    1.6	4.32831503225368\\
    1.65	4.3212554043164\\
    1.7	4.31372932817138\\
    1.75	4.30571845994028\\
    1.8	4.29720387895949\\
    1.85	4.28816603762167\\
    1.9	4.27858470688149\\
    1.95	4.26843891694111\\
    2	4.2577068925634\\
    2.05	4.24636598238375\\
    2.1	4.23439258150042\\
    2.15	4.22176204651851\\
    2.2	4.20844860209926\\
    2.25	4.19442523792143\\
    2.3	4.17966359479151\\
    2.35	4.16413383843723\\
    2.4	4.1478045192799\\
    2.45	4.1306424161952\\
    2.5	4.11261236193034\\
    2.55	4.09367704743449\\
    2.6	4.07379680186249\\
    2.65	4.05292934440889\\
    2.7	4.03102950339355\\
    2.75	4.00804889711701\\
    2.8	3.98393556988968\\
    2.85	3.95863357525606\\
    2.9	3.93208249670732\\
    2.95	3.90421689400247\\
    3	3.87496566046572\\
    3.05	3.84425127311386\\
    3.1	3.81198891294557\\
    3.15	3.77808542685497\\
    3.2	3.74243809494084\\
    3.25	3.70493315680415\\
    3.3	3.66544403681055\\
    3.35	3.62382918986462\\
    3.4	3.57992946398274\\
    3.45	3.53356484084942\\
    3.5	3.48453036602697\\
    3.55	3.4325910094841\\
    3.6	3.37747509342703\\
    3.65	3.3188657699515\\
    3.7	3.25638979571086\\
    3.75	3.18960248319805\\
    3.8	3.11796711754744\\
    3.85	3.04082614752338\\
    3.9	2.95735977133685\\
    3.95	2.86652450319574\\
    4	2.76695856664369\\
    4.05	2.65682942377063\\
    4.1	2.53357376911577\\
    4.15	2.39342092816138\\
    4.2	2.23043112043839\\
    4.25	2.03427387882488\\
    4.3	1.7839101015784\\
    4.35	1.42129436301186\\
    4.4	0\\
    };
    \addplot [color=black, forget plot]
      table[row sep=crcr]{%
    4.4	-0\\
    4.45	-1.43210263327108\\
    4.5	-1.81114484514709\\
    4.55	-2.08103675817557\\
    4.6	-2.2990544317303\\
    4.65	-2.48581971993629\\
    4.7	-2.65139359475723\\
    4.75	-2.80152446746804\\
    4.8	-2.939838638621\\
    4.85	-3.06878578458167\\
    4.9	-3.19010616025605\\
    4.95	-3.30508475380455\\
    5	-3.41469990581837\\
    };
    
    \addplot[area legend, draw=black, fill=black, forget plot]
    table[row sep=crcr] {%
    x	y\\
    3.64268943741856	3.34187519124151\\
    3.64268943741856	3.34187519124151\\
    3.64268943741856	3.34187519124151\\
    3.64268943741856	3.34187519124151\\
    3.64268943741856	3.34187519124151\\
    3.64268943741856	3.34187519124151\\
    3.64268943741856	3.34187519124151\\
    3.57088635031179	3.27007210413474\\
    3.39228231433004	3.59228231433004\\
    3.71449252452534	3.41367827834829\\
    3.64268943741856	3.34187519124151\\
    }--cycle;
    \addplot [color=black, forget plot]
      table[row sep=crcr]{%
    -5	6.17618334176118\\
    -4.95	6.14356852934671\\
    -4.9	6.11126646681883\\
    -4.85	6.07928226357011\\
    -4.8	6.04762103949539\\
    -4.75	6.01628791985994\\
    -4.7	5.98528802988077\\
    -4.65	5.9546264890184\\
    -4.6	5.92430840497767\\
    -4.55	5.89433886741701\\
    -4.5	5.86472294136692\\
    -4.45	5.83546566035971\\
    -4.4	5.80657201927403\\
    -4.35	5.77804696689878\\
    -4.3	5.74989539822316\\
    -4.25	5.72212214646078\\
    -4.2	5.6947319748176\\
    -4.15	5.66772956801541\\
    -4.1	5.64111952358409\\
    -4.05	5.61490634293814\\
    -4	5.58909442225448\\
    -3.95	5.56368804317094\\
    -3.9	5.53869136332627\\
    -3.85	5.51410840676505\\
    -3.8	5.48994305423219\\
    -3.75	5.46619903338409\\
    -3.7	5.44287990894502\\
    -3.65	5.41998907283918\\
    -3.6	5.3975297343305\\
    -3.55	5.37550491020363\\
    -3.5	5.35391741502115\\
    -3.45	5.33276985149294\\
    -3.4	5.31206460099498\\
    -3.35	5.29180381427528\\
    -3.3	5.27198940238559\\
    -3.25	5.25262302787746\\
    -3.2	5.23370609630174\\
    -3.15	5.2152397480499\\
    -3.1	5.19722485057551\\
    -3.05	5.17966199103314\\
    -3	5.16255146937099\\
    -2.95	5.14589329191211\\
    -2.9	5.12968716545745\\
    -2.85	5.11393249194193\\
    -2.8	5.09862836367258\\
    -2.75	5.08377355917515\\
    -2.7	5.06936653967282\\
    -2.65	5.05540544621769\\
    -2.6	5.04188809749251\\
    -2.55	5.02881198829654\\
    -2.5	5.01617428872595\\
    -2.45	5.00397184405552\\
    -2.4	4.99220117532457\\
    -2.35	4.98085848062601\\
    -2.3	4.96993963709383\\
    -2.25	4.95944020358025\\
    -2.2	4.94935542401002\\
    -2.15	4.93968023139562\\
    -2.1	4.93040925249353\\
    -2.05	4.92153681307815\\
    -2	4.91305694380694\\
    -1.95	4.90496338664699\\
    -1.9	4.89724960183065\\
    -1.85	4.88990877530508\\
    -1.8	4.88293382663862\\
    -1.75	4.87631741734448\\
    -1.7	4.87005195958094\\
    -1.65	4.86412962518571\\
    -1.6	4.85854235500106\\
    -1.55	4.8532818684457\\
    -1.5	4.84833967328888\\
    -1.45	4.84370707558215\\
    -1.4	4.83937518970431\\
    -1.35	4.8353349484757\\
    -1.3	4.83157711329864\\
    -1.25	4.82809228428169\\
    -1.2	4.82487091030687\\
    -1.15	4.82190329900017\\
    -1.1	4.81917962656736\\
    -1.05	4.81668994745888\\
    -1	4.81442420382937\\
    -0.95	4.8123722347595\\
    -0.899999999999999	4.81052378520972\\
    -0.85	4.80886851467779\\
    -0.8	4.80739600553389\\
    -0.75	4.80609577100958\\
    -0.7	4.80495726281873\\
    -0.649999999999999	4.8039698783909\\
    -0.6	4.8031229676995\\
    -0.55	4.80240583966936\\
    -0.5	4.80180776814992\\
    -0.45	4.80131799744245\\
    -0.399999999999999	4.80092574737107\\
    -0.35	4.80062021788932\\
    -0.3	4.80039059321517\\
    -0.25	4.80022604548897\\
    -0.199999999999999	4.80011573795004\\
    -0.149999999999999	4.8000488276283\\
    -0.0999999999999996	4.80001446754899\\
    -0.0499999999999998	4.80000180844839\\
    0	4.8\\
    0.0500000000000007	4.79999819155024\\
    0.100000000000001	4.7999855323638\\
    0.15	4.79995117137829\\
    0.2	4.79988425646833\\
    0.250000000000001	4.79977393321878\\
    0.300000000000001	4.79960934320654\\
    0.350000000000001	4.79937962178998\\
    0.4	4.79907389540437\\
    0.45	4.79868127836094\\
    0.500000000000001	4.79819086914608\\
    0.550000000000001	4.7975917462165\\
    0.600000000000001	4.79687296328433\\
    0.65	4.79602354408526\\
    0.7	4.79503247662074\\
    0.750000000000001	4.79388870686385\\
    0.800000000000001	4.79258113191607\\
    0.850000000000001	4.79109859260026\\
    0.9	4.78942986547258\\
    0.950000000000001	4.78756365423345\\
    1	4.78548858051474\\
    1.05	4.78319317401706\\
    1.1	4.78066586196769\\
    1.15	4.77789495786555\\
    1.2	4.77486864947567\\
    1.25	4.77157498603078\\
    1.3	4.76800186459246\\
    1.35	4.76413701551886\\
    1.4	4.75996798697958\\
    1.45	4.75548212845142\\
    1.5	4.7506665731213\\
    1.55	4.7455082191138\\
    1.6	4.73999370945179\\
    1.65	4.73410941064774\\
    1.7	4.72784138981183\\
    1.75	4.72117539014982\\
    1.8	4.7140968047088\\
    1.85	4.70659064821255\\
    1.9	4.69864152680957\\
    1.95	4.69023360553561\\
    2	4.68135057326899\\
    2.05	4.67197560492966\\
    2.1	4.66209132064226\\
    2.15	4.65167974154845\\
    2.2	4.64072224191325\\
    2.25	4.62919949712464\\
    2.3	4.61709142713204\\
    2.35	4.60437713480893\\
    2.4	4.59103483865373\\
    2.45	4.57704179916166\\
    2.5	4.56237423810478\\
    2.55	4.54700724984663\\
    2.6	4.5309147036877\\
    2.65	4.51406913608526\\
    2.7	4.49644163141122\\
    2.75	4.47800168969859\\
    2.8	4.45871707957417\\
    2.85	4.43855367427352\\
    2.9	4.41747526827306\\
    2.95	4.39544337163964\\
    3	4.37241697867311\\
    3.05	4.34835230677977\\
    3.1	4.32320250073683\\
    3.15	4.29691729655335\\
    3.2	4.26944263795511\\
    3.25	4.24072023705863\\
    3.3	4.21068706897219\\
    3.35	4.17927478776402\\
    3.4	4.14640904832589\\
    3.45	4.11200871494441\\
    3.5	4.07598493260902\\
    3.55	4.0382400308764\\
    3.6	3.99866622197514\\
    3.65	3.95714404406568\\
    3.7	3.91354048616405\\
    3.75	3.86770671173796\\
    3.8	3.8194752712541\\
    3.85	3.76865665681006\\
    3.9	3.71503499958271\\
    3.95	3.65836263567708\\
    4	3.59835315622033\\
    4.05	3.53467239404558\\
    4.1	3.46692655020797\\
    4.15	3.39464627442907\\
    4.2	3.31726488820732\\
    4.25	3.23408790150043\\
    4.3	3.14424918656787\\
    4.35	3.04664595911424\\
    4.4	2.939838638621\\
    4.45	2.82188944180945\\
    4.5	2.69008710671805\\
    4.55	2.54044224221422\\
    4.6	2.36666887081282\\
    4.65	2.15783154291702\\
    4.7	1.89165313589352\\
    4.75	1.50665561007714\\
    4.8	0\\
    };
    \addplot [color=black, forget plot]
      table[row sep=crcr]{%
    4.8	-0\\
    4.85	-1.51715490996357\\
    4.9	-1.91810939473754\\
    4.95	-2.20325791411787\\
    5	-2.43333124830413\\
    };
    
    \addplot[area legend, draw=black, fill=black, forget plot]
    table[row sep=crcr] {%
    x	y\\
    3.96493174019612	3.65459330925124\\
    3.96493174019612	3.65459330925124\\
    3.96493174019612	3.65459330925124\\
    3.96493174019612	3.65459330925124\\
    3.96493174019612	3.65459330925124\\
    3.96493174019612	3.65459330925124\\
    3.96493174019612	3.65459330925124\\
    3.8917631450717	3.58142471412683\\
    3.70976252472368	3.90976252472368\\
    4.03810033532053	3.72776190437566\\
    3.96493174019612	3.65459330925124\\
    }--cycle;
    \end{axis}
    \end{tikzpicture}%
            \end{center}
            \caption{Orbite per il problema di Cauchy}\label{fig:orbiteradiceterza}
        \end{figure}
    \end{enumerate}
}
\paragrafo{Esempio}{%
\[
    \begin{cases}
        x'=y^{2}\\ 
        y'=-xy
    \end{cases}    
\]
\begin{enumerate}
    \item \emph{Equilibri}
    
    Tutti i punti $ (x,0) $ sono equilibri.
    \item \emph{Tangente orizzontale e verticale}
    
    Tutta l'asse delle $ y $ è composta da punti a tangente orizzontale. 
    \item \emph{Orbite non singolari}
    
    Consideriamo $ f_1(x,y)\neq 0 $, ovvero quelli per i quali $ y\neq 0 $: troviamo tutte le orbite, infatti i punti $ (x,0) $ sono equilibri. Calcolando: \[
        \varphi'(x)=-\frac{x\,\varphi(x)}{\varphi^{2}(x)}=-\frac{x}{\varphi(x)}
    \]Da qui, possiamo dire \[
        \varphi(x)\,\varphi'(x)=-x \quad \implies \quad \varphi^{2}(x)=-x^{2}+c
    \]
    
    $c>0$, e $ \displaystyle \varphi(x)=\pm (c-x^{2})^{1/2}$. 

    Inoltre $ x'=y^{2}>0 $ lungo le orbite. Dunque il diagramma di fase è quello illustrato in figure \ref{fig:eqwsdcs}. 
    \begin{figure}
        \begin{center}
            \begin{tikzpicture}
                \draw [-Latex] (-3,0) -- (3,0);
                \draw [-Latex] (0,-3) -- (0,3);
                \draw [ultra thick] (-1.8,0) -- (1.8,0);
                \draw [ultra thick, dashed] (-1.8,0) -- (-2.9,0);
                \draw [ultra thick, dashed] (1.8,0) -- (2.9,0);
                \foreach \r in{0.5,1,...,2.5}{
                    \draw (0,0) circle (\r);
                    \fill [white] (\r,0) circle (0.05);
                    \draw (\r,0) circle (0.055);
                    \fill [white] (-\r,0) circle (0.05);
                    \draw (-\r,0) circle (0.055);
                    \draw [-Latex] (0,\r) -- (0.1, \r);
                    \draw [-Latex] (0,-\r) -- (0.1, -\r);
                };
            \end{tikzpicture}
        \end{center}
        \caption{Diagramma di fase per l'esempio \framref{jjdjdjdkjnsjdkjncskjndsckjn}}\label{fig:eqwsdcs}
    \end{figure}
\end{enumerate}
}{jjdjdjdkjnsjdkjncskjndsckjn}{} % Lezione 3
\days{7 marzo 2023}
\paragrafo{Esempio}{%
\[
    \begin{cases}
        x'=y^{2}-xy^{2}\\ 
        y'=-x\,y^{3}
    \end{cases}\qquad \begin{aligned}
    \bm{f}: \R^{2} &\longrightarrow \R^{2} \\
    (x,y) &\longmapsto (y^{2}-xy^{2}, -xy^{3})
    \end{aligned}
\]\begin{enumerate}
    \item \emph{Equilibri}: \[
        \begin{cases}
            y^{2}(1-x)=0\\ 
            x\,y^{3} = 0
        \end{cases}
    \]la prima equazione è soddisfatta per $ x = 1 $, oppure per $ y = 0 $, mentre la seconda è soddisfatta da $ x = 0 $ oppure $ y = 0 $. 

    Entrambe le equazioni sono soddisfatte se e solo se $ y=0 $ 
    
    $\implies$ $ \displaystyle \left\{(x,0): x \in \R\right\} $ è composto da equilibri. 
    \item \emph{Punti con tangente verticale}: $ x'=0 $;
    
    Oltre ai punti di equilibrio, sono tutti i punti sulla retta verticale $ x=1 $. Questa quindi è una retta invariante, composta da orbite.

    Se $ x = 1 $ abbiamo \[
        \begin{cases}
            x'=0\\ 
            y'=-y^{3}
        \end{cases}
    \]
    \item \emph{Punti con tangente orizzontale}: $ y'=0 $. 
    
    Oltre ai punti di equilibrio, sono tutti i punti sulla retta verticale $ x=0 $.

    Se $ x=0 $ abbiamo \[
        \begin{cases}
            x'=y^{2}>0 & y>0\\ 
            y'=0
        \end{cases}
    \]
    \begin{figure}
        \begin{center}
            \begin{tikzpicture}
                \draw [-Latex] (-3,0) -- (3,0);
                \draw [-Latex] (0,-3) -- (0,3);
                \node at (3,-0.2) {$x$};
                \node at (-0.2,3) {$y$};
                \foreach \i in{-2.4,-2,...,2.4}{
                    \fill (0,\i) circle (0.05);
                    \draw [-Latex, thin] (0,\i) -- (0.4,\i);
                };
                \draw [thick,-Latex] (1,3) -- (1,1.3);
                \draw [thick] (1,1.5) -- (1,-1.5);
                \draw [thick,-Latex] (1,-3) -- (1,-1.3);
                \fill (1,0) circle (0.1);
            \end{tikzpicture}
        \end{center}
        \caption{Punti con tangenti verticali e orizzontali per l'esempio \framref{dafasdflkjnsadfkjnasdfkjnadskfjnasdkfjnldskjn}}
    \end{figure}
    \item \emph{Equazioni delle orbite}: per $ y\neq 0 $ e $ x\neq 1 $ ho: \begin{equation}
        \frac{d\,y}{dx} =\frac{-xy^{3}}{y^{2}(1-x)}\quad\leadsto\quad \begin{cases}
            y'(x)=\displaystyle\frac{-x}{1-x}\,y(x) \label{eq:diff:orb}\\ 
            y(x_0)=y_0
        \end{cases}
    \end{equation}e integrando si ottiene \[
        \ln\frac{|y|}{|y_0|} = x-x_0 + \ln\frac{|x-1|}{|x_0-1|}
    \]Noto che poiché le orbite non intersecano l'asse $ x $ (poiché sono altre orbite), $ y(x) $ e $ y_0 $ hanno lo stesso segno, dunque posso ``eliminare'' i valori assoluti. Con lo stesso ragionamento ``elimino'' il secondo valore assoluto. Diventa: \[
        \ln\frac{y}{y_0} = x-x_0 + \ln\frac{x-1}{x_0-1}
    \]Facendo l'esponenziale da ambo le parti otteniamo \[
        y(x)= y_0\,\frac{x-1}{x_0-1}\,e^{x-x_0}
    \]Alcune osservazioni: \begin{enumerate}
        \item le orbite con dato iniziale $ (x_0,y_0) $ e $ (x_0,-y_0) $ sono simmetriche rispetto all'asse orizzontale 
        
        $\implies$ consideriamo solo $ y_0 >0$. 

        Dividiamo ancora i due casi: \begin{itemize}
            \item $ x_0>1 $ $ \,\implies\, $ $ \displaystyle y_0\,\frac{x-1}{x_0-1}>0 $\[
                \,\implies\,\lim_{x\to 1^{+}} y(x) = 0^{+},\qquad \lim_{x\to + \infty} y(x) = + \infty
            \]e guardando l'equazione differenziale delle orbite \eqref{eq:diff:orb}\[
                y'(x)=\frac{x}{x-1}\,y(x) >0
            \]e quindi $ y $ come funzione di $ x $ è monotona crescente.
            
            Noto inoltre che per $ x_0>1 $, $ x'=y^{2}(1-x)<0 $, dunque le frecce delle orbite sono rivolte verso $ x=1 $. 
            \item $ x_0<1 $ $ \,\implies\, $ $ \displaystyle \frac{y_0}{x_0-1}>0 $, $ x-1<0 $: \[
                \lim_{x\to 1^{-}} y(x) = 0^{+},\qquad \lim_{x\to - \infty} y(x) = 0^{+}
            \]e inoltre la funzione è sempre positiva. 

            Si inoltre che $ x'>0 $, dunque le frecce sono rivolte verso $ x=1 $.
        \end{itemize}
        Le orbite sono illustrate nella figura \ref{fig:orbitedivattelaapesce}
        \begin{figure}
            \begin{center}
                % This file was created by matlab2tikz.
%
%The latest updates can be retrieved from
%  http://www.mathworks.com/matlabcentral/fileexchange/22022-matlab2tikz-matlab2tikz
%where you can also make suggestions and rate matlab2tikz.
%
\begin{tikzpicture}


  \begin{axis}[%
    width=0.9*\textwidth,
    axis equal,
    axis lines=middle,
    xmin=-4,
    xmax=1,
    ymin=-5,
    ymax=5,
    axis background/.style={fill=white}
    ]
    \draw (1,5) -- (1,-5);
    \draw [ultra thick] (-4,0) -- (1,0);
    \draw [ultra thick, dashed] (-4,0) -- (-6,0);
    \draw [ultra thick, dashed] (3,0) -- (1,0);
    \fill (1,0) circle (0.13);
\addplot [color=black, forget plot]
  table[row sep=crcr]{%
-5	-0.121283045983538\\
-4.95	-0.12643884938358\\
-4.9	-0.131804520355361\\
-4.85	-0.137388025988912\\
-4.8	-0.143197598652949\\
-4.75	-0.149241742253831\\
-4.7	-0.155529238438998\\
-4.65	-0.162069152722712\\
-4.6	-0.168870840509844\\
-4.55	-0.175943952991147\\
-4.5	-0.183298442880998\\
-4.45	-0.190944569965966\\
-4.4	-0.198892906429709\\
-4.35	-0.207154341916702\\
-4.3	-0.215740088293995\\
-4.25	-0.224661684066738\\
-4.2	-0.233930998399452\\
-4.15	-0.243560234691002\\
-4.1	-0.253561933646947\\
-4.05	-0.263948975788327\\
-4	-0.274734583331013\\
-3.95	-0.285932321364496\\
-3.9	-0.297556098253325\\
-3.85	-0.309620165178388\\
-3.8	-0.322139114728785\\
-3.75	-0.33512787844813\\
-3.7	-0.348601723231785\\
-3.65	-0.362576246463637\\
-3.6	-0.377067369772637\\
-3.55	-0.392091331280368\\
-3.5	-0.4076646762013\\
-3.45	-0.423804245647207\\
-3.4	-0.440527163476304\\
-3.35	-0.457850821016028\\
-3.3	-0.475792859475996\\
-3.25	-0.494371149854456\\
-3.2	-0.513603770127414\\
-3.15	-0.53350897949465\\
-3.1	-0.554105189440761\\
-3.05	-0.575410931352362\\
-3	-0.597444820414367\\
-2.95	-0.620225515488924\\
-2.9	-0.643771674659965\\
-2.85	-0.668101906104384\\
-2.8	-0.693234713927485\\
-2.75	-0.71918843857546\\
-2.7	-0.745981191411223\\
-2.65	-0.773630783011704\\
-2.6	-0.802154644714806\\
-2.55	-0.831569742912281\\
-2.5	-0.861892485550937\\
-2.45	-0.893138620268485\\
-2.4	-0.925323123552008\\
-2.35	-0.958460080266274\\
-2.3	-0.992562552855757\\
-2.25	-1.02764243947818\\
-2.2	-1.06371032027841\\
-2.15	-1.10077529095955\\
-2.1	-1.13884478275273\\
-2.05	-1.17792436782841\\
-2	-1.21801754912951\\
-1.95	-1.25912553354065\\
-1.9	-1.30124698723692\\
-1.85	-1.34437777198152\\
-1.8	-1.38851066106133\\
-1.75	-1.43363503346617\\
-1.7	-1.47973654482715\\
-1.65	-1.52679677353499\\
-1.6	-1.57479284035831\\
-1.55	-1.62369699977458\\
-1.5	-1.67347620111322\\
-1.45	-1.72409161748941\\
-1.4	-1.77549814037957\\
-1.35	-1.82764383755354\\
-1.3	-1.88046937193469\\
-1.25	-1.93390737880628\\
-1.2	-1.98788179862053\\
-1.15	-2.04230716249489\\
-1.1	-2.0970878272979\\
-1.05	-2.1521171570336\\
-1	-2.20727664702865\\
-0.95	-2.26243498720883\\
-0.9	-2.31744706052141\\
-0.85	-2.37215287231543\\
-0.8	-2.426376406233\\
-0.75	-2.47992440189033\\
-0.7	-2.53258504933619\\
-0.65	-2.58412659496703\\
-0.6	-2.63429585325133\\
-0.55	-2.68281661826926\\
-0.5	-2.72938796870685\\
-0.45	-2.77368245955471\\
-0.4	-2.81534419334968\\
-0.35	-2.85398676336079\\
-0.3	-2.8891910606587\\
-0.25	-2.92050293651777\\
-0.2	-2.94743071108073\\
-0.15	-2.96944251866645\\
-0.1	-2.98596347951867\\
-0.05	-2.99637268717725\\
0	-3\\
0.0499999999999999	-2.99612262467167\\
0.1	-2.98396147880425\\
0.15	-2.96267731895712\\
0.2	-2.93136661958441\\
0.25	-2.88905718754742\\
0.3	-2.83470349590961\\
0.35	-2.76718171975685\\
0.4	-2.68528445575429\\
0.45	-2.58771510605878\\
0.5	-2.47308190605019\\
0.55	-2.33989157412098\\
0.6	-2.18654256046861\\
0.65	-2.01131787046459\\
0.7	-1.81237743672343\\
0.75	-1.58775001245951\\
0.8	-1.33532455709548\\
0.85	-1.0528410833667\\
0.9	-0.737880933347085\\
0.95	-0.387856448897377\\
1	0\\
};

\addplot[area legend, draw=black, fill=black, forget plot]
table[row sep=crcr] {%
x	y\\
0.0308755760368662	-3\\
0	-3\\
0	-3\\
0	-3\\
0	-3\\
0	-3\\
0.0617511520737324	-3\\
0.0617511520737324	-3.02513105922514\\
0.2	-3\\
0.0617511520737324	-2.97486894077486\\
0.0617511520737324	-3\\
}--cycle;
\addplot [color=black, forget plot]
  table[row sep=crcr]{%
-5	-0.101069204986282\\
-4.95	-0.10536570781965\\
-4.9	-0.109837100296134\\
-4.85	-0.114490021657427\\
-4.8	-0.11933133221079\\
-4.75	-0.124368118544859\\
-4.7	-0.129607698699165\\
-4.65	-0.135057627268927\\
-4.6	-0.14072570042487\\
-4.55	-0.146619960825956\\
-4.5	-0.152748702400832\\
-4.45	-0.159120474971638\\
-4.4	-0.165744088691424\\
-4.35	-0.172628618263918\\
-4.3	-0.179783406911662\\
-4.25	-0.187218070055615\\
-4.2	-0.19494249866621\\
-4.15	-0.202966862242502\\
-4.1	-0.211301611372456\\
-4.05	-0.219957479823606\\
-4	-0.228945486109177\\
-3.95	-0.238276934470413\\
-3.9	-0.247963415211104\\
-3.85	-0.258016804315323\\
-3.8	-0.268449262273987\\
-3.75	-0.279273232040108\\
-3.7	-0.290501436026488\\
-3.65	-0.302146872053031\\
-3.6	-0.314222808143865\\
-3.55	-0.326742776066974\\
-3.5	-0.339720563501083\\
-3.45	-0.353170204706006\\
-3.4	-0.367105969563587\\
-3.35	-0.38154235084669\\
-3.3	-0.39649404956333\\
-3.25	-0.411975958212046\\
-3.2	-0.428003141772845\\
-3.15	-0.444590816245542\\
-3.1	-0.461754324533968\\
-3.05	-0.479509109460302\\
-3	-0.497870683678639\\
-2.95	-0.51685459624077\\
-2.9	-0.536476395549971\\
-2.85	-0.55675158842032\\
-2.8	-0.577695594939571\\
-2.75	-0.599323698812883\\
-2.7	-0.621650992842686\\
-2.65	-0.64469231917642\\
-2.6	-0.668462203929005\\
-2.55	-0.692974785760234\\
-2.5	-0.718243737959114\\
-2.45	-0.744282183557071\\
-2.4	-0.771102602960006\\
-2.35	-0.798716733555228\\
-2.3	-0.827135460713131\\
-2.25	-0.856368699565148\\
-2.2	-0.886425266898671\\
-2.15	-0.917312742466289\\
-2.1	-0.94903731896061\\
-2.05	-0.981603639857007\\
-2	-1.0150146242746\\
-1.95	-1.04927127795054\\
-1.9	-1.0843724893641\\
-1.85	-1.1203148099846\\
-1.8	-1.15709221755111\\
-1.75	-1.19469586122181\\
-1.7	-1.23311378735596\\
-1.65	-1.2723306446125\\
-1.6	-1.31232736696526\\
-1.55	-1.35308083314549\\
-1.5	-1.39456350092769\\
-1.45	-1.43674301457451\\
-1.4	-1.47958178364964\\
-1.35	-1.52303653129461\\
-1.3	-1.56705780994557\\
-1.25	-1.61158948233857\\
-1.2	-1.65656816551711\\
-1.15	-1.70192263541241\\
-1.1	-1.74757318941492\\
-1.05	-1.79343096419467\\
-1	-1.83939720585721\\
-0.95	-1.88536248934069\\
-0.9	-1.93120588376785\\
-0.85	-1.97679406026286\\
-0.8	-2.0219803385275\\
-0.75	-2.06660366824194\\
-0.7	-2.11048754111349\\
-0.65	-2.15343882913919\\
-0.6	-2.19524654437611\\
-0.55	-2.23568051522439\\
-0.5	-2.27448997392238\\
-0.45	-2.31140204962893\\
-0.4	-2.34612016112474\\
-0.35	-2.37832230280066\\
-0.3	-2.40765921721558\\
-0.25	-2.43375244709814\\
-0.2	-2.45619225923395\\
-0.15	-2.47453543222204\\
-0.1	-2.48830289959889\\
-0.05	-2.49697723931437\\
0	-2.5\\
0.0499999999999999	-2.49676885389306\\
0.1	-2.48663456567021\\
0.15	-2.4688977657976\\
0.2	-2.44280551632034\\
0.25	-2.40754765628952\\
0.3	-2.36225291325801\\
0.35	-2.30598476646404\\
0.4	-2.23773704646191\\
0.45	-2.15642925504898\\
0.5	-2.06090158837516\\
0.55	-1.94990964510082\\
0.6	-1.82211880039051\\
0.65	-1.67609822538716\\
0.7	-1.51031453060286\\
0.75	-1.32312501038292\\
0.8	-1.11277046424623\\
0.85	-0.877367569472247\\
0.9	-0.614900777789237\\
0.95	-0.323213707414481\\
1	0\\
};

\addplot[area legend, draw=black, fill=black, forget plot]
table[row sep=crcr] {%
x	y\\
0.0308755760368662	-2.5\\
0	-2.5\\
0	-2.5\\
0	-2.5\\
0	-2.5\\
0	-2.5\\
0.0617511520737324	-2.5\\
0.0617511520737324	-2.53964221922933\\
0.2	-2.5\\
0.0617511520737324	-2.46035778077067\\
0.0617511520737324	-2.5\\
}--cycle;
\addplot [color=black, forget plot]
  table[row sep=crcr]{%
-5	-0.0808553639890256\\
-4.95	-0.0842925662557202\\
-4.9	-0.0878696802369072\\
-4.85	-0.0915920173259415\\
-4.8	-0.0954650657686323\\
-4.75	-0.0994944948358873\\
-4.7	-0.103686158959332\\
-4.65	-0.108046101815142\\
-4.6	-0.112580560339896\\
-4.55	-0.117295968660764\\
-4.5	-0.122198961920665\\
-4.45	-0.12729637997731\\
-4.4	-0.132595270953139\\
-4.35	-0.138102894611135\\
-4.3	-0.14382672552933\\
-4.25	-0.149774456044492\\
-4.2	-0.155953998932968\\
-4.15	-0.162373489794001\\
-4.1	-0.169041289097965\\
-4.05	-0.175965983858884\\
-4	-0.183156388887342\\
-3.95	-0.190621547576331\\
-3.9	-0.198370732168883\\
-3.85	-0.206413443452259\\
-3.8	-0.21475940981919\\
-3.75	-0.223418585632087\\
-3.7	-0.23240114882119\\
-3.65	-0.241717497642425\\
-3.6	-0.251378246515092\\
-3.55	-0.261394220853579\\
-3.5	-0.271776450800866\\
-3.45	-0.282536163764805\\
-3.4	-0.29368477565087\\
-3.35	-0.305233880677352\\
-3.3	-0.317195239650664\\
-3.25	-0.329580766569637\\
-3.2	-0.342402513418276\\
-3.15	-0.355672652996434\\
-3.1	-0.369403459627174\\
-3.05	-0.383607287568242\\
-3	-0.398296546942912\\
-2.95	-0.413483676992616\\
-2.9	-0.429181116439976\\
-2.85	-0.445401270736256\\
-2.8	-0.462156475951657\\
-2.75	-0.479458959050307\\
-2.7	-0.497320794274148\\
-2.65	-0.515753855341136\\
-2.6	-0.534769763143204\\
-2.55	-0.554379828608187\\
-2.5	-0.574594990367292\\
-2.45	-0.595425746845657\\
-2.4	-0.616882082368005\\
-2.35	-0.638973386844183\\
-2.3	-0.661708368570505\\
-2.25	-0.685094959652118\\
-2.2	-0.709140213518937\\
-2.15	-0.733850193973031\\
-2.1	-0.759229855168488\\
-2.05	-0.785282911885606\\
-2	-0.812011699419676\\
-1.95	-0.83941702236043\\
-1.9	-0.867497991491283\\
-1.85	-0.896251847987677\\
-1.8	-0.925673774040884\\
-1.75	-0.955756688977448\\
-1.7	-0.986491029884767\\
-1.65	-1.01786451569\\
-1.6	-1.04986189357221\\
-1.55	-1.08246466651639\\
-1.5	-1.11565080074215\\
-1.45	-1.14939441165961\\
-1.4	-1.18366542691971\\
-1.35	-1.21842922503569\\
-1.3	-1.25364624795646\\
-1.25	-1.28927158587086\\
-1.2	-1.32525453241369\\
-1.15	-1.36153810832993\\
-1.1	-1.39805855153193\\
-1.05	-1.43474477135574\\
-1	-1.47151776468577\\
-0.95	-1.50828999147255\\
-0.9	-1.54496470701428\\
-0.85	-1.58143524821029\\
-0.8	-1.617584270822\\
-0.75	-1.65328293459355\\
-0.7	-1.68839003289079\\
-0.65	-1.72275106331135\\
-0.6	-1.75619723550088\\
-0.55	-1.78854441217951\\
-0.5	-1.8195919791379\\
-0.45	-1.84912163970314\\
-0.4	-1.87689612889979\\
-0.35	-1.90265784224053\\
-0.3	-1.92612737377247\\
-0.25	-1.94700195767851\\
-0.2	-1.96495380738716\\
-0.15	-1.97962834577763\\
-0.1	-1.99064231967911\\
-0.05	-1.9975817914515\\
0	-2\\
0.0499999999999999	-1.99741508311445\\
0.1	-1.98930765253617\\
0.15	-1.97511821263808\\
0.2	-1.95424441305627\\
0.25	-1.92603812503161\\
0.3	-1.8898023306064\\
0.35	-1.84478781317123\\
0.4	-1.79018963716952\\
0.45	-1.72514340403919\\
0.5	-1.64872127070013\\
0.55	-1.55992771608066\\
0.6	-1.45769504031241\\
0.65	-1.34087858030973\\
0.7	-1.20825162448229\\
0.75	-1.05850000830634\\
0.8	-0.890216371396987\\
0.85	-0.701894055577797\\
0.9	-0.49192062223139\\
0.95	-0.258570965931585\\
1	0\\
};

\addplot[area legend, draw=black, fill=black, forget plot]
table[row sep=crcr] {%
x	y\\
0.0308755760368662	-2\\
0	-2\\
0	-2\\
0	-2\\
0	-2\\
0	-2\\
0.0617511520737324	-2\\
0.0617511520737324	-2.03964221922933\\
0.2	-2\\
0.0617511520737324	-1.96035778077067\\
0.0617511520737324	-2\\
}--cycle;
\addplot [color=black, forget plot]
  table[row sep=crcr]{%
-5	-0.0606415229917692\\
-4.95	-0.0632194246917902\\
-4.9	-0.0659022601776804\\
-4.85	-0.0686940129944562\\
-4.8	-0.0715987993264743\\
-4.75	-0.0746208711269155\\
-4.7	-0.0777646192194992\\
-4.65	-0.0810345763613562\\
-4.6	-0.0844354202549221\\
-4.55	-0.0879719764955733\\
-4.5	-0.091649221440499\\
-4.45	-0.0954722849829828\\
-4.4	-0.0994464532148544\\
-4.35	-0.103577170958351\\
-4.3	-0.107870044146997\\
-4.25	-0.112330842033369\\
-4.2	-0.116965499199726\\
-4.15	-0.121780117345501\\
-4.1	-0.126780966823474\\
-4.05	-0.131974487894163\\
-4	-0.137367291665506\\
-3.95	-0.142966160682248\\
-3.9	-0.148778049126662\\
-3.85	-0.154810082589194\\
-3.8	-0.161069557364392\\
-3.75	-0.167563939224065\\
-3.7	-0.174300861615893\\
-3.65	-0.181288123231819\\
-3.6	-0.188533684886319\\
-3.55	-0.196045665640184\\
-3.5	-0.20383233810065\\
-3.45	-0.211902122823604\\
-3.4	-0.220263581738152\\
-3.35	-0.228925410508014\\
-3.3	-0.237896429737998\\
-3.25	-0.247185574927228\\
-3.2	-0.256801885063707\\
-3.15	-0.266754489747325\\
-3.1	-0.277052594720381\\
-3.05	-0.287705465676181\\
-3	-0.298722410207184\\
-2.95	-0.310112757744462\\
-2.9	-0.321885837329982\\
-2.85	-0.334050953052192\\
-2.8	-0.346617356963742\\
-2.75	-0.35959421928773\\
-2.7	-0.372990595705611\\
-2.65	-0.386815391505852\\
-2.6	-0.401077322357403\\
-2.55	-0.415784871456141\\
-2.5	-0.430946242775469\\
-2.45	-0.446569310134243\\
-2.4	-0.462661561776004\\
-2.35	-0.479230040133137\\
-2.3	-0.496281276427879\\
-2.25	-0.513821219739089\\
-2.2	-0.531855160139203\\
-2.15	-0.550387645479773\\
-2.1	-0.569422391376366\\
-2.05	-0.588962183914204\\
-2	-0.609008774564757\\
-1.95	-0.629562766770323\\
-1.9	-0.650623493618462\\
-1.85	-0.672188885990758\\
-1.8	-0.694255330530663\\
-1.75	-0.716817516733086\\
-1.7	-0.739868272413575\\
-1.65	-0.763398386767497\\
-1.6	-0.787396420179156\\
-1.55	-0.811848499887292\\
-1.5	-0.836738100556612\\
-1.45	-0.862045808744706\\
-1.4	-0.887749070189783\\
-1.35	-0.913821918776768\\
-1.3	-0.940234685967343\\
-1.25	-0.966953689403142\\
-1.2	-0.993940899310267\\
-1.15	-1.02115358124745\\
-1.1	-1.04854391364895\\
-1.05	-1.0760585785168\\
-1	-1.10363832351433\\
-0.95	-1.13121749360442\\
-0.9	-1.15872353026071\\
-0.85	-1.18607643615772\\
-0.8	-1.2131882031165\\
-0.75	-1.23996220094516\\
-0.7	-1.26629252466809\\
-0.65	-1.29206329748351\\
-0.6	-1.31714792662566\\
-0.55	-1.34140830913463\\
-0.5	-1.36469398435343\\
-0.45	-1.38684122977736\\
-0.4	-1.40767209667484\\
-0.35	-1.42699338168039\\
-0.3	-1.44459553032935\\
-0.25	-1.46025146825888\\
-0.2	-1.47371535554037\\
-0.15	-1.48472125933322\\
-0.1	-1.49298173975933\\
-0.05	-1.49818634358862\\
0	-1.5\\
0.0499999999999999	-1.49806131233583\\
0.1	-1.49198073940212\\
0.15	-1.48133865947856\\
0.2	-1.4656833097922\\
0.25	-1.44452859377371\\
0.3	-1.4173517479548\\
0.35	-1.38359085987843\\
0.4	-1.34264222787714\\
0.45	-1.29385755302939\\
0.5	-1.2365409530251\\
0.55	-1.16994578706049\\
0.6	-1.09327128023431\\
0.65	-1.0056589352323\\
0.7	-0.906188718361715\\
0.75	-0.793875006229753\\
0.8	-0.66766227854774\\
0.85	-0.526420541683348\\
0.9	-0.368940466673542\\
0.95	-0.193928224448689\\
1	0\\
};

\addplot[area legend, draw=black, fill=black, forget plot]
table[row sep=crcr] {%
x	y\\
0.0308755760368662	-1.5\\
0	-1.5\\
0	-1.5\\
0	-1.5\\
0	-1.5\\
0	-1.5\\
0.0617511520737324	-1.5\\
0.0617511520737324	-1.53964221922933\\
0.2	-1.5\\
0.0617511520737324	-1.46035778077067\\
0.0617511520737324	-1.5\\
}--cycle;
\addplot [color=black, forget plot]
  table[row sep=crcr]{%
-5	-0.0404276819945128\\
-4.95	-0.0421462831278601\\
-4.9	-0.0439348401184536\\
-4.85	-0.0457960086629708\\
-4.8	-0.0477325328843162\\
-4.75	-0.0497472474179436\\
-4.7	-0.0518430794796661\\
-4.65	-0.0540230509075708\\
-4.6	-0.0562902801699481\\
-4.55	-0.0586479843303822\\
-4.5	-0.0610994809603327\\
-4.45	-0.0636481899886552\\
-4.4	-0.0662976354765696\\
-4.35	-0.0690514473055673\\
-4.3	-0.071913362764665\\
-4.25	-0.0748872280222461\\
-4.2	-0.0779769994664841\\
-4.15	-0.0811867448970006\\
-4.1	-0.0845206445489824\\
-4.05	-0.0879829919294422\\
-4	-0.0915781944436709\\
-3.95	-0.0953107737881653\\
-3.9	-0.0991853660844415\\
-3.85	-0.103206721726129\\
-3.8	-0.107379704909595\\
-3.75	-0.111709292816043\\
-3.7	-0.116200574410595\\
-3.65	-0.120858748821212\\
-3.6	-0.125689123257546\\
-3.55	-0.130697110426789\\
-3.5	-0.135888225400433\\
-3.45	-0.141268081882402\\
-3.4	-0.146842387825435\\
-3.35	-0.152616940338676\\
-3.3	-0.158597619825332\\
-3.25	-0.164790383284819\\
-3.2	-0.171201256709138\\
-3.15	-0.177836326498217\\
-3.1	-0.184701729813587\\
-3.05	-0.191803643784121\\
-3	-0.199148273471456\\
-2.95	-0.206741838496308\\
-2.9	-0.214590558219988\\
-2.85	-0.222700635368128\\
-2.8	-0.231078237975828\\
-2.75	-0.239729479525153\\
-2.7	-0.248660397137074\\
-2.65	-0.257876927670568\\
-2.6	-0.267384881571602\\
-2.55	-0.277189914304094\\
-2.5	-0.287297495183646\\
-2.45	-0.297712873422828\\
-2.4	-0.308441041184003\\
-2.35	-0.319486693422091\\
-2.3	-0.330854184285252\\
-2.25	-0.342547479826059\\
-2.2	-0.354570106759468\\
-2.15	-0.366925096986515\\
-2.1	-0.379614927584244\\
-2.05	-0.392641455942803\\
-2	-0.406005849709838\\
-1.95	-0.419708511180215\\
-1.9	-0.433748995745642\\
-1.85	-0.448125923993839\\
-1.8	-0.462836887020442\\
-1.75	-0.477878344488724\\
-1.7	-0.493245514942384\\
-1.65	-0.508932257844998\\
-1.6	-0.524930946786104\\
-1.55	-0.541232333258195\\
-1.5	-0.557825400371075\\
-1.45	-0.574697205829804\\
-1.4	-0.591832713459855\\
-1.35	-0.609214612517845\\
-1.3	-0.626823123978229\\
-1.25	-0.644635792935428\\
-1.2	-0.662627266206845\\
-1.15	-0.680769054164964\\
-1.1	-0.699029275765967\\
-1.05	-0.717372385677868\\
-1	-0.735758882342885\\
-0.95	-0.754144995736277\\
-0.9	-0.772482353507138\\
-0.85	-0.790717624105144\\
-0.8	-0.808792135410999\\
-0.75	-0.826641467296776\\
-0.7	-0.844195016445396\\
-0.65	-0.861375531655676\\
-0.6	-0.878098617750442\\
-0.55	-0.894272206089754\\
-0.5	-0.90979598956895\\
-0.45	-0.924560819851571\\
-0.4	-0.938448064449895\\
-0.35	-0.951328921120263\\
-0.3	-0.963063686886233\\
-0.25	-0.973500978839256\\
-0.2	-0.982476903693578\\
-0.15	-0.989814172888816\\
-0.1	-0.995321159839556\\
-0.05	-0.99879089572575\\
0	-1\\
0.0499999999999999	-0.998707541557223\\
0.1	-0.994653826268083\\
0.15	-0.987559106319041\\
0.2	-0.977122206528136\\
0.25	-0.963019062515806\\
0.3	-0.944901165303202\\
0.35	-0.922393906585617\\
0.4	-0.895094818584762\\
0.45	-0.862571702019593\\
0.5	-0.824360635350064\\
0.55	-0.779963858040328\\
0.6	-0.728847520156204\\
0.65	-0.670439290154864\\
0.7	-0.604125812241143\\
0.75	-0.529250004153169\\
0.8	-0.445108185698493\\
0.85	-0.350947027788899\\
0.9	-0.245960311115695\\
0.95	-0.129285482965792\\
1	0\\
};

\addplot[area legend, draw=black, fill=black, forget plot]
table[row sep=crcr] {%
x	y\\
0.0308755760368662	-1\\
0	-1\\
0	-1\\
0	-1\\
0	-1\\
0	-1\\
0.0617511520737324	-1\\
0.0617511520737324	-1.03964221922933\\
0.2	-1\\
0.0617511520737324	-0.960357780770672\\
0.0617511520737324	-1\\
}--cycle;
\addplot [color=black, forget plot]
  table[row sep=crcr]{%
-5	-0.0202138409972564\\
-4.95	-0.0210731415639301\\
-4.9	-0.0219674200592268\\
-4.85	-0.0228980043314854\\
-4.8	-0.0238662664421581\\
-4.75	-0.0248736237089718\\
-4.7	-0.0259215397398331\\
-4.65	-0.0270115254537854\\
-4.6	-0.028145140084974\\
-4.55	-0.0293239921651911\\
-4.5	-0.0305497404801663\\
-4.45	-0.0318240949943276\\
-4.4	-0.0331488177382848\\
-4.35	-0.0345257236527837\\
-4.3	-0.0359566813823325\\
-4.25	-0.037443614011123\\
-4.2	-0.038988499733242\\
-4.15	-0.0405933724485003\\
-4.1	-0.0422603222744912\\
-4.05	-0.0439914959647211\\
-4	-0.0457890972218354\\
-3.95	-0.0476553868940826\\
-3.9	-0.0495926830422208\\
-3.85	-0.0516033608630646\\
-3.8	-0.0536898524547974\\
-3.75	-0.0558546464080216\\
-3.7	-0.0581002872052976\\
-3.65	-0.0604293744106062\\
-3.6	-0.0628445616287729\\
-3.55	-0.0653485552133947\\
-3.5	-0.0679441127002166\\
-3.45	-0.0706340409412012\\
-3.4	-0.0734211939127174\\
-3.35	-0.0763084701693379\\
-3.3	-0.079298809912666\\
-3.25	-0.0823951916424093\\
-3.2	-0.085600628354569\\
-3.15	-0.0889181632491084\\
-3.1	-0.0923508649067935\\
-3.05	-0.0959018218920604\\
-3	-0.0995741367357279\\
-2.95	-0.103370919248154\\
-2.9	-0.107295279109994\\
-2.85	-0.111350317684064\\
-2.8	-0.115539118987914\\
-2.75	-0.119864739762577\\
-2.7	-0.124330198568537\\
-2.65	-0.128938463835284\\
-2.6	-0.133692440785801\\
-2.55	-0.138594957152047\\
-2.5	-0.143648747591823\\
-2.45	-0.148856436711414\\
-2.4	-0.154220520592001\\
-2.35	-0.159743346711046\\
-2.3	-0.165427092142626\\
-2.25	-0.17127373991303\\
-2.2	-0.177285053379734\\
-2.15	-0.183462548493258\\
-2.1	-0.189807463792122\\
-2.05	-0.196320727971401\\
-2	-0.203002924854919\\
-1.95	-0.209854255590108\\
-1.9	-0.216874497872821\\
-1.85	-0.224062961996919\\
-1.8	-0.231418443510221\\
-1.75	-0.238939172244362\\
-1.7	-0.246622757471192\\
-1.65	-0.254466128922499\\
-1.6	-0.262465473393052\\
-1.55	-0.270616166629097\\
-1.5	-0.278912700185537\\
-1.45	-0.287348602914902\\
-1.4	-0.295916356729928\\
-1.35	-0.304607306258923\\
-1.3	-0.313411561989114\\
-1.25	-0.322317896467714\\
-1.2	-0.331313633103422\\
-1.15	-0.340384527082482\\
-1.1	-0.349514637882983\\
-1.05	-0.358686192838934\\
-1	-0.367879441171442\\
-0.95	-0.377072497868139\\
-0.9	-0.386241176753569\\
-0.85	-0.395358812052572\\
-0.8	-0.404396067705499\\
-0.75	-0.413320733648388\\
-0.7	-0.422097508222698\\
-0.65	-0.430687765827838\\
-0.6	-0.439049308875221\\
-0.55	-0.447136103044877\\
-0.5	-0.454897994784475\\
-0.45	-0.462280409925786\\
-0.4	-0.469224032224948\\
-0.35	-0.475664460560132\\
-0.3	-0.481531843443117\\
-0.25	-0.486750489419628\\
-0.2	-0.491238451846789\\
-0.15	-0.494907086444408\\
-0.1	-0.497660579919778\\
-0.05	-0.499395447862875\\
0	-0.5\\
0.0499999999999999	-0.499353770778611\\
0.1	-0.497326913134041\\
0.15	-0.49377955315952\\
0.2	-0.488561103264068\\
0.25	-0.481509531257903\\
0.3	-0.472450582651601\\
0.35	-0.461196953292809\\
0.4	-0.447547409292381\\
0.45	-0.431285851009796\\
0.5	-0.412180317675032\\
0.55	-0.389981929020164\\
0.6	-0.364423760078102\\
0.65	-0.335219645077432\\
0.7	-0.302062906120572\\
0.75	-0.264625002076584\\
0.8	-0.222554092849247\\
0.85	-0.175473513894449\\
0.9	-0.122980155557847\\
0.95	-0.0646427414828962\\
1	0\\
};

\addplot[area legend, draw=black, fill=black, forget plot]
table[row sep=crcr] {%
x	y\\
0.0308755760368662	-0.5\\
0	-0.5\\
0	-0.5\\
0	-0.5\\
0	-0.5\\
0	-0.5\\
0.0617511520737324	-0.5\\
0.0617511520737324	-0.539642219229328\\
0.2	-0.5\\
0.0617511520737324	-0.460357780770672\\
0.0617511520737324	-0.5\\
}--cycle;
\addplot [color=black, forget plot]
  table[row sep=crcr]{%
-5	0\\
-4.95	0\\
-4.9	0\\
-4.85	0\\
-4.8	0\\
-4.75	0\\
-4.7	0\\
-4.65	0\\
-4.6	0\\
-4.55	0\\
-4.5	0\\
-4.45	0\\
-4.4	0\\
-4.35	0\\
-4.3	0\\
-4.25	0\\
-4.2	0\\
-4.15	0\\
-4.1	0\\
-4.05	0\\
-4	0\\
-3.95	0\\
-3.9	0\\
-3.85	0\\
-3.8	0\\
-3.75	0\\
-3.7	0\\
-3.65	0\\
-3.6	0\\
-3.55	0\\
-3.5	0\\
-3.45	0\\
-3.4	0\\
-3.35	0\\
-3.3	0\\
-3.25	0\\
-3.2	0\\
-3.15	0\\
-3.1	0\\
-3.05	0\\
-3	0\\
-2.95	0\\
-2.9	0\\
-2.85	0\\
-2.8	0\\
-2.75	0\\
-2.7	0\\
-2.65	0\\
-2.6	0\\
-2.55	0\\
-2.5	0\\
-2.45	0\\
-2.4	0\\
-2.35	0\\
-2.3	0\\
-2.25	0\\
-2.2	0\\
-2.15	0\\
-2.1	0\\
-2.05	0\\
-2	0\\
-1.95	0\\
-1.9	0\\
-1.85	0\\
-1.8	0\\
-1.75	0\\
-1.7	0\\
-1.65	0\\
-1.6	0\\
-1.55	0\\
-1.5	0\\
-1.45	0\\
-1.4	0\\
-1.35	0\\
-1.3	0\\
-1.25	0\\
-1.2	0\\
-1.15	0\\
-1.1	0\\
-1.05	0\\
-1	0\\
-0.95	0\\
-0.9	0\\
-0.85	0\\
-0.8	0\\
-0.75	0\\
-0.7	0\\
-0.65	0\\
-0.6	0\\
-0.55	0\\
-0.5	0\\
-0.45	0\\
-0.4	0\\
-0.35	0\\
-0.3	0\\
-0.25	0\\
-0.2	0\\
-0.15	0\\
-0.1	0\\
-0.05	0\\
0	0\\
0.0499999999999999	0\\
0.1	0\\
0.15	0\\
0.2	0\\
0.25	0\\
0.3	0\\
0.35	0\\
0.4	0\\
0.45	0\\
0.5	0\\
0.55	0\\
0.6	0\\
0.65	0\\
0.7	0\\
0.75	0\\
0.8	0\\
0.85	0\\
0.9	0\\
0.95	0\\
1	-0\\
};

\addplot[area legend, draw=black, fill=black, forget plot]
table[row sep=crcr] {%
x	y\\
0.0308755760368662	-2.22044604925031e-16\\
0	0\\
0	0\\
0	0\\
0	0\\
0	0\\
0.0617511520737324	-4.44089209850063e-16\\
0.0617511520737324	-0.0396422192293282\\
0.2	0\\
0.0617511520737315	0.0396422192293278\\
0.0617511520737324	-4.44089209850063e-16\\
}--cycle;
\addplot [color=black, forget plot]
  table[row sep=crcr]{%
-5	0.0202138409972564\\
-4.95	0.0210731415639301\\
-4.9	0.0219674200592268\\
-4.85	0.0228980043314854\\
-4.8	0.0238662664421581\\
-4.75	0.0248736237089718\\
-4.7	0.0259215397398331\\
-4.65	0.0270115254537854\\
-4.6	0.028145140084974\\
-4.55	0.0293239921651911\\
-4.5	0.0305497404801663\\
-4.45	0.0318240949943276\\
-4.4	0.0331488177382848\\
-4.35	0.0345257236527837\\
-4.3	0.0359566813823325\\
-4.25	0.037443614011123\\
-4.2	0.038988499733242\\
-4.15	0.0405933724485003\\
-4.1	0.0422603222744912\\
-4.05	0.0439914959647211\\
-4	0.0457890972218354\\
-3.95	0.0476553868940826\\
-3.9	0.0495926830422208\\
-3.85	0.0516033608630646\\
-3.8	0.0536898524547974\\
-3.75	0.0558546464080216\\
-3.7	0.0581002872052976\\
-3.65	0.0604293744106062\\
-3.6	0.0628445616287729\\
-3.55	0.0653485552133947\\
-3.5	0.0679441127002166\\
-3.45	0.0706340409412012\\
-3.4	0.0734211939127174\\
-3.35	0.0763084701693379\\
-3.3	0.079298809912666\\
-3.25	0.0823951916424093\\
-3.2	0.085600628354569\\
-3.15	0.0889181632491084\\
-3.1	0.0923508649067935\\
-3.05	0.0959018218920604\\
-3	0.0995741367357279\\
-2.95	0.103370919248154\\
-2.9	0.107295279109994\\
-2.85	0.111350317684064\\
-2.8	0.115539118987914\\
-2.75	0.119864739762577\\
-2.7	0.124330198568537\\
-2.65	0.128938463835284\\
-2.6	0.133692440785801\\
-2.55	0.138594957152047\\
-2.5	0.143648747591823\\
-2.45	0.148856436711414\\
-2.4	0.154220520592001\\
-2.35	0.159743346711046\\
-2.3	0.165427092142626\\
-2.25	0.17127373991303\\
-2.2	0.177285053379734\\
-2.15	0.183462548493258\\
-2.1	0.189807463792122\\
-2.05	0.196320727971401\\
-2	0.203002924854919\\
-1.95	0.209854255590108\\
-1.9	0.216874497872821\\
-1.85	0.224062961996919\\
-1.8	0.231418443510221\\
-1.75	0.238939172244362\\
-1.7	0.246622757471192\\
-1.65	0.254466128922499\\
-1.6	0.262465473393052\\
-1.55	0.270616166629097\\
-1.5	0.278912700185537\\
-1.45	0.287348602914902\\
-1.4	0.295916356729928\\
-1.35	0.304607306258923\\
-1.3	0.313411561989114\\
-1.25	0.322317896467714\\
-1.2	0.331313633103422\\
-1.15	0.340384527082482\\
-1.1	0.349514637882983\\
-1.05	0.358686192838934\\
-1	0.367879441171442\\
-0.95	0.377072497868139\\
-0.9	0.386241176753569\\
-0.85	0.395358812052572\\
-0.8	0.404396067705499\\
-0.75	0.413320733648388\\
-0.7	0.422097508222698\\
-0.65	0.430687765827838\\
-0.6	0.439049308875221\\
-0.55	0.447136103044877\\
-0.5	0.454897994784475\\
-0.45	0.462280409925786\\
-0.4	0.469224032224948\\
-0.35	0.475664460560132\\
-0.3	0.481531843443117\\
-0.25	0.486750489419628\\
-0.2	0.491238451846789\\
-0.15	0.494907086444408\\
-0.1	0.497660579919778\\
-0.05	0.499395447862875\\
0	0.5\\
0.0499999999999999	0.499353770778611\\
0.1	0.497326913134041\\
0.15	0.49377955315952\\
0.2	0.488561103264068\\
0.25	0.481509531257903\\
0.3	0.472450582651601\\
0.35	0.461196953292809\\
0.4	0.447547409292381\\
0.45	0.431285851009796\\
0.5	0.412180317675032\\
0.55	0.389981929020164\\
0.6	0.364423760078102\\
0.65	0.335219645077432\\
0.7	0.302062906120572\\
0.75	0.264625002076584\\
0.8	0.222554092849247\\
0.85	0.175473513894449\\
0.9	0.122980155557847\\
0.95	0.0646427414828962\\
1	-0\\
};

\addplot[area legend, draw=black, fill=black, forget plot]
table[row sep=crcr] {%
x	y\\
0.0308755760368662	0.5\\
0	0.5\\
0	0.5\\
0	0.5\\
0	0.5\\
0	0.5\\
0.0617511520737324	0.5\\
0.0617511520737324	0.460357780770672\\
0.2	0.5\\
0.0617511520737324	0.539642219229328\\
0.0617511520737324	0.5\\
}--cycle;
\addplot [color=black, forget plot]
  table[row sep=crcr]{%
-5	0.0404276819945128\\
-4.95	0.0421462831278601\\
-4.9	0.0439348401184536\\
-4.85	0.0457960086629708\\
-4.8	0.0477325328843162\\
-4.75	0.0497472474179436\\
-4.7	0.0518430794796661\\
-4.65	0.0540230509075708\\
-4.6	0.0562902801699481\\
-4.55	0.0586479843303822\\
-4.5	0.0610994809603327\\
-4.45	0.0636481899886552\\
-4.4	0.0662976354765696\\
-4.35	0.0690514473055673\\
-4.3	0.071913362764665\\
-4.25	0.0748872280222461\\
-4.2	0.0779769994664841\\
-4.15	0.0811867448970006\\
-4.1	0.0845206445489824\\
-4.05	0.0879829919294422\\
-4	0.0915781944436709\\
-3.95	0.0953107737881653\\
-3.9	0.0991853660844415\\
-3.85	0.103206721726129\\
-3.8	0.107379704909595\\
-3.75	0.111709292816043\\
-3.7	0.116200574410595\\
-3.65	0.120858748821212\\
-3.6	0.125689123257546\\
-3.55	0.130697110426789\\
-3.5	0.135888225400433\\
-3.45	0.141268081882402\\
-3.4	0.146842387825435\\
-3.35	0.152616940338676\\
-3.3	0.158597619825332\\
-3.25	0.164790383284819\\
-3.2	0.171201256709138\\
-3.15	0.177836326498217\\
-3.1	0.184701729813587\\
-3.05	0.191803643784121\\
-3	0.199148273471456\\
-2.95	0.206741838496308\\
-2.9	0.214590558219988\\
-2.85	0.222700635368128\\
-2.8	0.231078237975828\\
-2.75	0.239729479525153\\
-2.7	0.248660397137074\\
-2.65	0.257876927670568\\
-2.6	0.267384881571602\\
-2.55	0.277189914304094\\
-2.5	0.287297495183646\\
-2.45	0.297712873422828\\
-2.4	0.308441041184003\\
-2.35	0.319486693422091\\
-2.3	0.330854184285252\\
-2.25	0.342547479826059\\
-2.2	0.354570106759468\\
-2.15	0.366925096986515\\
-2.1	0.379614927584244\\
-2.05	0.392641455942803\\
-2	0.406005849709838\\
-1.95	0.419708511180215\\
-1.9	0.433748995745642\\
-1.85	0.448125923993839\\
-1.8	0.462836887020442\\
-1.75	0.477878344488724\\
-1.7	0.493245514942384\\
-1.65	0.508932257844998\\
-1.6	0.524930946786104\\
-1.55	0.541232333258195\\
-1.5	0.557825400371075\\
-1.45	0.574697205829804\\
-1.4	0.591832713459855\\
-1.35	0.609214612517845\\
-1.3	0.626823123978229\\
-1.25	0.644635792935428\\
-1.2	0.662627266206845\\
-1.15	0.680769054164964\\
-1.1	0.699029275765967\\
-1.05	0.717372385677868\\
-1	0.735758882342885\\
-0.95	0.754144995736277\\
-0.9	0.772482353507138\\
-0.85	0.790717624105144\\
-0.8	0.808792135410999\\
-0.75	0.826641467296776\\
-0.7	0.844195016445396\\
-0.65	0.861375531655676\\
-0.6	0.878098617750442\\
-0.55	0.894272206089754\\
-0.5	0.90979598956895\\
-0.45	0.924560819851571\\
-0.4	0.938448064449895\\
-0.35	0.951328921120263\\
-0.3	0.963063686886233\\
-0.25	0.973500978839256\\
-0.2	0.982476903693578\\
-0.15	0.989814172888816\\
-0.1	0.995321159839556\\
-0.05	0.99879089572575\\
0	1\\
0.0499999999999999	0.998707541557223\\
0.1	0.994653826268083\\
0.15	0.987559106319041\\
0.2	0.977122206528136\\
0.25	0.963019062515806\\
0.3	0.944901165303202\\
0.35	0.922393906585617\\
0.4	0.895094818584762\\
0.45	0.862571702019593\\
0.5	0.824360635350064\\
0.55	0.779963858040328\\
0.6	0.728847520156204\\
0.65	0.670439290154864\\
0.7	0.604125812241143\\
0.75	0.529250004153169\\
0.8	0.445108185698493\\
0.85	0.350947027788899\\
0.9	0.245960311115695\\
0.95	0.129285482965792\\
1	-0\\
};

\addplot[area legend, draw=black, fill=black, forget plot]
table[row sep=crcr] {%
x	y\\
0.0308755760368662	1\\
0	1\\
0	1\\
0	1\\
0	1\\
0	1\\
0.0617511520737324	1\\
0.0617511520737324	0.960357780770672\\
0.2	1\\
0.0617511520737324	1.03964221922933\\
0.0617511520737324	1\\
}--cycle;
\addplot [color=black, forget plot]
  table[row sep=crcr]{%
-5	0.0606415229917692\\
-4.95	0.0632194246917902\\
-4.9	0.0659022601776804\\
-4.85	0.0686940129944562\\
-4.8	0.0715987993264743\\
-4.75	0.0746208711269155\\
-4.7	0.0777646192194992\\
-4.65	0.0810345763613562\\
-4.6	0.0844354202549221\\
-4.55	0.0879719764955733\\
-4.5	0.091649221440499\\
-4.45	0.0954722849829828\\
-4.4	0.0994464532148544\\
-4.35	0.103577170958351\\
-4.3	0.107870044146997\\
-4.25	0.112330842033369\\
-4.2	0.116965499199726\\
-4.15	0.121780117345501\\
-4.1	0.126780966823474\\
-4.05	0.131974487894163\\
-4	0.137367291665506\\
-3.95	0.142966160682248\\
-3.9	0.148778049126662\\
-3.85	0.154810082589194\\
-3.8	0.161069557364392\\
-3.75	0.167563939224065\\
-3.7	0.174300861615893\\
-3.65	0.181288123231819\\
-3.6	0.188533684886319\\
-3.55	0.196045665640184\\
-3.5	0.20383233810065\\
-3.45	0.211902122823604\\
-3.4	0.220263581738152\\
-3.35	0.228925410508014\\
-3.3	0.237896429737998\\
-3.25	0.247185574927228\\
-3.2	0.256801885063707\\
-3.15	0.266754489747325\\
-3.1	0.277052594720381\\
-3.05	0.287705465676181\\
-3	0.298722410207184\\
-2.95	0.310112757744462\\
-2.9	0.321885837329982\\
-2.85	0.334050953052192\\
-2.8	0.346617356963742\\
-2.75	0.35959421928773\\
-2.7	0.372990595705611\\
-2.65	0.386815391505852\\
-2.6	0.401077322357403\\
-2.55	0.415784871456141\\
-2.5	0.430946242775469\\
-2.45	0.446569310134243\\
-2.4	0.462661561776004\\
-2.35	0.479230040133137\\
-2.3	0.496281276427879\\
-2.25	0.513821219739089\\
-2.2	0.531855160139203\\
-2.15	0.550387645479773\\
-2.1	0.569422391376366\\
-2.05	0.588962183914204\\
-2	0.609008774564757\\
-1.95	0.629562766770323\\
-1.9	0.650623493618462\\
-1.85	0.672188885990758\\
-1.8	0.694255330530663\\
-1.75	0.716817516733086\\
-1.7	0.739868272413575\\
-1.65	0.763398386767497\\
-1.6	0.787396420179156\\
-1.55	0.811848499887292\\
-1.5	0.836738100556612\\
-1.45	0.862045808744706\\
-1.4	0.887749070189783\\
-1.35	0.913821918776768\\
-1.3	0.940234685967343\\
-1.25	0.966953689403142\\
-1.2	0.993940899310267\\
-1.15	1.02115358124745\\
-1.1	1.04854391364895\\
-1.05	1.0760585785168\\
-1	1.10363832351433\\
-0.95	1.13121749360442\\
-0.9	1.15872353026071\\
-0.85	1.18607643615772\\
-0.8	1.2131882031165\\
-0.75	1.23996220094516\\
-0.7	1.26629252466809\\
-0.65	1.29206329748351\\
-0.6	1.31714792662566\\
-0.55	1.34140830913463\\
-0.5	1.36469398435343\\
-0.45	1.38684122977736\\
-0.4	1.40767209667484\\
-0.35	1.42699338168039\\
-0.3	1.44459553032935\\
-0.25	1.46025146825888\\
-0.2	1.47371535554037\\
-0.15	1.48472125933322\\
-0.1	1.49298173975933\\
-0.05	1.49818634358862\\
0	1.5\\
0.0499999999999999	1.49806131233583\\
0.1	1.49198073940212\\
0.15	1.48133865947856\\
0.2	1.4656833097922\\
0.25	1.44452859377371\\
0.3	1.4173517479548\\
0.35	1.38359085987843\\
0.4	1.34264222787714\\
0.45	1.29385755302939\\
0.5	1.2365409530251\\
0.55	1.16994578706049\\
0.6	1.09327128023431\\
0.65	1.0056589352323\\
0.7	0.906188718361715\\
0.75	0.793875006229753\\
0.8	0.66766227854774\\
0.85	0.526420541683348\\
0.9	0.368940466673542\\
0.95	0.193928224448689\\
1	-0\\
};

\addplot[area legend, draw=black, fill=black, forget plot]
table[row sep=crcr] {%
x	y\\
0.0308755760368662	1.5\\
0	1.5\\
0	1.5\\
0	1.5\\
0	1.5\\
0	1.5\\
0.0617511520737324	1.5\\
0.0617511520737324	1.46035778077067\\
0.2	1.5\\
0.0617511520737324	1.53964221922933\\
0.0617511520737324	1.5\\
}--cycle;
\addplot [color=black, forget plot]
  table[row sep=crcr]{%
-5	0.0808553639890256\\
-4.95	0.0842925662557202\\
-4.9	0.0878696802369072\\
-4.85	0.0915920173259415\\
-4.8	0.0954650657686323\\
-4.75	0.0994944948358873\\
-4.7	0.103686158959332\\
-4.65	0.108046101815142\\
-4.6	0.112580560339896\\
-4.55	0.117295968660764\\
-4.5	0.122198961920665\\
-4.45	0.12729637997731\\
-4.4	0.132595270953139\\
-4.35	0.138102894611135\\
-4.3	0.14382672552933\\
-4.25	0.149774456044492\\
-4.2	0.155953998932968\\
-4.15	0.162373489794001\\
-4.1	0.169041289097965\\
-4.05	0.175965983858884\\
-4	0.183156388887342\\
-3.95	0.190621547576331\\
-3.9	0.198370732168883\\
-3.85	0.206413443452259\\
-3.8	0.21475940981919\\
-3.75	0.223418585632087\\
-3.7	0.23240114882119\\
-3.65	0.241717497642425\\
-3.6	0.251378246515092\\
-3.55	0.261394220853579\\
-3.5	0.271776450800866\\
-3.45	0.282536163764805\\
-3.4	0.29368477565087\\
-3.35	0.305233880677352\\
-3.3	0.317195239650664\\
-3.25	0.329580766569637\\
-3.2	0.342402513418276\\
-3.15	0.355672652996434\\
-3.1	0.369403459627174\\
-3.05	0.383607287568242\\
-3	0.398296546942912\\
-2.95	0.413483676992616\\
-2.9	0.429181116439976\\
-2.85	0.445401270736256\\
-2.8	0.462156475951657\\
-2.75	0.479458959050307\\
-2.7	0.497320794274148\\
-2.65	0.515753855341136\\
-2.6	0.534769763143204\\
-2.55	0.554379828608187\\
-2.5	0.574594990367292\\
-2.45	0.595425746845657\\
-2.4	0.616882082368005\\
-2.35	0.638973386844183\\
-2.3	0.661708368570505\\
-2.25	0.685094959652118\\
-2.2	0.709140213518937\\
-2.15	0.733850193973031\\
-2.1	0.759229855168488\\
-2.05	0.785282911885606\\
-2	0.812011699419676\\
-1.95	0.83941702236043\\
-1.9	0.867497991491283\\
-1.85	0.896251847987677\\
-1.8	0.925673774040884\\
-1.75	0.955756688977448\\
-1.7	0.986491029884767\\
-1.65	1.01786451569\\
-1.6	1.04986189357221\\
-1.55	1.08246466651639\\
-1.5	1.11565080074215\\
-1.45	1.14939441165961\\
-1.4	1.18366542691971\\
-1.35	1.21842922503569\\
-1.3	1.25364624795646\\
-1.25	1.28927158587086\\
-1.2	1.32525453241369\\
-1.15	1.36153810832993\\
-1.1	1.39805855153193\\
-1.05	1.43474477135574\\
-1	1.47151776468577\\
-0.95	1.50828999147255\\
-0.9	1.54496470701428\\
-0.85	1.58143524821029\\
-0.8	1.617584270822\\
-0.75	1.65328293459355\\
-0.7	1.68839003289079\\
-0.65	1.72275106331135\\
-0.6	1.75619723550088\\
-0.55	1.78854441217951\\
-0.5	1.8195919791379\\
-0.45	1.84912163970314\\
-0.4	1.87689612889979\\
-0.35	1.90265784224053\\
-0.3	1.92612737377247\\
-0.25	1.94700195767851\\
-0.2	1.96495380738716\\
-0.15	1.97962834577763\\
-0.1	1.99064231967911\\
-0.05	1.9975817914515\\
0	2\\
0.0499999999999999	1.99741508311445\\
0.1	1.98930765253617\\
0.15	1.97511821263808\\
0.2	1.95424441305627\\
0.25	1.92603812503161\\
0.3	1.8898023306064\\
0.35	1.84478781317123\\
0.4	1.79018963716952\\
0.45	1.72514340403919\\
0.5	1.64872127070013\\
0.55	1.55992771608066\\
0.6	1.45769504031241\\
0.65	1.34087858030973\\
0.7	1.20825162448229\\
0.75	1.05850000830634\\
0.8	0.890216371396987\\
0.85	0.701894055577797\\
0.9	0.49192062223139\\
0.95	0.258570965931585\\
1	-0\\
};

\addplot[area legend, draw=black, fill=black, forget plot]
table[row sep=crcr] {%
x	y\\
0.0265975597542343	2\\
0	2\\
0	2\\
0	2\\
0	2\\
0	2\\
0.0531951195084686	2\\
0.0531951195084686	1.95790437791218\\
0.2	2\\
0.0531951195084686	2.04209562208782\\
0.0531951195084686	2\\
}--cycle;
\addplot [color=black, forget plot]
  table[row sep=crcr]{%
-5	0.101069204986282\\
-4.95	0.10536570781965\\
-4.9	0.109837100296134\\
-4.85	0.114490021657427\\
-4.8	0.11933133221079\\
-4.75	0.124368118544859\\
-4.7	0.129607698699165\\
-4.65	0.135057627268927\\
-4.6	0.14072570042487\\
-4.55	0.146619960825956\\
-4.5	0.152748702400832\\
-4.45	0.159120474971638\\
-4.4	0.165744088691424\\
-4.35	0.172628618263918\\
-4.3	0.179783406911662\\
-4.25	0.187218070055615\\
-4.2	0.19494249866621\\
-4.15	0.202966862242502\\
-4.1	0.211301611372456\\
-4.05	0.219957479823606\\
-4	0.228945486109177\\
-3.95	0.238276934470413\\
-3.9	0.247963415211104\\
-3.85	0.258016804315323\\
-3.8	0.268449262273987\\
-3.75	0.279273232040108\\
-3.7	0.290501436026488\\
-3.65	0.302146872053031\\
-3.6	0.314222808143865\\
-3.55	0.326742776066974\\
-3.5	0.339720563501083\\
-3.45	0.353170204706006\\
-3.4	0.367105969563587\\
-3.35	0.38154235084669\\
-3.3	0.39649404956333\\
-3.25	0.411975958212046\\
-3.2	0.428003141772845\\
-3.15	0.444590816245542\\
-3.1	0.461754324533968\\
-3.05	0.479509109460302\\
-3	0.497870683678639\\
-2.95	0.51685459624077\\
-2.9	0.536476395549971\\
-2.85	0.55675158842032\\
-2.8	0.577695594939571\\
-2.75	0.599323698812883\\
-2.7	0.621650992842686\\
-2.65	0.64469231917642\\
-2.6	0.668462203929005\\
-2.55	0.692974785760234\\
-2.5	0.718243737959114\\
-2.45	0.744282183557071\\
-2.4	0.771102602960006\\
-2.35	0.798716733555228\\
-2.3	0.827135460713131\\
-2.25	0.856368699565148\\
-2.2	0.886425266898671\\
-2.15	0.917312742466289\\
-2.1	0.94903731896061\\
-2.05	0.981603639857007\\
-2	1.0150146242746\\
-1.95	1.04927127795054\\
-1.9	1.0843724893641\\
-1.85	1.1203148099846\\
-1.8	1.15709221755111\\
-1.75	1.19469586122181\\
-1.7	1.23311378735596\\
-1.65	1.2723306446125\\
-1.6	1.31232736696526\\
-1.55	1.35308083314549\\
-1.5	1.39456350092769\\
-1.45	1.43674301457451\\
-1.4	1.47958178364964\\
-1.35	1.52303653129461\\
-1.3	1.56705780994557\\
-1.25	1.61158948233857\\
-1.2	1.65656816551711\\
-1.15	1.70192263541241\\
-1.1	1.74757318941492\\
-1.05	1.79343096419467\\
-1	1.83939720585721\\
-0.95	1.88536248934069\\
-0.9	1.93120588376785\\
-0.85	1.97679406026286\\
-0.8	2.0219803385275\\
-0.75	2.06660366824194\\
-0.7	2.11048754111349\\
-0.65	2.15343882913919\\
-0.6	2.19524654437611\\
-0.55	2.23568051522439\\
-0.5	2.27448997392238\\
-0.45	2.31140204962893\\
-0.4	2.34612016112474\\
-0.35	2.37832230280066\\
-0.3	2.40765921721558\\
-0.25	2.43375244709814\\
-0.2	2.45619225923395\\
-0.15	2.47453543222204\\
-0.1	2.48830289959889\\
-0.05	2.49697723931437\\
0	2.5\\
0.0499999999999999	2.49676885389306\\
0.1	2.48663456567021\\
0.15	2.4688977657976\\
0.2	2.44280551632034\\
0.25	2.40754765628952\\
0.3	2.36225291325801\\
0.35	2.30598476646404\\
0.4	2.23773704646191\\
0.45	2.15642925504898\\
0.5	2.06090158837516\\
0.55	1.94990964510082\\
0.6	1.82211880039051\\
0.65	1.67609822538716\\
0.7	1.51031453060286\\
0.75	1.32312501038292\\
0.8	1.11277046424623\\
0.85	0.877367569472247\\
0.9	0.614900777789237\\
0.95	0.323213707414481\\
1	-0\\
};

\addplot[area legend, draw=black, fill=black, forget plot]
table[row sep=crcr] {%
x	y\\
0.0192940248433371	2.5\\
0	2.5\\
0	2.5\\
0	2.5\\
0	2.5\\
0	2.5\\
0.0385880496866742	2.5\\
0.0385880496866742	2.45371586804132\\
0.2	2.5\\
0.0385880496866742	2.54628413195868\\
0.0385880496866742	2.5\\
}--cycle;
\addplot [color=black, forget plot]
  table[row sep=crcr]{%
-5	0.121283045983538\\
-4.95	0.12643884938358\\
-4.9	0.131804520355361\\
-4.85	0.137388025988912\\
-4.8	0.143197598652949\\
-4.75	0.149241742253831\\
-4.7	0.155529238438998\\
-4.65	0.162069152722712\\
-4.6	0.168870840509844\\
-4.55	0.175943952991147\\
-4.5	0.183298442880998\\
-4.45	0.190944569965966\\
-4.4	0.198892906429709\\
-4.35	0.207154341916702\\
-4.3	0.215740088293995\\
-4.25	0.224661684066738\\
-4.2	0.233930998399452\\
-4.15	0.243560234691002\\
-4.1	0.253561933646947\\
-4.05	0.263948975788327\\
-4	0.274734583331013\\
-3.95	0.285932321364496\\
-3.9	0.297556098253325\\
-3.85	0.309620165178388\\
-3.8	0.322139114728785\\
-3.75	0.33512787844813\\
-3.7	0.348601723231785\\
-3.65	0.362576246463637\\
-3.6	0.377067369772637\\
-3.55	0.392091331280368\\
-3.5	0.4076646762013\\
-3.45	0.423804245647207\\
-3.4	0.440527163476304\\
-3.35	0.457850821016028\\
-3.3	0.475792859475996\\
-3.25	0.494371149854456\\
-3.2	0.513603770127414\\
-3.15	0.53350897949465\\
-3.1	0.554105189440761\\
-3.05	0.575410931352362\\
-3	0.597444820414367\\
-2.95	0.620225515488924\\
-2.9	0.643771674659965\\
-2.85	0.668101906104384\\
-2.8	0.693234713927485\\
-2.75	0.71918843857546\\
-2.7	0.745981191411223\\
-2.65	0.773630783011704\\
-2.6	0.802154644714806\\
-2.55	0.831569742912281\\
-2.5	0.861892485550937\\
-2.45	0.893138620268485\\
-2.4	0.925323123552008\\
-2.35	0.958460080266274\\
-2.3	0.992562552855757\\
-2.25	1.02764243947818\\
-2.2	1.06371032027841\\
-2.15	1.10077529095955\\
-2.1	1.13884478275273\\
-2.05	1.17792436782841\\
-2	1.21801754912951\\
-1.95	1.25912553354065\\
-1.9	1.30124698723692\\
-1.85	1.34437777198152\\
-1.8	1.38851066106133\\
-1.75	1.43363503346617\\
-1.7	1.47973654482715\\
-1.65	1.52679677353499\\
-1.6	1.57479284035831\\
-1.55	1.62369699977458\\
-1.5	1.67347620111322\\
-1.45	1.72409161748941\\
-1.4	1.77549814037957\\
-1.35	1.82764383755354\\
-1.3	1.88046937193469\\
-1.25	1.93390737880628\\
-1.2	1.98788179862053\\
-1.15	2.04230716249489\\
-1.1	2.0970878272979\\
-1.05	2.1521171570336\\
-1	2.20727664702865\\
-0.95	2.26243498720883\\
-0.9	2.31744706052141\\
-0.85	2.37215287231543\\
-0.8	2.426376406233\\
-0.75	2.47992440189033\\
-0.7	2.53258504933619\\
-0.65	2.58412659496703\\
-0.6	2.63429585325133\\
-0.55	2.68281661826926\\
-0.5	2.72938796870685\\
-0.45	2.77368245955471\\
-0.4	2.81534419334968\\
-0.35	2.85398676336079\\
-0.3	2.8891910606587\\
-0.25	2.92050293651777\\
-0.2	2.94743071108073\\
-0.15	2.96944251866645\\
-0.1	2.98596347951867\\
-0.05	2.99637268717725\\
0	3\\
0.0499999999999999	2.99612262467167\\
0.1	2.98396147880425\\
0.15	2.96267731895712\\
0.2	2.93136661958441\\
0.25	2.88905718754742\\
0.3	2.83470349590961\\
0.35	2.76718171975685\\
0.4	2.68528445575429\\
0.45	2.58771510605878\\
0.5	2.47308190605019\\
0.55	2.33989157412098\\
0.6	2.18654256046861\\
0.65	2.01131787046459\\
0.7	1.81237743672343\\
0.75	1.58775001245951\\
0.8	1.33532455709548\\
0.85	1.0528410833667\\
0.9	0.737880933347085\\
0.95	0.387856448897377\\
1	-0\\
};

\addplot[area legend, draw=black, fill=black, forget plot]
table[row sep=crcr] {%
x	y\\
0.0119904899324408	3\\
0	3\\
0	3\\
0	3\\
0	3\\
0	3\\
0.0239809798648816	3\\
0.0239809798648825	2.94952735817047\\
0.2	3\\
0.0239809798648807	3.05047264182953\\
0.0239809798648816	3\\
}--cycle;
\addplot [color=black, forget plot]
  table[row sep=crcr]{%
1	-0\\
1.05	-0.0580111535181752\\
1.1	-0.12197089792218\\
1.15	-0.192336719376927\\
1.2	-0.269597378470333\\
1.25	-0.354274914555761\\
1.3	-0.446926773412269\\
1.35	-0.548148065599067\\
1.4	-0.658573963312831\\
1.45	-0.778882244013657\\
1.5	-0.90979598956895\\
1.55	-1.05208645017593\\
1.6	-1.20657608286415\\
1.65	-1.37414177495149\\
1.7	-1.55571826343161\\
1.75	-1.75230176191066\\
1.8	-1.96495380738716\\
1.85	-2.1948053398839\\
1.9	-2.44306102869709\\
1.95	-2.71100385982703\\
2	-3\\
};

\addplot[area legend, draw=black, fill=black, forget plot]
table[row sep=crcr] {%
x	y\\
1.43520090604112	-0.759828460426891\\
1.5	-0.90979598956895\\
1.5	-0.90979598956895\\
1.5	-0.90979598956895\\
1.5	-0.90979598956895\\
1.5	-0.90979598956895\\
1.37040181208223	-0.609860931284832\\
1.32368119412978	-0.630048326048471\\
1.3	-0.446926773412268\\
1.41712243003469	-0.589673536521194\\
1.37040181208223	-0.609860931284832\\
}--cycle;
\addplot [color=black, forget plot]
  table[row sep=crcr]{%
1	-0\\
1.05	-0.0386741023454502\\
1.1	-0.0813139319481199\\
1.15	-0.128224479584618\\
1.2	-0.179731585646889\\
1.25	-0.236183276370507\\
1.3	-0.297951182274846\\
1.35	-0.365432043732711\\
1.4	-0.439049308875221\\
1.45	-0.519254829342438\\
1.5	-0.606530659712633\\
1.55	-0.701390966783951\\
1.6	-0.804384055242767\\
1.65	-0.916094516634327\\
1.7	-1.0371455089544\\
1.75	-1.16820117460711\\
1.8	-1.30996920492477\\
1.85	-1.4632035599226\\
1.9	-1.62870735246473\\
1.95	-1.80733590655136\\
2	-2\\
};

\addplot[area legend, draw=black, fill=black, forget plot]
table[row sep=crcr] {%
x	y\\
1.44826807681646	-0.526713610598493\\
1.5	-0.606530659712633\\
1.5	-0.606530659712633\\
1.5	-0.606530659712633\\
1.5	-0.606530659712633\\
1.5	-0.606530659712633\\
1.39653615363292	-0.446896561484353\\
1.3538267534145	-0.474577858097498\\
1.3	-0.297951182274846\\
1.43924555385135	-0.419215264871208\\
1.39653615363292	-0.446896561484353\\
}--cycle;
\addplot [color=black, forget plot]
  table[row sep=crcr]{%
1	-0\\
1.05	-0.0193370511727251\\
1.1	-0.04065696597406\\
1.15	-0.064112239792309\\
1.2	-0.0898657928234443\\
1.25	-0.118091638185254\\
1.3	-0.148975591137423\\
1.35	-0.182716021866356\\
1.4	-0.219524654437611\\
1.45	-0.259627414671219\\
1.5	-0.303265329856317\\
1.55	-0.350695483391975\\
1.6	-0.402192027621384\\
1.65	-0.458047258317164\\
1.7	-0.518572754477202\\
1.75	-0.584100587303554\\
1.8	-0.654984602462386\\
1.85	-0.731601779961299\\
1.9	-0.814353676232363\\
1.95	-0.903667953275678\\
2	-1\\
};

\addplot[area legend, draw=black, fill=black, forget plot]
table[row sep=crcr] {%
x	y\\
1.47026743190549	-0.280328179042599\\
1.5	-0.303265329856317\\
1.5	-0.303265329856317\\
1.5	-0.303265329856317\\
1.5	-0.303265329856317\\
1.5	-0.303265329856317\\
1.44053486381098	-0.257391028228882\\
1.40944723747998	-0.297688751964923\\
1.3	-0.148975591137423\\
1.47162249014198	-0.21709330449284\\
1.44053486381098	-0.257391028228882\\
}--cycle;
\addplot [color=black, forget plot]
  table[row sep=crcr]{%
1	0\\
1.05	0\\
1.1	0\\
1.15	0\\
1.2	0\\
1.25	0\\
1.3	0\\
1.35	0\\
1.4	0\\
1.45	0\\
1.5	0\\
1.55	0\\
1.6	0\\
1.65	0\\
1.7	0\\
1.75	0\\
1.8	0\\
1.85	0\\
1.9	0\\
1.95	0\\
2	0\\
};

\addplot[area legend, draw=black, fill=black, forget plot]
table[row sep=crcr] {%
x	y\\
1.48874676747085	0\\
1.5	0\\
1.5	0\\
1.5	0\\
1.5	0\\
1.5	0\\
1.47749353494171	0\\
1.47749353494171	-0.0508954521465541\\
1.3	0\\
1.47749353494171	0.0508954521465546\\
1.47749353494171	0\\
}--cycle;
\addplot [color=black, forget plot]
  table[row sep=crcr]{%
1	0\\
1.05	0.0193370511727251\\
1.1	0.04065696597406\\
1.15	0.064112239792309\\
1.2	0.0898657928234443\\
1.25	0.118091638185254\\
1.3	0.148975591137423\\
1.35	0.182716021866356\\
1.4	0.219524654437611\\
1.45	0.259627414671219\\
1.5	0.303265329856317\\
1.55	0.350695483391975\\
1.6	0.402192027621384\\
1.65	0.458047258317164\\
1.7	0.518572754477202\\
1.75	0.584100587303554\\
1.8	0.654984602462386\\
1.85	0.731601779961299\\
1.9	0.814353676232363\\
1.95	0.903667953275678\\
2	1\\
};

\addplot[area legend, draw=black, fill=black, forget plot]
table[row sep=crcr] {%
x	y\\
1.47026743190549	0.280328179042599\\
1.5	0.303265329856317\\
1.5	0.303265329856317\\
1.5	0.303265329856317\\
1.5	0.303265329856317\\
1.5	0.303265329856317\\
1.44053486381098	0.257391028228882\\
1.47162249014198	0.21709330449284\\
1.3	0.148975591137423\\
1.40944723747998	0.297688751964923\\
1.44053486381098	0.257391028228882\\
}--cycle;
\addplot [color=black, forget plot]
  table[row sep=crcr]{%
1	0\\
1.05	0.0386741023454502\\
1.1	0.0813139319481199\\
1.15	0.128224479584618\\
1.2	0.179731585646889\\
1.25	0.236183276370507\\
1.3	0.297951182274846\\
1.35	0.365432043732711\\
1.4	0.439049308875221\\
1.45	0.519254829342438\\
1.5	0.606530659712633\\
1.55	0.701390966783951\\
1.6	0.804384055242767\\
1.65	0.916094516634327\\
1.7	1.0371455089544\\
1.75	1.16820117460711\\
1.8	1.30996920492477\\
1.85	1.4632035599226\\
1.9	1.62870735246473\\
1.95	1.80733590655136\\
2	2\\
};

\addplot[area legend, draw=black, fill=black, forget plot]
table[row sep=crcr] {%
x	y\\
1.44826807681646	0.526713610598494\\
1.5	0.606530659712634\\
1.5	0.606530659712634\\
1.5	0.606530659712634\\
1.5	0.606530659712634\\
1.5	0.606530659712634\\
1.39653615363292	0.446896561484353\\
1.3538267534145	0.474577858097497\\
1.3	0.297951182274846\\
1.43924555385135	0.41921526487121\\
1.39653615363292	0.446896561484353\\
}--cycle;
\addplot [color=black, forget plot]
  table[row sep=crcr]{%
1	0\\
1.05	0.0580111535181752\\
1.1	0.12197089792218\\
1.15	0.192336719376927\\
1.2	0.269597378470333\\
1.25	0.354274914555761\\
1.3	0.446926773412269\\
1.35	0.548148065599067\\
1.4	0.658573963312831\\
1.45	0.778882244013657\\
1.5	0.90979598956895\\
1.55	1.05208645017593\\
1.6	1.20657608286415\\
1.65	1.37414177495149\\
1.7	1.55571826343161\\
1.75	1.75230176191066\\
1.8	1.96495380738716\\
1.85	2.1948053398839\\
1.9	2.44306102869709\\
1.95	2.71100385982703\\
2	3\\
};

\addplot[area legend, draw=black, fill=black, forget plot]
table[row sep=crcr] {%
x	y\\
1.43520090604112	0.759828460426891\\
1.5	0.90979598956895\\
1.5	0.90979598956895\\
1.5	0.90979598956895\\
1.5	0.90979598956895\\
1.5	0.90979598956895\\
1.37040181208223	0.609860931284831\\
1.32368119412978	0.63004832604847\\
1.3	0.446926773412268\\
1.41712243003469	0.589673536521192\\
1.37040181208223	0.609860931284831\\
}--cycle;

\addplot[area legend, draw=black, fill=black, forget plot]
table[row sep=crcr] {%
x	y\\
1	0.788746767470854\\
1	0.8\\
1	0.8\\
1	0.8\\
1	0.8\\
1	0.8\\
1	0.777493534941709\\
0.949104547853445	0.777493534941709\\
1	0.600000000000001\\
1.05089545214656	0.777493534941709\\
1	0.777493534941709\\
}--cycle;

\addplot[area legend, draw=black, fill=black, forget plot]
table[row sep=crcr] {%
x	y\\
1	-0.788746767470854\\
1	-0.8\\
1	-0.8\\
1	-0.8\\
1	-0.8\\
1	-0.8\\
1	-0.777493534941708\\
0.949104547853445	-0.777493534941708\\
1	-0.6\\
1.05089545214656	-0.777493534941708\\
1	-0.777493534941708\\
}--cycle;
\end{axis}
\end{tikzpicture}%
            \end{center}
            \caption{Orbite per l'esempio \framref{dafasdflkjnsadfkjnasdfkjnadskfjnasdkfjnldskjn}.}\label{fig:orbitedivattelaapesce}
        \end{figure}
    \end{enumerate}
\end{enumerate}
}{dafasdflkjnsadfkjnasdfkjnadskfjnasdkfjnldskjn}{}
%% BEGIN Pendolo Piano Senza Attrito
\paragrafo{Pendolo Piano Senza Attrito}{% 
    Consideriamo un pendolo piano senza attrito, e sia $ \theta $ l'angolo rispetto alla verticale. La legge che regola il moto è \[
        \theta''(t)= -\frac{g}{l}\,\sin \theta(t)
    \]Se $ \alpha= \sqrt{g/l} $, definendo $ x(t)= \theta(\alpha\,t) $, l'equazione del moto diventa: \[
        x ''(t)= -\sin x(t)
    \]equazione non lineare del secondo ordine. La trasformo, tramite la trasformazione $ x'(t)=y(t) $, in un sistema di equazioni del primo ordine: \[
        \begin{cases}
            x'(t)=y(t)\\ 
            y'(t) = -\sin x(t)
        \end{cases}
    \]\begin{enumerate}
        \item \emph{Equilibri:} sono i punti che soddisfano il sistema \[
            \begin{cases}
                y=0\\ 
                \sin x = 0
            \end{cases}
        \]ovvero tutti quelli nella forma $ (k\, \pi,0) $, per $ k \in \Z $.
        \item \emph{Punti con tangente orizzontale:} $ x'=0 $ $ \leadsto $ $ y=0 $; 
        
        Noto inoltre che $ y'>0 $ quando $ x \in [\pi +2k\,\pi, 2(k+1)\,\pi] $. 
        \item \emph{Punti con tangente verticale:} $ y'=0 $ $ \leadsto $ $ \sin x = 0 $ 
        
        $\implies$ $ x = k\, \pi $, $ k \in \Z $. 
        \begin{figure}
            \begin{center}
                \begin{tikzpicture}[scale=1.2]
                    \draw [-Latex] (-3.5,0) -- (3.5,0);
                    \draw [-Latex] (0,-2) -- (0,2);
                    \foreach \k in{-3,-2,...,3}{
                        \draw [dashed] (\k, -2) -- (\k,2);
                        \fill [black] (\k,0) circle (0.07);
                        \draw [-Latex, thick] (\k, 1) -- (\k+0.5, 1);
                        \draw [-Latex, thick] (\k, -1) -- (\k-0.5, -1);
                    };
                    \foreach \k in{-2.5,-0.5,1.5}{
                        \draw [-Latex, thick] (\k, -0.3) -- (\k,0.3);
                        \draw [-Latex, thick] (\k+1, +0.3) -- (\k+1,-0.3);
                    };
                    \node at (-3,-2.3) {$-3\,\pi$};
                    \node at (-2,-2.3) {$-2\,\pi$};
                    \node at (-1,-2.3) {$-\pi$};
                    \node at (1,-2.3) {$\pi$};
                    \node at (2,-2.3) {$2\,\pi$};
                    \node at (3,-2.3) {$3\,\pi$};
                \end{tikzpicture}
            \end{center}
            \caption{Punti di equilibrio e a tangente orizzontale/verticale per il pendolo senza attrito}
        \end{figure}
        \item \emph{Equazione delle orbite:} quando $ y\neq 0 $, si ha che \begin{align*}
            y'(x) &= \frac{-\sin x}{y(x)}\\ 
            y(x)\,y'(x) &= -\sin x\\ 
            y_{c}(x) &= \pm \sqrt{2\left(\cos x + c\right)} 
        \end{align*}Le orbite variano al variare di $ c $: \begin{itemize}
            \item $ c<-1 $: non ci sono soluzioni;
            \item $ c=-1 $: ottengo i valori per cui $ \cos x = 1 $, ovvero i punti di equilibrio $ (2n\,\pi, 0) $, $ n \in \Z $;
            \item $ c \in (-1,1) $: ottengo soluzioni definite sull'intervallo $ [-x_{c}, x_{c}  ] $ modulo $ 2\pi $, dove \[
                x_{c}=\arccos{-c};
            \]
            \item $ c\ge1 $: si creano infinite eterocline. 
        \end{itemize}
        In definitiva, le orbite sono quelle illustrate nella figura \ref{fig:sodaufnaisdufjaiuhfaoisfjdaosi}
        \begin{figure}
            \begin{center}
                % This file was created by matlab2tikz.
%
%The latest updates can be retrieved from
%  http://www.mathworks.com/matlabcentral/fileexchange/22022-matlab2tikz-matlab2tikz
%where you can also make suggestions and rate matlab2tikz.
%
\begin{tikzpicture}


  \begin{axis}[%
    width=\textwidth,
    axis equal,
    axis lines=middle,
    xmin=-5,
    xmax=5,
    ymin=-5,
    ymax=5,
    axis background/.style={fill=white}
    ]
    \fill [black] (0,0) circle (0.1);
    \fill [black] (3.14156,0) circle (0.1);
    \fill [black] (-3.14156,0) circle (0.1);

  \addplot [color=black, forget plot]
    table[row sep=crcr]{%
  -5	0\\
  -4.95	0\\
  -4.9	0\\
  -4.85	0\\
  -4.8	0\\
  -4.75	0\\
  -4.7	0\\
  -4.65	0\\
  -4.6	0\\
  -4.55	0\\
  -4.5	0\\
  -4.45	0\\
  -4.4	0\\
  -4.35	0\\
  -4.3	0\\
  -4.25	0\\
  -4.2	0\\
  -4.15	0\\
  -4.1	0\\
  -4.05	0\\
  -4	0\\
  -3.95	0\\
  -3.9	0\\
  -3.85	0\\
  -3.8	0\\
  -3.75	0\\
  -3.7	0\\
  -3.65	0\\
  -3.6	0\\
  -3.55	0\\
  -3.5	0\\
  -3.45	0\\
  -3.4	0\\
  -3.35	0\\
  -3.3	0\\
  -3.25	0\\
  -3.2	0\\
  -3.15	0\\
  -3.1	0\\
  -3.05	0\\
  -3	0\\
  -2.95	0\\
  -2.9	0\\
  -2.85	0\\
  -2.8	0\\
  -2.75	0\\
  -2.7	0\\
  -2.65	0\\
  -2.6	0\\
  -2.55	0\\
  -2.5	0\\
  -2.45	0\\
  -2.4	0\\
  -2.35	0\\
  -2.3	0\\
  -2.25	0\\
  -2.2	0\\
  -2.15	0\\
  -2.1	0\\
  -2.05	0\\
  -2	0\\
  -1.95	0\\
  -1.9	0\\
  -1.85	0\\
  -1.8	0\\
  -1.75	0\\
  -1.7	0\\
  -1.65	0\\
  -1.6	0\\
  -1.55	0\\
  -1.5	0\\
  -1.45	0\\
  -1.4	0\\
  -1.35	0\\
  -1.3	0\\
  -1.25	0\\
  -1.2	0\\
  -1.15	0\\
  -1.1	0\\
  -1.05	0\\
  -1	0\\
  -0.95	0\\
  -0.899999999999999	0\\
  -0.85	0\\
  -0.8	0\\
  -0.75	0\\
  -0.7	0\\
  -0.649999999999999	0\\
  -0.6	0\\
  -0.55	0\\
  -0.5	0\\
  -0.45	0\\
  -0.399999999999999	0\\
  -0.35	0\\
  -0.3	0\\
  -0.25	0\\
  -0.199999999999999	0\\
  -0.149999999999999	0\\
  -0.0999999999999996	0\\
  -0.0499999999999998	0\\
  0	0\\
  0.0499999999999998	0\\
  0.0999999999999996	0\\
  0.149999999999999	0\\
  0.199999999999999	0\\
  0.25	0\\
  0.3	0\\
  0.35	0\\
  0.399999999999999	0\\
  0.45	0\\
  0.5	0\\
  0.55	0\\
  0.6	0\\
  0.649999999999999	0\\
  0.7	0\\
  0.75	0\\
  0.8	0\\
  0.85	0\\
  0.899999999999999	0\\
  0.95	0\\
  1	0\\
  1.05	0\\
  1.1	0\\
  1.15	0\\
  1.2	0\\
  1.25	0\\
  1.3	0\\
  1.35	0\\
  1.4	0\\
  1.45	0\\
  1.5	0\\
  1.55	0\\
  1.6	0\\
  1.65	0\\
  1.7	0\\
  1.75	0\\
  1.8	0\\
  1.85	0\\
  1.9	0\\
  1.95	0\\
  2	0\\
  2.05	0\\
  2.1	0\\
  2.15	0\\
  2.2	0\\
  2.25	0\\
  2.3	0\\
  2.35	0\\
  2.4	0\\
  2.45	0\\
  2.5	0\\
  2.55	0\\
  2.6	0\\
  2.65	0\\
  2.7	0\\
  2.75	0\\
  2.8	0\\
  2.85	0\\
  2.9	0\\
  2.95	0\\
  3	0\\
  3.05	0\\
  3.1	0\\
  3.15	0\\
  3.2	0\\
  3.25	0\\
  3.3	0\\
  3.35	0\\
  3.4	0\\
  3.45	0\\
  3.5	0\\
  3.55	0\\
  3.6	0\\
  3.65	0\\
  3.7	0\\
  3.75	0\\
  3.8	0\\
  3.85	0\\
  3.9	0\\
  3.95	0\\
  4	0\\
  4.05	0\\
  4.1	0\\
  4.15	0\\
  4.2	0\\
  4.25	0\\
  4.3	0\\
  4.35	0\\
  4.4	0\\
  4.45	0\\
  4.5	0\\
  4.55	0\\
  4.6	0\\
  4.65	0\\
  4.7	0\\
  4.75	0\\
  4.8	0\\
  4.85	0\\
  4.9	0\\
  4.95	0\\
  5	0\\
  };
  \addplot [color=black, forget plot]
    table[row sep=crcr]{%
  -5	-0\\
  -4.95	-0\\
  -4.9	-0\\
  -4.85	-0\\
  -4.8	-0\\
  -4.75	-0\\
  -4.7	-0\\
  -4.65	-0\\
  -4.6	-0\\
  -4.55	-0\\
  -4.5	-0\\
  -4.45	-0\\
  -4.4	-0\\
  -4.35	-0\\
  -4.3	-0\\
  -4.25	-0\\
  -4.2	-0\\
  -4.15	-0\\
  -4.1	-0\\
  -4.05	-0\\
  -4	-0\\
  -3.95	-0\\
  -3.9	-0\\
  -3.85	-0\\
  -3.8	-0\\
  -3.75	-0\\
  -3.7	-0\\
  -3.65	-0\\
  -3.6	-0\\
  -3.55	-0\\
  -3.5	-0\\
  -3.45	-0\\
  -3.4	-0\\
  -3.35	-0\\
  -3.3	-0\\
  -3.25	-0\\
  -3.2	-0\\
  -3.15	-0\\
  -3.1	-0\\
  -3.05	-0\\
  -3	-0\\
  -2.95	-0\\
  -2.9	-0\\
  -2.85	-0\\
  -2.8	-0\\
  -2.75	-0\\
  -2.7	-0\\
  -2.65	-0\\
  -2.6	-0\\
  -2.55	-0\\
  -2.5	-0\\
  -2.45	-0\\
  -2.4	-0\\
  -2.35	-0\\
  -2.3	-0\\
  -2.25	-0\\
  -2.2	-0\\
  -2.15	-0\\
  -2.1	-0\\
  -2.05	-0\\
  -2	-0\\
  -1.95	-0\\
  -1.9	-0\\
  -1.85	-0\\
  -1.8	-0\\
  -1.75	-0\\
  -1.7	-0\\
  -1.65	-0\\
  -1.6	-0\\
  -1.55	-0\\
  -1.5	-0\\
  -1.45	-0\\
  -1.4	-0\\
  -1.35	-0\\
  -1.3	-0\\
  -1.25	-0\\
  -1.2	-0\\
  -1.15	-0\\
  -1.1	-0\\
  -1.05	-0\\
  -1	-0\\
  -0.95	-0\\
  -0.899999999999999	-0\\
  -0.85	-0\\
  -0.8	-0\\
  -0.75	-0\\
  -0.7	-0\\
  -0.649999999999999	-0\\
  -0.6	-0\\
  -0.55	-0\\
  -0.5	-0\\
  -0.45	-0\\
  -0.399999999999999	-0\\
  -0.35	-0\\
  -0.3	-0\\
  -0.25	-0\\
  -0.199999999999999	-0\\
  -0.149999999999999	-0\\
  -0.0999999999999996	-0\\
  -0.0499999999999998	-0\\
  0	-0\\
  0.0499999999999998	-0\\
  0.0999999999999996	-0\\
  0.149999999999999	-0\\
  0.199999999999999	-0\\
  0.25	-0\\
  0.3	-0\\
  0.35	-0\\
  0.399999999999999	-0\\
  0.45	-0\\
  0.5	-0\\
  0.55	-0\\
  0.6	-0\\
  0.649999999999999	-0\\
  0.7	-0\\
  0.75	-0\\
  0.8	-0\\
  0.85	-0\\
  0.899999999999999	-0\\
  0.95	-0\\
  1	-0\\
  1.05	-0\\
  1.1	-0\\
  1.15	-0\\
  1.2	-0\\
  1.25	-0\\
  1.3	-0\\
  1.35	-0\\
  1.4	-0\\
  1.45	-0\\
  1.5	-0\\
  1.55	-0\\
  1.6	-0\\
  1.65	-0\\
  1.7	-0\\
  1.75	-0\\
  1.8	-0\\
  1.85	-0\\
  1.9	-0\\
  1.95	-0\\
  2	-0\\
  2.05	-0\\
  2.1	-0\\
  2.15	-0\\
  2.2	-0\\
  2.25	-0\\
  2.3	-0\\
  2.35	-0\\
  2.4	-0\\
  2.45	-0\\
  2.5	-0\\
  2.55	-0\\
  2.6	-0\\
  2.65	-0\\
  2.7	-0\\
  2.75	-0\\
  2.8	-0\\
  2.85	-0\\
  2.9	-0\\
  2.95	-0\\
  3	-0\\
  3.05	-0\\
  3.1	-0\\
  3.15	-0\\
  3.2	-0\\
  3.25	-0\\
  3.3	-0\\
  3.35	-0\\
  3.4	-0\\
  3.45	-0\\
  3.5	-0\\
  3.55	-0\\
  3.6	-0\\
  3.65	-0\\
  3.7	-0\\
  3.75	-0\\
  3.8	-0\\
  3.85	-0\\
  3.9	-0\\
  3.95	-0\\
  4	-0\\
  4.05	-0\\
  4.1	-0\\
  4.15	-0\\
  4.2	-0\\
  4.25	-0\\
  4.3	-0\\
  4.35	-0\\
  4.4	-0\\
  4.45	-0\\
  4.5	-0\\
  4.55	-0\\
  4.6	-0\\
  4.65	-0\\
  4.7	-0\\
  4.75	-0\\
  4.8	-0\\
  4.85	-0\\
  4.9	-0\\
  4.95	-0\\
  5	-0\\
  };
  
  \addplot[area legend, draw=black, fill=black, forget plot]
  table[row sep=crcr] {%
  x	y\\
  -0.0806451612903221	-1.11022302462516e-16\\
  -0.0999999999999996	-1.11022302462516e-16\\
  -0.0999999999999996	-1.11022302462516e-16\\
  -0.0999999999999996	-1.11022302462516e-16\\
  -0.0999999999999996	-1.11022302462516e-16\\
  -0.0999999999999996	-1.11022302462516e-16\\
  -0.0612903225806445	-1.11022302462516e-16\\
  -0.0612903225806445	-0.0117278276383972\\
  0.0999999999999996	-1.11022302462516e-16\\
  -0.0612903225806436	0.0117278276383972\\
  -0.0612903225806445	-1.11022302462516e-16\\
  }--cycle;
  
  \addplot[area legend, draw=black, fill=black, forget plot]
  table[row sep=crcr] {%
  x	y\\
  0.080645161290323	-1.73472347597681e-18\\
  0.0999999999999996	-1.73472347597681e-18\\
  0.0999999999999996	-1.73472347597681e-18\\
  0.0999999999999996	-1.73472347597681e-18\\
  0.0999999999999996	-1.73472347597681e-18\\
  0.0999999999999996	-1.73472347597681e-18\\
  0.0612903225806463	-1.73472347597681e-18\\
  0.0612903225806463	-0.000175917414575958\\
  -0.0999999999999996	-1.73472347597681e-18\\
  0.0612903225806463	0.000175917414575958\\
  0.0612903225806463	-1.73472347597681e-18\\
  }--cycle;
  \addplot [color=black, forget plot]
    table[row sep=crcr]{%
  -5	0\\
  -4.95	0\\
  -4.9	0\\
  -4.85	0\\
  -4.8	0\\
  -4.75	0\\
  -4.7	0\\
  -4.65	0\\
  -4.6	0\\
  -4.55	0\\
  -4.5	0\\
  -4.45	0\\
  -4.4	0\\
  -4.35	0\\
  -4.3	0\\
  -4.25	0\\
  -4.2	0\\
  -4.15	0\\
  -4.1	0\\
  -4.05	0\\
  -4	0\\
  -3.95	0\\
  -3.9	0\\
  -3.85	0\\
  -3.8	0\\
  -3.75	0\\
  -3.7	0\\
  -3.65	0\\
  -3.6	0\\
  -3.55	0\\
  -3.5	0\\
  -3.45	0\\
  -3.4	0\\
  -3.35	0\\
  -3.3	0\\
  -3.25	0\\
  -3.2	0\\
  -3.15	0\\
  -3.1	0\\
  -3.05	0\\
  -3	0\\
  -2.95	0\\
  -2.9	0\\
  -2.85	0\\
  -2.8	0\\
  -2.75	0\\
  -2.7	0\\
  -2.65	0\\
  -2.6	0\\
  -2.55	0\\
  -2.5	0\\
  -2.45	0\\
  -2.4	0\\
  -2.35	0\\
  -2.3	0\\
  -2.25	0\\
  -2.2	0\\
  -2.15	0\\
  -2.1	0\\
  -2.05	0\\
  -2	0\\
  -1.95	0\\
  -1.9	0\\
  -1.85	0\\
  -1.8	0\\
  -1.75	0\\
  -1.7	0\\
  -1.65	0\\
  -1.6	0\\
  -1.55	0\\
  -1.5	0\\
  -1.45	0\\
  -1.4	0\\
  -1.35	0\\
  -1.3	0\\
  -1.25	0\\
  -1.2	0\\
  -1.15	0\\
  -1.1	0\\
  -1.05	0\\
  -1	0\\
  -0.95	0\\
  -0.899999999999999	0\\
  -0.85	0\\
  -0.8	0\\
  -0.75	0.251749355009386\\
  -0.7	0.360117167834271\\
  -0.649999999999999	0.438369247436579\\
  -0.6	0.500670779873718\\
  -0.55	0.552312451533561\\
  -0.5	0.595957317079626\\
  -0.45	0.633162068277431\\
  -0.399999999999999	0.664922542861776\\
  -0.35	0.691914319619675\\
  -0.3	0.714613866539974\\
  -0.25	0.733365422842726\\
  -0.199999999999999	0.748420440449406\\
  -0.149999999999999	0.75996194369987\\
  -0.0999999999999996	0.76811999749782\\
  -0.0499999999999998	0.772981578557945\\
  0	0.774596669241483\\
  0.0499999999999998	0.772981578557945\\
  0.0999999999999996	0.76811999749782\\
  0.149999999999999	0.75996194369987\\
  0.199999999999999	0.748420440449406\\
  0.25	0.733365422842726\\
  0.3	0.714613866539974\\
  0.35	0.691914319619675\\
  0.399999999999999	0.664922542861776\\
  0.45	0.633162068277431\\
  0.5	0.595957317079626\\
  0.55	0.552312451533561\\
  0.6	0.500670779873718\\
  0.649999999999999	0.438369247436579\\
  0.7	0.360117167834271\\
  0.75	0.251749355009386\\
  0.8	0\\
  0.85	0\\
  0.899999999999999	0\\
  0.95	0\\
  1	0\\
  1.05	0\\
  1.1	0\\
  1.15	0\\
  1.2	0\\
  1.25	0\\
  1.3	0\\
  1.35	0\\
  1.4	0\\
  1.45	0\\
  1.5	0\\
  1.55	0\\
  1.6	0\\
  1.65	0\\
  1.7	0\\
  1.75	0\\
  1.8	0\\
  1.85	0\\
  1.9	0\\
  1.95	0\\
  2	0\\
  2.05	0\\
  2.1	0\\
  2.15	0\\
  2.2	0\\
  2.25	0\\
  2.3	0\\
  2.35	0\\
  2.4	0\\
  2.45	0\\
  2.5	0\\
  2.55	0\\
  2.6	0\\
  2.65	0\\
  2.7	0\\
  2.75	0\\
  2.8	0\\
  2.85	0\\
  2.9	0\\
  2.95	0\\
  3	0\\
  3.05	0\\
  3.1	0\\
  3.15	0\\
  3.2	0\\
  3.25	0\\
  3.3	0\\
  3.35	0\\
  3.4	0\\
  3.45	0\\
  3.5	0\\
  3.55	0\\
  3.6	0\\
  3.65	0\\
  3.7	0\\
  3.75	0\\
  3.8	0\\
  3.85	0\\
  3.9	0\\
  3.95	0\\
  4	0\\
  4.05	0\\
  4.1	0\\
  4.15	0\\
  4.2	0\\
  4.25	0\\
  4.3	0\\
  4.35	0\\
  4.4	0\\
  4.45	0\\
  4.5	0\\
  4.55	0\\
  4.6	0\\
  4.65	0\\
  4.7	0\\
  4.75	0\\
  4.8	0\\
  4.85	0\\
  4.9	0\\
  4.95	0\\
  5	0\\
  };
  \addplot [color=black, forget plot]
    table[row sep=crcr]{%
  -5	-0\\
  -4.95	-0\\
  -4.9	-0\\
  -4.85	-0\\
  -4.8	-0\\
  -4.75	-0\\
  -4.7	-0\\
  -4.65	-0\\
  -4.6	-0\\
  -4.55	-0\\
  -4.5	-0\\
  -4.45	-0\\
  -4.4	-0\\
  -4.35	-0\\
  -4.3	-0\\
  -4.25	-0\\
  -4.2	-0\\
  -4.15	-0\\
  -4.1	-0\\
  -4.05	-0\\
  -4	-0\\
  -3.95	-0\\
  -3.9	-0\\
  -3.85	-0\\
  -3.8	-0\\
  -3.75	-0\\
  -3.7	-0\\
  -3.65	-0\\
  -3.6	-0\\
  -3.55	-0\\
  -3.5	-0\\
  -3.45	-0\\
  -3.4	-0\\
  -3.35	-0\\
  -3.3	-0\\
  -3.25	-0\\
  -3.2	-0\\
  -3.15	-0\\
  -3.1	-0\\
  -3.05	-0\\
  -3	-0\\
  -2.95	-0\\
  -2.9	-0\\
  -2.85	-0\\
  -2.8	-0\\
  -2.75	-0\\
  -2.7	-0\\
  -2.65	-0\\
  -2.6	-0\\
  -2.55	-0\\
  -2.5	-0\\
  -2.45	-0\\
  -2.4	-0\\
  -2.35	-0\\
  -2.3	-0\\
  -2.25	-0\\
  -2.2	-0\\
  -2.15	-0\\
  -2.1	-0\\
  -2.05	-0\\
  -2	-0\\
  -1.95	-0\\
  -1.9	-0\\
  -1.85	-0\\
  -1.8	-0\\
  -1.75	-0\\
  -1.7	-0\\
  -1.65	-0\\
  -1.6	-0\\
  -1.55	-0\\
  -1.5	-0\\
  -1.45	-0\\
  -1.4	-0\\
  -1.35	-0\\
  -1.3	-0\\
  -1.25	-0\\
  -1.2	-0\\
  -1.15	-0\\
  -1.1	-0\\
  -1.05	-0\\
  -1	-0\\
  -0.95	-0\\
  -0.899999999999999	-0\\
  -0.85	-0\\
  -0.8	-0\\
  -0.75	-0.251749355009386\\
  -0.7	-0.360117167834271\\
  -0.649999999999999	-0.438369247436579\\
  -0.6	-0.500670779873718\\
  -0.55	-0.552312451533561\\
  -0.5	-0.595957317079626\\
  -0.45	-0.633162068277431\\
  -0.399999999999999	-0.664922542861776\\
  -0.35	-0.691914319619675\\
  -0.3	-0.714613866539974\\
  -0.25	-0.733365422842726\\
  -0.199999999999999	-0.748420440449406\\
  -0.149999999999999	-0.75996194369987\\
  -0.0999999999999996	-0.76811999749782\\
  -0.0499999999999998	-0.772981578557945\\
  0	-0.774596669241483\\
  0.0499999999999998	-0.772981578557945\\
  0.0999999999999996	-0.76811999749782\\
  0.149999999999999	-0.75996194369987\\
  0.199999999999999	-0.748420440449406\\
  0.25	-0.733365422842726\\
  0.3	-0.714613866539974\\
  0.35	-0.691914319619675\\
  0.399999999999999	-0.664922542861776\\
  0.45	-0.633162068277431\\
  0.5	-0.595957317079626\\
  0.55	-0.552312451533561\\
  0.6	-0.500670779873718\\
  0.649999999999999	-0.438369247436579\\
  0.7	-0.360117167834271\\
  0.75	-0.251749355009386\\
  0.8	-0\\
  0.85	-0\\
  0.899999999999999	-0\\
  0.95	-0\\
  1	-0\\
  1.05	-0\\
  1.1	-0\\
  1.15	-0\\
  1.2	-0\\
  1.25	-0\\
  1.3	-0\\
  1.35	-0\\
  1.4	-0\\
  1.45	-0\\
  1.5	-0\\
  1.55	-0\\
  1.6	-0\\
  1.65	-0\\
  1.7	-0\\
  1.75	-0\\
  1.8	-0\\
  1.85	-0\\
  1.9	-0\\
  1.95	-0\\
  2	-0\\
  2.05	-0\\
  2.1	-0\\
  2.15	-0\\
  2.2	-0\\
  2.25	-0\\
  2.3	-0\\
  2.35	-0\\
  2.4	-0\\
  2.45	-0\\
  2.5	-0\\
  2.55	-0\\
  2.6	-0\\
  2.65	-0\\
  2.7	-0\\
  2.75	-0\\
  2.8	-0\\
  2.85	-0\\
  2.9	-0\\
  2.95	-0\\
  3	-0\\
  3.05	-0\\
  3.1	-0\\
  3.15	-0\\
  3.2	-0\\
  3.25	-0\\
  3.3	-0\\
  3.35	-0\\
  3.4	-0\\
  3.45	-0\\
  3.5	-0\\
  3.55	-0\\
  3.6	-0\\
  3.65	-0\\
  3.7	-0\\
  3.75	-0\\
  3.8	-0\\
  3.85	-0\\
  3.9	-0\\
  3.95	-0\\
  4	-0\\
  4.05	-0\\
  4.1	-0\\
  4.15	-0\\
  4.2	-0\\
  4.25	-0\\
  4.3	-0\\
  4.35	-0\\
  4.4	-0\\
  4.45	-0\\
  4.5	-0\\
  4.55	-0\\
  4.6	-0\\
  4.65	-0\\
  4.7	-0\\
  4.75	-0\\
  4.8	-0\\
  4.85	-0\\
  4.9	-0\\
  4.95	-0\\
  5	-0\\
  };
  
  \addplot[area legend, draw=black, fill=black, forget plot]
  table[row sep=crcr] {%
  x	y\\
  -0.0806451612903221	0.774596669241484\\
  -0.0999999999999996	0.774596669241484\\
  -0.0999999999999996	0.774596669241484\\
  -0.0999999999999996	0.774596669241484\\
  -0.0999999999999996	0.774596669241484\\
  -0.0999999999999996	0.774596669241484\\
  -0.0612903225806445	0.774596669241484\\
  -0.0612903225806445	0.728347413473933\\
  0.0999999999999996	0.774596669241484\\
  -0.0612903225806436	0.820845925009034\\
  -0.0612903225806445	0.774596669241484\\
  }--cycle;
  
  \addplot[area legend, draw=black, fill=black, forget plot]
  table[row sep=crcr] {%
  x	y\\
  0.080645161290323	-0.774596669241483\\
  0.0999999999999996	-0.774596669241483\\
  0.0999999999999996	-0.774596669241483\\
  0.0999999999999996	-0.774596669241483\\
  0.0999999999999996	-0.774596669241483\\
  0.0999999999999996	-0.774596669241483\\
  0.0612903225806463	-0.774596669241483\\
  0.0612903225806463	-0.820845925009033\\
  -0.0999999999999996	-0.774596669241483\\
  0.0612903225806463	-0.728347413473934\\
  0.0612903225806463	-0.774596669241483\\
  }--cycle;
  \addplot [color=black, forget plot]
    table[row sep=crcr]{%
  -5	0\\
  -4.95	0\\
  -4.9	0\\
  -4.85	0\\
  -4.8	0\\
  -4.75	0\\
  -4.7	0\\
  -4.65	0\\
  -4.6	0\\
  -4.55	0\\
  -4.5	0\\
  -4.45	0\\
  -4.4	0\\
  -4.35	0\\
  -4.3	0\\
  -4.25	0\\
  -4.2	0\\
  -4.15	0\\
  -4.1	0\\
  -4.05	0\\
  -4	0\\
  -3.95	0\\
  -3.9	0\\
  -3.85	0\\
  -3.8	0\\
  -3.75	0\\
  -3.7	0\\
  -3.65	0\\
  -3.6	0\\
  -3.55	0\\
  -3.5	0\\
  -3.45	0\\
  -3.4	0\\
  -3.35	0\\
  -3.3	0\\
  -3.25	0\\
  -3.2	0\\
  -3.15	0\\
  -3.1	0\\
  -3.05	0\\
  -3	0\\
  -2.95	0\\
  -2.9	0\\
  -2.85	0\\
  -2.8	0\\
  -2.75	0\\
  -2.7	0\\
  -2.65	0\\
  -2.6	0\\
  -2.55	0\\
  -2.5	0\\
  -2.45	0\\
  -2.4	0\\
  -2.35	0\\
  -2.3	0\\
  -2.25	0\\
  -2.2	0\\
  -2.15	0\\
  -2.1	0\\
  -2.05	0\\
  -2	0\\
  -1.95	0\\
  -1.9	0\\
  -1.85	0\\
  -1.8	0\\
  -1.75	0\\
  -1.7	0\\
  -1.65	0\\
  -1.6	0\\
  -1.55	0\\
  -1.5	0\\
  -1.45	0\\
  -1.4	0\\
  -1.35	0\\
  -1.3	0\\
  -1.25	0\\
  -1.2	0\\
  -1.15	0.130287688475599\\
  -1.1	0.327402264578539\\
  -1.05	0.441748905809006\\
  -1	0.529721258527803\\
  -0.95	0.602798622201284\\
  -0.899999999999999	0.665747652298774\\
  -0.85	0.721086882261746\\
  -0.8	0.770333316619716\\
  -0.75	0.81448004134395\\
  -0.7	0.854215648749762\\
  -0.649999999999999	0.890037975087643\\
  -0.6	0.922318399371582\\
  -0.55	0.95134065618947\\
  -0.5	0.977325495308879\\
  -0.45	1.00044700244708\\
  -0.399999999999999	1.02084376277948\\
  -0.35	1.03862670180135\\
  -0.3	1.05388470823483\\
  -0.25	1.06668872845891\\
  -0.199999999999999	1.07709477562677\\
  -0.149999999999999	1.08514614493721\\
  -0.0999999999999996	1.090875029761\\
  -0.0499999999999998	1.09430366936693\\
  0	1.09544511501033\\
  0.0499999999999998	1.09430366936693\\
  0.0999999999999996	1.090875029761\\
  0.149999999999999	1.08514614493721\\
  0.199999999999999	1.07709477562677\\
  0.25	1.06668872845891\\
  0.3	1.05388470823483\\
  0.35	1.03862670180135\\
  0.399999999999999	1.02084376277948\\
  0.45	1.00044700244708\\
  0.5	0.977325495308879\\
  0.55	0.95134065618947\\
  0.6	0.922318399371582\\
  0.649999999999999	0.890037975087643\\
  0.7	0.854215648749762\\
  0.75	0.81448004134395\\
  0.8	0.770333316619716\\
  0.85	0.721086882261746\\
  0.899999999999999	0.665747652298774\\
  0.95	0.602798622201284\\
  1	0.529721258527803\\
  1.05	0.441748905809006\\
  1.1	0.327402264578539\\
  1.15	0.130287688475599\\
  1.2	0\\
  1.25	0\\
  1.3	0\\
  1.35	0\\
  1.4	0\\
  1.45	0\\
  1.5	0\\
  1.55	0\\
  1.6	0\\
  1.65	0\\
  1.7	0\\
  1.75	0\\
  1.8	0\\
  1.85	0\\
  1.9	0\\
  1.95	0\\
  2	0\\
  2.05	0\\
  2.1	0\\
  2.15	0\\
  2.2	0\\
  2.25	0\\
  2.3	0\\
  2.35	0\\
  2.4	0\\
  2.45	0\\
  2.5	0\\
  2.55	0\\
  2.6	0\\
  2.65	0\\
  2.7	0\\
  2.75	0\\
  2.8	0\\
  2.85	0\\
  2.9	0\\
  2.95	0\\
  3	0\\
  3.05	0\\
  3.1	0\\
  3.15	0\\
  3.2	0\\
  3.25	0\\
  3.3	0\\
  3.35	0\\
  3.4	0\\
  3.45	0\\
  3.5	0\\
  3.55	0\\
  3.6	0\\
  3.65	0\\
  3.7	0\\
  3.75	0\\
  3.8	0\\
  3.85	0\\
  3.9	0\\
  3.95	0\\
  4	0\\
  4.05	0\\
  4.1	0\\
  4.15	0\\
  4.2	0\\
  4.25	0\\
  4.3	0\\
  4.35	0\\
  4.4	0\\
  4.45	0\\
  4.5	0\\
  4.55	0\\
  4.6	0\\
  4.65	0\\
  4.7	0\\
  4.75	0\\
  4.8	0\\
  4.85	0\\
  4.9	0\\
  4.95	0\\
  5	0\\
  };
  \addplot [color=black, forget plot]
    table[row sep=crcr]{%
  -5	-0\\
  -4.95	-0\\
  -4.9	-0\\
  -4.85	-0\\
  -4.8	-0\\
  -4.75	-0\\
  -4.7	-0\\
  -4.65	-0\\
  -4.6	-0\\
  -4.55	-0\\
  -4.5	-0\\
  -4.45	-0\\
  -4.4	-0\\
  -4.35	-0\\
  -4.3	-0\\
  -4.25	-0\\
  -4.2	-0\\
  -4.15	-0\\
  -4.1	-0\\
  -4.05	-0\\
  -4	-0\\
  -3.95	-0\\
  -3.9	-0\\
  -3.85	-0\\
  -3.8	-0\\
  -3.75	-0\\
  -3.7	-0\\
  -3.65	-0\\
  -3.6	-0\\
  -3.55	-0\\
  -3.5	-0\\
  -3.45	-0\\
  -3.4	-0\\
  -3.35	-0\\
  -3.3	-0\\
  -3.25	-0\\
  -3.2	-0\\
  -3.15	-0\\
  -3.1	-0\\
  -3.05	-0\\
  -3	-0\\
  -2.95	-0\\
  -2.9	-0\\
  -2.85	-0\\
  -2.8	-0\\
  -2.75	-0\\
  -2.7	-0\\
  -2.65	-0\\
  -2.6	-0\\
  -2.55	-0\\
  -2.5	-0\\
  -2.45	-0\\
  -2.4	-0\\
  -2.35	-0\\
  -2.3	-0\\
  -2.25	-0\\
  -2.2	-0\\
  -2.15	-0\\
  -2.1	-0\\
  -2.05	-0\\
  -2	-0\\
  -1.95	-0\\
  -1.9	-0\\
  -1.85	-0\\
  -1.8	-0\\
  -1.75	-0\\
  -1.7	-0\\
  -1.65	-0\\
  -1.6	-0\\
  -1.55	-0\\
  -1.5	-0\\
  -1.45	-0\\
  -1.4	-0\\
  -1.35	-0\\
  -1.3	-0\\
  -1.25	-0\\
  -1.2	-0\\
  -1.15	-0.130287688475599\\
  -1.1	-0.327402264578539\\
  -1.05	-0.441748905809006\\
  -1	-0.529721258527803\\
  -0.95	-0.602798622201284\\
  -0.899999999999999	-0.665747652298774\\
  -0.85	-0.721086882261746\\
  -0.8	-0.770333316619716\\
  -0.75	-0.81448004134395\\
  -0.7	-0.854215648749762\\
  -0.649999999999999	-0.890037975087643\\
  -0.6	-0.922318399371582\\
  -0.55	-0.95134065618947\\
  -0.5	-0.977325495308879\\
  -0.45	-1.00044700244708\\
  -0.399999999999999	-1.02084376277948\\
  -0.35	-1.03862670180135\\
  -0.3	-1.05388470823483\\
  -0.25	-1.06668872845891\\
  -0.199999999999999	-1.07709477562677\\
  -0.149999999999999	-1.08514614493721\\
  -0.0999999999999996	-1.090875029761\\
  -0.0499999999999998	-1.09430366936693\\
  0	-1.09544511501033\\
  0.0499999999999998	-1.09430366936693\\
  0.0999999999999996	-1.090875029761\\
  0.149999999999999	-1.08514614493721\\
  0.199999999999999	-1.07709477562677\\
  0.25	-1.06668872845891\\
  0.3	-1.05388470823483\\
  0.35	-1.03862670180135\\
  0.399999999999999	-1.02084376277948\\
  0.45	-1.00044700244708\\
  0.5	-0.977325495308879\\
  0.55	-0.95134065618947\\
  0.6	-0.922318399371582\\
  0.649999999999999	-0.890037975087643\\
  0.7	-0.854215648749762\\
  0.75	-0.81448004134395\\
  0.8	-0.770333316619716\\
  0.85	-0.721086882261746\\
  0.899999999999999	-0.665747652298774\\
  0.95	-0.602798622201284\\
  1	-0.529721258527803\\
  1.05	-0.441748905809006\\
  1.1	-0.327402264578539\\
  1.15	-0.130287688475599\\
  1.2	-0\\
  1.25	-0\\
  1.3	-0\\
  1.35	-0\\
  1.4	-0\\
  1.45	-0\\
  1.5	-0\\
  1.55	-0\\
  1.6	-0\\
  1.65	-0\\
  1.7	-0\\
  1.75	-0\\
  1.8	-0\\
  1.85	-0\\
  1.9	-0\\
  1.95	-0\\
  2	-0\\
  2.05	-0\\
  2.1	-0\\
  2.15	-0\\
  2.2	-0\\
  2.25	-0\\
  2.3	-0\\
  2.35	-0\\
  2.4	-0\\
  2.45	-0\\
  2.5	-0\\
  2.55	-0\\
  2.6	-0\\
  2.65	-0\\
  2.7	-0\\
  2.75	-0\\
  2.8	-0\\
  2.85	-0\\
  2.9	-0\\
  2.95	-0\\
  3	-0\\
  3.05	-0\\
  3.1	-0\\
  3.15	-0\\
  3.2	-0\\
  3.25	-0\\
  3.3	-0\\
  3.35	-0\\
  3.4	-0\\
  3.45	-0\\
  3.5	-0\\
  3.55	-0\\
  3.6	-0\\
  3.65	-0\\
  3.7	-0\\
  3.75	-0\\
  3.8	-0\\
  3.85	-0\\
  3.9	-0\\
  3.95	-0\\
  4	-0\\
  4.05	-0\\
  4.1	-0\\
  4.15	-0\\
  4.2	-0\\
  4.25	-0\\
  4.3	-0\\
  4.35	-0\\
  4.4	-0\\
  4.45	-0\\
  4.5	-0\\
  4.55	-0\\
  4.6	-0\\
  4.65	-0\\
  4.7	-0\\
  4.75	-0\\
  4.8	-0\\
  4.85	-0\\
  4.9	-0\\
  4.95	-0\\
  5	-0\\
  };
  
  \addplot[area legend, draw=black, fill=black, forget plot]
  table[row sep=crcr] {%
  x	y\\
  -0.0806451612903221	1.09544511501033\\
  -0.0999999999999996	1.09544511501033\\
  -0.0999999999999996	1.09544511501033\\
  -0.0999999999999996	1.09544511501033\\
  -0.0999999999999996	1.09544511501033\\
  -0.0999999999999996	1.09544511501033\\
  -0.0612903225806445	1.09544511501033\\
  -0.0612903225806445	1.04919585924278\\
  0.0999999999999996	1.09544511501033\\
  -0.0612903225806436	1.14169437077788\\
  -0.0612903225806445	1.09544511501033\\
  }--cycle;
  
  \addplot[area legend, draw=black, fill=black, forget plot]
  table[row sep=crcr] {%
  x	y\\
  0.080645161290323	-1.09544511501033\\
  0.0999999999999996	-1.09544511501033\\
  0.0999999999999996	-1.09544511501033\\
  0.0999999999999996	-1.09544511501033\\
  0.0999999999999996	-1.09544511501033\\
  0.0999999999999996	-1.09544511501033\\
  0.0612903225806463	-1.09544511501033\\
  0.0612903225806463	-1.14169437077788\\
  -0.0999999999999996	-1.09544511501033\\
  0.0612903225806463	-1.04919585924278\\
  0.0612903225806463	-1.09544511501033\\
  }--cycle;
  \addplot [color=black, forget plot]
    table[row sep=crcr]{%
  -5	1.08042786474917\\
  -4.95	1.03477673239637\\
  -4.9	0.986420163442106\\
  -4.85	0.935069101297767\\
  -4.8	0.880339688347\\
  -4.75	0.821708163386462\\
  -4.7	0.758434356470103\\
  -4.65	0.689422200676782\\
  -4.6	0.612939594193335\\
  -4.55	0.52597297201722\\
  -4.5	0.422384186657645\\
  -4.45	0.284996481418188\\
  -4.4	0\\
  -4.35	0\\
  -4.3	0\\
  -4.25	0\\
  -4.2	0\\
  -4.15	0\\
  -4.1	0\\
  -4.05	0\\
  -4	0\\
  -3.95	0\\
  -3.9	0\\
  -3.85	0\\
  -3.8	0\\
  -3.75	0\\
  -3.7	0\\
  -3.65	0\\
  -3.6	0\\
  -3.55	0\\
  -3.5	0\\
  -3.45	0\\
  -3.4	0\\
  -3.35	0\\
  -3.3	0\\
  -3.25	0\\
  -3.2	0\\
  -3.15	0\\
  -3.1	0\\
  -3.05	0\\
  -3	0\\
  -2.95	0\\
  -2.9	0\\
  -2.85	0\\
  -2.8	0\\
  -2.75	0\\
  -2.7	0\\
  -2.65	0\\
  -2.6	0\\
  -2.55	0\\
  -2.5	0\\
  -2.45	0\\
  -2.4	0\\
  -2.35	0\\
  -2.3	0\\
  -2.25	0\\
  -2.2	0\\
  -2.15	0\\
  -2.1	0\\
  -2.05	0\\
  -2	0\\
  -1.95	0\\
  -1.9	0\\
  -1.85	0.220951366483611\\
  -1.8	0.381570190939788\\
  -1.75	0.493465184892527\\
  -1.7	0.585073509406255\\
  -1.65	0.664648946727919\\
  -1.6	0.73593542882336\\
  -1.55	0.800992918574306\\
  -1.5	0.861089079791055\\
  -1.45	0.917063541274395\\
  -1.4	0.96950208138017\\
  -1.35	1.01882941368322\\
  -1.3	1.06536268812512\\
  -1.25	1.10934427694496\\
  -1.2	1.15096286167424\\
  -1.15	1.19036754062278\\
  -1.1	1.22767758098417\\
  -1.05	1.26298934903801\\
  -1	1.29638135274165\\
  -0.95	1.32791798652167\\
  -0.899999999999999	1.35765236218309\\
  -0.85	1.38562848259191\\
  -0.8	1.41188293377827\\
  -0.75	1.43644621818836\\
  -0.7	1.45934381643565\\
  -0.649999999999999	1.48059704075691\\
  -0.6	1.50022372658859\\
  -0.55	1.51823879680339\\
  -0.5	1.53465472461422\\
  -0.45	1.5494819149333\\
  -0.399999999999999	1.56272901937789\\
  -0.35	1.57440319667319\\
  -0.3	1.58451032759374\\
  -0.25	1.59305519158041\\
  -0.199999999999999	1.6000416106097\\
  -0.149999999999999	1.60547256465879\\
  -0.0999999999999996	1.60935028211886\\
  -0.0499999999999998	1.61167630769641\\
  0	1.61245154965971\\
  0.0499999999999998	1.61167630769641\\
  0.0999999999999996	1.60935028211886\\
  0.149999999999999	1.60547256465879\\
  0.199999999999999	1.6000416106097\\
  0.25	1.59305519158041\\
  0.3	1.58451032759374\\
  0.35	1.57440319667319\\
  0.399999999999999	1.56272901937789\\
  0.45	1.5494819149333\\
  0.5	1.53465472461422\\
  0.55	1.51823879680339\\
  0.6	1.50022372658859\\
  0.649999999999999	1.48059704075691\\
  0.7	1.45934381643565\\
  0.75	1.43644621818836\\
  0.8	1.41188293377827\\
  0.85	1.38562848259191\\
  0.899999999999999	1.35765236218309\\
  0.95	1.32791798652167\\
  1	1.29638135274165\\
  1.05	1.26298934903801\\
  1.1	1.22767758098417\\
  1.15	1.19036754062278\\
  1.2	1.15096286167424\\
  1.25	1.10934427694496\\
  1.3	1.06536268812512\\
  1.35	1.01882941368322\\
  1.4	0.96950208138017\\
  1.45	0.917063541274395\\
  1.5	0.861089079791055\\
  1.55	0.800992918574306\\
  1.6	0.73593542882336\\
  1.65	0.664648946727919\\
  1.7	0.585073509406255\\
  1.75	0.493465184892527\\
  1.8	0.381570190939788\\
  1.85	0.220951366483611\\
  1.9	0\\
  1.95	0\\
  2	0\\
  2.05	0\\
  2.1	0\\
  2.15	0\\
  2.2	0\\
  2.25	0\\
  2.3	0\\
  2.35	0\\
  2.4	0\\
  2.45	0\\
  2.5	0\\
  2.55	0\\
  2.6	0\\
  2.65	0\\
  2.7	0\\
  2.75	0\\
  2.8	0\\
  2.85	0\\
  2.9	0\\
  2.95	0\\
  3	0\\
  3.05	0\\
  3.1	0\\
  3.15	0\\
  3.2	0\\
  3.25	0\\
  3.3	0\\
  3.35	0\\
  3.4	0\\
  3.45	0\\
  3.5	0\\
  3.55	0\\
  3.6	0\\
  3.65	0\\
  3.7	0\\
  3.75	0\\
  3.8	0\\
  3.85	0\\
  3.9	0\\
  3.95	0\\
  4	0\\
  4.05	0\\
  4.1	0\\
  4.15	0\\
  4.2	0\\
  4.25	0\\
  4.3	0\\
  4.35	0\\
  4.4	0\\
  4.45	0.284996481418188\\
  4.5	0.422384186657645\\
  4.55	0.52597297201722\\
  4.6	0.612939594193335\\
  4.65	0.689422200676782\\
  4.7	0.758434356470103\\
  4.75	0.821708163386462\\
  4.8	0.880339688347\\
  4.85	0.935069101297767\\
  4.9	0.986420163442106\\
  4.95	1.03477673239637\\
  5	1.08042786474917\\
  };
  \addplot [color=black, forget plot]
    table[row sep=crcr]{%
  -5	-1.08042786474917\\
  -4.95	-1.03477673239637\\
  -4.9	-0.986420163442106\\
  -4.85	-0.935069101297767\\
  -4.8	-0.880339688347\\
  -4.75	-0.821708163386462\\
  -4.7	-0.758434356470103\\
  -4.65	-0.689422200676782\\
  -4.6	-0.612939594193335\\
  -4.55	-0.52597297201722\\
  -4.5	-0.422384186657645\\
  -4.45	-0.284996481418188\\
  -4.4	-0\\
  -4.35	-0\\
  -4.3	-0\\
  -4.25	-0\\
  -4.2	-0\\
  -4.15	-0\\
  -4.1	-0\\
  -4.05	-0\\
  -4	-0\\
  -3.95	-0\\
  -3.9	-0\\
  -3.85	-0\\
  -3.8	-0\\
  -3.75	-0\\
  -3.7	-0\\
  -3.65	-0\\
  -3.6	-0\\
  -3.55	-0\\
  -3.5	-0\\
  -3.45	-0\\
  -3.4	-0\\
  -3.35	-0\\
  -3.3	-0\\
  -3.25	-0\\
  -3.2	-0\\
  -3.15	-0\\
  -3.1	-0\\
  -3.05	-0\\
  -3	-0\\
  -2.95	-0\\
  -2.9	-0\\
  -2.85	-0\\
  -2.8	-0\\
  -2.75	-0\\
  -2.7	-0\\
  -2.65	-0\\
  -2.6	-0\\
  -2.55	-0\\
  -2.5	-0\\
  -2.45	-0\\
  -2.4	-0\\
  -2.35	-0\\
  -2.3	-0\\
  -2.25	-0\\
  -2.2	-0\\
  -2.15	-0\\
  -2.1	-0\\
  -2.05	-0\\
  -2	-0\\
  -1.95	-0\\
  -1.9	-0\\
  -1.85	-0.220951366483611\\
  -1.8	-0.381570190939788\\
  -1.75	-0.493465184892527\\
  -1.7	-0.585073509406255\\
  -1.65	-0.664648946727919\\
  -1.6	-0.73593542882336\\
  -1.55	-0.800992918574306\\
  -1.5	-0.861089079791055\\
  -1.45	-0.917063541274395\\
  -1.4	-0.96950208138017\\
  -1.35	-1.01882941368322\\
  -1.3	-1.06536268812512\\
  -1.25	-1.10934427694496\\
  -1.2	-1.15096286167424\\
  -1.15	-1.19036754062278\\
  -1.1	-1.22767758098417\\
  -1.05	-1.26298934903801\\
  -1	-1.29638135274165\\
  -0.95	-1.32791798652167\\
  -0.899999999999999	-1.35765236218309\\
  -0.85	-1.38562848259191\\
  -0.8	-1.41188293377827\\
  -0.75	-1.43644621818836\\
  -0.7	-1.45934381643565\\
  -0.649999999999999	-1.48059704075691\\
  -0.6	-1.50022372658859\\
  -0.55	-1.51823879680339\\
  -0.5	-1.53465472461422\\
  -0.45	-1.5494819149333\\
  -0.399999999999999	-1.56272901937789\\
  -0.35	-1.57440319667319\\
  -0.3	-1.58451032759374\\
  -0.25	-1.59305519158041\\
  -0.199999999999999	-1.6000416106097\\
  -0.149999999999999	-1.60547256465879\\
  -0.0999999999999996	-1.60935028211886\\
  -0.0499999999999998	-1.61167630769641\\
  0	-1.61245154965971\\
  0.0499999999999998	-1.61167630769641\\
  0.0999999999999996	-1.60935028211886\\
  0.149999999999999	-1.60547256465879\\
  0.199999999999999	-1.6000416106097\\
  0.25	-1.59305519158041\\
  0.3	-1.58451032759374\\
  0.35	-1.57440319667319\\
  0.399999999999999	-1.56272901937789\\
  0.45	-1.5494819149333\\
  0.5	-1.53465472461422\\
  0.55	-1.51823879680339\\
  0.6	-1.50022372658859\\
  0.649999999999999	-1.48059704075691\\
  0.7	-1.45934381643565\\
  0.75	-1.43644621818836\\
  0.8	-1.41188293377827\\
  0.85	-1.38562848259191\\
  0.899999999999999	-1.35765236218309\\
  0.95	-1.32791798652167\\
  1	-1.29638135274165\\
  1.05	-1.26298934903801\\
  1.1	-1.22767758098417\\
  1.15	-1.19036754062278\\
  1.2	-1.15096286167424\\
  1.25	-1.10934427694496\\
  1.3	-1.06536268812512\\
  1.35	-1.01882941368322\\
  1.4	-0.96950208138017\\
  1.45	-0.917063541274395\\
  1.5	-0.861089079791055\\
  1.55	-0.800992918574306\\
  1.6	-0.73593542882336\\
  1.65	-0.664648946727919\\
  1.7	-0.585073509406255\\
  1.75	-0.493465184892527\\
  1.8	-0.381570190939788\\
  1.85	-0.220951366483611\\
  1.9	-0\\
  1.95	-0\\
  2	-0\\
  2.05	-0\\
  2.1	-0\\
  2.15	-0\\
  2.2	-0\\
  2.25	-0\\
  2.3	-0\\
  2.35	-0\\
  2.4	-0\\
  2.45	-0\\
  2.5	-0\\
  2.55	-0\\
  2.6	-0\\
  2.65	-0\\
  2.7	-0\\
  2.75	-0\\
  2.8	-0\\
  2.85	-0\\
  2.9	-0\\
  2.95	-0\\
  3	-0\\
  3.05	-0\\
  3.1	-0\\
  3.15	-0\\
  3.2	-0\\
  3.25	-0\\
  3.3	-0\\
  3.35	-0\\
  3.4	-0\\
  3.45	-0\\
  3.5	-0\\
  3.55	-0\\
  3.6	-0\\
  3.65	-0\\
  3.7	-0\\
  3.75	-0\\
  3.8	-0\\
  3.85	-0\\
  3.9	-0\\
  3.95	-0\\
  4	-0\\
  4.05	-0\\
  4.1	-0\\
  4.15	-0\\
  4.2	-0\\
  4.25	-0\\
  4.3	-0\\
  4.35	-0\\
  4.4	-0\\
  4.45	-0.284996481418188\\
  4.5	-0.422384186657645\\
  4.55	-0.52597297201722\\
  4.6	-0.612939594193335\\
  4.65	-0.689422200676782\\
  4.7	-0.758434356470103\\
  4.75	-0.821708163386462\\
  4.8	-0.880339688347\\
  4.85	-0.935069101297767\\
  4.9	-0.986420163442106\\
  4.95	-1.03477673239637\\
  5	-1.08042786474917\\
  };
  
  \addplot[area legend, draw=black, fill=black, forget plot]
  table[row sep=crcr] {%
  x	y\\
  -0.0806451612903221	1.61245154965971\\
  -0.0999999999999996	1.61245154965971\\
  -0.0999999999999996	1.61245154965971\\
  -0.0999999999999996	1.61245154965971\\
  -0.0999999999999996	1.61245154965971\\
  -0.0999999999999996	1.61245154965971\\
  -0.0612903225806445	1.61245154965971\\
  -0.0612903225806445	1.56620229389216\\
  0.0999999999999996	1.61245154965971\\
  -0.0612903225806436	1.65870080542726\\
  -0.0612903225806445	1.61245154965971\\
  }--cycle;
  
  \addplot[area legend, draw=black, fill=black, forget plot]
  table[row sep=crcr] {%
  x	y\\
  0.080645161290323	-1.61245154965971\\
  0.0999999999999996	-1.61245154965971\\
  0.0999999999999996	-1.61245154965971\\
  0.0999999999999996	-1.61245154965971\\
  0.0999999999999996	-1.61245154965971\\
  0.0999999999999996	-1.61245154965971\\
  0.0612903225806463	-1.61245154965971\\
  0.0612903225806463	-1.65870080542726\\
  -0.0999999999999996	-1.61245154965971\\
  0.0612903225806463	-1.56620229389216\\
  0.0612903225806463	-1.61245154965971\\
  }--cycle;
  \addplot [color=black, forget plot]
    table[row sep=crcr]{%
  -5	1.3294075262787\\
  -4.95	1.29257993405008\\
  -4.9	1.25420283002597\\
  -4.85	1.21422988935449\\
  -4.8	1.17260307302978\\
  -4.75	1.12924944355796\\
  -4.7	1.08407687599829\\
  -4.65	1.03696816286037\\
  -4.6	0.987772719875321\\
  -4.55	0.936294594287838\\
  -4.5	0.882274561085403\\
  -4.45	0.825362341290628\\
  -4.4	0.765071408460126\\
  -4.35	0.700700984660173\\
  -4.3	0.63119066520351\\
  -4.25	0.554819808741914\\
  -4.2	0.468485172997611\\
  -4.15	0.365489381413046\\
  -4.1	0.224392751517204\\
  -4.05	0\\
  -4	0\\
  -3.95	0\\
  -3.9	0\\
  -3.85	0\\
  -3.8	0\\
  -3.75	0\\
  -3.7	0\\
  -3.65	0\\
  -3.6	0\\
  -3.55	0\\
  -3.5	0\\
  -3.45	0\\
  -3.4	0\\
  -3.35	0\\
  -3.3	0\\
  -3.25	0\\
  -3.2	0\\
  -3.15	0\\
  -3.1	0\\
  -3.05	0\\
  -3	0\\
  -2.95	0\\
  -2.9	0\\
  -2.85	0\\
  -2.8	0\\
  -2.75	0\\
  -2.7	0\\
  -2.65	0\\
  -2.6	0\\
  -2.55	0\\
  -2.5	0\\
  -2.45	0\\
  -2.4	0\\
  -2.35	0\\
  -2.3	0\\
  -2.25	0\\
  -2.2	0.151650141738506\\
  -2.15	0.324475991468488\\
  -2.1	0.436242811746263\\
  -2.05	0.527119167974922\\
  -2	0.606387934333884\\
  -1.95	0.677966324604273\\
  -1.9	0.743922621159616\\
  -1.85	0.805493331040658\\
  -1.8	0.863478899923922\\
  -1.75	0.918426855389702\\
  -1.7	0.970727052991185\\
  -1.65	1.02066557813347\\
  -1.6	1.06845727822755\\
  -1.55	1.11426642038885\\
  -1.5	1.15822036043898\\
  -1.45	1.20041890135683\\
  -1.4	1.24094088731111\\
  -1.35	1.27984896538071\\
  -1.3	1.3171930979356\\
  -1.25	1.35301320200157\\
  -1.2	1.38734116530627\\
  -1.15	1.42020240873205\\
  -1.1	1.45161711303331\\
  -1.05	1.48160119323098\\
  -1	1.5101670807352\\
  -0.95	1.53732435709832\\
  -0.899999999999999	1.56308027194426\\
  -0.85	1.58744016950875\\
  -0.8	1.6104078423475\\
  -0.75	1.63198582645427\\
  -0.7	1.6521756488246\\
  -0.649999999999999	1.67097803609087\\
  -0.6	1.68839309102453\\
  -0.55	1.70442044229674\\
  -0.5	1.71905937180213\\
  -0.45	1.73230892299998\\
  -0.399999999999999	1.74416799305737\\
  -0.35	1.75463541104548\\
  -0.3	1.76371000401177\\
  -0.25	1.77139065240316\\
  -0.199999999999999	1.7776763360304\\
  -0.149999999999999	1.7825661715269\\
  -0.0999999999999996	1.78605944205563\\
  -0.0499999999999998	1.78815561984687\\
  0	1.78885438199983\\
  0.0499999999999998	1.78815561984687\\
  0.0999999999999996	1.78605944205563\\
  0.149999999999999	1.7825661715269\\
  0.199999999999999	1.7776763360304\\
  0.25	1.77139065240316\\
  0.3	1.76371000401177\\
  0.35	1.75463541104548\\
  0.399999999999999	1.74416799305737\\
  0.45	1.73230892299998\\
  0.5	1.71905937180213\\
  0.55	1.70442044229674\\
  0.6	1.68839309102453\\
  0.649999999999999	1.67097803609087\\
  0.7	1.6521756488246\\
  0.75	1.63198582645427\\
  0.8	1.6104078423475\\
  0.85	1.58744016950875\\
  0.899999999999999	1.56308027194426\\
  0.95	1.53732435709832\\
  1	1.5101670807352\\
  1.05	1.48160119323098\\
  1.1	1.45161711303331\\
  1.15	1.42020240873205\\
  1.2	1.38734116530627\\
  1.25	1.35301320200157\\
  1.3	1.3171930979356\\
  1.35	1.27984896538071\\
  1.4	1.24094088731111\\
  1.45	1.20041890135683\\
  1.5	1.15822036043898\\
  1.55	1.11426642038885\\
  1.6	1.06845727822755\\
  1.65	1.02066557813347\\
  1.7	0.970727052991185\\
  1.75	0.918426855389702\\
  1.8	0.863478899923922\\
  1.85	0.805493331040658\\
  1.9	0.743922621159616\\
  1.95	0.677966324604273\\
  2	0.606387934333884\\
  2.05	0.527119167974922\\
  2.1	0.436242811746263\\
  2.15	0.324475991468488\\
  2.2	0.151650141738506\\
  2.25	0\\
  2.3	0\\
  2.35	0\\
  2.4	0\\
  2.45	0\\
  2.5	0\\
  2.55	0\\
  2.6	0\\
  2.65	0\\
  2.7	0\\
  2.75	0\\
  2.8	0\\
  2.85	0\\
  2.9	0\\
  2.95	0\\
  3	0\\
  3.05	0\\
  3.1	0\\
  3.15	0\\
  3.2	0\\
  3.25	0\\
  3.3	0\\
  3.35	0\\
  3.4	0\\
  3.45	0\\
  3.5	0\\
  3.55	0\\
  3.6	0\\
  3.65	0\\
  3.7	0\\
  3.75	0\\
  3.8	0\\
  3.85	0\\
  3.9	0\\
  3.95	0\\
  4	0\\
  4.05	0\\
  4.1	0.224392751517204\\
  4.15	0.365489381413046\\
  4.2	0.468485172997611\\
  4.25	0.554819808741914\\
  4.3	0.63119066520351\\
  4.35	0.700700984660173\\
  4.4	0.765071408460126\\
  4.45	0.825362341290628\\
  4.5	0.882274561085403\\
  4.55	0.936294594287838\\
  4.6	0.987772719875321\\
  4.65	1.03696816286037\\
  4.7	1.08407687599829\\
  4.75	1.12924944355796\\
  4.8	1.17260307302978\\
  4.85	1.21422988935449\\
  4.9	1.25420283002597\\
  4.95	1.29257993405008\\
  5	1.3294075262787\\
  };
  \addplot [color=black, forget plot]
    table[row sep=crcr]{%
  -5	-1.3294075262787\\
  -4.95	-1.29257993405008\\
  -4.9	-1.25420283002597\\
  -4.85	-1.21422988935449\\
  -4.8	-1.17260307302978\\
  -4.75	-1.12924944355796\\
  -4.7	-1.08407687599829\\
  -4.65	-1.03696816286037\\
  -4.6	-0.987772719875321\\
  -4.55	-0.936294594287838\\
  -4.5	-0.882274561085403\\
  -4.45	-0.825362341290628\\
  -4.4	-0.765071408460126\\
  -4.35	-0.700700984660173\\
  -4.3	-0.63119066520351\\
  -4.25	-0.554819808741914\\
  -4.2	-0.468485172997611\\
  -4.15	-0.365489381413046\\
  -4.1	-0.224392751517204\\
  -4.05	-0\\
  -4	-0\\
  -3.95	-0\\
  -3.9	-0\\
  -3.85	-0\\
  -3.8	-0\\
  -3.75	-0\\
  -3.7	-0\\
  -3.65	-0\\
  -3.6	-0\\
  -3.55	-0\\
  -3.5	-0\\
  -3.45	-0\\
  -3.4	-0\\
  -3.35	-0\\
  -3.3	-0\\
  -3.25	-0\\
  -3.2	-0\\
  -3.15	-0\\
  -3.1	-0\\
  -3.05	-0\\
  -3	-0\\
  -2.95	-0\\
  -2.9	-0\\
  -2.85	-0\\
  -2.8	-0\\
  -2.75	-0\\
  -2.7	-0\\
  -2.65	-0\\
  -2.6	-0\\
  -2.55	-0\\
  -2.5	-0\\
  -2.45	-0\\
  -2.4	-0\\
  -2.35	-0\\
  -2.3	-0\\
  -2.25	-0\\
  -2.2	-0.151650141738506\\
  -2.15	-0.324475991468488\\
  -2.1	-0.436242811746263\\
  -2.05	-0.527119167974922\\
  -2	-0.606387934333884\\
  -1.95	-0.677966324604273\\
  -1.9	-0.743922621159616\\
  -1.85	-0.805493331040658\\
  -1.8	-0.863478899923922\\
  -1.75	-0.918426855389702\\
  -1.7	-0.970727052991185\\
  -1.65	-1.02066557813347\\
  -1.6	-1.06845727822755\\
  -1.55	-1.11426642038885\\
  -1.5	-1.15822036043898\\
  -1.45	-1.20041890135683\\
  -1.4	-1.24094088731111\\
  -1.35	-1.27984896538071\\
  -1.3	-1.3171930979356\\
  -1.25	-1.35301320200157\\
  -1.2	-1.38734116530627\\
  -1.15	-1.42020240873205\\
  -1.1	-1.45161711303331\\
  -1.05	-1.48160119323098\\
  -1	-1.5101670807352\\
  -0.95	-1.53732435709832\\
  -0.899999999999999	-1.56308027194426\\
  -0.85	-1.58744016950875\\
  -0.8	-1.6104078423475\\
  -0.75	-1.63198582645427\\
  -0.7	-1.6521756488246\\
  -0.649999999999999	-1.67097803609087\\
  -0.6	-1.68839309102453\\
  -0.55	-1.70442044229674\\
  -0.5	-1.71905937180213\\
  -0.45	-1.73230892299998\\
  -0.399999999999999	-1.74416799305737\\
  -0.35	-1.75463541104548\\
  -0.3	-1.76371000401177\\
  -0.25	-1.77139065240316\\
  -0.199999999999999	-1.7776763360304\\
  -0.149999999999999	-1.7825661715269\\
  -0.0999999999999996	-1.78605944205563\\
  -0.0499999999999998	-1.78815561984687\\
  0	-1.78885438199983\\
  0.0499999999999998	-1.78815561984687\\
  0.0999999999999996	-1.78605944205563\\
  0.149999999999999	-1.7825661715269\\
  0.199999999999999	-1.7776763360304\\
  0.25	-1.77139065240316\\
  0.3	-1.76371000401177\\
  0.35	-1.75463541104548\\
  0.399999999999999	-1.74416799305737\\
  0.45	-1.73230892299998\\
  0.5	-1.71905937180213\\
  0.55	-1.70442044229674\\
  0.6	-1.68839309102453\\
  0.649999999999999	-1.67097803609087\\
  0.7	-1.6521756488246\\
  0.75	-1.63198582645427\\
  0.8	-1.6104078423475\\
  0.85	-1.58744016950875\\
  0.899999999999999	-1.56308027194426\\
  0.95	-1.53732435709832\\
  1	-1.5101670807352\\
  1.05	-1.48160119323098\\
  1.1	-1.45161711303331\\
  1.15	-1.42020240873205\\
  1.2	-1.38734116530627\\
  1.25	-1.35301320200157\\
  1.3	-1.3171930979356\\
  1.35	-1.27984896538071\\
  1.4	-1.24094088731111\\
  1.45	-1.20041890135683\\
  1.5	-1.15822036043898\\
  1.55	-1.11426642038885\\
  1.6	-1.06845727822755\\
  1.65	-1.02066557813347\\
  1.7	-0.970727052991185\\
  1.75	-0.918426855389702\\
  1.8	-0.863478899923922\\
  1.85	-0.805493331040658\\
  1.9	-0.743922621159616\\
  1.95	-0.677966324604273\\
  2	-0.606387934333884\\
  2.05	-0.527119167974922\\
  2.1	-0.436242811746263\\
  2.15	-0.324475991468488\\
  2.2	-0.151650141738506\\
  2.25	-0\\
  2.3	-0\\
  2.35	-0\\
  2.4	-0\\
  2.45	-0\\
  2.5	-0\\
  2.55	-0\\
  2.6	-0\\
  2.65	-0\\
  2.7	-0\\
  2.75	-0\\
  2.8	-0\\
  2.85	-0\\
  2.9	-0\\
  2.95	-0\\
  3	-0\\
  3.05	-0\\
  3.1	-0\\
  3.15	-0\\
  3.2	-0\\
  3.25	-0\\
  3.3	-0\\
  3.35	-0\\
  3.4	-0\\
  3.45	-0\\
  3.5	-0\\
  3.55	-0\\
  3.6	-0\\
  3.65	-0\\
  3.7	-0\\
  3.75	-0\\
  3.8	-0\\
  3.85	-0\\
  3.9	-0\\
  3.95	-0\\
  4	-0\\
  4.05	-0\\
  4.1	-0.224392751517204\\
  4.15	-0.365489381413046\\
  4.2	-0.468485172997611\\
  4.25	-0.554819808741914\\
  4.3	-0.63119066520351\\
  4.35	-0.700700984660173\\
  4.4	-0.765071408460126\\
  4.45	-0.825362341290628\\
  4.5	-0.882274561085403\\
  4.55	-0.936294594287838\\
  4.6	-0.987772719875321\\
  4.65	-1.03696816286037\\
  4.7	-1.08407687599829\\
  4.75	-1.12924944355796\\
  4.8	-1.17260307302978\\
  4.85	-1.21422988935449\\
  4.9	-1.25420283002597\\
  4.95	-1.29257993405008\\
  5	-1.3294075262787\\
  };
  
  \addplot[area legend, draw=black, fill=black, forget plot]
  table[row sep=crcr] {%
  x	y\\
  -0.0806451612903221	1.78885438199983\\
  -0.0999999999999996	1.78885438199983\\
  -0.0999999999999996	1.78885438199983\\
  -0.0999999999999996	1.78885438199983\\
  -0.0999999999999996	1.78885438199983\\
  -0.0999999999999996	1.78885438199983\\
  -0.0612903225806445	1.78885438199983\\
  -0.0612903225806445	1.74260512623228\\
  0.0999999999999996	1.78885438199983\\
  -0.0612903225806436	1.83510363776738\\
  -0.0612903225806445	1.78885438199983\\
  }--cycle;
  
  \addplot[area legend, draw=black, fill=black, forget plot]
  table[row sep=crcr] {%
  x	y\\
  0.080645161290323	-1.78885438199983\\
  0.0999999999999996	-1.78885438199983\\
  0.0999999999999996	-1.78885438199983\\
  0.0999999999999996	-1.78885438199983\\
  0.0999999999999996	-1.78885438199983\\
  0.0999999999999996	-1.78885438199983\\
  0.0612903225806463	-1.78885438199983\\
  0.0612903225806463	-1.83510363776738\\
  -0.0999999999999996	-1.78885438199983\\
  0.0612903225806463	-1.74260512623228\\
  0.0612903225806463	-1.78885438199983\\
  }--cycle;
  \addplot [color=black, forget plot]
    table[row sep=crcr]{%
  -5	1.53861118250403\\
  -4.95	1.5069050686453\\
  -4.9	1.47411829201226\\
  -4.85	1.44026185959422\\
  -4.8	1.40534620890331\\
  -4.75	1.36938099365222\\
  -4.7	1.33237482454234\\
  -4.65	1.29433495308827\\
  -4.6	1.25526688243174\\
  -4.55	1.21517388356261\\
  -4.5	1.17405638754637\\
  -4.45	1.131911213135\\
  -4.4	1.0887305727512\\
  -4.35	1.0445007754443\\
  -4.3	0.999200508326556\\
  -4.25	0.952798520240462\\
  -4.2	0.90525043900492\\
  -4.15	0.856494301163581\\
  -4.1	0.806444112715482\\
  -4.05	0.754980295735492\\
  -4	0.701935010006465\\
  -3.95	0.647068626096943\\
  -3.9	0.590029992118807\\
  -3.85	0.530284717604594\\
  -3.8	0.466973849558159\\
  -3.75	0.398599153688111\\
  -3.7	0.322179975447239\\
  -3.65	0.230126497023102\\
  -3.6	0.0805181180338028\\
  -3.55	0\\
  -3.5	0\\
  -3.45	0\\
  -3.4	0\\
  -3.35	0\\
  -3.3	0\\
  -3.25	0\\
  -3.2	0\\
  -3.15	0\\
  -3.1	0\\
  -3.05	0\\
  -3	0\\
  -2.95	0\\
  -2.9	0\\
  -2.85	0\\
  -2.8	0\\
  -2.75	0\\
  -2.7	0\\
  -2.65	0.191926048892348\\
  -2.6	0.293636668796841\\
  -2.55	0.374022632376111\\
  -2.5	0.444649040149793\\
  -2.45	0.509448222987759\\
  -2.4	0.570274117348411\\
  -2.35	0.628151133448705\\
  -2.3	0.683701658210913\\
  -2.25	0.737328118651745\\
  -2.2	0.78930207493032\\
  -2.15	0.839812282024655\\
  -2.1	0.888992570722774\\
  -2.05	0.936938961323828\\
  -2	0.983720654914654\\
  -1.95	1.02938736017955\\
  -1.9	1.07397433222261\\
  -1.85	1.11750593123749\\
  -1.8	1.15999819422869\\
  -1.75	1.20146073123553\\
  -1.7	1.24189814856491\\
  -1.65	1.28131113410699\\
  -1.6	1.31969729688191\\
  -1.55	1.35705182495223\\
  -1.5	1.39336800714506\\
  -1.45	1.42863765130796\\
  -1.4	1.46285142300935\\
  -1.35	1.49599912238814\\
  -1.3	1.52806991242193\\
  -1.25	1.5590525086701\\
  -1.2	1.58893533819138\\
  -1.15	1.61770667358712\\
  -1.1	1.64535474681029\\
  -1.05	1.67186784638722\\
  -1	1.69723440094062\\
  -0.95	1.72144305131705\\
  -0.899999999999999	1.74448271316781\\
  -0.85	1.76634263147611\\
  -0.8	1.78701242824283\\
  -0.75	1.80648214432018\\
  -0.7	1.82474227620477\\
  -0.649999999999999	1.84178380845802\\
  -0.6	1.85759824230627\\
  -0.55	1.8721776208787\\
  -0.5	1.88551455146354\\
  -0.45	1.89760222510023\\
  -0.399999999999999	1.90843443377177\\
  -0.35	1.91800558541803\\
  -0.3	1.92631071695384\\
  -0.25	1.9333455054442\\
  -0.199999999999999	1.93910627756255\\
  -0.149999999999999	1.94359001743477\\
  -0.0999999999999996	1.94679437295161\\
  -0.0499999999999998	1.94871766061426\\
  0	1.94935886896179\\
  0.0499999999999998	1.94871766061426\\
  0.0999999999999996	1.94679437295161\\
  0.149999999999999	1.94359001743477\\
  0.199999999999999	1.93910627756255\\
  0.25	1.9333455054442\\
  0.3	1.92631071695384\\
  0.35	1.91800558541803\\
  0.399999999999999	1.90843443377177\\
  0.45	1.89760222510023\\
  0.5	1.88551455146354\\
  0.55	1.8721776208787\\
  0.6	1.85759824230627\\
  0.649999999999999	1.84178380845802\\
  0.7	1.82474227620477\\
  0.75	1.80648214432018\\
  0.8	1.78701242824283\\
  0.85	1.76634263147611\\
  0.899999999999999	1.74448271316781\\
  0.95	1.72144305131705\\
  1	1.69723440094062\\
  1.05	1.67186784638722\\
  1.1	1.64535474681029\\
  1.15	1.61770667358712\\
  1.2	1.58893533819138\\
  1.25	1.5590525086701\\
  1.3	1.52806991242193\\
  1.35	1.49599912238814\\
  1.4	1.46285142300935\\
  1.45	1.42863765130796\\
  1.5	1.39336800714506\\
  1.55	1.35705182495223\\
  1.6	1.31969729688191\\
  1.65	1.28131113410699\\
  1.7	1.24189814856491\\
  1.75	1.20146073123553\\
  1.8	1.15999819422869\\
  1.85	1.11750593123749\\
  1.9	1.07397433222261\\
  1.95	1.02938736017955\\
  2	0.983720654914654\\
  2.05	0.936938961323828\\
  2.1	0.888992570722774\\
  2.15	0.839812282024655\\
  2.2	0.78930207493032\\
  2.25	0.737328118651745\\
  2.3	0.683701658210913\\
  2.35	0.628151133448705\\
  2.4	0.570274117348411\\
  2.45	0.509448222987759\\
  2.5	0.444649040149793\\
  2.55	0.374022632376111\\
  2.6	0.293636668796841\\
  2.65	0.191926048892348\\
  2.7	0\\
  2.75	0\\
  2.8	0\\
  2.85	0\\
  2.9	0\\
  2.95	0\\
  3	0\\
  3.05	0\\
  3.1	0\\
  3.15	0\\
  3.2	0\\
  3.25	0\\
  3.3	0\\
  3.35	0\\
  3.4	0\\
  3.45	0\\
  3.5	0\\
  3.55	0\\
  3.6	0.0805181180338028\\
  3.65	0.230126497023102\\
  3.7	0.322179975447239\\
  3.75	0.398599153688111\\
  3.8	0.466973849558159\\
  3.85	0.530284717604594\\
  3.9	0.590029992118807\\
  3.95	0.647068626096943\\
  4	0.701935010006465\\
  4.05	0.754980295735492\\
  4.1	0.806444112715482\\
  4.15	0.856494301163581\\
  4.2	0.90525043900492\\
  4.25	0.952798520240462\\
  4.3	0.999200508326556\\
  4.35	1.0445007754443\\
  4.4	1.0887305727512\\
  4.45	1.131911213135\\
  4.5	1.17405638754637\\
  4.55	1.21517388356261\\
  4.6	1.25526688243174\\
  4.65	1.29433495308827\\
  4.7	1.33237482454234\\
  4.75	1.36938099365222\\
  4.8	1.40534620890331\\
  4.85	1.44026185959422\\
  4.9	1.47411829201226\\
  4.95	1.5069050686453\\
  5	1.53861118250403\\
  };
  \addplot [color=black, forget plot]
    table[row sep=crcr]{%
  -5	-1.53861118250403\\
  -4.95	-1.5069050686453\\
  -4.9	-1.47411829201226\\
  -4.85	-1.44026185959422\\
  -4.8	-1.40534620890331\\
  -4.75	-1.36938099365222\\
  -4.7	-1.33237482454234\\
  -4.65	-1.29433495308827\\
  -4.6	-1.25526688243174\\
  -4.55	-1.21517388356261\\
  -4.5	-1.17405638754637\\
  -4.45	-1.131911213135\\
  -4.4	-1.0887305727512\\
  -4.35	-1.0445007754443\\
  -4.3	-0.999200508326556\\
  -4.25	-0.952798520240462\\
  -4.2	-0.90525043900492\\
  -4.15	-0.856494301163581\\
  -4.1	-0.806444112715482\\
  -4.05	-0.754980295735492\\
  -4	-0.701935010006465\\
  -3.95	-0.647068626096943\\
  -3.9	-0.590029992118807\\
  -3.85	-0.530284717604594\\
  -3.8	-0.466973849558159\\
  -3.75	-0.398599153688111\\
  -3.7	-0.322179975447239\\
  -3.65	-0.230126497023102\\
  -3.6	-0.0805181180338028\\
  -3.55	-0\\
  -3.5	-0\\
  -3.45	-0\\
  -3.4	-0\\
  -3.35	-0\\
  -3.3	-0\\
  -3.25	-0\\
  -3.2	-0\\
  -3.15	-0\\
  -3.1	-0\\
  -3.05	-0\\
  -3	-0\\
  -2.95	-0\\
  -2.9	-0\\
  -2.85	-0\\
  -2.8	-0\\
  -2.75	-0\\
  -2.7	-0\\
  -2.65	-0.191926048892348\\
  -2.6	-0.293636668796841\\
  -2.55	-0.374022632376111\\
  -2.5	-0.444649040149793\\
  -2.45	-0.509448222987759\\
  -2.4	-0.570274117348411\\
  -2.35	-0.628151133448705\\
  -2.3	-0.683701658210913\\
  -2.25	-0.737328118651745\\
  -2.2	-0.78930207493032\\
  -2.15	-0.839812282024655\\
  -2.1	-0.888992570722774\\
  -2.05	-0.936938961323828\\
  -2	-0.983720654914654\\
  -1.95	-1.02938736017955\\
  -1.9	-1.07397433222261\\
  -1.85	-1.11750593123749\\
  -1.8	-1.15999819422869\\
  -1.75	-1.20146073123553\\
  -1.7	-1.24189814856491\\
  -1.65	-1.28131113410699\\
  -1.6	-1.31969729688191\\
  -1.55	-1.35705182495223\\
  -1.5	-1.39336800714506\\
  -1.45	-1.42863765130796\\
  -1.4	-1.46285142300935\\
  -1.35	-1.49599912238814\\
  -1.3	-1.52806991242193\\
  -1.25	-1.5590525086701\\
  -1.2	-1.58893533819138\\
  -1.15	-1.61770667358712\\
  -1.1	-1.64535474681029\\
  -1.05	-1.67186784638722\\
  -1	-1.69723440094062\\
  -0.95	-1.72144305131705\\
  -0.899999999999999	-1.74448271316781\\
  -0.85	-1.76634263147611\\
  -0.8	-1.78701242824283\\
  -0.75	-1.80648214432018\\
  -0.7	-1.82474227620477\\
  -0.649999999999999	-1.84178380845802\\
  -0.6	-1.85759824230627\\
  -0.55	-1.8721776208787\\
  -0.5	-1.88551455146354\\
  -0.45	-1.89760222510023\\
  -0.399999999999999	-1.90843443377177\\
  -0.35	-1.91800558541803\\
  -0.3	-1.92631071695384\\
  -0.25	-1.9333455054442\\
  -0.199999999999999	-1.93910627756255\\
  -0.149999999999999	-1.94359001743477\\
  -0.0999999999999996	-1.94679437295161\\
  -0.0499999999999998	-1.94871766061426\\
  0	-1.94935886896179\\
  0.0499999999999998	-1.94871766061426\\
  0.0999999999999996	-1.94679437295161\\
  0.149999999999999	-1.94359001743477\\
  0.199999999999999	-1.93910627756255\\
  0.25	-1.9333455054442\\
  0.3	-1.92631071695384\\
  0.35	-1.91800558541803\\
  0.399999999999999	-1.90843443377177\\
  0.45	-1.89760222510023\\
  0.5	-1.88551455146354\\
  0.55	-1.8721776208787\\
  0.6	-1.85759824230627\\
  0.649999999999999	-1.84178380845802\\
  0.7	-1.82474227620477\\
  0.75	-1.80648214432018\\
  0.8	-1.78701242824283\\
  0.85	-1.76634263147611\\
  0.899999999999999	-1.74448271316781\\
  0.95	-1.72144305131705\\
  1	-1.69723440094062\\
  1.05	-1.67186784638722\\
  1.1	-1.64535474681029\\
  1.15	-1.61770667358712\\
  1.2	-1.58893533819138\\
  1.25	-1.5590525086701\\
  1.3	-1.52806991242193\\
  1.35	-1.49599912238814\\
  1.4	-1.46285142300935\\
  1.45	-1.42863765130796\\
  1.5	-1.39336800714506\\
  1.55	-1.35705182495223\\
  1.6	-1.31969729688191\\
  1.65	-1.28131113410699\\
  1.7	-1.24189814856491\\
  1.75	-1.20146073123553\\
  1.8	-1.15999819422869\\
  1.85	-1.11750593123749\\
  1.9	-1.07397433222261\\
  1.95	-1.02938736017955\\
  2	-0.983720654914654\\
  2.05	-0.936938961323828\\
  2.1	-0.888992570722774\\
  2.15	-0.839812282024655\\
  2.2	-0.78930207493032\\
  2.25	-0.737328118651745\\
  2.3	-0.683701658210913\\
  2.35	-0.628151133448705\\
  2.4	-0.570274117348411\\
  2.45	-0.509448222987759\\
  2.5	-0.444649040149793\\
  2.55	-0.374022632376111\\
  2.6	-0.293636668796841\\
  2.65	-0.191926048892348\\
  2.7	-0\\
  2.75	-0\\
  2.8	-0\\
  2.85	-0\\
  2.9	-0\\
  2.95	-0\\
  3	-0\\
  3.05	-0\\
  3.1	-0\\
  3.15	-0\\
  3.2	-0\\
  3.25	-0\\
  3.3	-0\\
  3.35	-0\\
  3.4	-0\\
  3.45	-0\\
  3.5	-0\\
  3.55	-0\\
  3.6	-0.0805181180338028\\
  3.65	-0.230126497023102\\
  3.7	-0.322179975447239\\
  3.75	-0.398599153688111\\
  3.8	-0.466973849558159\\
  3.85	-0.530284717604594\\
  3.9	-0.590029992118807\\
  3.95	-0.647068626096943\\
  4	-0.701935010006465\\
  4.05	-0.754980295735492\\
  4.1	-0.806444112715482\\
  4.15	-0.856494301163581\\
  4.2	-0.90525043900492\\
  4.25	-0.952798520240462\\
  4.3	-0.999200508326556\\
  4.35	-1.0445007754443\\
  4.4	-1.0887305727512\\
  4.45	-1.131911213135\\
  4.5	-1.17405638754637\\
  4.55	-1.21517388356261\\
  4.6	-1.25526688243174\\
  4.65	-1.29433495308827\\
  4.7	-1.33237482454234\\
  4.75	-1.36938099365222\\
  4.8	-1.40534620890331\\
  4.85	-1.44026185959422\\
  4.9	-1.47411829201226\\
  4.95	-1.5069050686453\\
  5	-1.53861118250403\\
  };
  
  \addplot[area legend, draw=black, fill=black, forget plot]
  table[row sep=crcr] {%
  x	y\\
  -0.0806451612903221	1.94935886896179\\
  -0.0999999999999996	1.94935886896179\\
  -0.0999999999999996	1.94935886896179\\
  -0.0999999999999996	1.94935886896179\\
  -0.0999999999999996	1.94935886896179\\
  -0.0999999999999996	1.94935886896179\\
  -0.0612903225806445	1.94935886896179\\
  -0.0612903225806445	1.90310961319424\\
  0.0999999999999996	1.94935886896179\\
  -0.0612903225806436	1.99560812472934\\
  -0.0612903225806445	1.94935886896179\\
  }--cycle;
  
  \addplot[area legend, draw=black, fill=black, forget plot]
  table[row sep=crcr] {%
  x	y\\
  0.080645161290323	-1.94935886896179\\
  0.0999999999999996	-1.94935886896179\\
  0.0999999999999996	-1.94935886896179\\
  0.0999999999999996	-1.94935886896179\\
  0.0999999999999996	-1.94935886896179\\
  0.0999999999999996	-1.94935886896179\\
  0.0612903225806463	-1.94935886896179\\
  0.0612903225806463	-1.99560812472934\\
  -0.0999999999999996	-1.94935886896179\\
  0.0612903225806463	-1.90310961319424\\
  0.0612903225806463	-1.94935886896179\\
  }--cycle;
  \addplot [color=black, forget plot]
    table[row sep=crcr]{%
  -5	1.72259234031922\\
  -4.95	1.69433257830595\\
  -4.9	1.66524014449723\\
  -4.85	1.63534529204135\\
  -4.8	1.60468002009089\\
  -4.75	1.57327820355332\\
  -4.7	1.54117574373406\\
  -4.65	1.50841074339386\\
  -4.6	1.47502371036194\\
  -4.55	1.44105779457058\\
  -4.5	1.40655906421964\\
  -4.45	1.37157682775\\
  -4.4	1.33616400941021\\
  -4.35	1.30037758743518\\
  -4.3	1.26427910519792\\
  -4.25	1.22793526709367\\
  -4.2	1.19141863226936\\
  -4.15	1.15480842044284\\
  -4.1	1.11819144467013\\
  -4.05	1.08166318553829\\
  -4	1.04532901914793\\
  -3.95	1.00930560628532\\
  -3.9	0.973722440739516\\
  -3.85	0.938723538495218\\
  -3.8	0.904469223451614\\
  -3.75	0.871137925543871\\
  -3.7	0.838927849447844\\
  -3.65	0.808058292842864\\
  -3.6	0.778770291762408\\
  -3.55	0.751326152923914\\
  -3.5	0.726007317744392\\
  -3.45	0.703109927971004\\
  -3.4	0.682937489702445\\
  -3.35	0.665790239366823\\
  -3.3	0.65195127132499\\
  -3.25	0.641670201769497\\
  -3.2	0.635146005584931\\
  -3.15	0.63251140942857\\
  -3.1	0.633821504410698\\
  -3.05	0.639048786026448\\
  -3	0.64808564773424\\
  -2.95	0.660753807407185\\
  -2.9	0.676818786456773\\
  -2.85	0.696006842562499\\
  -2.8	0.718021809322449\\
  -2.75	0.742559925349512\\
  -2.7	0.769321594631191\\
  -2.65	0.798019804418053\\
  -2.6	0.828385473835765\\
  -2.55	0.860170291006122\\
  -2.5	0.893147674747089\\
  -2.45	0.927112448360707\\
  -2.4	0.96187970605347\\
  -2.35	0.997283232814476\\
  -2.3	1.03317373052181\\
  -2.25	1.06941701620767\\
  -2.2	1.10589229380139\\
  -2.15	1.14249055533928\\
  -2.1	1.17911313740467\\
  -2.05	1.21567043940641\\
  -2	1.25208079887271\\
  -1.95	1.2882695126787\\
  -1.9	1.32416799020101\\
  -1.85	1.35971302352775\\
  -1.8	1.39484616019611\\
  -1.75	1.42951316492749\\
  -1.7	1.46366355813382\\
  -1.65	1.49725022036617\\
  -1.6	1.53022905324576\\
  -1.55	1.56255868869178\\
  -1.5	1.59420023941016\\
  -1.45	1.62511708462336\\
  -1.4	1.65527468590578\\
  -1.35	1.68464042875211\\
  -1.3	1.71318348615937\\
  -1.25	1.74087470105994\\
  -1.2	1.76768648491562\\
  -1.15	1.79359273018384\\
  -1.1	1.81856873470627\\
  -1.05	1.84259113635756\\
  -1	1.86563785653494\\
  -0.95	1.88768805127536\\
  -0.899999999999999	1.90872206896167\\
  -0.85	1.92872141372723\\
  -0.8	1.9476687137946\\
  -0.75	1.9655476940913\\
  -0.7	1.98234315257701\\
  -0.649999999999999	1.99804093979531\\
  -0.6	2.01262794122991\\
  -0.55	2.02609206210355\\
  -0.5	2.03842221430712\\
  -0.45	2.04960830519037\\
  -0.399999999999999	2.05964122798262\\
  -0.35	2.06851285364504\\
  -0.3	2.07621602398479\\
  -0.25	2.08274454588682\\
  -0.199999999999999	2.08809318654185\\
  -0.149999999999999	2.09225766956943\\
  -0.0999999999999996	2.09523467195349\\
  -0.0499999999999998	2.09702182172478\\
  0	2.0976176963403\\
  0.0499999999999998	2.09702182172478\\
  0.0999999999999996	2.09523467195349\\
  0.149999999999999	2.09225766956943\\
  0.199999999999999	2.08809318654185\\
  0.25	2.08274454588682\\
  0.3	2.07621602398479\\
  0.35	2.06851285364504\\
  0.399999999999999	2.05964122798262\\
  0.45	2.04960830519037\\
  0.5	2.03842221430712\\
  0.55	2.02609206210355\\
  0.6	2.01262794122991\\
  0.649999999999999	1.99804093979531\\
  0.7	1.98234315257701\\
  0.75	1.9655476940913\\
  0.8	1.9476687137946\\
  0.85	1.92872141372723\\
  0.899999999999999	1.90872206896167\\
  0.95	1.88768805127536\\
  1	1.86563785653494\\
  1.05	1.84259113635756\\
  1.1	1.81856873470627\\
  1.15	1.79359273018384\\
  1.2	1.76768648491562\\
  1.25	1.74087470105994\\
  1.3	1.71318348615937\\
  1.35	1.68464042875211\\
  1.4	1.65527468590578\\
  1.45	1.62511708462336\\
  1.5	1.59420023941016\\
  1.55	1.56255868869178\\
  1.6	1.53022905324576\\
  1.65	1.49725022036617\\
  1.7	1.46366355813382\\
  1.75	1.42951316492749\\
  1.8	1.39484616019611\\
  1.85	1.35971302352775\\
  1.9	1.32416799020101\\
  1.95	1.2882695126787\\
  2	1.25208079887271\\
  2.05	1.21567043940641\\
  2.1	1.17911313740467\\
  2.15	1.14249055533928\\
  2.2	1.10589229380139\\
  2.25	1.06941701620767\\
  2.3	1.03317373052181\\
  2.35	0.997283232814476\\
  2.4	0.96187970605347\\
  2.45	0.927112448360707\\
  2.5	0.893147674747089\\
  2.55	0.860170291006122\\
  2.6	0.828385473835765\\
  2.65	0.798019804418053\\
  2.7	0.769321594631191\\
  2.75	0.742559925349512\\
  2.8	0.718021809322449\\
  2.85	0.696006842562499\\
  2.9	0.676818786456773\\
  2.95	0.660753807407185\\
  3	0.64808564773424\\
  3.05	0.639048786026448\\
  3.1	0.633821504410698\\
  3.15	0.63251140942857\\
  3.2	0.635146005584931\\
  3.25	0.641670201769497\\
  3.3	0.65195127132499\\
  3.35	0.665790239366823\\
  3.4	0.682937489702445\\
  3.45	0.703109927971004\\
  3.5	0.726007317744392\\
  3.55	0.751326152923914\\
  3.6	0.778770291762408\\
  3.65	0.808058292842864\\
  3.7	0.838927849447844\\
  3.75	0.871137925543871\\
  3.8	0.904469223451614\\
  3.85	0.938723538495218\\
  3.9	0.973722440739516\\
  3.95	1.00930560628532\\
  4	1.04532901914793\\
  4.05	1.08166318553829\\
  4.1	1.11819144467013\\
  4.15	1.15480842044284\\
  4.2	1.19141863226936\\
  4.25	1.22793526709367\\
  4.3	1.26427910519792\\
  4.35	1.30037758743518\\
  4.4	1.33616400941021\\
  4.45	1.37157682775\\
  4.5	1.40655906421964\\
  4.55	1.44105779457058\\
  4.6	1.47502371036194\\
  4.65	1.50841074339386\\
  4.7	1.54117574373406\\
  4.75	1.57327820355332\\
  4.8	1.60468002009089\\
  4.85	1.63534529204135\\
  4.9	1.66524014449723\\
  4.95	1.69433257830595\\
  5	1.72259234031922\\
  };
  \addplot [color=black, forget plot]
    table[row sep=crcr]{%
  -5	-1.72259234031922\\
  -4.95	-1.69433257830595\\
  -4.9	-1.66524014449723\\
  -4.85	-1.63534529204135\\
  -4.8	-1.60468002009089\\
  -4.75	-1.57327820355332\\
  -4.7	-1.54117574373406\\
  -4.65	-1.50841074339386\\
  -4.6	-1.47502371036194\\
  -4.55	-1.44105779457058\\
  -4.5	-1.40655906421964\\
  -4.45	-1.37157682775\\
  -4.4	-1.33616400941021\\
  -4.35	-1.30037758743518\\
  -4.3	-1.26427910519792\\
  -4.25	-1.22793526709367\\
  -4.2	-1.19141863226936\\
  -4.15	-1.15480842044284\\
  -4.1	-1.11819144467013\\
  -4.05	-1.08166318553829\\
  -4	-1.04532901914793\\
  -3.95	-1.00930560628532\\
  -3.9	-0.973722440739516\\
  -3.85	-0.938723538495218\\
  -3.8	-0.904469223451614\\
  -3.75	-0.871137925543871\\
  -3.7	-0.838927849447844\\
  -3.65	-0.808058292842864\\
  -3.6	-0.778770291762408\\
  -3.55	-0.751326152923914\\
  -3.5	-0.726007317744392\\
  -3.45	-0.703109927971004\\
  -3.4	-0.682937489702445\\
  -3.35	-0.665790239366823\\
  -3.3	-0.65195127132499\\
  -3.25	-0.641670201769497\\
  -3.2	-0.635146005584931\\
  -3.15	-0.63251140942857\\
  -3.1	-0.633821504410698\\
  -3.05	-0.639048786026448\\
  -3	-0.64808564773424\\
  -2.95	-0.660753807407185\\
  -2.9	-0.676818786456773\\
  -2.85	-0.696006842562499\\
  -2.8	-0.718021809322449\\
  -2.75	-0.742559925349512\\
  -2.7	-0.769321594631191\\
  -2.65	-0.798019804418053\\
  -2.6	-0.828385473835765\\
  -2.55	-0.860170291006122\\
  -2.5	-0.893147674747089\\
  -2.45	-0.927112448360707\\
  -2.4	-0.96187970605347\\
  -2.35	-0.997283232814476\\
  -2.3	-1.03317373052181\\
  -2.25	-1.06941701620767\\
  -2.2	-1.10589229380139\\
  -2.15	-1.14249055533928\\
  -2.1	-1.17911313740467\\
  -2.05	-1.21567043940641\\
  -2	-1.25208079887271\\
  -1.95	-1.2882695126787\\
  -1.9	-1.32416799020101\\
  -1.85	-1.35971302352775\\
  -1.8	-1.39484616019611\\
  -1.75	-1.42951316492749\\
  -1.7	-1.46366355813382\\
  -1.65	-1.49725022036617\\
  -1.6	-1.53022905324576\\
  -1.55	-1.56255868869178\\
  -1.5	-1.59420023941016\\
  -1.45	-1.62511708462336\\
  -1.4	-1.65527468590578\\
  -1.35	-1.68464042875211\\
  -1.3	-1.71318348615937\\
  -1.25	-1.74087470105994\\
  -1.2	-1.76768648491562\\
  -1.15	-1.79359273018384\\
  -1.1	-1.81856873470627\\
  -1.05	-1.84259113635756\\
  -1	-1.86563785653494\\
  -0.95	-1.88768805127536\\
  -0.899999999999999	-1.90872206896167\\
  -0.85	-1.92872141372723\\
  -0.8	-1.9476687137946\\
  -0.75	-1.9655476940913\\
  -0.7	-1.98234315257701\\
  -0.649999999999999	-1.99804093979531\\
  -0.6	-2.01262794122991\\
  -0.55	-2.02609206210355\\
  -0.5	-2.03842221430712\\
  -0.45	-2.04960830519037\\
  -0.399999999999999	-2.05964122798262\\
  -0.35	-2.06851285364504\\
  -0.3	-2.07621602398479\\
  -0.25	-2.08274454588682\\
  -0.199999999999999	-2.08809318654185\\
  -0.149999999999999	-2.09225766956943\\
  -0.0999999999999996	-2.09523467195349\\
  -0.0499999999999998	-2.09702182172478\\
  0	-2.0976176963403\\
  0.0499999999999998	-2.09702182172478\\
  0.0999999999999996	-2.09523467195349\\
  0.149999999999999	-2.09225766956943\\
  0.199999999999999	-2.08809318654185\\
  0.25	-2.08274454588682\\
  0.3	-2.07621602398479\\
  0.35	-2.06851285364504\\
  0.399999999999999	-2.05964122798262\\
  0.45	-2.04960830519037\\
  0.5	-2.03842221430712\\
  0.55	-2.02609206210355\\
  0.6	-2.01262794122991\\
  0.649999999999999	-1.99804093979531\\
  0.7	-1.98234315257701\\
  0.75	-1.9655476940913\\
  0.8	-1.9476687137946\\
  0.85	-1.92872141372723\\
  0.899999999999999	-1.90872206896167\\
  0.95	-1.88768805127536\\
  1	-1.86563785653494\\
  1.05	-1.84259113635756\\
  1.1	-1.81856873470627\\
  1.15	-1.79359273018384\\
  1.2	-1.76768648491562\\
  1.25	-1.74087470105994\\
  1.3	-1.71318348615937\\
  1.35	-1.68464042875211\\
  1.4	-1.65527468590578\\
  1.45	-1.62511708462336\\
  1.5	-1.59420023941016\\
  1.55	-1.56255868869178\\
  1.6	-1.53022905324576\\
  1.65	-1.49725022036617\\
  1.7	-1.46366355813382\\
  1.75	-1.42951316492749\\
  1.8	-1.39484616019611\\
  1.85	-1.35971302352775\\
  1.9	-1.32416799020101\\
  1.95	-1.2882695126787\\
  2	-1.25208079887271\\
  2.05	-1.21567043940641\\
  2.1	-1.17911313740467\\
  2.15	-1.14249055533928\\
  2.2	-1.10589229380139\\
  2.25	-1.06941701620767\\
  2.3	-1.03317373052181\\
  2.35	-0.997283232814476\\
  2.4	-0.96187970605347\\
  2.45	-0.927112448360707\\
  2.5	-0.893147674747089\\
  2.55	-0.860170291006122\\
  2.6	-0.828385473835765\\
  2.65	-0.798019804418053\\
  2.7	-0.769321594631191\\
  2.75	-0.742559925349512\\
  2.8	-0.718021809322449\\
  2.85	-0.696006842562499\\
  2.9	-0.676818786456773\\
  2.95	-0.660753807407185\\
  3	-0.64808564773424\\
  3.05	-0.639048786026448\\
  3.1	-0.633821504410698\\
  3.15	-0.63251140942857\\
  3.2	-0.635146005584931\\
  3.25	-0.641670201769497\\
  3.3	-0.65195127132499\\
  3.35	-0.665790239366823\\
  3.4	-0.682937489702445\\
  3.45	-0.703109927971004\\
  3.5	-0.726007317744392\\
  3.55	-0.751326152923914\\
  3.6	-0.778770291762408\\
  3.65	-0.808058292842864\\
  3.7	-0.838927849447844\\
  3.75	-0.871137925543871\\
  3.8	-0.904469223451614\\
  3.85	-0.938723538495218\\
  3.9	-0.973722440739516\\
  3.95	-1.00930560628532\\
  4	-1.04532901914793\\
  4.05	-1.08166318553829\\
  4.1	-1.11819144467013\\
  4.15	-1.15480842044284\\
  4.2	-1.19141863226936\\
  4.25	-1.22793526709367\\
  4.3	-1.26427910519792\\
  4.35	-1.30037758743518\\
  4.4	-1.33616400941021\\
  4.45	-1.37157682775\\
  4.5	-1.40655906421964\\
  4.55	-1.44105779457058\\
  4.6	-1.47502371036194\\
  4.65	-1.50841074339386\\
  4.7	-1.54117574373406\\
  4.75	-1.57327820355332\\
  4.8	-1.60468002009089\\
  4.85	-1.63534529204135\\
  4.9	-1.66524014449723\\
  4.95	-1.69433257830595\\
  5	-1.72259234031922\\
  };
  
  \addplot[area legend, draw=black, fill=black, forget plot]
  table[row sep=crcr] {%
  x	y\\
  -0.0806451612903221	2.0976176963403\\
  -0.0999999999999996	2.0976176963403\\
  -0.0999999999999996	2.0976176963403\\
  -0.0999999999999996	2.0976176963403\\
  -0.0999999999999996	2.0976176963403\\
  -0.0999999999999996	2.0976176963403\\
  -0.0612903225806445	2.0976176963403\\
  -0.0612903225806445	2.05136844057275\\
  0.0999999999999996	2.0976176963403\\
  -0.0612903225806436	2.14386695210785\\
  -0.0612903225806445	2.0976176963403\\
  }--cycle;
  
  \addplot[area legend, draw=black, fill=black, forget plot]
  table[row sep=crcr] {%
  x	y\\
  0.080645161290323	-2.0976176963403\\
  0.0999999999999996	-2.0976176963403\\
  0.0999999999999996	-2.0976176963403\\
  0.0999999999999996	-2.0976176963403\\
  0.0999999999999996	-2.0976176963403\\
  0.0999999999999996	-2.0976176963403\\
  0.0612903225806463	-2.0976176963403\\
  0.0612903225806463	-2.14386695210785\\
  -0.0999999999999996	-2.0976176963403\\
  0.0612903225806463	-2.05136844057275\\
  0.0612903225806463	-2.0976176963403\\
  }--cycle;
  \addplot [color=black, forget plot]
    table[row sep=crcr]{%
  -5	1.88873618351702\\
  -4.95	1.86299835907306\\
  -4.9	1.83657963041224\\
  -4.85	1.80951767722833\\
  -4.8	1.78185239761291\\
  -4.75	1.75362604502099\\
  -4.7	1.72488337955765\\
  -4.65	1.69567183463842\\
  -4.6	1.6660416999973\\
  -4.55	1.63604632186642\\
  -4.5	1.60574232090284\\
  -4.45	1.57518982805907\\
  -4.4	1.54445273804127\\
  -4.35	1.51359897922261\\
  -4.3	1.4827007978146\\
  -4.25	1.45183505267383\\
  -4.2	1.42108351525116\\
  -4.15	1.39053316678377\\
  -4.1	1.36027648179826\\
  -4.05	1.3304116832578\\
  -4	1.30104295020294\\
  -3.95	1.27228055352543\\
  -3.9	1.24424088969931\\
  -3.85	1.21704637616033\\
  -3.8	1.19082516608072\\
  -3.75	1.16571063532975\\
  -3.7	1.14184059157975\\
  -3.65	1.11935615629348\\
  -3.6	1.09840027646196\\
  -3.55	1.07911583626015\\
  -3.5	1.0616433607471\\
  -3.45	1.04611833499437\\
  -3.4	1.03266820171877\\
  -3.35	1.02140914565914\\
  -3.3	1.0124428182284\\
  -3.25	1.00585319397957\\
  -3.2	1.0017037727844\\
  -3.15	1.00003534090417\\
  -3.1	1.00086447606729\\
  -3.05	1.0041829270217\\
  -3	1.00995792328151\\
  -2.95	1.01813338713702\\
  -2.9	1.02863194083249\\
  -2.85	1.04135753941373\\
  -2.8	1.05619852237289\\
  -2.75	1.0730308675593\\
  -2.7	1.09172144614177\\
  -2.65	1.11213111108512\\
  -2.6	1.13411749535139\\
  -2.55	1.15753744195579\\
  -2.5	1.18224902998739\\
  -2.45	1.20811319498853\\
  -2.4	1.23499496716283\\
  -2.35	1.26276436695565\\
  -2.3	1.2912970058977\\
  -2.25	1.32047444297666\\
  -2.2	1.35018434500231\\
  -2.15	1.38032049504434\\
  -2.1	1.41078268730527\\
  -2.05	1.44147654065079\\
  -2	1.47231325705697\\
  -1.95	1.50320934579899\\
  -1.9	1.53408632947204\\
  -1.85	1.56487044395086\\
  -1.8	1.59549234113293\\
  -1.75	1.62588680070324\\
  -1.7	1.65599245511837\\
  -1.65	1.6857515304417\\
  -1.6	1.71510960448521\\
  -1.55	1.74401538284678\\
  -1.5	1.77242049281072\\
  -1.45	1.80027929464701\\
  -1.4	1.82754870955619\\
  -1.35	1.85418806332747\\
  -1.3	1.88015894467707\\
  -1.25	1.90542507719158\\
  -1.2	1.92995220380023\\
  -1.15	1.95370798272626\\
  -1.1	1.97666189391387\\
  -1.05	1.99878515498376\\
  -1	2.02005064583448\\
  -0.95	2.04043284107264\\
  -0.899999999999999	2.05990774952213\\
  -0.85	2.07845286012697\\
  -0.8	2.09604709362512\\
  -0.75	2.11267075942932\\
  -0.7	2.12830551720588\\
  -0.649999999999999	2.14293434269417\\
  -0.6	2.15654149735621\\
  -0.55	2.16911250148973\\
  -0.5	2.18063411047813\\
  -0.45	2.19109429388727\\
  -0.399999999999999	2.20048221715282\\
  -0.35	2.20878822563295\\
  -0.3	2.21600383082954\\
  -0.25	2.22212169860728\\
  -0.199999999999999	2.22713563926459\\
  -0.149999999999999	2.23104059933299\\
  -0.0999999999999996	2.23383265500262\\
  -0.0499999999999998	2.23550900709211\\
  0	2.23606797749979\\
  0.0499999999999998	2.23550900709211\\
  0.0999999999999996	2.23383265500262\\
  0.149999999999999	2.23104059933299\\
  0.199999999999999	2.22713563926459\\
  0.25	2.22212169860728\\
  0.3	2.21600383082954\\
  0.35	2.20878822563295\\
  0.399999999999999	2.20048221715282\\
  0.45	2.19109429388727\\
  0.5	2.18063411047813\\
  0.55	2.16911250148973\\
  0.6	2.15654149735621\\
  0.649999999999999	2.14293434269417\\
  0.7	2.12830551720588\\
  0.75	2.11267075942932\\
  0.8	2.09604709362512\\
  0.85	2.07845286012697\\
  0.899999999999999	2.05990774952213\\
  0.95	2.04043284107264\\
  1	2.02005064583448\\
  1.05	1.99878515498376\\
  1.1	1.97666189391387\\
  1.15	1.95370798272626\\
  1.2	1.92995220380023\\
  1.25	1.90542507719158\\
  1.3	1.88015894467707\\
  1.35	1.85418806332747\\
  1.4	1.82754870955619\\
  1.45	1.80027929464701\\
  1.5	1.77242049281072\\
  1.55	1.74401538284678\\
  1.6	1.71510960448521\\
  1.65	1.6857515304417\\
  1.7	1.65599245511837\\
  1.75	1.62588680070324\\
  1.8	1.59549234113293\\
  1.85	1.56487044395086\\
  1.9	1.53408632947204\\
  1.95	1.50320934579899\\
  2	1.47231325705697\\
  2.05	1.44147654065079\\
  2.1	1.41078268730527\\
  2.15	1.38032049504434\\
  2.2	1.35018434500231\\
  2.25	1.32047444297666\\
  2.3	1.2912970058977\\
  2.35	1.26276436695565\\
  2.4	1.23499496716283\\
  2.45	1.20811319498853\\
  2.5	1.18224902998739\\
  2.55	1.15753744195579\\
  2.6	1.13411749535139\\
  2.65	1.11213111108512\\
  2.7	1.09172144614177\\
  2.75	1.0730308675593\\
  2.8	1.05619852237289\\
  2.85	1.04135753941373\\
  2.9	1.02863194083249\\
  2.95	1.01813338713702\\
  3	1.00995792328151\\
  3.05	1.0041829270217\\
  3.1	1.00086447606729\\
  3.15	1.00003534090417\\
  3.2	1.0017037727844\\
  3.25	1.00585319397957\\
  3.3	1.0124428182284\\
  3.35	1.02140914565914\\
  3.4	1.03266820171877\\
  3.45	1.04611833499437\\
  3.5	1.0616433607471\\
  3.55	1.07911583626015\\
  3.6	1.09840027646196\\
  3.65	1.11935615629348\\
  3.7	1.14184059157975\\
  3.75	1.16571063532975\\
  3.8	1.19082516608072\\
  3.85	1.21704637616033\\
  3.9	1.24424088969931\\
  3.95	1.27228055352543\\
  4	1.30104295020294\\
  4.05	1.3304116832578\\
  4.1	1.36027648179826\\
  4.15	1.39053316678377\\
  4.2	1.42108351525116\\
  4.25	1.45183505267383\\
  4.3	1.4827007978146\\
  4.35	1.51359897922261\\
  4.4	1.54445273804127\\
  4.45	1.57518982805907\\
  4.5	1.60574232090284\\
  4.55	1.63604632186642\\
  4.6	1.6660416999973\\
  4.65	1.69567183463842\\
  4.7	1.72488337955765\\
  4.75	1.75362604502099\\
  4.8	1.78185239761291\\
  4.85	1.80951767722833\\
  4.9	1.83657963041224\\
  4.95	1.86299835907306\\
  5	1.88873618351702\\
  };
  \addplot [color=black, forget plot]
    table[row sep=crcr]{%
  -5	-1.88873618351702\\
  -4.95	-1.86299835907306\\
  -4.9	-1.83657963041224\\
  -4.85	-1.80951767722833\\
  -4.8	-1.78185239761291\\
  -4.75	-1.75362604502099\\
  -4.7	-1.72488337955765\\
  -4.65	-1.69567183463842\\
  -4.6	-1.6660416999973\\
  -4.55	-1.63604632186642\\
  -4.5	-1.60574232090284\\
  -4.45	-1.57518982805907\\
  -4.4	-1.54445273804127\\
  -4.35	-1.51359897922261\\
  -4.3	-1.4827007978146\\
  -4.25	-1.45183505267383\\
  -4.2	-1.42108351525116\\
  -4.15	-1.39053316678377\\
  -4.1	-1.36027648179826\\
  -4.05	-1.3304116832578\\
  -4	-1.30104295020294\\
  -3.95	-1.27228055352543\\
  -3.9	-1.24424088969931\\
  -3.85	-1.21704637616033\\
  -3.8	-1.19082516608072\\
  -3.75	-1.16571063532975\\
  -3.7	-1.14184059157975\\
  -3.65	-1.11935615629348\\
  -3.6	-1.09840027646196\\
  -3.55	-1.07911583626015\\
  -3.5	-1.0616433607471\\
  -3.45	-1.04611833499437\\
  -3.4	-1.03266820171877\\
  -3.35	-1.02140914565914\\
  -3.3	-1.0124428182284\\
  -3.25	-1.00585319397957\\
  -3.2	-1.0017037727844\\
  -3.15	-1.00003534090417\\
  -3.1	-1.00086447606729\\
  -3.05	-1.0041829270217\\
  -3	-1.00995792328151\\
  -2.95	-1.01813338713702\\
  -2.9	-1.02863194083249\\
  -2.85	-1.04135753941373\\
  -2.8	-1.05619852237289\\
  -2.75	-1.0730308675593\\
  -2.7	-1.09172144614177\\
  -2.65	-1.11213111108512\\
  -2.6	-1.13411749535139\\
  -2.55	-1.15753744195579\\
  -2.5	-1.18224902998739\\
  -2.45	-1.20811319498853\\
  -2.4	-1.23499496716283\\
  -2.35	-1.26276436695565\\
  -2.3	-1.2912970058977\\
  -2.25	-1.32047444297666\\
  -2.2	-1.35018434500231\\
  -2.15	-1.38032049504434\\
  -2.1	-1.41078268730527\\
  -2.05	-1.44147654065079\\
  -2	-1.47231325705697\\
  -1.95	-1.50320934579899\\
  -1.9	-1.53408632947204\\
  -1.85	-1.56487044395086\\
  -1.8	-1.59549234113293\\
  -1.75	-1.62588680070324\\
  -1.7	-1.65599245511837\\
  -1.65	-1.6857515304417\\
  -1.6	-1.71510960448521\\
  -1.55	-1.74401538284678\\
  -1.5	-1.77242049281072\\
  -1.45	-1.80027929464701\\
  -1.4	-1.82754870955619\\
  -1.35	-1.85418806332747\\
  -1.3	-1.88015894467707\\
  -1.25	-1.90542507719158\\
  -1.2	-1.92995220380023\\
  -1.15	-1.95370798272626\\
  -1.1	-1.97666189391387\\
  -1.05	-1.99878515498376\\
  -1	-2.02005064583448\\
  -0.95	-2.04043284107264\\
  -0.899999999999999	-2.05990774952213\\
  -0.85	-2.07845286012697\\
  -0.8	-2.09604709362512\\
  -0.75	-2.11267075942932\\
  -0.7	-2.12830551720588\\
  -0.649999999999999	-2.14293434269417\\
  -0.6	-2.15654149735621\\
  -0.55	-2.16911250148973\\
  -0.5	-2.18063411047813\\
  -0.45	-2.19109429388727\\
  -0.399999999999999	-2.20048221715282\\
  -0.35	-2.20878822563295\\
  -0.3	-2.21600383082954\\
  -0.25	-2.22212169860728\\
  -0.199999999999999	-2.22713563926459\\
  -0.149999999999999	-2.23104059933299\\
  -0.0999999999999996	-2.23383265500262\\
  -0.0499999999999998	-2.23550900709211\\
  0	-2.23606797749979\\
  0.0499999999999998	-2.23550900709211\\
  0.0999999999999996	-2.23383265500262\\
  0.149999999999999	-2.23104059933299\\
  0.199999999999999	-2.22713563926459\\
  0.25	-2.22212169860728\\
  0.3	-2.21600383082954\\
  0.35	-2.20878822563295\\
  0.399999999999999	-2.20048221715282\\
  0.45	-2.19109429388727\\
  0.5	-2.18063411047813\\
  0.55	-2.16911250148973\\
  0.6	-2.15654149735621\\
  0.649999999999999	-2.14293434269417\\
  0.7	-2.12830551720588\\
  0.75	-2.11267075942932\\
  0.8	-2.09604709362512\\
  0.85	-2.07845286012697\\
  0.899999999999999	-2.05990774952213\\
  0.95	-2.04043284107264\\
  1	-2.02005064583448\\
  1.05	-1.99878515498376\\
  1.1	-1.97666189391387\\
  1.15	-1.95370798272626\\
  1.2	-1.92995220380023\\
  1.25	-1.90542507719158\\
  1.3	-1.88015894467707\\
  1.35	-1.85418806332747\\
  1.4	-1.82754870955619\\
  1.45	-1.80027929464701\\
  1.5	-1.77242049281072\\
  1.55	-1.74401538284678\\
  1.6	-1.71510960448521\\
  1.65	-1.6857515304417\\
  1.7	-1.65599245511837\\
  1.75	-1.62588680070324\\
  1.8	-1.59549234113293\\
  1.85	-1.56487044395086\\
  1.9	-1.53408632947204\\
  1.95	-1.50320934579899\\
  2	-1.47231325705697\\
  2.05	-1.44147654065079\\
  2.1	-1.41078268730527\\
  2.15	-1.38032049504434\\
  2.2	-1.35018434500231\\
  2.25	-1.32047444297666\\
  2.3	-1.2912970058977\\
  2.35	-1.26276436695565\\
  2.4	-1.23499496716283\\
  2.45	-1.20811319498853\\
  2.5	-1.18224902998739\\
  2.55	-1.15753744195579\\
  2.6	-1.13411749535139\\
  2.65	-1.11213111108512\\
  2.7	-1.09172144614177\\
  2.75	-1.0730308675593\\
  2.8	-1.05619852237289\\
  2.85	-1.04135753941373\\
  2.9	-1.02863194083249\\
  2.95	-1.01813338713702\\
  3	-1.00995792328151\\
  3.05	-1.0041829270217\\
  3.1	-1.00086447606729\\
  3.15	-1.00003534090417\\
  3.2	-1.0017037727844\\
  3.25	-1.00585319397957\\
  3.3	-1.0124428182284\\
  3.35	-1.02140914565914\\
  3.4	-1.03266820171877\\
  3.45	-1.04611833499437\\
  3.5	-1.0616433607471\\
  3.55	-1.07911583626015\\
  3.6	-1.09840027646196\\
  3.65	-1.11935615629348\\
  3.7	-1.14184059157975\\
  3.75	-1.16571063532975\\
  3.8	-1.19082516608072\\
  3.85	-1.21704637616033\\
  3.9	-1.24424088969931\\
  3.95	-1.27228055352543\\
  4	-1.30104295020294\\
  4.05	-1.3304116832578\\
  4.1	-1.36027648179826\\
  4.15	-1.39053316678377\\
  4.2	-1.42108351525116\\
  4.25	-1.45183505267383\\
  4.3	-1.4827007978146\\
  4.35	-1.51359897922261\\
  4.4	-1.54445273804127\\
  4.45	-1.57518982805907\\
  4.5	-1.60574232090284\\
  4.55	-1.63604632186642\\
  4.6	-1.6660416999973\\
  4.65	-1.69567183463842\\
  4.7	-1.72488337955765\\
  4.75	-1.75362604502099\\
  4.8	-1.78185239761291\\
  4.85	-1.80951767722833\\
  4.9	-1.83657963041224\\
  4.95	-1.86299835907306\\
  5	-1.88873618351702\\
  };
  
  \addplot[area legend, draw=black, fill=black, forget plot]
  table[row sep=crcr] {%
  x	y\\
  -0.0806451612903221	2.23606797749979\\
  -0.0999999999999996	2.23606797749979\\
  -0.0999999999999996	2.23606797749979\\
  -0.0999999999999996	2.23606797749979\\
  -0.0999999999999996	2.23606797749979\\
  -0.0999999999999996	2.23606797749979\\
  -0.0612903225806445	2.23606797749979\\
  -0.0612903225806445	2.18981872173224\\
  0.0999999999999996	2.23606797749979\\
  -0.0612903225806436	2.28231723326734\\
  -0.0612903225806445	2.23606797749979\\
  }--cycle;
  
  \addplot[area legend, draw=black, fill=black, forget plot]
  table[row sep=crcr] {%
  x	y\\
  0.080645161290323	-2.23606797749979\\
  0.0999999999999996	-2.23606797749979\\
  0.0999999999999996	-2.23606797749979\\
  0.0999999999999996	-2.23606797749979\\
  0.0999999999999996	-2.23606797749979\\
  0.0999999999999996	-2.23606797749979\\
  0.0612903225806463	-2.23606797749979\\
  0.0612903225806463	-2.28231723326734\\
  -0.0999999999999996	-2.23606797749979\\
  0.0612903225806463	-2.18981872173224\\
  0.0612903225806463	-2.23606797749979\\
  }--cycle;
  \addplot [color=black, forget plot]
    table[row sep=crcr]{%
  -5	2.0414025499461\\
  -4.95	2.01761316557682\\
  -4.9	1.99324477644999\\
  -4.85	1.96833793445176\\
  -4.8	1.94293539956399\\
  -4.75	1.91708223761422\\
  -4.7	1.89082592352501\\
  -4.65	1.86421644955354\\
  -4.6	1.83730643773158\\
  -4.55	1.81015125536311\\
  -4.5	1.78280913199884\\
  -4.45	1.75534127576969\\
  -4.4	1.72781198631193\\
  -4.35	1.70028876074146\\
  -4.3	1.67284238822432\\
  -4.25	1.64554702763926\\
  -4.2	1.6184802616401\\
  -4.15	1.59172311911516\\
  -4.1	1.56536005664303\\
  -4.05	1.53947888811404\\
  -4	1.51417065031415\\
  -3.95	1.48952939107591\\
  -3.9	1.46565186575794\\
  -3.85	1.44263712752895\\
  -3.8	1.42058599745709\\
  -3.75	1.39960040201512\\
  -3.7	1.37978256858796\\
  -3.65	1.36123407415188\\
  -3.6	1.34405474863627\\
  -3.55	1.32834144257696\\
  -3.5	1.31418667829894\\
  -3.45	1.30167721452417\\
  -3.4	1.29089256518158\\
  -3.35	1.28190352321699\\
  -3.3	1.27477074808856\\
  -3.25	1.26954348008995\\
  -3.2	1.26625844455644\\
  -3.15	1.26493900369042\\
  -3.1	1.26559460312275\\
  -3.05	1.26822054506378\\
  -3	1.27279810134959\\
  -2.95	1.27929495973489\\
  -2.9	1.28766597753486\\
  -2.85	1.29785420016804\\
  -2.8	1.30979208986109\\
  -2.75	1.32340290264721\\
  -2.7	1.338602149993\\
  -2.65	1.35529908442507\\
  -2.6	1.37339815540218\\
  -2.55	1.39280039112917\\
  -2.5	1.41340467273394\\
  -2.45	1.43510887806653\\
  -2.4	1.45781088242526\\
  -2.35	1.48140941216562\\
  -2.3	1.5058047540901\\
  -2.25	1.53089932868054\\
  -2.2	1.5565981387273\\
  -2.15	1.58280910694861\\
  -2.1	1.60944331705105\\
  -2.05	1.63641517264005\\
  -2	1.66364248770753\\
  -1.95	1.69104652132856\\
  -1.9	1.71855196787091\\
  -1.85	1.74608691259942\\
  -1.8	1.77358276114024\\
  -1.75	1.80097414992582\\
  -1.7	1.82819884350936\\
  -1.65	1.85519762353948\\
  -1.6	1.88191417322826\\
  -1.55	1.90829496032615\\
  -1.5	1.93428912092671\\
  -1.45	1.95984834585096\\
  -1.4	1.98492677089118\\
  -1.35	2.00948087181393\\
  -1.3	2.03346936471863\\
  -1.25	2.05685311210853\\
  -1.2	2.07959503484533\\
  -1.15	2.10166003001635\\
  -1.1	2.12301489463714\\
  -1.05	2.14362825503478\\
  -1	2.16347050170236\\
  -0.95	2.18251372937899\\
  -0.899999999999999	2.20073168208697\\
  -0.85	2.218099702847\\
  -0.8	2.23459468778889\\
  -0.75	2.25019504437896\\
  -0.7	2.26488065349346\\
  -0.649999999999999	2.27863283507855\\
  -0.6	2.29143431715146\\
  -0.55	2.30326920791275\\
  -0.5	2.31412297075604\\
  -0.45	2.32398240197841\\
  -0.399999999999999	2.33283561101201\\
  -0.35	2.34067200301425\\
  -0.3	2.34748226367127\\
  -0.25	2.35325834608555\\
  -0.199999999999999	2.35799345963522\\
  -0.149999999999999	2.36168206070844\\
  -0.0999999999999996	2.36431984523162\\
  -0.0499999999999998	2.36590374292572\\
  0	2.36643191323985\\
  0.0499999999999998	2.36590374292572\\
  0.0999999999999996	2.36431984523162\\
  0.149999999999999	2.36168206070844\\
  0.199999999999999	2.35799345963522\\
  0.25	2.35325834608555\\
  0.3	2.34748226367127\\
  0.35	2.34067200301425\\
  0.399999999999999	2.33283561101201\\
  0.45	2.32398240197841\\
  0.5	2.31412297075604\\
  0.55	2.30326920791275\\
  0.6	2.29143431715146\\
  0.649999999999999	2.27863283507855\\
  0.7	2.26488065349346\\
  0.75	2.25019504437896\\
  0.8	2.23459468778889\\
  0.85	2.218099702847\\
  0.899999999999999	2.20073168208697\\
  0.95	2.18251372937899\\
  1	2.16347050170236\\
  1.05	2.14362825503478\\
  1.1	2.12301489463714\\
  1.15	2.10166003001635\\
  1.2	2.07959503484533\\
  1.25	2.05685311210853\\
  1.3	2.03346936471863\\
  1.35	2.00948087181393\\
  1.4	1.98492677089118\\
  1.45	1.95984834585096\\
  1.5	1.93428912092671\\
  1.55	1.90829496032615\\
  1.6	1.88191417322826\\
  1.65	1.85519762353948\\
  1.7	1.82819884350936\\
  1.75	1.80097414992582\\
  1.8	1.77358276114024\\
  1.85	1.74608691259942\\
  1.9	1.71855196787091\\
  1.95	1.69104652132856\\
  2	1.66364248770753\\
  2.05	1.63641517264005\\
  2.1	1.60944331705105\\
  2.15	1.58280910694861\\
  2.2	1.5565981387273\\
  2.25	1.53089932868054\\
  2.3	1.5058047540901\\
  2.35	1.48140941216562\\
  2.4	1.45781088242526\\
  2.45	1.43510887806653\\
  2.5	1.41340467273394\\
  2.55	1.39280039112917\\
  2.6	1.37339815540218\\
  2.65	1.35529908442507\\
  2.7	1.338602149993\\
  2.75	1.32340290264721\\
  2.8	1.30979208986109\\
  2.85	1.29785420016804\\
  2.9	1.28766597753486\\
  2.95	1.27929495973489\\
  3	1.27279810134959\\
  3.05	1.26822054506378\\
  3.1	1.26559460312275\\
  3.15	1.26493900369042\\
  3.2	1.26625844455644\\
  3.25	1.26954348008995\\
  3.3	1.27477074808856\\
  3.35	1.28190352321699\\
  3.4	1.29089256518158\\
  3.45	1.30167721452417\\
  3.5	1.31418667829894\\
  3.55	1.32834144257696\\
  3.6	1.34405474863627\\
  3.65	1.36123407415188\\
  3.7	1.37978256858796\\
  3.75	1.39960040201512\\
  3.8	1.42058599745709\\
  3.85	1.44263712752895\\
  3.9	1.46565186575794\\
  3.95	1.48952939107591\\
  4	1.51417065031415\\
  4.05	1.53947888811404\\
  4.1	1.56536005664303\\
  4.15	1.59172311911516\\
  4.2	1.6184802616401\\
  4.25	1.64554702763926\\
  4.3	1.67284238822432\\
  4.35	1.70028876074146\\
  4.4	1.72781198631193\\
  4.45	1.75534127576969\\
  4.5	1.78280913199884\\
  4.55	1.81015125536311\\
  4.6	1.83730643773158\\
  4.65	1.86421644955354\\
  4.7	1.89082592352501\\
  4.75	1.91708223761422\\
  4.8	1.94293539956399\\
  4.85	1.96833793445176\\
  4.9	1.99324477644999\\
  4.95	2.01761316557682\\
  5	2.0414025499461\\
  };
  \addplot [color=black, forget plot]
    table[row sep=crcr]{%
  -5	-2.0414025499461\\
  -4.95	-2.01761316557682\\
  -4.9	-1.99324477644999\\
  -4.85	-1.96833793445176\\
  -4.8	-1.94293539956399\\
  -4.75	-1.91708223761422\\
  -4.7	-1.89082592352501\\
  -4.65	-1.86421644955354\\
  -4.6	-1.83730643773158\\
  -4.55	-1.81015125536311\\
  -4.5	-1.78280913199884\\
  -4.45	-1.75534127576969\\
  -4.4	-1.72781198631193\\
  -4.35	-1.70028876074146\\
  -4.3	-1.67284238822432\\
  -4.25	-1.64554702763926\\
  -4.2	-1.6184802616401\\
  -4.15	-1.59172311911516\\
  -4.1	-1.56536005664303\\
  -4.05	-1.53947888811404\\
  -4	-1.51417065031415\\
  -3.95	-1.48952939107591\\
  -3.9	-1.46565186575794\\
  -3.85	-1.44263712752895\\
  -3.8	-1.42058599745709\\
  -3.75	-1.39960040201512\\
  -3.7	-1.37978256858796\\
  -3.65	-1.36123407415188\\
  -3.6	-1.34405474863627\\
  -3.55	-1.32834144257696\\
  -3.5	-1.31418667829894\\
  -3.45	-1.30167721452417\\
  -3.4	-1.29089256518158\\
  -3.35	-1.28190352321699\\
  -3.3	-1.27477074808856\\
  -3.25	-1.26954348008995\\
  -3.2	-1.26625844455644\\
  -3.15	-1.26493900369042\\
  -3.1	-1.26559460312275\\
  -3.05	-1.26822054506378\\
  -3	-1.27279810134959\\
  -2.95	-1.27929495973489\\
  -2.9	-1.28766597753486\\
  -2.85	-1.29785420016804\\
  -2.8	-1.30979208986109\\
  -2.75	-1.32340290264721\\
  -2.7	-1.338602149993\\
  -2.65	-1.35529908442507\\
  -2.6	-1.37339815540218\\
  -2.55	-1.39280039112917\\
  -2.5	-1.41340467273394\\
  -2.45	-1.43510887806653\\
  -2.4	-1.45781088242526\\
  -2.35	-1.48140941216562\\
  -2.3	-1.5058047540901\\
  -2.25	-1.53089932868054\\
  -2.2	-1.5565981387273\\
  -2.15	-1.58280910694861\\
  -2.1	-1.60944331705105\\
  -2.05	-1.63641517264005\\
  -2	-1.66364248770753\\
  -1.95	-1.69104652132856\\
  -1.9	-1.71855196787091\\
  -1.85	-1.74608691259942\\
  -1.8	-1.77358276114024\\
  -1.75	-1.80097414992582\\
  -1.7	-1.82819884350936\\
  -1.65	-1.85519762353948\\
  -1.6	-1.88191417322826\\
  -1.55	-1.90829496032615\\
  -1.5	-1.93428912092671\\
  -1.45	-1.95984834585096\\
  -1.4	-1.98492677089118\\
  -1.35	-2.00948087181393\\
  -1.3	-2.03346936471863\\
  -1.25	-2.05685311210853\\
  -1.2	-2.07959503484533\\
  -1.15	-2.10166003001635\\
  -1.1	-2.12301489463714\\
  -1.05	-2.14362825503478\\
  -1	-2.16347050170236\\
  -0.95	-2.18251372937899\\
  -0.899999999999999	-2.20073168208697\\
  -0.85	-2.218099702847\\
  -0.8	-2.23459468778889\\
  -0.75	-2.25019504437896\\
  -0.7	-2.26488065349346\\
  -0.649999999999999	-2.27863283507855\\
  -0.6	-2.29143431715146\\
  -0.55	-2.30326920791275\\
  -0.5	-2.31412297075604\\
  -0.45	-2.32398240197841\\
  -0.399999999999999	-2.33283561101201\\
  -0.35	-2.34067200301425\\
  -0.3	-2.34748226367127\\
  -0.25	-2.35325834608555\\
  -0.199999999999999	-2.35799345963522\\
  -0.149999999999999	-2.36168206070844\\
  -0.0999999999999996	-2.36431984523162\\
  -0.0499999999999998	-2.36590374292572\\
  0	-2.36643191323985\\
  0.0499999999999998	-2.36590374292572\\
  0.0999999999999996	-2.36431984523162\\
  0.149999999999999	-2.36168206070844\\
  0.199999999999999	-2.35799345963522\\
  0.25	-2.35325834608555\\
  0.3	-2.34748226367127\\
  0.35	-2.34067200301425\\
  0.399999999999999	-2.33283561101201\\
  0.45	-2.32398240197841\\
  0.5	-2.31412297075604\\
  0.55	-2.30326920791275\\
  0.6	-2.29143431715146\\
  0.649999999999999	-2.27863283507855\\
  0.7	-2.26488065349346\\
  0.75	-2.25019504437896\\
  0.8	-2.23459468778889\\
  0.85	-2.218099702847\\
  0.899999999999999	-2.20073168208697\\
  0.95	-2.18251372937899\\
  1	-2.16347050170236\\
  1.05	-2.14362825503478\\
  1.1	-2.12301489463714\\
  1.15	-2.10166003001635\\
  1.2	-2.07959503484533\\
  1.25	-2.05685311210853\\
  1.3	-2.03346936471863\\
  1.35	-2.00948087181393\\
  1.4	-1.98492677089118\\
  1.45	-1.95984834585096\\
  1.5	-1.93428912092671\\
  1.55	-1.90829496032615\\
  1.6	-1.88191417322826\\
  1.65	-1.85519762353948\\
  1.7	-1.82819884350936\\
  1.75	-1.80097414992582\\
  1.8	-1.77358276114024\\
  1.85	-1.74608691259942\\
  1.9	-1.71855196787091\\
  1.95	-1.69104652132856\\
  2	-1.66364248770753\\
  2.05	-1.63641517264005\\
  2.1	-1.60944331705105\\
  2.15	-1.58280910694861\\
  2.2	-1.5565981387273\\
  2.25	-1.53089932868054\\
  2.3	-1.5058047540901\\
  2.35	-1.48140941216562\\
  2.4	-1.45781088242526\\
  2.45	-1.43510887806653\\
  2.5	-1.41340467273394\\
  2.55	-1.39280039112917\\
  2.6	-1.37339815540218\\
  2.65	-1.35529908442507\\
  2.7	-1.338602149993\\
  2.75	-1.32340290264721\\
  2.8	-1.30979208986109\\
  2.85	-1.29785420016804\\
  2.9	-1.28766597753486\\
  2.95	-1.27929495973489\\
  3	-1.27279810134959\\
  3.05	-1.26822054506378\\
  3.1	-1.26559460312275\\
  3.15	-1.26493900369042\\
  3.2	-1.26625844455644\\
  3.25	-1.26954348008995\\
  3.3	-1.27477074808856\\
  3.35	-1.28190352321699\\
  3.4	-1.29089256518158\\
  3.45	-1.30167721452417\\
  3.5	-1.31418667829894\\
  3.55	-1.32834144257696\\
  3.6	-1.34405474863627\\
  3.65	-1.36123407415188\\
  3.7	-1.37978256858796\\
  3.75	-1.39960040201512\\
  3.8	-1.42058599745709\\
  3.85	-1.44263712752895\\
  3.9	-1.46565186575794\\
  3.95	-1.48952939107591\\
  4	-1.51417065031415\\
  4.05	-1.53947888811404\\
  4.1	-1.56536005664303\\
  4.15	-1.59172311911516\\
  4.2	-1.6184802616401\\
  4.25	-1.64554702763926\\
  4.3	-1.67284238822432\\
  4.35	-1.70028876074146\\
  4.4	-1.72781198631193\\
  4.45	-1.75534127576969\\
  4.5	-1.78280913199884\\
  4.55	-1.81015125536311\\
  4.6	-1.83730643773158\\
  4.65	-1.86421644955354\\
  4.7	-1.89082592352501\\
  4.75	-1.91708223761422\\
  4.8	-1.94293539956399\\
  4.85	-1.96833793445176\\
  4.9	-1.99324477644999\\
  4.95	-2.01761316557682\\
  5	-2.0414025499461\\
  };
  
  \addplot[area legend, draw=black, fill=black, forget plot]
  table[row sep=crcr] {%
  x	y\\
  -0.0806451612903221	2.36643191323985\\
  -0.0999999999999996	2.36643191323985\\
  -0.0999999999999996	2.36643191323985\\
  -0.0999999999999996	2.36643191323985\\
  -0.0999999999999996	2.36643191323985\\
  -0.0999999999999996	2.36643191323985\\
  -0.0612903225806445	2.36643191323985\\
  -0.0612903225806445	2.3201826574723\\
  0.0999999999999996	2.36643191323985\\
  -0.0612903225806436	2.4126811690074\\
  -0.0612903225806445	2.36643191323985\\
  }--cycle;
  
  \addplot[area legend, draw=black, fill=black, forget plot]
  table[row sep=crcr] {%
  x	y\\
  0.080645161290323	-2.36643191323985\\
  0.0999999999999996	-2.36643191323985\\
  0.0999999999999996	-2.36643191323985\\
  0.0999999999999996	-2.36643191323985\\
  0.0999999999999996	-2.36643191323985\\
  0.0999999999999996	-2.36643191323985\\
  0.0612903225806463	-2.36643191323985\\
  0.0612903225806463	-2.4126811690074\\
  -0.0999999999999996	-2.36643191323985\\
  0.0612903225806463	-2.3201826574723\\
  0.0612903225806463	-2.36643191323985\\
  }--cycle;
  \end{axis}
  \end{tikzpicture}%
            \end{center}
            \caption{Orbite per il pendolo piano senza attrito}\label{fig:sodaufnaisdufjaiuhfaoisfjdaosi}
        \end{figure}

        Per una comprensione più profonda, si rimanda alle dispense del corso, \cite{dispense}, pp. 27-31.
    \end{enumerate}
}{dsfsdfsdfsdfsdfsdfsdfds}{}
\osservazione{
    L'equazione caratterizzante del pendolo piano \[
        \ddot{x}=-\sin x
    \]può essere considerato come sistema conservativo in dimensione 1. Moltiplicando da entrambe le parti per $ \dot{x} $, otteniamo \begin{align*}
        \dot{x}\,\ddot{x} &= -(\sin x)\,\dot{x}\\
        \frac{d}{dt}\left(\frac{1}{2}\,\dot{x}^{2}\right) &= \frac{d}{dt} \left(\cos x\right)\\ 
        \frac{d}{dt}\left(\frac{1}{2}\,\dot{x}^{2}-\cos x\right) &=0\\ 
        \frac{1}{2}\,\dot{x}^{2}-\cos x &= c
    \end{align*}Abbiamo trovato una costante del moto, pari alla somma tra \emph{energia cinetica} ed \emph{energia potenziale}: l'\emph{energia totale} è conservata.
}
%% END % Lezione 4
\chapter{Sistemi di E.D.O. lineari}\stepcounter{capitoloeccolo}
\paragrafo{Perché è utile studiarli}{%
    Sia data l'equazione autonoma $ \bm{x}'=\bm{f}(\bm{x}) $; sia $ \bm{x}^{\star} $ tale che $ \bm{f}(\bm{x}^{\star})=\bm{0} $ 
    
    $\implies$ $ \{\bm{x}^{\star}\} $ è orbita, e può essere stabile, instabile,\dots

    Considero ora $ \bm{x}^{\star}+ \bm{\eta}=\bm{x} $; 
    
    $\implies$ $ \bm{f}(\bm{x})=\bm{f}(\bm{x}^{\star} + \bm{\eta}) $. 

    Sviluppo ora $ \bm{f}(\bm{x})$ al primo ordine: \[
        \bm{f}(\bm{x})= \parentesi{=\bm{0}}{\bm{f}(\bm{x}^{\star})} + J_{\bm{f}}(\bm{x}^{\star})\,\bm{\eta} + o\left(\norma{\bm{\eta}}\right),\qquad \norma{\bm{\eta}}\to \bm{0}
    \]Dunque \begin{align*}
        \bm{x}(t) &= \bm{x}^{\star} + \bm{\eta}(t)\\ 
        \bm{f}(\bm{x})=\bm{x}'(t) &= \bm{\eta}'(t) 
    \end{align*}Sostituisco e ottengo \[
        \bm{\eta}'(t) = J_{\bm{f}}(\bm{x}^{\star})\,\bm{\eta}(t) + o\left(\norma{\bm{\eta}(t)}\right)
    \]dove $ J_{\bm{f}}(\bm{x}^{\star}) $ è una matrice costante.
}{}{} % Lezione 6
\days{16 marzo 2023}

\paragrafo{Ripasso}{%
    Un \emph{sistema lineare di equazioni differenziali ordinarie} è un sistema della forma \[
        \bm{x}'=A(t)\bm{x}+\bm{b}(t)
    \]dove \[
        \bm{x}=\bm{x}(t)=\begin{bmatrix}
            x_1(t)\\ 
            \vdots\\ 
            x_{n}(t) 
        \end{bmatrix}, \quad \bm{b}(t)=\begin{bmatrix}
            b_1(t)\\ 
            \vdots\\ 
            b_{n}(t) 
        \end{bmatrix},\quad A(t)=\begin{bmatrix}
            a_{11}(t) & \dots & a_{1n}(t)\\ 
            \vdots & \ddots & \vdots\\ 
            a_{n1}(t)  & \dots & a_{nn}(t) 
        \end{bmatrix}.
    \]e tutte le funzioni sono continue su un intervallo aperto $ I \subseteq \R $. 
    
    Se $ b_1,\dots,b_{n}  $ sono tutte nulle, il sistema si dice \emph{omogeneo}.
}{}{}
\paragrafo{Caso monodimensionale}{%
    Ponendo $ f(t,x)=A(t)\,x + b(t) $, $ f \in C^{1}(I\times \R) $: per ogni condizione iniziale $ (t_0,x_0) \in I\times \R $ il problema di Cauchy associato al sistema ammette un'unica soluzione locale. 

    Essendo \[
        \pd{f}{x}(t,x)=A(t)
    \]continua su $ I $, abbiamo che su ogni striscia $ [a,b]\times \R \subseteq I\times \R$ tale funzione è limitata, e deduciamo che ogni problema di Cauchy associato al sistema ammette un'unica soluzione definita sull'intero intervallo $ I $.
}{}{}
\paragrafo{Alcuni Risultati}{%
    Posto \[
        S_{\bm{b}}\coloneqq \{\text{soluzioni di } \bm{x}'=A(t)\bm{x}+\bm{b}(t)\} 
    \]valgono i seguenti risultati. \begin{itemize}
        \item \emph{Principio di Sovrapposizione}. Se $ \bm{x}_1 \in S_{\bm{b}_1}  $ e $ \bm{x}_2 \in S_{\bm{b}_2} $, allora \[
            \bm{x}_1+\bm{x}_2 \in S_{\bm{b}_1+\bm{b}_2} 
        \]
        \item \emph{Soluzioni di un sistema omogeneo}. $ S_{\bm{0}}  $ è isomorfo a $ \R^{n} $. 
        \item \emph{Soluzioni di un sistema non omogeneo}. Se $ \bm{x}_P $ risolve $ \bm{x}'=A(t)\,\bm{x}+\bm{b}(t) $, allora \[
            S_{\bm{b}}=\{\bm{x}_0+\bm{x}_P: \bm{x}_0 \in S_{\bm{0}} \} 
        \]
        \item \emph{Lemma}. Siano $ \bm{y}_1,\dots,\bm{y}_n $ delle $ n $ soluzioni del problema omogenero $ \bm{x}'=A(t)\,\bm{x} $. Allora $ \bm{y}_1,\dots,\bm{y}_n $ sono funzioni linearmente indipendente se e solo se esiste $ t_0 \in I$ tale che i vettori \[
            \bm{y}_1(t_0),\dots,\bm{y}_n(t_0)
        \]sono linearmente indipendenti in $ \R^{n} $.
    \end{itemize}
}{}{}
\paragrafo{Matrice Wronskiana}{%
    Se $ \bm{\varphi}_1,\dots,\bm{\varphi}_n $ sono $ n $ soluzioni linearmente indipendenti del sistema omogenero $ \bm{x}'=A(t) \, \bm{x}$, allora $ \{\bm{\varphi}_1,\dots,\bm{\varphi}_n\} $ si dice \emph{insieme fondamentale}. 

    La matrice \[
        W(t)=\begin{bmatrix}
            \bm{\varphi}_1 & \dots & \bm{\varphi}_n
        \end{bmatrix}
    \]si dice \emph{matrice wronskiana}.

    Si ha che $ \displaystyle S_{\bm{0}} = \{W(t)\,\bm{c}: \bm{c} \in \R^{n}\} $. Per selezionare in $ S_{\bm{0}}  $ la soluzione di \[
        \begin{cases}
            \bm{x}'=A(t)\,\bm{x}\\ 
            \bm{x}(t_0)=\bm{x}_0
        \end{cases}
    \]imponiamo $ W(t_0)\,\bm{c} = \bm{x}_0 $. Essendo le colonne di $ W(t_0) $ linearmente indipendenti, $ W(t_0) $ è invertibile e $ \bm{c}=\left[W(t_0)\right]^{-1}\,\bm{x}_0 $. La soluzione dunque è \[
        \bm{x}(t)=W(t)\,\left[W(t_0)\right]^{-1}\,\bm{x}_0
    \]
}{}{}
\paragrafo{Matrice risolvente}{%
    Se $ W $ è una matrice wronskiana e se, per qualche $ t_0 \in I$, si ha $ W(t_0)=\id_{n}  $, allora $ W $ viene detta \emph{matrice risolvente} o \emph{di transizione} o \emph{di monodromia}. 

    Se $ W(t) $ è una matrice wronskiana per \[
        \begin{cases}
            \bm{x}'=A(t)\,\bm{x}\\ 
            \bm{x}(t_0)=\bm{x}_0
        \end{cases}
    \]allora \[
        \Phi(t)=W(t)\,\left[W(t_0)\right]^{-1}
    \]è di monodromia.
}{}{}
\paragrafo{Equazioni differenziali lineari di grado $ n $}{%
    Consideriamo l'equazione differenziale lineare \[
        y^{(n)}(t)=a_{n-1}(t)\,y^{(n-1)}(t)+ a_{n-2}(t)\,y^{(n-2)}(t) + \dots+ a_0(t)\,y(t) + b(t).
    \]Supponiamo che tutte le $ a_{i}  $ e $ b $ siano continue su $ I \subseteq \R $ intervallo. 

    Definendo \[
        x_1(t)=y(t), \quad x_2(t)=y'(t), \quad\dots,\quad x_{n}(t)=y^{(n-1)}(t) 
    \]otteniamo il sistema lineare del primo ordine $ \bm{x}'=A(t)\,\bm{x}+\bm{B}(t) $ con \[
        A(t)=\begin{bmatrix}
            0 & 1 & 0 & \dots & 0\\ 
            0 & 0 & 1 & \dots & 0\\ 
            \vdots\\ 
            0 & & & & 1\\ 
            a_0(t) & a_1(t) & \dots & a_{n-1}(t) 
        \end{bmatrix},\quad \bm{B}(t)=\begin{bmatrix}
            0\\ 
            0\\ 
            \vdots\\ 
            0\\ 
            b(t)
        \end{bmatrix}
    \]
}{}{}
\paragrafo{Obiettivo}{%
    L'obiettivo è risolvere l'equazione \[
        \bm{x}'=A\,\bm{x},\qquad A \in \R^{n,n}, \bm{x} \in \R^{n}
    \]L'esistenza delle soluzioni è garantita su tutto $ \R $. 

    Per gradi:\begin{enumerate}
        \item $ A $ diagonale;
        \item $ A $ diagonalizzabile;
        \item $ A $ con autovalori in $ \C $, tutti autovalori regolari\footnote{Un autovalore è regolare se la molteplicità algebrica e geometrica coincidono};
        \item caso generale.
    \end{enumerate}
}{}{}

\osservazione{
    Per una equazione nella forma $ \bm{x}'=A\,\bm{x}$, $A \in \R^{n,n} $, $ \bm{x}=\bm{0} $ è sempre soluzione (e quindi è equilibrio).

    Ogni punto di $ \ker A $ è un punto di equilibrio.
}

\section{Matrice diagonale}

\paragrafo{Risoluzione generica}{%
    Il sistema $ \bm{x}'=A\,\bm{x} $, $ A \in \R^{n,n} $, \[
        A=\begin{pmatrix}
            \lambda_1\\ 
            & \lambda_2\\ 
            & & \lambda 3\\ 
            & & & \ddots\\ 
            & & & & \lambda_{n} 
        \end{pmatrix}
    \]diventa: \[
        \begin{cases}
            x_1'(t)=\lambda_1\,x_1(t) \\
            x_2'(t)=\lambda_2\,x_2(t) \\
            \vdots
        \end{cases}
    \] 
    
    $\implies$ $ \displaystyle x_{i}(t) = c_{i}\,e^{\lambda_{i}\,t }   $ per ogni $ i=1,\dots,m $, dove $ c_{i}  $ è una costante arbitraria. 
    
    $\implies$ $ \displaystyle \bm{x}(t) = \parentesi{W(t)}{\begin{pmatrix}
        e^{t\,\lambda_1}\\ 
            & e^{t\,\lambda_2}\\ 
            & & e^{t\,\lambda 3}\\ 
            & & & \ddots\\ 
            & & & & e^{t\,\lambda_{n} }
    \end{pmatrix}}\begin{pmatrix}
        c_1\\ \vdots\\ c_{n} 
    \end{pmatrix} $

    In questo caso $ W(t) $ è anche di monodromia.
}{}{}
\paragrafo{Ritratto di fase per $ n=2 $}{%
    Siamo nel caso \[
        A=\begin{pmatrix}
            \lambda_1 & 0\\ 
            0 & \lambda_2
        \end{pmatrix}\,\leadsto\quad \begin{cases}
            x_1(t)= c_1\,e^{\lambda_1\,t}\\ 
            x_2(t)= c_2\,e^{\lambda_2\,t}
        \end{cases}
    \]\begin{itemize}
        \item Supponiamo che $ \lambda_1 \cdot \lambda_2\neq 0 $. L'equazione delle orbite è \[
            x_2=c\,x_1^{\lambda_2/\lambda_1}
        \]\begin{itemize}
            \item Se $ \lambda=\lambda_1=\lambda_2 $, allora le orbite sono di equazione $ \displaystyle x_2=c\,x_1 $: sono tutte rette. In particolare, tutte le rette passanti per l'origine sono orbite, \emph{anche gli assi}. 
            
            Per stabilire il verso di percorrenza delle orbite, si studia nuovamente il sistema  \[
                \begin{cases}
                    x_1(t)= c_1\,e^{\lambda\,t}\\ 
                    x_2(t)= c_2\,e^{\lambda\,t}
                \end{cases}
            \]si ha che \begin{itemize}
                \item se $\lambda>0$, $ \norma{\bm{x}(t)} \displaystyle \xrightarrow[t\to \infty]{}  + \infty  $, e quindi le semirette vengono percorse verso l'esterno;
                \item se $\lambda<0$, $ \norma{\bm{x}(t)} \xrightarrow[t\to \infty]{} 0$, e quindi le semirette vengono percorse verso l'interno.
            \end{itemize} 
            \item Consideriamo $ \lambda_1\neq \lambda_2 $, ma di segno concorde $ \leadsto \lambda_1 \cdot \lambda_2 > 0$. \begin{itemize}
                \item Se $ \lambda_2/\lambda_1>1 $ e sono entrambe positive, le orbite sono \emph{uscenti} e tangenti a $ x_1 $;
                \item Se $ \lambda_2/\lambda_1>1 $ e sono entrambe negative, le orbite sono \emph{entranti} e tangenti a $ x_1 $;
                \item se $ \lambda_2/\lambda_1<1 $ e sono entrambe positive, le orbite sono \emph{uscenti} e tangenti a $ x_2 $
                \item se $ \lambda_2/\lambda_1<1 $ e sono entrambe negative, le orbite sono \emph{entranti} e tangenti a $ x_2 $
            \end{itemize}
            \item Consideriamo $\lambda_1\neq \lambda_2$, ma di segno discorde $ \leadsto $ $\lambda_1 \cdot \lambda_2<0$. 
            
            In questo caso l'origine si chiama \emph{sella}, e si ha che \[
                x_2= c\,x_1^{\lambda_2/\lambda_1}.
            \]Essendo $ \lambda_2/\lambda_1<0 $, le orbite sono quelle di equazione $ y=c\,x^{\beta} $, eventualmente simmetrizzate rispetto all'asse delle $ y $.
        \end{itemize}
        \item Se $ \lambda_1 \cdot \lambda_2 = 0 $, suppongo che uno $\lambda_1 \neq 0$ (se fossero entrambi nulli, allora tutti i punti di $ \R^{2} $ sarebbero di equilibrio) \[
            \begin{cases}
                x_1'=\lambda\,x_1\\ 
                x_2'=0
            \end{cases}\,\leadsto\quad \begin{cases}
                x_1(t)=c_1\,e^{\lambda_1\,t}\\ 
                x_2\equiv c_2
            \end{cases}
        \]Dunque, quando $ c_1=0 $ ottengo infiniti punti di equilibrio, in quanto \[
            A=\begin{pmatrix}
                \lambda_1 & 0 \\ 
                 0 & 0
            \end{pmatrix}
        \]ha come nucleo tutto l'asse $ x_2 $. 

        Inoltre, si ha che se \begin{itemize}
            \item $\lambda_1>0$: tutti gli equilibri sono instabili;
            \item $\lambda_1<0$: tutti gli equilibri sono stabili, non asintotici.
        \end{itemize}
    \end{itemize}
}{}{}

\section{Matrice diagonalizzabile}

\todo{Manca un pezzo}%TODO finire gli appunti % Lezione 7
\days{21 marzo 2023}

\section{Matrice esponenziale\footnote{Dal \cite{paganisalsa}}}

%% BEGIN Esponenziale di una matrice
\definizione{
    Data $ A \in \R^{n,n} $, definiamo \[
        e^{A} \coloneqq \displaystyle \sum_{k=0}^{\infty} \frac{A^{k}}{k!}
    \]
}
\paragrafo{Ripasso}{%
    \begin{itemize}
        \item Si definisce la norma euclidea matriciale come: \[
            \norma{A} = \left( \sum_{i,j=1}^{n}a_{ij}^{2}  \right)^{1/2}
        \]e vale la disuguaglianza: $ \norma{AB}\le \norma{A} \cdot \norma{B} $. 
        \item Criterio di Weierstrass: \[
            \sum_{k}\norma{B_{k} }< \infty \,\implies\, \sum_{k} B_{k}    
        \]
    \end{itemize}
}{}{}
\paragrafo{Buona definizione}{%
    Dunque, affinché $ e^{A} $ sia ben definita, vogliamo che \[
        \sum_{k=0}^{\infty} \norm{\frac{A^{k}}{k!}}
    \]sia convergenti, e in effetti si ha: \[
        \norm{\frac{A^{k}}{k!}} = \frac{1}{k!}\,\norm{A^{k}}\le \frac{1}{k!}\,\norm{A}^{k}
    \]e la serie \emph{di numeri reali} converge: \[
        \sum_{k=0}^{\infty}\frac{\norm{A}^{k}}{k!}
    \]dunque la matrice esponenziale è sempre ben definità.
}{}{}
\paragrafo{Proprietà}{%
    $ e^{A} $ soddisfa queste proprietà: \begin{romanen}
        \item $ e^{\bm{0}_n} = \I_{n}  $
        \item $ e^{A+B}=e^{A}\,e^{B} $;
        \item $ A\,e^{A}=e^{A}\,A $
    \end{romanen}
}{}{}
%% END
\teorema{dkdkdkkkdkkkkdkkdkkteroremamiatricedimonodromia}{
    La matrice $ e^{tA} $, $ t \in \R $, è la matrice di monodromia (risolvente) per \[
        \bm{x}'=A\,\bm{x}.
    \]In particolare, l'unica soluzione di \[
        \begin{cases}
            \bm{x}'=A\,\bm{x}\\ 
            \bm{x}(0)=\bm{x}_0  
        \end{cases}
    \]è $ \bm{x}(t)=e^{tA}\,\bm{x}_0 $.
}
\osservazione{
    Nella risoluzione dei sistemi lineari per matrici diagonali e diagonalizzabile, è stata calcolata esplicitamente la matrice risolvente. Per l'unicità della soluzione, segue che: \begin{enumerate}
        \item per $ A $ diagonale, \[
            e^{tA}= \begin{pmatrix}
                e^{\lambda_1\,t}\\ 
                & e^{\lambda_2\,t}\\ 
                & & \ddots\\ 
                & & & e^{\lambda_{n}\,t }
            \end{pmatrix},\qquad A=\begin{pmatrix}
                \lambda_1\\ 
                & \lambda_2\\ 
                & & \ddots\\ 
                & & & \lambda_{n} 
            \end{pmatrix}
        \]
        \item per $ A $ diagonalizzabile, $ A=Q\,D\,Q^{-1} $, $ Q $ matrice degli autovettori, \[
            D=\begin{pmatrix}
                \lambda_1\\ 
                & \lambda_2\\ 
                & & \ddots\\ 
                & & & \lambda_{n} 
            \end{pmatrix},\qquad e^{tA}= Q\,e^{tD}\,Q^{-1}
        \]
    \end{enumerate}
}
\dimostrazione{dkdkdkkkdkkkkdkkdkkteroremamiatricedimonodromia}{
    Dimostro che la soluzione è \[
        \bm{x}(t)=e^{t\,A}\,\bm{x}_0.
    \]La condizione iniziale è soddisfatta, devo verificare che $ \bm{x}'(t)=A\,\bm{x}(t) $. \begin{align*}
        \bm{x}'(t) &= \lim_{h\to 0} \frac{\bm{x}(t+h)-\bm{x}(t)}{h} = \lim_{h\to 0} \frac{e^{(t+h)A}\,\bm{x}_0-e^{tA}\,\bm{x}_0}{h}\\ 
        &= \lim_{h\to 0} \frac{e^{hA}\,e^{tA}\,\bm{x}_0-e^{tA}\,\bm{x}_0}{h}\\ 
        &= \lim_{h\to 0} \frac{e^{hA}-\I_{n} }{h}\,e^{tA}\,\bm{x}_0
    \end{align*}La tesi da dimostrare, ora, è che \[
        \lim_{h\to 0} \frac{e^{hA}-\I_{n} }{h} = A
    \]ovvero che \[
        \lim_{h\to 0} \left(\frac{e^{hA}-\I_{n} }{h} - A\right) = 0
    \]Svolgendo i passaggi: \begin{align*}
        \lim_{h\to 0} \left(\frac{e^{hA}-\I_{n} }{h} - A\right) &= \lim_{h\to 0} \frac{e^{hA}-\I_{n}-hA }{h}\\ 
        &= \lim_{h\to 0} \frac{\displaystyle \sum_{k=2}^{\infty} \frac{(hA)^{k}}{k!}}{h} = 0
    \end{align*}Mostro l'ultima uguaglianza mostrando che la norma di quel termine tende a $ 0 $. \begin{align*}
        \norm{\frac{1}{h}\sum_{k=2}^{\infty} \frac{(hA)^{k}}{k!}} &\le \frac{1}{|h|} \sum_{k=2}^{ \infty} \frac{|h|^{k}\,\norm{A}^{k}}{k!}\\ 
        &= \frac{1}{|h|}\left(e^{|h|\,\norm{A}}-1-|h|\,\norm{A}\right)\\ 
        &= \frac{e^{|h|\,\norm{A}}-1}{|h|} - \norm{A}\\ 
        &= \norm{A}\, \parentesi{\xrightarrow[|h|\to 0]{} 1 }{\frac{e^{|h|\,\norm{A}}-1}{|h|\,\norm{A}}}-\norm{A} \displaystyle \xrightarrow[|h|\to 0]{} 0 
        \qedd
    \end{align*}
}
\section{Matrice  con autovalori in $ \C $}
\paragrafo{Caso $2\times 2$ base}{%
    Consideriamo $ A \in \R^{2,2} $ in forma canonica, \[
        A=\begin{pmatrix}
            a & -b\\ 
            b & a
        \end{pmatrix}\qquad a,b \in \R, b \neq 0
    \]I due autovalori sono \begin{align*}
        \mu &= a+i\,b\\ 
        \overline{\mu} &= a-i\,b
    \end{align*}Il sistema $ \bm{x}'=A\,\bm{x} $ diventa: \[
        \begin{cases}
            x_1'=a\,x_1-b\,x_2\\ 
            x_2'=b\,x_1+a\,x_1
        \end{cases}
    \]Definisco la funzione complessa \[
        z(t)\coloneqq x_1(t)+i\,x_2(t)
    \]e la derivo: \begin{align*}
        z'(t)&= x_1'(t)+i\,x_2'(t)\\ 
        &= a\,x_1(t)-b\,x_2(t) + i\,b\,x_1(t)+i\,a\,x_2(t)\\ 
        &= \mu\,x_1(t)+i\,\mu\,x_2(t) = \mu\,z(t)
    \end{align*}Dunque si ha che \[
        z(t)=c\,e^{\mu\,t}, \quad c=c_1+i\,c_2 \in \C
    \]Esplicitando ora la funzione: \begin{align*}
        z(t)&= (c_1+i\,c_2) e^{(a+i\,b)t} = (c_1+i\,c_2)e^{a\,t}\,e^{i\,b\,t}\\ 
        &= (c_1+i\,c_2)e^{a\,t}\,\left(\cos(bt)+i\,\sin(bt)\right)\\ 
        &= e^{a\,t}\left[c_1\,\cos(bt)-c_2\,\sin(bt)\right]+i\,e^{a\,t}\left[c_1\,\sin(bt)+c_2\,\cos(bt)\right]
    \end{align*}Da qui posso scrivere, ricordando la definizione di $ z(t) $ \[
        \begin{cases}
            x_1 = e^{a\,t}\left[c_1\,\cos(bt)-c_2\,\sin(bt)\right]\\
            x_2=e^{a\,t}\left[c_1\,\sin(bt)+c_2\,\cos(bt)\right]
        \end{cases}
    \]Posso ora quindi scrivere la soluzione dell'equazione differenziale iniziale: \[
        \bm{x}(t) = \begin{pmatrix}
            x_1(t)\\ x_2(t)
        \end{pmatrix}= e^{at}\,\parentesi{R_{bt} }{%
            \begin{pmatrix}
                \cos(bt) & -\sin(bt)\\ 
            \sin(bt) & \cos(bt)
            \end{pmatrix}
        }\,\parentesi{\bm{x}(0)}{\begin{pmatrix}
            c_1\\c_2
        \end{pmatrix}}
    \]Dunque le orbite sono contraddistinte da: \begin{itemize}
        \item una dilatazione se $ a>0 $ (e in questo caso l'origine si chiama \emph{sorgente});
        \item una contrazione se $ a<0 $ (e in questo caso l'origine si chiama \straniero{sink}, o \emph{pozzo});
    \end{itemize}mentre per quanto riguarda la rotazione, questa sarà: \begin{itemize}
        \item antioraria se $ b>0 $;
        \item oraria se $ b<0 $.
    \end{itemize}
    \begin{figure}

        \caption{Diagramma di fase per \framref{oijdoijcidididicoijcididiciciciciciciciici}}
    \end{figure}
}{oijdoijcidididicoijcididiciciciciciciciici}{}
\teorema{dojnskjndjdjdicidjisijcoijsdlkjnckjndkjncskjndkjncjdjdjcjdj}{
    Le soluzioni di $ \bm{x}'=A\,\bm{x} $ con $ A \in \R^{2,2} $ con autovalori complessi \[
        \mu,\overline{\mu} = a\pm i\,b
    \]e autovettori $ \bm{z}_{\mu} = \bm{w}+i\,\bm{v}   $, definita la matrice $ Q=[\bm{v}, \bm{w}] $ sono \[
        \bm{x}(t)= e^{at} Q\,R_{bt}\, \bm{c}, \quad \bm{c} \in \R^{2} 
    \]In particolare, l'unica soluzione di \[
        \begin{cases}
            \bm{x}'=A\,\bm{x}\\ 
            \bm{x}(0)=\bm{x}_0
        \end{cases}
    \]è $ \displaystyle \bm{x}(t)=e^{at}\,Q\,R_{bt}\, Q^{-1}\,\bm{x}_0$
}
\osservazione{
    Per il teorema precedente, se $ A \in \R^{2,2} $ con autovalori complessi, allora \[
        e^{tA}=e^{at}\,Q\,R_{bt}\, Q^{-1}
    \]
}
\section{Matrice con autovalori regolari in $ \R $ o in $ \C $}
\paragrafo{Ipotesi}{%
    Consideriamo $ A \in \R^{n,n} $, con \begin{itemize}
        \item $ \{\lambda_1,\dots,\lambda_{h} \} $ autovalori reali con autovettori $ \{\bm{u}_1,\dots,\bm{u}_h\} $;
        \item $ \{\mu_1, \overline{\mu}_1, \dots, \mu_{k}, \overline{\mu}_k \} $ autovalori complessi con autovettori $ \{\bm{z}_1,\dots,\bm{z}_k\} $
    \end{itemize}tali per cui $ h+2k=n $. Scrivo \begin{align*}
        \mu_{j} &= a_{j} + i\,b_{j}\\   
        \bm{z}_j &= \bm{w}_j+i\,\bm{v}_j
    \end{align*}

    Costruisco la matrice \[
        Q=\begin{pmatrix}
            \bm{u}_1 & \dots & \bm{u}_h & \bm{v}_1 & \bm{w}_1 & \dots & \bm{v}_k & \bm{w}_k 
        \end{pmatrix}
    \]e la matrice pseudo diagonale: \[
        \tilde{D}=\begin{pmatrix}
            \lambda_1\\ 
            & \lambda_2\\ 
            & & \ddots\\ 
            & & & \lambda_{h}\\ 
            & & & & \begin{bmatrix}
                a_1 & -b_1\\ 
                b_1 & a_1
            \end{bmatrix}\\
            & & & & &\ddots\\ 
            & & & & & &\begin{bmatrix}
                a_k & -b_k\\ 
                b_k & a_k
            \end{bmatrix}
        \end{pmatrix}
    \]e dunque si ha che \[
        \tilde{R}=e^{t\tilde{D}}=\begin{pmatrix}
            e^{\lambda_1\,t}\\ 
            & \ddots\\ 
            & & e^{\lambda_h\,t}\\ 
            & & &\begin{bmatrix}
                e^{a_1\,t}R_{b_1\,t} 
            \end{bmatrix}\\ 
            & & & &\ddots\\ 
            & & & & &\begin{bmatrix}
                e^{a_k\,t}R_{b_k\,t} 
            \end{bmatrix}
        \end{pmatrix}
    \]dove $ R_{b_{j}\,t }$ è la matrice \[
        \begin{pmatrix}
            \cos(b_j\,t) & -\sin(b_j\,t)\\ 
            \sin(b_j\,t) & \cos(b_j\,t)
        \end{pmatrix}
    \]
}{}{}
\teorema{ddoodoodododokcpokspokdpokcspokdpokpok}{
    Nelle ipotesi precedenti, le soluzioni di $ \bm{x}'(t)=A\,\bm{x}(t) $ sono: $ \displaystyle \bm{x}(t)=Q\,e^{t\tilde{D}}\, \bm{c} $, al variare di $ \bm{c} \in \R^{n} $. In particolare, la soluzione di \[
        \begin{cases}
            \bm{x}'(t)=A\,\bm{x}(t)\\ 
            \bm{x}(0)= \bm{x}_0
        \end{cases}
    \]è $ \displaystyle \bm{x}(t)= Q\,e^{t\,\tilde{D}}\,Q^{-1}\,\bm{x}_0 $
}
\osservazione{
    Per il teorema \teoref{dkdkdkkkdkkkkdkkdkkteroremamiatricedimonodromia}, nelle ipotesi precedenti si ha che \[
        e^{tA}=Q\,e^{t\,\tilde{D}}\,Q^{-1}
    \]
}
\sesercizio{
    Trovare la soluzione di \[
        \begin{cases}
            \bm{x}'=A\,\bm{x}\\ 
            \bm{x}(0)=\bm{p}
        \end{cases}
    \]con \[
        A=\begin{pmatrix}
            1 & 0 & 0\\ 
            -2 & 3 & 1\\ 
            0 & -1 & 4
        \end{pmatrix},\qquad \bm{p}=\begin{pmatrix}
            1\\0\\ 0
        \end{pmatrix}
    \]
}
 % Lezione 8
\section{Matrice generica}
\days{23 marzo 2023}

\paragrafo{Matrice $ 2\times 2 $ in forma canonica}{%
    Consideriamo una matrice $ A \in \R^{2,2} $ con autovalori \emph{non regolari}, scritta in forma canonica: \[
        A=\begin{pmatrix}
            \lambda & 1 \\
            0 & \lambda
        \end{pmatrix}
    \]con $\lambda_1=\lambda_2$ e $ \bm{u}=\left(\begin{smallmatrix}
        1\\0
    \end{smallmatrix}\right) $. 

    Risolviamo il sistema associato: $ \bm{x}'=A\,\bm{x} $ \[
        \begin{cases}
            x'=\lambda\,x+y\\ 
            y'=\lambda\,y
        \end{cases}\quad\leadsto\quad \begin{cases}
            x'=\lambda\,x + c_2\,e^{\lambda \,t}\\ 
            y(t)= c_2\,e^{\lambda\,t}
        \end{cases}
    \]da cui otteniamo \[
        \begin{cases}
            x(t)=(c_1+c_2\,t)\,e^{\lambda\,t}\\ 
            y(t)= c_2\,e^{\lambda\,t}
        \end{cases}
    \]Lo si vuole scrivere in forma matriciale come \[
        \begin{pmatrix}
            x(t)\\ y(t)
        \end{pmatrix} = \parentesi{\Phi(t)\coloneqq}{\begin{pmatrix}
            e^{\lambda\,t} & t\,e^{\lambda\,t}\\ 
            0 & e^{\lambda\,t}
        \end{pmatrix}}\,\begin{pmatrix}
            c_1\\ c_2
        \end{pmatrix}
    \]Poiché $ \Phi(0)=\I_{2}  $, allora $ \Phi $ è la risolvente, e posso scrivere:\[
        e^{tA} = \begin{pmatrix}
            e^{\lambda\,t} & t\,e^{\lambda\,t}\\ 
            0 & e^{\lambda\,t}
        \end{pmatrix}
    \]\todo{Manca il diagramma di fase}%TODO aggiungere diagramma di fase
}{}{}
\teorema{dididififoisdisodpokspteoreminonoinoindkjnfjscdoij}{
    Sia $ A \in \R^{2,2} $ con autovalori \emph{non regolari} qualsiasi e polinomio caratteristico $ p_{A}(t)=(t-\lambda)^{2}  $. Sia $ \bm{u} \in \R^{2} $ l'unico autovettore di $ A $, e $ \bm{v} \in \R^{2} $ tale che \begin{itemize}
        \item $ \bm{v}\perp \bm{u} $
        \item $ (A-\lambda\I)\,\bm{v}=\bm{u} $.
    \end{itemize}

    Allora, le soluzioni di $ \bm{x}'=A\,\bm{x} $ sono nella forma: \[
        \bm{x}(t) = e^{\lambda\,t} (c_1+c_2\,t) \bm{u}+e^{\lambda\,t}c_2\,\bm{v},\qquad c_1,c_2 \in \R
    \]
}
\dimostrazione{dididififoisdisodpokspteoreminonoinoindkjnfjscdoij}{
    Si ha che $ \{\bm{u},\bm{v}\} $ sono una base di $ \R^{2} $, dunque necessariamente una qualsiasi funzione deve essere nella forma\[
        \bm{x}(t)=y_1(t)\,\bm{u}+y_2(t)\,\bm{v}
    \]Imponiamo che $ \bm{x}(t) $ risolva $ \bm{x}'(t)=A\,\bm{x} $, e determiniamo $ y_1 $ e $ y_2 $. \begin{align*}
        \bm{x}'(t) &= y_1'(t)\,\bm{u} + y_2'(t) \,\bm{v}\\ 
        A\,\bm{x}(t) &= y_1(t) A\,\bm{u} + y_2(t) A\,\bm{v}
    \end{align*}ma $ A\bm{u}=\lambda \bm{u} $, e $ A\bm{v}-\lambda\bm{v}= \bm{u} $, e quindi $ A\bm{v}=\lambda\bm{v}+\bm{u} $\begin{align*}
        \bm{x}'(t)=A\,\bm{x}(t) &= y_1(t) \lambda \bm{u} + y_2(t)(\lambda\bm{v}+\bm{u})\\ 
        &= \left(\lambda\,y_1(t)+y_2(t)\right) \,\bm{u} + \lambda\,y_2(t)\,\bm{v}
    \end{align*}Imponendo l'uguaglianza con $ \bm{x}'(t) $, si ottiene il sistema \[
        \begin{cases}
            y_1'=\lambda\,y_1+y_2\\ 
            y_2'= \lambda\,y_2
        \end{cases}
    \]che ha proprio come soluzione \[
        \begin{cases}
            y_1=e^{\lambda\,t} (c_1+c_2\,t)\\ 
            y_2=e^{\lambda\,t}c_2
        \end{cases}\qedd
    \]
}

\teorema{didididiocoijscjkdkjncjdjdkjncjd}{
    Considero il sistema $ \bm{x}'(t)=A\,\bm{x}(t) $, con $ A \in \R^{n,n} $. Siano \begin{itemize}
        \item $ \{\lambda_1,\dots, \lambda_{h} \} $ autovalori reali, di molteplicità algebrica, rispettivamente $ m_1,\dots,m_{k}  $; 
        \item $ \{\mu_1, \overline{\mu}_1,\dots,\mu_{k},\overline{\mu}_k \} $ autovalori complessi, $ \mu_{j}, \overline{\mu}_i=a_{j} \pm b_{j}\,i    $, di molteplicità algebrica, rispettivamente $ n_1,\dots,n_{k}  $
    \end{itemize}tali per cui \[
        \sum_{i=1}^{h} m_{i} + 2\,\sum_{i=1}^{k} n_{i} = n   
    \]

    Per ciascun autovalore, sia $ F $ l'insieme:\begin{align*}
        F_{\lambda_{i} } &= \left\{t^{j}\, e^{\lambda_{i}\,t } : j=0,\dots, m_{i}-1 \right\} \\ 
        F_{\mu_{i} } &= \{t^{j}\,e^{a_{i}\,t }\,\cos(b_{i} \,t), t^{j}\,e^{a_{i}\,t }\,\sin(b_{i}\,t ): j=0,\dots,n_{i}-1 \} 
    \end{align*}e definiamo $ \displaystyle F_{A}=\bigcup_{j} F_{\lambda_{j} }\cup\bigcup_{j} F_{\mu_{j} }       $. 

    Allora \emph{ogni} componente di $ \bm{x}(t) $ è combinazione lineare di elementi di $ F_{A}$.
}
\paragrafo{Corollario sulla stabilità dell'origine}{%
    Considero il sistema $ \bm{x}'(t)=A\,\bm{x}(t) $.\begin{itemize}
        \item Se tutti gli autovalori di $ A $ hanno parte reale $ <0 $ 
        
        $\implies$ $ \bm{0} $ è un punto di equilibrio asintoticamente stabile. 
        \item Se esiste almeno un autovalore di $A$ con parte reale $ >0 $ 
        
        $\implies$ $ \bm{0} $ è un punto di equilibrio \emph{instabile}.
        \item Se tutti gli autovalori di $ A $ hanno parte reale $ \le 0 $ 
        
        $\implies$ $ \bm{0} $ è un punto di equilibrio stabile.
    \end{itemize} 
}{}{}
\esercizio{
    Stabilire per quali $ a \in \R $ tutte le soluzioni del seguente sistema si mantengono limitate. \[
        \begin{cases}
            x_1'= a\,x_2+x_4\\
            x_2' = -x_1\\
            x_3' =x_4\\
            x_4'=-a\,x_1-x_3
        \end{cases}
    \]
}{
    Scriviamo la matrice \[
        A=\begin{pmatrix}
            0 & a & 0 & 1\\ 
            -1 & 0 & 0 & 0\\ 
            0 & 0 & 0 & 1\\ 
            -a & 0 & -1 & 0
        \end{pmatrix}
    \]Essendo le soluzioni combinazioni lineari di elementi di $ F_{A}$, le uniche funzioni ammesse per avere limitatezza sono seni e coseni. 
    
    $\implies$ imponiamo che gli autovalori di $ A $ siano tutti in $ i\R $, e che siano \emph{semplici} (ovvero con molteplicità algebrica 1). 


    Calcolo il polinomio caratteristico: \[
        p_{A}(t)= \dots = t^{4}+(2a+1)t^{2} + a 
    \]da cui ricavo che, per $ \lambda $ autovalore: \[
        \lambda_{\pm}^{2} = \frac{-(2a+1)\pm\sqrt{(2a+1)^{2}-4a}}{2}=\frac{-(2a+1)\pm \sqrt{4a^{2}+1}}{2}
    \]Per avere autovalori immaginari puri, voglio che siano entrambi \emph{strettamente} minori di $ 0 $. 

    Noto che $ \lambda_{-}^{2}< \lambda_{+}^{2}$, dunque impongo soltanto $ \lambda_{+}^{2}<0$ $ \iff $ \begin{align*}
        -(2a+1)+\sqrt{4a^{2}+1}&<0\\ 
        \sqrt{4a^{2}+1}&<0
    \end{align*}da cui ricavo il sistema: \[
        \begin{cases}
            2a+1>0\\ 
            4a^{2}+1<(2a+1)^{2}
        \end{cases}\quad \leadsto \quad \begin{cases}
            a>-1/2\\ 
            a>0
        \end{cases}
    \] 
    
    $\implies$ se $ a>0 $ ho quattro (e sono certo essere distinti, poiché la radice quadrata è iniettiva) autovalori in $ i\R $.
}

\section{Metodo di linearizzazione}

{\renewcommand{\Re}{\operatorname{Re}}
    \paragrafo{Classificazione di un punto di equilibrio per un sistema lineare}{%
    Per il sistema $ \bm{x}'=A\,\bm{x} $, l'origine è un punto di equilibrio: \begin{itemize}
        \item \emph{iperbolico}:\begin{itemize}
            \item \emph{attrattore}: se $ \Re \lambda<0 $ per ogni autovalore $ \lambda $ di $ A $;
            \item \emph{repulsore}: se $ \Re\lambda>0 $ per ogni autovalore $ \lambda $ di $ A $;
            \item \emph{sella}: se tutti gli autovalori di $ A $ hanno $ \Re\lambda\neq 0 $ e c'è almeno una coppia di autovalori di segno opposto;
        \end{itemize}
        \item \emph{centro}: se esiste almeno un autovalore $ \lambda $ di $ A $ nullo o con $ \Re \lambda = 0 $.
    \end{itemize}
}{}{}}
\teorema[Teorema di Hartman-Grobman]{duuudiuhciuhduchdiuhcuducud}{
    Se $ \bm{x}^{*} $ è un equilibrio dell'equazione autonoma $ \bm{x}'=\bm{f}(\bm{x}) $ e nel sistema linearizzato \[
        \bm{x}'=\parentesi{A}{J_{\bm{f}}(\bm{x}^{*}) }\,\bm{x}
    \]l'origine è iperbolica, allora la stabilità di $ \bm{x}^{*} $ come equilibrio di $ \bm{x}'=\bm{f}(\bm{x}) $ è la stessa di quella dell'origine per il sistema linearizzato.
}
\esempio{
    Considero il sistema: \[
        \begin{cases}
            x'=-x+x^{3}\\ 
            y'=-2y
        \end{cases}
    \]i cui equilibri sono \[
        (0,0),\,(1,0),\,(-1,0)
    \]
    
    Calcoliamo la matrice Jacobiana in un punto generico: \[
        J_{\bm{f}}(x,y)=\begin{pmatrix}
            -1+3x^{2} & 0\\ 
            0 & -2
        \end{pmatrix} 
    \]e quindi: \begin{itemize}
        \item $ J_{\bm{f}}(0,0)=\left(\begin{smallmatrix}
            -1 & 0\\ 0 & -2
        \end{smallmatrix}\right)  $ che ha due autovalori reali negativi 
        
        $\implies$ per il teorema di Hartman-Grobman $ (0,0) $ è asintoticamente stabile;
        \item $ J_{\bm{f}}(\pm 1,0)=\left(\begin{smallmatrix}
            2 & 0\\ 0 & 2
        \end{smallmatrix}\right) $ il cui punto di equilibrio è sella 
        
        $\implies$ per il teorema di Hartman-Grobman $ (\pm 1,0) $ sono instabili.
    \end{itemize}
} % Lezione 9
\days{28 marzo 2023}
\osservazione{
    Il metodo di linearizzazione: \begin{itemize}
        \item è un metodo locale e non globale;
        \item fornisce la stabilità senza conoscere un ritratto di fase;
        \item non funziona quando la parte reale degli autovalori è nulla.
    \end{itemize}
}
{\chapter{Metodo diretto di Lyapunov per lo studio della stabilità degli equilibri}\stepcounter{capitoloeccolo}

\paragrafo{Ipotesi}{%
    \begin{itemize}
        \item Sia $ \bm{x}'=\bm{f}(\bm{x}) $, $ \bm{f} \in  C^{1}(\Omega) $. $ \Omega \subseteq \R^{n} $ aperto. 
        \item $ \bm{p} \in \Omega $ tale che $ \bm{f}(\bm{p})=\bm{0} $ (equilibrio).
        \item $ V:B_{r}(\bm{p})\longrightarrow \R  $, con $ r>0 $ e $ B_{r}(\bm{p}) \subseteq \Omega  $, $ V $ di classe $ C^{1} $.
    \end{itemize}
}{dkjnckjnskjndkjnckjnskjnckjndkjnckjndkjncjndkjncjdjckjnskjncjdjskjncjdjskjn}{}
\newcommand{\ipotesi}{\framref{dkjnckjnskjndkjnckjnskjnckjndkjnckjndkjncjndkjncjdjckjnskjncjdjskjncjdjskjn}}
\paragrafo{Idea di base}{%
    Sotto le ipotesi \ipotesi, se troviamo una funzione $ V $ tale che
    \begin{itemize}
        \item $ V $ sia positiva in un intorno di $ \bm{p} $ e nulla in $ \bm{p} $;
        \item $ V $ sia decrescente lungo le traiettorie vicino a $ \bm{p} $.
    \end{itemize}Allora $ \bm{p} $ è stabile o asintoticamente stabile.
}{}{}
\paragrafo{Monotonia di $ V $}{%
    Sotto le ipotesi \ipotesi, supponiamo che $ \bm{x}=\bm{x}(t) $ è una soluzione dell'equazione differenziale. 
    \begin{itemize}
        \item Valutiamo $ V $ lungo $ \bm{x}(t) $, ovvero $ V\left(\bm{x}(t)\right) $.
        \item Ne calcoliamo la monotonia in $ t $: \[
            \od{}{t}V\left(\bm{x}(t)\right) =: \dot{V}\left(\bm{x}(t)\right)
        \]Dunque: \begin{align*}
            \dot{V}\left(\bm{x}(t)\right) &= \scalare{\nabla V\left(\bm{x}(t)\right)}{\bm{x}'(t)}\\ 
            &= \scalare{\nabla V\left(\bm{x}(t)\right)}{\bm{f}\left(\bm{x}(t)\right)}
        \end{align*}e quindi \begin{align*}
        \dot{V}:B_{r} (\bm{p}) &\longrightarrow \R \\
        \xi &\longmapsto \scalare{\nabla V (\xi)}{\bm{f}(\xi)}
        \end{align*}
    \end{itemize}
    
    Poiché sia $ V $ che $ \bm{f} $ sono di classe $ C^{1} $, allora anche $ \nabla V $ è continua, e quindi $ \dot{V} $ è continua. 

    Questa è la \emph{derivata totale di $ V $ rispetto al campo $ \bm{f} $.}
}{}{}
\teorema[Teorema di Lyapunov]{dididicoijcoijdoijcoijcoijsoij}{
    Supponiamo le ipotesi \ipotesi.

    ($ H_1 $): \parbox{16em}{\begin{itemize}
            \item $ V $ è definita positiva in $ \bm{p} $;
            \item $ \dot{V} $ è semi definita negativa in $ \bm{p} $
        \end{itemize}} \hspace{2em}$\implies$ $ \bm{p} $ è stabile.\vspace{1em} 

    ($ H_2 $): \parbox{14em}{\begin{itemize}
            \item $ V $ è definita positiva in $ \bm{p} $;
            \item $ \dot{V} $ è definita negativa in $ \bm{p} $.
        \end{itemize}}\hspace{2em}$\implies$ \parbox{7em}{$ \bm{p} $ è asintoticamente stabile. }\vspace{1em}

    ($ H_3 $) \parbox{17em}{\begin{itemize}
            \item $ V(\bm{p}) =0 $;
            \item $ \forall\, \varepsilon \in (0,r) $,\\ $ \exists\, \bm{x}_{\epsilon} \in B_{ \varepsilon}(\bm{p})  $ tale che $ V(\bm{x}_{ \varepsilon})>0 $;
            \item $ \dot{V} $ è definita positiva in $ \bm{p} $.
        \end{itemize}}\hspace{2em} $\implies$ $ \bm{p} $ è instabile.
}
\osservazione{
    Questo metodo: \begin{itemize}
        \item è locale;
        \item determina la stabilità semplice;
        \item non stabilisce come determinare la funzione $ V $; 
        \item non determina necessariamente l'asintotica stabilità di alcuni punti di equilibrio stabili.
    \end{itemize}
}
\definizione{
    Una funzione $ V $ che soddisfa ($ H_1 $) o ($ H_2 $) o ($ H_3 $) si dice \emph{funzione di Lyapunov}.
}
\dimostrazione{dididicoijcoijdoijcoijcoijsoij}{
    Dimostriamo solo la prima parte. La tesi è la stabilità di $ \bm{p} $, ovvero: $ \forall\, \varepsilon>0 $, $ \exists\, \delta = \delta( \varepsilon)>0 $ tale che, se $ \bm{q} \in B_{\delta}(\bm{p}) $, la soluzione $ \bm{\psi}_{\bm{q}}(t) $ del sistema \[
            \begin{cases*}
                \bm{x}'=\bm{f}(\bm{x})\\ 
                \bm{x}(0)=\bm{q}
            \end{cases*}
        \]soddisfi: \begin{itemize}
            \item $ T_{\max}= + \infty  $;
            \item $ \exists\, T\ge 0 $ tale che $ \forall\, t > T $, \[
                \bm{\psi}_{\bm{q}}(t) \in B_{ \varepsilon}(\bm{p}) 
            \]
        \end{itemize}
        
        Fisso $ \varepsilon< r $, in modo che $ V $ sia definita su $ B_{ \varepsilon}(\bm{p})  $, e definisco \[
            m_{ \varepsilon} = \min_{x \in \partial B_{ \varepsilon}(\bm{p}) } V(x)>0  
        \]poiché $ V $ è definita positiva. 

        $ V(\bm{p})=0 $, e inoltre $ V $ è continua 
        
        $\implies$ $ \exists\, \delta \in (0, \varepsilon) $ tale che $ V(x)< \frac{1}{2} m_{ \varepsilon}  $, $ \forall\, \bm{x} \in B_{ \delta}(\bm{p})  $. 

        Dimostro che questo $ \delta $ soddisfa la condizione di stabilità.

        Per assurdo, supponiamo che $ \exists\, \bm{q} \in B_{\delta}(\bm{p}) $ tale la soluzione $ \bm{\psi}_{\bm{q}}(t)$ di \[
            \begin{cases}
                \bm{x}'=\bm{f}(\bm{x})\\ 
                \bm{x}(0)=\bm{q}
            \end{cases}
        \]per qualche $\tau$ sia \[
            \norma{\bm{\psi}_{\bm{q}}(\tau)- \bm{p}} = \varepsilon
        \]e che $ \norma{\bm{\psi}_{\bm{q}}(\tau)- \bm{p}} < \varepsilon $, $ \forall\, t \in [0,\tau) $. 

        Per ipotesi $ \dot{V}(\bm{x}) $ è semidefinita negativa, cioè $ V $ decresce lungo le soluzioni: \[
            m_{ \varepsilon} \le V\left(\bm{\psi}_{\bm{q}}(\tau)\right) \le V(\bm{q}) = V\left(\bm{\psi}_{\bm{q}}(\tau)\right) < \frac{m_{ \varepsilon} }{2}.
        \]Assurdo.\qed 
}
\paragrafo{Corollario - Applicazione ai sistemi $ \ddot{\bm{x}}=-\nabla U (\bm{x})$}{%
    Sia\\ $U: \Omega \subseteq \R^{n}\longrightarrow \R $, $ \Omega $ aperto e $ \bm{p} \in  \Omega$- \begin{itemize}
        \item $ U \in C^{1} (\Omega)$;
        \item $ \bm{p} $ sia un minimo stretto di $ U $. 
    \end{itemize} 

    Allora $ (\bm{p},0) $ è un punto di equilibrio stabile per \[
        \begin{cases}
            \bm{x}'=\bm{y}\\ 
            \bm{y}' = -\nabla U(\bm{x})
        \end{cases}
    \]
}{dkjnckjnskjnckjndjdjjjdjjdjdjcjdjcjsoijcoij}{}
\dimframmento{dkjnckjnskjnckjndjdjjjdjjdjdjcjdjcjsoijcoij}{
    Considero $ V(\bm{x},\bm{y}) $, \[
        V(\bm{x},\bm{y}) = \frac{1}{2}\norma{\bm{y}}^{2} + U(\bm{x})-U(\bm{p})
    \]Si ha che \begin{itemize}
        \item $ V(\bm{p},0) = 0 $;
        \item $ V(\bm{x},\bm{y})>0 $ in un intorno di $ (\bm{p},0) $, poiché $ U(\bm{x})>U(\bm{p}) $ per ipotesi;
        \item $ \dot{V}(\bm{x},\bm{y}) = 0 $.
    \end{itemize} 
    
    $\implies$ $ V $ soddisfa le ipotesi ($ H_1 $) 
    
    $\implies$ $ (\bm{p},0) $ è stabile.\qed
}
} % Lezione 10
\days{30 marzo 2023}
\section{Applicazione del metodo di Lyapunov}
\paragrafo{Come trovare $ V $}{%
    Per applicare il teorema di Lyapunov è necessario trovare la funzione $ V $. Alcuni candidati possono essere:
    \begin{itemize}
        \item la funzione distanza, ad una potenza pari\[
            V(\bm{x})= \norma{\bm{x}-\bm{p}}^{2k}
        \]
        \item combinazioni lineari di potenze pari; ad esempio, in $ \R^{2} $, una espressione nella forma: \[
            V(x,y)= \alpha x^{2k}+\beta y^{2l}
        \]
    \end{itemize}
}{}{}
\esempio{%TODO sistemare gli spazi di tutto l'esempio
    Consideriamo \[
        \begin{cases}
            x'=-y-3x^{3}\\ 
            y'=x^{5}-2y^{3}
        \end{cases}
    \]Mostrare che $ (0,0) $ è l'unico equilibrio e che è asintoticamente stabile, sfruttando un'opportuna funzione di Lyapunov della forma \[
        V(x,y)= \alpha x^{2m}+\beta y^{2n}
    \]\begin{itemize}
        \item \emph{Equilibri}: \[
            \begin{cases}
                -y-3x^{3}=0\\ 
                x^{5}-2y^{3}=0
            \end{cases}\qquad \begin{cases}
                y=-3x^{3}\\ 
                x^{5}-2(-3x^{3})^{3}=0
            \end{cases} 
        \]Dalla seconda equazione ottengo $ x^{5}\left(1+54x^{4}\right)=0 $, ovvero $ x=0 $ 
        
        $\implies$ $ y=0 $

        Dunque l'unico equilibrio è $ (0,0) $. 
        \item Vogliamo che $ V $ sia definita positiva in $ (0,0) $ e che $ \dot{V} $ sia definita negativa. \begin{itemize}
            \item $ V(x,y)= \alpha x^{2m}+\beta y^{2n} $ è definita positiva  $ \iff $ $ \alpha,\beta>0 $. 
            \item $ \displaystyle \dot{V}(x,y)=\pd{V}{x}x'(t)+\pd{V}{y}y'(t) $:\begin{align*}
                \dot{V}(x,y )&=\alpha\, 2 m\, x^{2m-1}(-y-3x^{3})+\beta\, 2n\, y^{2n-1} (x^{5}-2y^{3})\\ 
                &= -2\alpha m\, x^{2m-1}y \uline{- 6\alpha m x^{2n+2}}\\ 
                &\quad +2\beta nx^{5}y^{2n-1}\uline{-4\beta xy^{2n+2}}
            \end{align*} 

            La parte sottolineata, poiché $ m,n,\alpha,\beta>0 $, è già definita negativa in $ (0,0) $. Richiediamo ora, per esempio, che l'altro pezzo di $ \dot{V} $ sia nullo. In particolare, vorremmo che fossero monomi simili: \[
            \begin{cases}
                2m-1=5\\ 
                2n-1=1
            \end{cases} \,\implies\,\begin{cases}
                m=3\\ 
                n=1
            \end{cases}
            \]e dunque \[
                -6\alpha x^{5}y+2\beta x^{5}y= (-6\alpha +2\beta)x^{5}y=0.
            \]
        \end{itemize}

        Dunque, scegliendo $ m=3 , n=1, \beta=3\,\alpha$ si ottiene una funzione che soddisfa tutte le ipotesi.
    \end{itemize}
}
\esempio{
    Consideriamo il sistema: \[
        \begin{cases}
            x'=-x+y-(x+y)(x^{2}+y^{2})\\ 
            y'=-x+y+(x+y)(x^{2}+y^{2})
        \end{cases}
    \]\begin{itemize}
        \item \emph{Equilibri}: \[
            \begin{cases}
                0=-x+y-(x+y)(x^{2}+y^{2})\\ 
                0=-x+y+(x+y)(x^{2}+y^{2})
            \end{cases} \,\implies\, \begin{cases}
                (x+y)(x^{2}+y^{2})=y-x\\ 
                (x+y)(x^{2}+y^{2})=x-y
            \end{cases}
        \] 
        
        $\implies$ $ y-x=0 $ e $ x=y $. 

        Sostituendo nella prima equazione, $ (2x)(2x^{2})=0 $ 
        
        $\implies$ $ x=0 $ 
        
        $\implies$ $ (0,0) $ è l'unico equilibrio. 
        \item \emph{Linearizzazione}: la matrice associata al sistema linearizzato è \[
            A=\begin{pmatrix}
                -1 & 1\\ 
                -1 & 1
            \end{pmatrix}
        \]che ha entrambi gli autovalori nulli: il metodo non funziona. 
        \item \emph{Lyapunov}\footnote{Suggerimento: utilizzare $ \displaystyle V(x,y)=-\frac{x^{2}}{2}+\frac{y^{2}}{2} $}: osservo che la funzione $ V(x,y) $ suggerita ha una sella in $ (0,0) $, dunque se il metodo di Lyapunov funziona dimostrerà l'instabilità. 
        
        Il segno di $ V $ su $ \R^{2} $ è mostrato in figura \ref{fig:graficodiVSegno}
        \begin{figure}
            \begin{center}
                \begin{tikzpicture}
                    \draw [black, -Latex] (-3,0) -- (3,0);
                    \draw [black, -Latex] (0,-3) -- (0,3);
                    \draw [black,, thick] (-3,-3) -- (3,3);
                    \draw [black,, thick] (3,-3) -- (-3,3);
                    \node at (3.5,3.3) {$V(x,y)=0$};
                    \node at (3.5,-3.3) {$V(x,y)=0$};
                    \fill [pattern=north east lines] (-2.6,-2.6) -- (0,0) -- (2.6,-2.6) -- cycle;
                    \fill [pattern=north east lines] (-2.6,2.6) -- (0,0) -- (2.6,2.6) -- cycle;
                    \fill [white!3!black, opacity=0.3] (-2.6,2.6) -- (0,0) -- (-2.6,-2.6) -- cycle; 
                    \fill [white!3!black, opacity=0.3] (2.6,2.6) -- (0,0) -- (2.6,-2.6) -- cycle;
                    \node [fill=white] at (1.5,0) {\Huge $-$};
                    \node [fill=white] at (0,1.5) {\Huge $+$};
                    \node [fill=white] at (-1.5,0) {\Huge $-$};
                    \node [fill=white] at (0,-1.5) {\Huge $+$};
                \end{tikzpicture}
            \end{center}
            \caption{Segno su $ \R^{2} $ di $ V(x,y)=-\frac{x^{2}}{2}+\frac{y^{2}}{2} $}\label{fig:graficodiVSegno}
        \end{figure}
       
        In ogni intorno di $ (0,0) $ la funzione $ V $ cambia di segno 
        
        $\implies$ $ \exists\, \{(x_{n},y_{n} ) \}_{n \in \N} $ tale che \begin{align*}
            (x_{n},y_{n}  ) \displaystyle &\xrightarrow{n\to + \infty} \bm{0} \\ 
            (x_{n},y_{n}  )&\ge 0, \quad\forall\, n
        \end{align*}

        Verifico ora il segno di $ \dot{V} $: \begin{align*}
            \dot{V}(x,y) &= \pd{V}{x}\,x' + \pd{V}{y}\,y'\\ 
            &=-x(-x+y) + y(-x+y)\\&\qquad +x(x+y)(x^{2}+y^{2})+y(x+y)(x^{2}+y^{2})\\ 
            &= x^{2}-xy-xy+y^{2}+ (x+y)^{2}(x^{2}+y^{2})\ge 0
        \end{align*}e si annulla \emph{solo} nell'origine. 
        
        $\implies$ sono soddisfatte le ipotesi ($ H_3 $) del teorema di Lyapunov.
    \end{itemize}
}
\esercizio{Considero il sistema \[
    \begin{cases}
        x'=y\\ 
        y' =-x^{3}-y
    \end{cases}
\]e le funzioni \begin{align*}
    V_1(x,y)&= x^{2}+y^{2}\\ 
    V_2(x,y) &= \frac{x^{4}}{4}+\frac{y^{2}}{2}\\ 
    V_3(x,y) &= \frac{x^{4}}{4}+\frac{y^{2}}{2}+x^{3}y
\end{align*}

Stabilire se $ V_1,V_2 $ e $ V_3 $ sono funzioni di Lyapunov per il sistema.}{
    \begin{itemize}
        \item $ (0,0) $ è l'unico equilibrio. 
        \item $ V_1(x,y) = x^{2}+y^{2}$ è definita positiva in $ (0,0) $. \begin{align*}
            \dot{V}_1(x,y) &= \pd{V_1}{x}\,x'+\pd{V_2}{y}\,y'\\ 
            &= 
        \end{align*}
        \todo{Manca la fine dell'esercizio}%TODO finire l'esercizio
    \end{itemize}
}
\osservazione{
    Il sistema precedente può essere visto come \[
        x''=-x^{3}-x'
    \]Se considero solo $ x''=-x^{3} $, posso vederlo nella forma \[
        x''=-\nabla U(x),\qquad U(x)=\frac{1}{4}x^{4}
    \]Quindi il termine $ -x' $ è \emph{dissipativo}
}
\teorema[Teorema di La Salle-Krasovskii]{dididicoijsidiiccdididiterorfelkddislassalledoijsdoicj}{
    Sia $ \bm{p} $ un punto di equilibrio stabile per $ \bm{x}'=\bm{f}(\bm{x}) $, e $ V $ una funzione di Lyapunov che soddisfa le ipotesi ($ H_1 $). 

    Sia \[
        \mathscr{E}=\{\bm{x} \in B_{\bm{r}}(\bm{p}): \dot{V}(\bm{x}) =0 \}.
    \]Se $ \mathscr{E} $ non contiene orbite complete distinte da $ \{\bm{p}\} $, allora $ \bm{p} $ è asintoticamente stabile.
}
\esempio{
    Se consideriamo \[
        \begin{cases}
            x'=y\\ 
            y'=-x^{3}-y
        \end{cases}\qquad \begin{aligned}
            V_2(x,y)&=\frac{1}{4}x^{4}+\frac{1}{2}y^{2}\\ 
            \dot{V}_2(x,y)&= -y^{2}
        \end{aligned}
    \]l'insieme \[
        \mathscr{E}=\left\{(x,y): -y^{2}=0\right\} =\left\{(x,0): x \in \R\right\}
    \]Ci chiediamo se questo insieme contiene orbite distinte da $ \bm{p} $. 

    Sostituisco $ y=0 $ nel sistema \[
        \begin{cases}
            x'=y\\ 
            y'=-x^{3}-y
        \end{cases}\quad \leadsto\quad \begin{cases}
            x'=0\\ 
            y'=-x^{3}
        \end{cases}
    \] 
    
    $\implies$ l'asse $ x $ non contiene orbite distinte da $ \bm{p} $, poiché in ogni punto tranne l'origine l'orbita passante per quel punto è perpendicolare all'asse $ x $.
}
\paragrafo{Modello preda-predatore Lodka-Volterra}{%
    Siano: \begin{itemize}
        \item $ C(t) $ il numero di prede al tempo $ t $ (dove $ c $ sta per coniglio);
        \item $ L(t) $ il numero di predatori al tempo $ t $ (dove $ l $ sta per lupo).
    \end{itemize}

    Supponiamo che entrambe queste popolazioni crescono proporizionalmente a sé stessi: \[
        \begin{aligned}
            C'(t)&=c\,C(t)\\ 
            L'(t) &= l\,L(t)
        \end{aligned}
    \]dove $ c,l $ sono tassi di crescita. 

    Consideriamo però dei tassi di crescita sifatti: \[
        x= \gamma- \delta\,L(t),\qquad l=-\alpha+\beta\,C(t)
    \]dove $ \alpha,\beta,\delta, \gamma>0 $. 

    Ottengo il modello \[
        \begin{cases}
            C'(t) = \left(\gamma- \delta\,L(t)\right)\,C(t)\\ 
            L'(t)=\left(-\alpha+\beta\,C(t)\right)\,L(t)
        \end{cases}\qquad \alpha,\beta,\delta, \gamma>0.
    \]Riscaliamo le variabili, o equivalentemente supponiamo $ \alpha=\beta=\delta= \gamma=1 $: \[
        \begin{cases}
            C'=(1-L)\,C\\ 
            L'=(C-1)\,L
        \end{cases}
    \]Ha senso supporre che $ C(t), L(t)>0 $, poiché sono numeri di individui. \begin{itemize}
        \item \emph{Equilibri}: facilmente si nota che i punti di equilibri sono $ (0,0),(1,1) $. 
        
        È interessante notare che due specie in competizione trovano \emph{sempre} un equilibrio, che non prevede l'annichilimento di ambo le specie.
        \item \emph{Rette invarianti}: sono presenti due rette invarianti: \begin{itemize}
            \item $ L=0 $, orbita uscente dall'origine; 
            \item $ C=0 $, orbita entrante nell'origine. 
            \item \emph{Equazioni delle orbite}: le soluzioni si trovano sugli insiemi di livello della funzione \[
                g(C,L)= (C-\log C) +(L-\log L)
            \]
        \end{itemize}
    \end{itemize}
}{}{} % Lezione 11
\days{4 aprile 2023}
\section{Alcuni risultati teorici}
\osservazione{
    Consideriamo $ \bm{x}'=\bm{f}(\bm{x}) $ e $ \bm{p} \in \Omega $ punto di equilibrio asintoticamente stabile, $ \bm{f} \in C^{1}(\Omega) $, $ \Omega \subseteq \R^{n} $ aperto. 
    
    Nei sistemi lineari, se $ \operatorname{Re}\lambda<0 $ per tutti gli autovalori $\lambda$ 
    
    $\implies$ $ \bm{p} $ attrae tutte le orbite.
}
\definizione{%TODO verificare che la notazione per \phi_t(q) sia coerente
Consideriamo $ \bm{x}'=\bm{f}(\bm{x}) $ e $ \bm{p} \in \Omega $ punto di equilibrio, $ \bm{f} \in C^{1}(\Omega) $, $ \Omega \subseteq \R^{n} $ aperto. 

$ \bm{p} $ si dice \emph{attrattore globale} se \[
        \lim_{t\to + \infty} \bm{\phi}_{t}(\bm{q})  =\bm{p},\qquad \forall\, \bm{q} \in \Omega
    \]dove $ \bm{\phi}_t(\bm{q}) $ è soluzione di \[
        \begin{cases}
            \bm{x}'=\bm{f}(\bm{x})\\ 
            \bm{x}(0)=\bm{q}. 
        \end{cases}
    \]
}
\teorema[Teorema di Barbashin-Krasovskii]{doijcoijoijoijoijoijcoijsoijoijoij}{
    Se $ V: \R^{2}\longrightarrow \R$ è una funzione di Lyapunov che soddisfa ($ H_2 $ ) e \[
        V(\bm{x}) \longrightarrow + \infty,\qquad \norma{\bm{x}}\longrightarrow + \infty
    \]allora $ \bm{p} $ è un attrattore globale.
}
\osservazione{
    Se $ \bm{x}'=\bm{f}(\bm{x}) $ ammette più di un punto di equilibrio, allora nessuno dei due può essere attrattore globale.
}
\esempio{
    \todo{Manca un esempio}%TODO manca l'esempio

    In questo sistema l'orbita periodica di raggio 1 è un ciclo limite.
}
\definizione{
    Un \emph{ciclo limite} è un'orbita chiusa che ammette un'intorno che non contiene altre orbite chiuse.
}
\definizione{
    Se $ \gamma $ è un ciclo limite che ammette un intorno $ I_{\gamma}  $ tale che \[
        \forall\,\bm{q} \in I_{\gamma},\qquad \lim_{t\to + \infty} d\left(\bm{\varphi}_t (\bm{q}), \gamma\right) = 0
    \]allora $\gamma$ si dice \emph{stabile}. Altrimenti $\gamma$ è \emph{instabile}.
}
\teorema[Teorema di Poincaré-Bendixon]{dpokscpokdpokpcokpoktppokincarre}{
    Sia $ \bm{f}:\Omega \subseteq \R^{2}\longrightarrow \R^{2} $, $ \Omega $ aperto e $ \bm{f} \in C^{1}(\Omega) $. Supponiamo che \begin{itemize}
        \item $ K \subseteq \Omega $ compatto;
        \item $ \forall\, \bm{p} \in K $, $ \bm{f}(\bm{p})\neq \bm{0} $; 
        \item $ \exists\, \bm{q} \in K $ tale che $ \bm{\phi}_{t}(\bm{q}) \in K $, $ \forall\, t \ge 0 $.
    \end{itemize} 

    Allora $ K $ contiene un ciclo limite.
}
\definizione{
    Il compatto $ K $ che soddisfa questo teorema si chiama \emph{trapping region}.
}
\esempio{
    Torniamo all'esempio di prima, scritto direttamente in forma polare: \[
        \begin{cases}
            r'=r(1-r^{2})\\ 
            \theta=1
        \end{cases}
    \]Per un compatto nella forma $ \{r \in [r_1,r_2]\} $, con $ r_1<1<r_2 $, questo teorema garantisce l'esistenza di un ciclo limite, come trovato ``a mano''.
}
\osservazione{
    Questo teorema è peculiare per la dimensione due.
} % Lezione 12
\cleardoublepage
\chapter{Campi di vettori e forme differenziali su spazi affini}
	\definizione{ Si chiama \underline{spazio affine} la terna $(A,E,\delta)$,dove:
    \begin{itemize}
        \item $A$ è un insieme di elementi che chiamiamo punti
        \item $E$ è uno spazio vettoriale
        \item $\delta \colon A \times A \to E$ è un'applicazione tale che:
        \begin{enumerate}
            \item $\forall\, (P,Q) \in A\times E, \exists! \, Q \in A \colon \delta(P,Q)=\mathbf{v}$
            \item $\forall\, P,Q,R \in A, \delta(P,Q)+\delta(Q,R)=\delta(P,R)$
        \end{enumerate}
    \end{itemize}
    La \underline{dimensione dello spazio affine} $A$ è la dimensione dello spazio vettoriale soggiaciente $E$.}
    \notazione[]{Uno spazio affine $(A,E,\delta)$ è indicato più brevemente con $A$, mentre il vettore $\delta(P,Q)$ è indicato semplicemente con $PQ$.}
    \definizione{Un vettore $\mathbf{v}\in E$ è chiamto \underline{vettore libero}, mentre con la coppia $(P,\mathbf{v})$ indicheremo il vettore $\mathbf{v}$ applicato nel punto $P$.}
    \definizione{ Sia $A$ uno spazio affine. Un \underline{riferimento cartesiano} è una coppia $(O,\mathbf{c}_\alpha)$, dove:
    \begin{itemize}
        \item $O$ è l'origine
        \item $(\mathbf{c}_\alpha)=(\mathbf{c}_1,\dots,\mathbf{c}_n)$ è una base di $E$
    \end{itemize}}

    Un riferimento cartesiano stabilisce una corrispondenza biunivoca
        \begin{align*}
            \Phi \colon A \leftrightarrow \mathbb{R}^n\\
            P\leftrightarrow (x^\alpha)
        \end{align*}
    dove ad ogni punto $P\in A$ si fa corrispondere l'ennupla reale $(x^\alpha)$ costituita dalle componenti secondo la base $(\mathbf{c}_\alpha)$ del vettore $OP$:
        \begin{align*}
            OP=\mathbf{x}=x^\alpha\mathbf{c}_\alpha
        \end{align*}
    Risultano così definite anche delle mappe $x^\alpha\colon A \to \mathbb{R}$ dette \underline{coordinate cartesiane} o \underline{affini}, tali che $x^\alpha(P)\mathbf{c}_\alpha=OP$
    \osservazione{Questa corrispondennza biunivoca istituisce anche una topologia di $A$ indotta dalla topologia di $\mathbb{R}^n$.}
    \paragrafo{Trasformazioni affini}{ Se si considerano due riferimenti affini $(O,\mathbf{c}_\alpha)$ e $(O',\mathbf{c}_{\alpha'})$, 
    allora vi è un legame tra i due sistemi di coordinate indotti $(x^\alpha)$ e $(x^{\alpha'})$, ovvero:
        \begin{align*}
            x^\alpha=a^\alpha_{\alpha'}x^{\alpha'}+b^\alpha && x^{\alpha'}=a^{\alpha'}_\alpha x^\alpha+b^{\alpha'}
        \end{align*}
    Dove $a^\alpha_{\alpha'}$ e $a^{\alpha'}_\alpha$ compongono le matrici dei cambiamenti di base:
    \begin{align*}
        \mathbf{c}_{\alpha'}=a^\alpha_{\alpha'}\mathbf{c}_\alpha && \mathbf{c}_\alpha=a^{\alpha'}_\alpha \mathbf{c}_{\alpha'}
    \end{align*}
    Mentre $(b^\alpha)$ $(b^{\alpha'})$ sono le componenti rispettivamente del vettore $OO'$ e $O'O$.}{}{}{}
    \definizione{ Sia $A$ uno spazio affine. Una funzione $f$ del tipo:
    \begin{align*}
        f:A\to \mathbb{R}\\
        P\mapsto f(P)
    \end{align*}
    è detta \underline{campo scalare}.}
    \definizione{Possiamo vedere $f=g\circ \Phi$ con:
    \begin{align*}
    A\xrightarrow[]{\Phi} \mathbb{R}^n\xrightarrow[]{g} \mathbb{R}\\P\xmapsto[]{\Phi}(x^\alpha)\xmapsto[]{g} g(x^\alpha)
    \end{align*}
    Dove $g\colon\mathbb{R}^n\to \mathbb{R}$ è detta \underline{rappresentazione del campo $f$}.}
    \proprieta{ L'insieme dei campi scalari su $A$ ha una struttura di anello commutativo ed algebra associativa e commutativa.\\
     Siano $f,g$ due campi scalari e $P\in A$, allora vale:
    \begin{enumerate}
        \item \textit{Somma di campi}: $(f+g)(P)=f(P)+g(P)$
        \item \textit{Prodotto numerico}: $(fg)(P)=f(P)g(P)$
        \item \textit{Prodotto per uno scalare}: $(af)(P)=af(P), a \in \mathbb{R}$
    \end{enumerate}
    Denoteremo con $\mathcal{F}(A)$ l'anello dei campi scalari sullo spazio affine $A$.}
    \definizione{ Un \underline{campo vettoriale} è una mappa:
    \begin{align*}
        \mathbf{X}\colon A \to A\times E\\
        P\mapsto (P,\mathbf{X}(P))
    \end{align*}
    ovvero ad un punto $P$ associa un vettore $\mathbf{X(P)}$ applicato in $P$.}
    \definizione{Fissato un riferimento cartesiano, ogni campo vettoriale $\mathbf{X}$ risulta rappresentato da un insieme di $n$ funzioni reali $X^\alpha\colon A \to \mathbb{R}$, dette \underline{componenti cartesiane}, tali che:
    \begin{align*}
        \mathbf{X}(P)=X^\alpha(P)\mathbf{c}_\alpha
    \end{align*}
    Queste componenti in quanto campi scalari hanno anche loro una funzione rappresentativa, tale per cui si possono anche denotare come $X^\alpha(x^{\beta})$.}
    \proprieta[]{Siano $\mathbf{X}$ e $\mathbf{Y}$ due campi vettoriali. Sono definite le operazioni di:
    \begin{enumerate}
        \item \textit{Somma}: $(\mathbf{X}+\mathbf{Y})(P)=\mathbf{X}(P)+\mathbf{Y}(P)$
        \item \textit{Prodotto per un numero reale}: $(a\mathbf{X})(P)=a(\mathbf{X}(P)), a \in \mathbb{R}$
    \end{enumerate}
    Con queste due operazioni l'insieme dei campi vettoriali su $A$, che denoteremo con $\mathcal{X}(A)$, ha una struttura di modulo.}
    \definizione{ Sia $f\in \mathcal{F}(A)$ e $X\in \mathcal{X}(A)$. La \underline{derivata di un campo scalare $f$} \underline{rispetto ad un campo vettoriale $\textbf{X}$} è il campo scalare:
    \begin{align*}
        \hat{\textbf{X}} (f)=X^\alpha\cfrac{\partial f}{\partial x^\alpha}
    \end{align*}
    Dove con: 
    \begin{align*}
            \frac{\partial f}{\partial x^\alpha}
    \end{align*}
    s'intende la derivata parziale rispetto alla $x^\alpha$ della funzione rappresentativa $f(x^1,\dots,x^n)$ in un qualunque sistema di coordinate cartesiane.}
   \osservazione{ Dunque è una mappa che, una volta fissato un $\mathbf{X}\in \mathcal{X}(A)$, lavora:
    \begin{align*}
        \hat{\textbf{X}} \colon \mathcal{F}(A)\to \mathcal{F}(A)\\
        f\mapsto \hat{\textbf{X}} (f)=\textbf{X}^\alpha\cfrac{\partial f}{\partial x^\alpha}
    \end{align*}}
    \proprieta{ Siano $f,g\in \mathcal{F}(A)$. Si verifica facilmente che $\hat{\textbf{X}}$ soddisfa:
    \begin{itemize}
        \item \textit{$\mathbb{R}$-lineare}: $\hat{\textbf{X}}(af+bg)=a\hat{\textbf{X}}(f)+b\hat{\textbf{X}}(g)$$\quad a,b\in \mathbb{R}$
        \item \textit{Regola di Leibnitz}: $\hat{\textbf{X}}(f\cdot g)=\hat{\textbf{X}}(f)\cdot g+f\cdot \hat{\textbf{X}}(g)$
    \end{itemize} Una funzione che soddisfa queste due proprietà è per l'appunto chiamata \underline{derivazione}.}
    \paragrafo{Invarianza per cambiamenti di coordinate affini} {Un'altro fatto interessante in merito alla derivazione di un campo scalare $f$ rispetto ad un campo vettoriale $\mathbf{X}$ è la sua indipendenza dalle coordinate affini stabilite.\\
    Sia $X\in \mathcal{X}(A)$ e siano $(\mathbf{c}_\alpha)$ e $(\mathbf{c}_{\alpha'})$ due basi. Si consideri la rappresentazione del campo secondo le due basi:
    \begin{align*}
        \mathbf{X}=X^\alpha\mathbf{c}_\alpha =X^{\alpha'}\mathbf{c}_{\alpha'}
    \end{align*}
    Tenuto conto delle relazioni tra le basi, si ha la relazione:
    \begin{align*}
        X^\alpha=a^\alpha_{\alpha'}X^{\alpha'}
    \end{align*}
    D'altra parte, interpretando la $f$ come funzione delle $(x^\alpha)$ per il tramite delle $(x^{\alpha'})$ dalle relazioni precedenti tra le basi si ha:
    \begin{align*}
        \frac{\partial f}{\partial x^\alpha}=\frac{\partial f}{\partial x^{\alpha'}}\frac{\partial x^{\alpha'} }{\partial x^\alpha}=\frac{\partial f}{\partial x^{\alpha'}}a^{\alpha'}_\alpha
    \end{align*}
    Si ha quindi:
    \begin{align*}
        X^\alpha\frac{\partial f}{\partial x^\alpha}=X^\alpha\frac{\partial f}{\partial x^{\alpha'}}a^{\alpha'}_\alpha= X^{\alpha'}\frac{\partial f}{\partial x^{\alpha'}}
    \end{align*}
    Ciò mostra l'indipendenza della definizione dalla scelta delle coordinate cartesiane.}{}{}
    %%
    %%
    %%COMMUTATORE DI DUE CAMPI VETTORIALI
    %%
    %%
    %%
    \definizione{ Siano $\mathbf{X},\mathbf{Y}\in \mathcal{X}(A)$ .Definiamo il \underline{commutatore di $\mathbf{X}$ e $\mathbf{Y}$} come il campo vettoriale:
    \begin{align*}
    [\cdot,\cdot]\colon \mathcal{X}(A)\times \mathcal{X}(A)\to \mathcal{X}(A)\\
        (\mathbf{X},\mathbf{Y})\mapsto[\mathbf{X},\mathbf{Y}]_f=\mathbf{X}(\hat{\mathbf{Y}}(f))-\mathbf{Y}(\hat{\mathbf{X}}(f)) && f\in \mathcal{F}(A)
    \end{align*}}
    \proprieta{ Siano $\mathbf{X},\mathbf{Y},\mathbf{Z}\in \mathcal{X}(A)$. Il commutatore è un'operazione binaria interna $[\cdot,\cdot]$:
    \begin{itemize}
        \item \textit{Anticommutativa}: $[\mathbf{X},\mathbf{Y}]=-[\mathbf{Y},\mathbf{X}]$
        \item \textit{Bilineare}:
    $[a\mathbf{X}+b\mathbf{Y},\mathbf{Z}]=a[\mathbf{X},\mathbf{Z}]+b[\mathbf{Y},\mathbf{Z}]$
    \item Soddisfa l'\textit{Identità di Jacobi}:
    \begin{align*}
        [\mathbf{X},[\mathbf{Y},\mathbf{Z}]]+[\mathbf{Z},[\mathbf{X},\mathbf{Y}]]+[\mathbf{Y},[\mathbf{Z},\mathbf{X}]]=0
    \end{align*}
    \end{itemize}
    Ovvero $(\mathcal{X}(A),[\cdot,\cdot])$ è un'\underline{algebra di Lie}.}
\definizione{ Sia $\mathbf{X}\in \mathcal{X}(A)$. La \underline{divergenza} di $\mathbf{X}$ è una funzione:
\begin{align*}div\colon\mathcal{X}(A)\to \mathcal{F}(A)\\
     X\mapsto div(X)
    \end{align*}
    Dove $div(X)=\cfrac{\partial X^1}{\partial x^1}+\dots +\cfrac{\partial X^n}{\partial x^n}=\cfrac{\partial X^\alpha}{\partial x^\alpha}$.}

\proprieta{Siano $\mathbf{X},\mathbf{Y}\in \mathcal{X}(A)$. La divergenza gode delle seguenti proprietà:
\begin{itemize}
    \item \textit{Somma}: $div(X+Y)=div(X)+div(Y)$
    \item \textit{Prodotto per campo scalare}: $div(fX)=fdiv(V)+ X(f),\: f\in \mathcal{F}(A)$
    \item $X=$ costante\footnote{Considerato $X=$ costante nelle coordinate cartesiane affini $(x^\alpha)$} $\Rightarrow \: div(X)=0$
\end{itemize}}

%%
%%
%%
%%CARTA E COORDINATE NON AFFINI
%%
%%
%%
\definizione{ Una \underline{carta} di dimensione $n$ su un insieme $A$ è una coppia $(U,\varphi)$, dove 
\begin{itemize}
    \item $U\subseteq A$  
    \item $\varphi$ è una mappa biettiva:
    \begin{align*}
        \varphi\colon U\to \varphi(U)\subseteq \mathbb{R}^n
    \end{align*}
    la cui immagine $\varphi(U)$ è un aperto di $\mathbb{R}^n$.
\end{itemize}}
\definizione{ Possiamo definire le \underline{coordinate associate} alla carta $(U,\varphi)$ come le $n$ funzioni:
\begin{align*}
    q^i\colon U\to \mathbb{R}&&
    q^i=pr_i\circ \varphi
\end{align*}
dove:\begin{minipage}{4cm}
\begin{align*}
pr_i\colon\mathbb{R}^n\to \mathbb{R}\\
    (r^1,...r^n)\mapsto r^i
\end{align*}
\end{minipage}
\begin{minipage}{7cm}
è la proiezione della i-esima coordinata.
\end{minipage}}
\osservazione[]{ Siano $A$ uno spazio affine, $(x^\alpha)$ delle coordinate affini su $A$ e $(U,\varphi)$ una carta di dimensione $n$.\\
Le coordinate $q^i$ si possono rappresentare come funzioni delle $n$ $(x^\alpha)$:
\begin{align*}
    q^i=q^i(x^\alpha)
\end{align*}
Essendo tutte applicazioni biettive si può invertire su $\varphi (U)$, ovvero:
\begin{align*}
    x^\alpha=x^\alpha(q^i)
\end{align*}
Quindi ricapitolando}
\definizione{ Si possono definire i \underline{cambiamenti} (o \underline{trasformazioni}) \underline{di coordinate}, come:
\begin{align*}
    q^i=q^i(x^\alpha) && x^\alpha=x^\alpha(q^i)
\end{align*}
Con le matrici Jacobiane delle trasformazioni, rispettivamente:
\begin{align*}
    E^i_\alpha= \frac{\partial q^i}{\partial x^\alpha} (x^\beta) && E^\alpha_i= \frac{\partial x^\alpha}{\partial q^i} (q^j)
\end{align*}
Queste sono regolari e una l'inversa dell'altra.}
\paragrafo{Coordinate non affini - Cerchio}{
Si considerino le coordinate del piano affine $(x,y)$ e le trasformazioni di queste in coordinate polari piane $(r,\theta)$:
\begin{align*}
x^\alpha=x^\alpha(q^i)\colon\begin{cases}
    x=r\,cos\theta\\
    y=r\,sen\theta
    \end{cases} &&r>0,\: -\pi<\theta<\pi
\end{align*}%% disegno disegno
Queste hanno come trasformazione inversa di coordinate:
\begin{align*}
q^i=q^i(x^\alpha)\colon \begin{cases}
r=\sqrt{x^2+y^2}\\
\theta=\begin{cases}
\arcsin\cfrac{y}{\sqrt{x^2+y^2}} \quad x\ge 0 \\
\pi-\arcsin\cfrac{y}{\sqrt{x^2+y^2}}\quad x<0,\, y>0\\
-\pi-\arcsin\cfrac{y}{\sqrt{x^2+y^2}}\quad x<0,\, y<0
\end{cases}
    \end{cases}
\end{align*}}{}{}
\paragrafo{ESEMPIO coordinate non affini - Sfera}{ Consideriamo le coordinate dello spazio affine $(O,(x,y,z))$ e la trasformazione in coordinate polari sferiche:
\begin{align*}
    x^\alpha=x^\alpha(q^i)\colon\begin{cases}
x=r\:sen\varphi\:cos\theta\\
y=r\:cos\varphi\:sin\theta\\
 z=r\:cos\varphi
    \end{cases}
\end{align*}
Dove $(q^1,q^2,q^3)=(r,\varphi,\theta)$ sono definite sul dominio aperto $U$ in $\mathbb{R}^3$, asportando il semiasse positivo delle $x$ e tutto l'asse $z$.\\
 La carta è a valori nell'aperto $\varphi(V)=\left\{(r,\varphi,\theta)\in \mathbb{R}^3\colon r>0, 0<\varphi<\pi, 0<\theta<2\pi\right\}$, dove:
\begin{align*}
    \begin{cases}
        r\equiv\text{raggio}\\
        \varphi\equiv\text{colatitudine}\\
        \theta\equiv \text{longitudine}
    \end{cases}
\end{align*}}{}{}
\notazione[]{In generale $x^\alpha=x^\alpha(q^i)$ si può anche scrivere come $\textbf{x}=\mathbf{x}(q^i)$ con $\mathbf{x}$ un vettore in $\mathbb{R}^3$ che rappresenta il vettore posizione $OP=x^\alpha \mathbf{e}_\alpha$.}
\definizione{ Si possono definire $n$ campi vettoriali su $\varphi(U)$, come segue:
\begin{align*}
    \mathbf{E}_i=\frac{\partial \mathbf{x}}{\partial q^i}
\end{align*}
Questi ovviamente hanno componenti rispetto alla base $(\mathbf{c}_\alpha)$:
\begin{align*}
    \mathbf{E}_i=E^\alpha_i\mathbf{c}_\alpha
\end{align*}
E vengono chiamati \underline{riferimento naturale associato alle coordinate non affini} $(q^i)$.}


Le loro caratteristiche principali sono:
\begin{enumerate}
   \item Non essere, in generale,  costanti, a causa della loro dipendenza $\mathbf{E}_i=\mathbf{E}_i(q^j)$.
   \item Essere tra loro indipendenti e costituire così una base dei campi di vettori su $U$.
\end{enumerate}
\paragrafo{Trasformazione di $\bm{X}$:coordinate affini$\leadsto$ coordinate non affini}{ Sia $\mathbf{X}\in \mathcal{X}(A)$. Sappiamo sia che:
\begin{align*}
    \mathbf{X}=X^i(q^j)\mathbf{E}_i=\boxed{X^i\mathbf{E}_i}=X^iE^\alpha_i\mathbf{c}_\alpha=\boxed{X^\alpha\mathbf{c}_\alpha}
\end{align*}
Uguagliando cosi i due termini evidenziati abbiamo le relazioni tra i coefficienti delle due rappresentazioni del campo vettoriale $\mathbf{X}$:
\begin{align*}
    X^i=E^i_\alpha X^\alpha\\
    X^\alpha=E^\alpha_iX^i
\end{align*}}{}{}
\paragrafo{Interpretazione degli $\mathbf{E}_i$ come derivazioni}{ Consideriamo i campi vettoriale $\mathbf{E}_i$. Questi possiamo interpretarli come delle derivazioni del tipo:
\begin{align*}
    \mathbf{E}_i(f)=\frac{\partial f}{\partial q^i}
\end{align*}
Ovvero come la derivata della funzione rappresentativa $f$ rispetto alle coordinate non affini $(q^i)$. Per fare ciò, quello che sta accadendo è:
\begin{align*}
    \varphi (U)\xrightarrow[]{\varphi^{-1}} U\xrightarrow[]{f} \mathbb{R}\\
    (q^i)\mapsto\varphi^{-1}(q^i)\mapsto f(\varphi^{-1}(q^i))\equiv f(q^i)
\end{align*}
Quindi procedendo con questa interpretazione:
\begin{align*}
    \mathbf{E}_i(f)=\frac{\partial}{\partial q^i}(f)=\frac{\partial x^\alpha}{\partial q^i}\frac{\partial}{\partial x^\alpha}(f)=E^\alpha_i\frac{\partial}{\partial x^\alpha}(f)
\end{align*}
Concludendo, dunque, per un generico campo vettoriale $\mathbf{X}$:
\begin{align*}
    \mathbf{X}(f)=X^i\frac{\partial}{\partial q^i}(f)=X^\alpha \frac{\partial}{\partial x^\alpha}(f)
\end{align*}}{}{}
%%
%%
%%SIMBOLI DI CHRISTOFFEL
%%
%%
\section{Simboli di Christoffel} Consideriamo i campi vettoriali $\mathbf{E_i}$. Visto che questi possono essere espressi in funzione delle coordinate $q^i$, ha senso considerarne le derivate parziali:
\begin{align*}
    \frac{\partial \mathbf{E}_i}{\partial q^i}=\frac{\partial^2\mathbf{x}}{\partial q^k\partial q^i}
\end{align*}
Queste derivate parziali sono a loro volta dei campi vettoriali e quindi possiamo considerarne la rappresentazione secondo il riferimento $(\mathbf{E}_i)$:
\begin{align*}
    \partial_j\mathbf{E}_i=\partial_j\partial_i\mathbf{x}=\Gamma^h_{ji}\mathbf{E}_h
\end{align*}
\definizione{Le componenti $\Gamma^h_{ji}$, definite come precede, sono delle funzioni sopra il dominio $U$ della carta, denominate \underline{simboli di Christoffel}.}
\proprieta{ I simboli di Christoffel hanno le seguenti proprietà:
\begin{itemize}
    \item $\boxed{\Gamma^h_{ji}=\Gamma^h_{ij}}$: Questo vale per definizione stessa dei simboli di Christoffel. Essendo definiti tramite le derivate secondo di funzioni regolari, sono simmetrici rispetto agli indici in basso.
    \item $\boxed{\Gamma^h_{ji}=0\iff \text{le coordinate sono cartesiane}}$: Dalla definizione si vede che sono identicamente nulli se e solo se i campi $\mathbf{E}_i$ sono costanti e ciò accade se e solo se le coordinate sono cartesiane.
\end{itemize}}
\osservazione[]{ Le componenti del commutatore sono le stesse in ogni sistema di coordinate:
\begin{align*}
    [\mathbf{X},\mathbf{Y}]^i=X^j\frac{\partial Y^i}{\partial q^j}-Y^i\frac{\partial X^j}{\partial q^j}
\end{align*}}
\definizione{ La \underline{divergenza in coordinate non affini} è il campo scalare:
\begin{align*}
    div\mathbf{X}=\frac{\partial}{\partial q^i}X^i+\Gamma^k_{ji}X^i
\end{align*}
Si verifica facilmente che soddisfa le condizioni precedentemente enunciate per la divergenza.}
\osservazione[]{ Questa, euristicamente, può essere vista come:
\begin{itemize}
    \item La traccia di un'opportuna matrice
    \item Un prodotto che \textit{assomiglia} al prodotto scalare
\end{itemize}}
\newpage
\section{Forme differenziali}
\definizione{ Sia $A$ uno spazio affine. Una \underline{forma lineare} o \underline{1-forma} su $A$ è un'applicazione:
\begin{align*}
    \varphi \colon \mathcal{X}(A)\to \mathcal{F}(A)
\end{align*}
tale che $\varphi$ sia \textit{$\mathcal{F}(A)$-lineare}, ovvero:
\begin{align*}
    \varphi(f\mathbf{X}+g\mathbf{Y})=f\varphi(\mathbf{X})+g\varphi(\mathbf{Y})&& \forall\, f,g\in \mathcal{F}(A)\:\:\forall \, \mathbf{X},\mathbf{Y}\in \mathcal{X}(A)
\end{align*}}
\proprieta{L'insieme delle forme lineari su $A$, $\Phi^1(A)$, è un modulo sull'anello $\mathcal{F(A)}$. Le operazioni sono così definite:
\begin{itemize}
    \item \textit{Somma di 1-forme}: $(\varphi+ \psi)(\mathbf{X})=\varphi(\mathbf{X})+\psi(\mathbf{X})$, $\forall\, \varphi, \psi\in \Phi^1(A)$
    \item \textit{Prodotto per un campo scalare}: $(f\varphi)(\mathbf{X})=f\cdot \varphi(\mathbf{X})$, $\forall\, f \in \mathcal{F}(A)$
\end{itemize}}

\notazione[]{ Denotiamo con $\langle \mathbf{X},\varphi\rangle $ il valore della forma lineare $\varphi$ sul campo vettoriale $\mathbf{X}$. In tal modo}
\definizione{ Definiamo un'applicazione lineare:
\begin{align*}
    \langle \cdot,\cdot\rangle  \colon \mathcal{X}(A)\times \Phi^1(A)\to \mathcal{F}(A)
\end{align*}
che prende il nome di \underline{valutazione} tra una forma lineare e un campo vettoriale.}
\osservazione[]{Una forma lineare può anche essere interpretata come \underline{campo di covettori}, cioè come un'applicazione:
\begin{align*}
    \varphi \colon A \to A \times E^*
\end{align*}
che associa ad ogni punto $P\in A$ un covettore applicato in $P$.\\
Il collegamento tra questa e la definizione precedente è dato dalla formula:
\begin{align*}
    \langle \mathbf{X},\varphi(P)\rangle =\langle \mathbf{X}(P),\varphi(P)\rangle 
\end{align*} 
Assume cosi senso valutare una 1-forma $\varphi$ su di un vettore applicato $(P,\mathbf{v})$. Il risultato $\langle \mathbf{v},\varphi(P)\rangle $ è un numero reale.}
\paragrafo{Il differenziale}{Un esempio fondamentale di 1-forma è il \ul{differenziale $df$ di un campo scalare $f$}.\\
Questo è definito dall'uguaglianza:
\begin{align*}
    \langle \mathbf{X},df\rangle = \hat{\mathbf{X}}(f)
\end{align*}
La linearità dell'applicazione:
\begin{align*}
    df\colon \mathcal{X}(A)\to \mathcal{F}(A)\\
    \mathbf{X}\mapsto \langle \mathbf{X},df\rangle 
\end{align*}
segue dal fatto che $\hat{\mathbf{X}}(f)$ è lineare rispetto al campo vettoriale $\mathbf{X}$, una volta fissato il campo scalare $f$. Inoltre dalla regola di Leibnitz per la derivata rispetto ad un campo vettoriale segue la regola di Leibnitz per il differenziale:
\begin{align*}
    d(fg)=gdf+fdg
\end{align*}}{}{}
\paragraph{Differenziale di $\mathbf{(q^i)}$} Il nostro obiettivo adesso è quello di studiare il differenziale delle $(q^i)$, generiche coordinate su un aperto $U$.\\
Essendo delle funzioni reali su $U$ possiamo considerarne i differenziali $(dq^i)$. Queste, come già visto, fanno corrispondere ad un campo $\mathbf{X}$ le sue componenti $X^i$:
\begin{align*}
    \langle\mathbf{X},dq^i\rangle=X^i
\end{align*}
E quindi in particolare:
\begin{align*}
    \langle \mathbf{E}_k,dq^i\rangle=\delta_k^i
\end{align*}
D'altra parte, nel dominio $U$, ogni forma lineare è combinazione lineare dei differenziali delle coordinate, ovvero ammette una rappresentazione locale:
\begin{equation}
    \label{sus1}
    \mathbf{\varphi}=\varphi_idq^i
\end{equation}
Dove le $(\varphi_i)$ sono funzioni reali su $U$ dette \underline{componenti} di $\mathbf{\varphi}$ nelle coordinate $(q^i)$, definite da:
\begin{equation}
    \label{sus2}
    \varphi_i=\langle\mathbf{E}_i,\mathbf{\varphi}\rangle
\end{equation}
Si noti come \ref{sus1}$\leadsto$\ref{sus2}, infatti:
\begin{align*}
    \langle \mathbf{E}_k,\mathbf{\varphi}\rangle = \varphi_i\langle \mathbf{E}_k,dq^i\rangle = \varphi_i \delta_k^i=\varphi_k
\end{align*}
Viceversa, \ref{sus2}$\leadsto$\ref{sus1}:
\begin{align*}
    \langle \mathbf{X},\varphi_idq^i\rangle=\varphi_i \langle \mathbf{X},dq^i\rangle= \langle \mathbf{E}_i, \mathbf{\varphi}\rangle X^i=\langle X^i\mathbf{E}_i, \mathbf{\varphi}\rangle =\langle \mathbf{X}, \mathbf{\varphi}\rangle
\end{align*}
Dalle formule precedenti segue che la valutazione di una forma lineare sopra un campo vettoriale è data, \textit{qualunque siano le coordinate scelte}, dalla somma dei prodotti delle componenti omologhe:
\begin{align*}
    \boxed{\langle \mathbf{X},\mathbf{\varphi}\rangle = X^i\varphi_i}
\end{align*}
%%
%%
%%
%%
%% FORME DIFFERENZIALI
%%
%%
%%
%%
\definizione{ Una \underline{forma differenziale} o \underline{p-forma} su uno spazio affine $A$ è un'applicazione \textit{p-lineare antisimmetrica} dello spazio $\mathcal{X}(A)^p$ nello spazio $\mathcal{F}(A)$:
\begin{align*}
    \phi \colon \underbrace{\mathcal{X}(A)\times \mathcal{X}(A)\times \dots \times \mathcal{X}(A)}_{p \text{ volte}}\to \mathcal{F}(A)
\end{align*}}
\notazione{Indicheremo con:
\begin{align*}
    \Phi^p(A)=\text{ spazio delle $p$ forme sopra } A
\end{align*}
definendo:
\begin{align*}
    \Phi^0(A)\colon =\mathcal{F}(A)
\end{align*}}
\paragrafo{Prodotto esterno}{ Sulle $p$-forme differenziali è definita l'operazione fondamentale chiamata \underline{prodotto esterno}.
Questa è una generalizzazione dell'operatore di differenziale applicabile sulle $0$-forme alle $p$-forme e per questo è indicato con $d$.
\begin{align*}
    d\colon \Phi^p(A)\to \Phi ^{p+1}(A)
\end{align*}
La sua proprietà fondamentale è $d^2\equiv 0$.}{}{}
\osservazione[]{ Per l'antisimmetria, se $p>n=dim(A)$, allora $\Phi^p(A)\equiv 0$.}
\paragrafo{Invarianza della rappresentazione differenziale 1-forma}{
Sia $\varphi$ una 1-forma su $A$ scritta in rappresentazione locale come:
\begin{align*}
    \varphi = \varphi_idq^i
\end{align*}
Il contenuto di questa paragrafo sarà quello di dimostrare la rappresentazione del suo differenziale in coordinate locali:
\begin{align*}
    d\varphi=d\varphi _i\wedge dq^i
\end{align*}
e il fatto che questa non dipenda dalle coordinate scelte. Ovvero presa:
\begin{align*}
    \varphi = \varphi_{i'}dq^{i'}\longrightarrow d\varphi = \varphi_{i'}dq^{i'}
\end{align*}
Mostriamo questo secondo fatto.\\
Siano $(q^i)$ e $(q^{i'})$ due sistemi di coordinate generiche. Ricordando che la matrice Jacobiana della trasformazione di coordinate è:
\begin{align*}
    E^{i'}_i=\frac{\partial q^{i'}}{\partial q^i}
\end{align*}
Si noti come:
\begin{align*}
    dq^{i'}=\frac{\partial q^{i'}}{\partial q^i}dq^i
\end{align*}
E quindi:
\begin{align*}
    \varphi_{i'}=E^i_{i'}\varphi_i
\end{align*}
Iniziamo allora i calcoli. Tenendo a mente che:
\begin{align*}
    d\varphi_{i'}=d(E^i_{i'}\varphi_i)=dE^i_{i'}\varphi_i+E^i_{i'}d\varphi_i
\end{align*}
Studiamo il membro destro della tesi:
\begin{align*}
    d\varphi_{i'}\wedge dq^{i'}=\varphi_i\frac{\partial E^{i}_{i'}}{\partial q^k}dq^k\wedge dq^{i'}+E^i_{i'}d\varphi_i\wedge dq^{i'}=\\
    =\underbrace{\varphi_i\underbrace{\frac{\partial q^i}{\partial q^{i'} \partial q^{k'}}}_{\text{simmetrico in $i',k'$}}\overbrace{dq^{k'}\wedge dq^{i'}}^{\text{antisimmetrico in $i',k'$}}}_{\equiv 0}+d\varphi_i\wedge E^i_{i'}dq^{i'}=\\
    =d\varphi_i\wedge dq^i
\end{align*}}{}{}
%%
%%
%%
%%CHAPTER 2 CURVE, SUPERFICI ETC
%%
%%
%%
\chapter{Curve negli spazi affini, rappresentazione in coordinate non affini, e sistemi dinamici}
%%
%%
%%
%%
%%CURVA PARAMETRIZZATA
%%
%%
\definizione{Chiamiamo \underline{curva parametrizzata} in uno spazio affine $A$ un'applicazione $\gamma \colon I \to A$ da un intervallo aperto $I\subseteq \mathbb{R}$ nello spazio affine.}
\definizione{Considerata un'origine $O\in A$, per la curva $\gamma$ vi è una \underline{rappresentazione vettoriale}:
\begin{align*}
    \mathbf{x}=\gamma(t)&& \text{con }\mathbf{x}=OP
\end{align*}
che dunque identifica i punti $P\in A$ con il loro vettore posizione rispetto al punto $O$.}


Siano $(x^\alpha)$ delle coordinate cartesiane aventi origine in $O$. Si possono allora considerare le equazioni parametriche:
\begin{align*}
    x^\alpha= \gamma^\alpha(t) && \alpha=1,\dots, n
\end{align*}
\paragrafo{Interpretazione cinematica}{Una curva può essere interpretata come moto di un punto $P$ nello spazio affine, se il parametro $t$ viene inteso come tempo.\\
Nel caso in cui la curva rapresenti il moto di un punto nello spazio affine tridimensionale euclideo, il generico vettore $OP=\mathbf{x}$ è chiamato \underline{vettore posizione}.}{}{}
\definizione{L'immagine della curva, cioè l'insieme 
\begin{align*}
    \gamma(I)=\{P\in A | \exists\,t \in I: \gamma(t)=P\}
\end{align*}
 è detta \underline{traiettoria} o \underline{orbita}\footnote{In geometria è questa in realtà la vera e propria curva, ovvero il luogo dei punti definito da $n-1$ equazioni.}.}
\definizione{Il \underline{vettore tangente} alla curva $\gamma$ nel punto $\gamma(t)$ è il vettore denotato con $\dot{\gamma}(t)$ definito dal limite:
\begin{align*}
    \dot{\gamma}(t)=\lim_{h\to 0}\frac{\gamma(t+h)-\gamma(t)}{h}
\end{align*}
Questo nel contesto cinematico prende il nome di \underline{velocità istantanea} e lo si denota con $\mathbf{v}(t)$.}
\paragrafo{Campo tangente come curva}{ Conviene interpretare il vettore tangente $\dot{\gamma}(t)=\mathbf{v}(t)$ come vettore applicato nel punto $\gamma(t)$. Ovvero come un'applicazione:
\begin{align*}
    \hat{\gamma}(t)\colon I \to A\times E\\
    t\mapsto (\gamma(t),\dot{\gamma}(t))
\end{align*}
che viene detta \underline{curva tangente} della curva $\gamma\colon I \to A$.}{}{}
\definizione{ Siano $\gamma \colon I \to \mathbb{R}$ una curva e $F\colon A \to \mathbb{R}$ un campo scalare, entrambi almeno di classe $C^1$.
Definiamo la \ul{derivata del campo scalare $F$ lungo la curva $\gamma$} come:
\begin{align*}
    \frac{d}{dt}(F\circ \gamma)(t)=\langle \mathbf{v}(t),dF\rangle \quad \forall t \in I
\end{align*}
Questa è la definizione naturale, infatti:
\begin{align*}
    \frac{d}{dt}(F\circ \gamma)(t)=\frac{\partial F}{\partial x^\alpha}\frac{d\gamma^\alpha}{dt}(t)=\frac{\partial F}{\partial x^\alpha}v^\alpha(t)=\langle\mathbf{v}(t),dF\rangle
\end{align*}}
%%
%%
%%CURVA INTEGRALE
%%
%%
\definizione{La \underline{curva integrale di un campo vettoriale $\mathbf{X}$} è la curva:
\begin{align*}
    \gamma \colon I \to A
\end{align*}
tale che:
\begin{itemize}
    \item $0\in I\subseteq \mathbb{R}$;
    \item $\forall \,\gamma(t)\in A$, il vettore tangente $\dot{\gamma}(t)$ coincide con il valore del campo $\mathbf{X}$ in quel punto ovvero:
    \begin{align*}
        \dot{\gamma}=\mathbf{X}\circ \gamma=\mathbf{X}(\gamma(t))\\
        I\xlongrightarrow{\gamma}A\xlongrightarrow[]{\mathbf{X}}A\times E\\
        \dot{\gamma}(t)\colon I \to A\times E
    \end{align*}
\end{itemize}}
\definizione{Diciamo inoltre che la curva integrale è \underline{basata nel punto} $P_0$ se $\gamma(0)=P_0$.}
\notazione{In rappresentazione vettoriale sarebbe:
\begin{align*}
    \mathbf{x}=\gamma(t) \text{ curva integrale} \iff \frac{d\mathbf{x}}{dt}=\mathbf{X}(\mathbf{x})
\end{align*}}


Le curve integrali rappresentano i moti delle particelle del fluido secondo l'interpretazione del campo $\mathbf{X}$ come campo di velocità.\\
\definizione{ Un campo vettoriale interpretato come campo di velocità viene detto \underline{sistema dinamico}.}
\notazione[]{ Un sistema dinamico si può rappresentare dunque come un'equazione differenziale:
\begin{itemize}
        \item vettoriale:
        \begin{align*}
            \frac{d\mathbf{x}}{dt}=\mathbf{X}(\mathbf{x})
        \end{align*}
        \item in coordinate affini:
        \begin{align*}
            \frac{dx^\alpha}{dt}=X^\alpha (x^\beta)
        \end{align*}
        \item in coordinate generiche:
        \begin{align*}
            \frac{dq^i}{dt}=X^i (x^i)
        \end{align*}
\end{itemize}
In particolare sono $n$ equazioni differenziali ordinarie in forma normale autonome.}
\section{Risoluzione sistemi dinamici} Integrare significa trovare tutte le soluzioni del sistema dinamico e queste costituiscono lo spazio delle soluzioni/spazio dei moti.
\definizione{Dato un sistema dinamico e un punto $P_0\in A$, parliamo di \ul{curva integrale massimale} $\gamma_{P_0}\colon I_{P_0}\to A$, quando presa un'altra curva integrale $\gamma\colon I\to A$, allora:
\begin{align*}
    I \subseteq I_{P_0}&& \gamma_{P_0|_I}=\gamma
\end{align*}}
Una volta fissate le condizioni iniziali/dati iniziali riusciamo ad individuare un'unica soluzione del sistema dinamico grazie al
\teorema[Teorema di Cauchy]{fgdslkjlsdaljkfsd}{ Sia $\mathbf{X}$ un campo vettoriale di classe $C^k$($k\ge1$) su un dominio $M$.
Fissato un punto $P_0\in M$ esiste una e una sola curva integrale massimale basata in $P_0$
\begin{align*}
    \gamma_{P_0}\colon I_{P_0}\to M
\end{align*}
Se $I_{P_0}=\mathbb{R}$, $\mathbf{X}$ si dice \ul{completo}.}
\definizione{Il \underline{flusso del campo di vettori $\mathbf{X}$} descrive lo spazio delle soluzioni (o insieme di tutte le curve integrali del campo $\mathbf{X}$).
Si definisce come la funzione:
\begin{align*}
    \varphi\colon D\subseteq \mathbb{R}\times M \to M\\
    (t,P_0)\mapsto \gamma_{P_0}(t)
\end{align*}}
\teorema[]{sdljfkdslkgjldgdd}{Sia $\mathbf{X}$ campo vettoriale di classe $C^k$($k\ge1$) su un dominio $M$, allora:
\begin{enumerate}
    \item Il dominio $D$ del flusso $\varphi$ è un aperto di $\mathbb{R}\times M$ e $\varphi\in C^k(D)$
    \item Sia $V\subseteq M$ aperto, $\delta >0$ e si consideri $(-\delta,\delta)\times V\subseteq D$. Allora $\forall \, t \in (-\delta,\delta)$:
    \begin{align*}
       \varphi_t\colon V\to V_t\\
       P\mapsto \varphi(t,P) 
    \end{align*}
    è un omemomorfismo $C^k$ di $V$ su $V_t\subseteq M$ con $\varphi_t\colon P\mapsto\varphi(-t,P)$ omeomorfismo inverso
    \item Vale:
    \begin{align*}
      \varphi(t,\varphi(s,P))=\varphi(t+s,P)  
    \end{align*}
    $\forall\, t,s,P$ per cui i due membri hanno significato.
\end{enumerate}}
\osservazione[]{Se $\mathbf{X}$ è completo, allora:
\begin{itemize}
    \item $D=\mathbb{R}\times M$
    \item $\varphi_t\colon M\to M,P\mapsto\varphi_t(P)=\gamma_P(t)$ è una trasformazione $C^k$ di M.
\end{itemize}}
\definizione{ Al variare del parametro $t\in \mathbb{R}$ i diversi flussi $\varphi_t$ costiutiscono un \underline{gruppo ad un parametro}, ovvero un'insieme $\{\varphi_t|t\in \mathbb{R}\}$ tale che valgono:
\begin{itemize}
    \item $\varphi_t\circ \varphi_s=\varphi_{t+s}$
    \item $\varphi_t\circ\varphi_s=\varphi_s\circ \varphi_t$
    \item $\varphi_0=id_M$
    \item $(\varphi_t)^{-1}=\varphi_{-t}$
\end{itemize}}
\osservazione[]{ Ad ogni gruppo ad un parametro si può associare il campo vettoriale $\mathbf{X}$ corrispondente e viceversa. Ovvero questo sono condizioni necessarie e sufficienti affinché un insieme di curve $\varphi(t,P)$ sia lo spazio delle soluzioni di un certo campo vettoriale.}
\definizione{ Sia $\mathbf{X}\in \mathcal{X}(A)$. Definiamo l'\underline{integrale primo di $\mathbf{X}$} come il campo scalare $F\colon A\to \mathbb{R}$ tale che $\forall\, \gamma \colon I \to A$ curva integrale, vale:


\parbox{10em}{\centering
   $F\circ \gamma (t_1)=F\circ \gamma (t_2)$
   $\forall\, t_1,t_2\in I$
    }$\iff$
\parbox{10em}{
    $\cfrac{d}{dt}(F\circ\gamma)(t)=0, \forall \, t\in I$
  }
    $\iff$
\parbox{10em}{
         $\langle \mathbf{X},dF\rangle =0$}
}
%%
%%
%%
%%
%%
%%%VELOCITA' E ACCELERAZIONE
%%
%
%
%%
\section{Velocità e accelerazione in coordinate non affini}
\paragrafo{Velocità in coordinate non affini}{ Consideriamo una curva $\gamma$ e delle coordinate non affini $(q^i)$. La curva è rappresentata da:
\begin{align*}
    q^i(t)=\gamma^i(t)
\end{align*}
Ricordando che $\mathbf{x}=x^i\mathbf{E}_i$, abbiamo che:
\begin{align*}
    \mathbf{v}=\frac{d\mathbf{x}(t)}{dt}=\underbrace{\frac{\partial \mathbf{x}}{\partial q^i}}_{\mathbf{E}_i}\cdot \frac{dq^i}{dt}=v^i\mathbf{E}_i
\end{align*}
Quindi la velocità in componenti rispetto alle cordinate $(q^i)$ è data da 
\begin{align*}
    v^i(t)=\frac{dq^i(t)}{dt}
\end{align*}}{}{}
\osservazione[]{ Come già visto l'espressione del campo di vettori $\mathbf{v}$ non cambia nei due sistemi $(\mathbf{c}_\alpha)$ e $(\mathbf{E}_i)$, ovvero:
\begin{align*}
    \mathbf{v}=v^\alpha \mathbf{c}_\alpha = v^i\mathbf{E}_i
\end{align*}}
\paragrafo{Accelerazione in coordinate non affini}{ Nel riferimento affine $(\mathbf{c}_\alpha)$, l'accelerazione è definita come:
\begin{align*}
    \mathbf{a}=\frac{d\mathbf{v}}{dt}=\frac{dv^\alpha}{dt}(t)\mathbf{c}_\alpha=a^\alpha(t)\mathbf{c}_\alpha
\end{align*}
Ora allora analizziamo $\cfrac{d\mathbf{v}}{dt}$ passando per le coordinate non affini:
\begin{align*}
    \frac{d\mathbf{v}}{dt}=\frac{dv^i}{dt}\mathbf{E}_i+v^i\frac{d}{dt}\mathbf{E}_i=\frac{dv^i}{dt}\mathbf{E}_i+v^i\frac{\partial}{\partial q^j}(\mathbf{E}_i)\cdot\overbrace{\frac{dq^j}{dt}}^{v^j}=\frac{dv^i}{dt}\mathbf{E}_i+\boxed{v^iv^j\Gamma^k_{ji}\mathbf{E}_k}
\end{align*}
Ora, scambiando $i\leftrightarrow k$ in $\square$ e raccogliendo, otteniamo:
\begin{align*}
    \left(\frac{dv^i}{dt}+v^kv^j\Gamma_{jk}^i\right)\mathbf{E}_i=a^i\mathbf{E}_i
\end{align*}
Quindi si noti che \underline{a meno che} $\Gamma_{jk}^i=0$, abbiamo:
\begin{align*}
    a^i\ne \frac{dv^i}{dt}
\end{align*}}{}{}
\paragrafo{Coordinate polari}{ Consideriamo le coordinate polari. Per quanto riguarda la velocità abbiamo:
\begin{align*}
    \mathbf{v}=\dot{r}\mathbf{E}_r+\dot{\theta}\mathbf{E}_\theta=v^r\mathbf{E}_r+v^\theta\mathbf{E}_\theta
\end{align*}
Ora analizziamo l'accelerazione sfruttando l'espressione ottenuta precedentemente:
\begin{align*}
    \mathbf{a}=(\frac{dv^r}{dt}+v^rv^\theta\Gamma^r_{r\theta}+v^\theta v^\theta\Gamma^r_{\theta \theta}+v^\theta v^r\Gamma^r_{\theta r}+v^rv^r\Gamma^r_{rr})\mathbf{E}_r+(\frac{dv^\theta}{dt}+v^rv^\theta\Gamma^\theta_{r\theta}+v^r v^r\Gamma^\theta_{rr}+v^\theta v^r\Gamma^\theta_{\theta r}+v^\theta v^\theta \Gamma^\theta_{\theta \theta})\mathbf{E}_\theta
\end{align*}
Ricordando inoltre che $\Gamma^r_{\theta \theta}=-r$ e $\Gamma^\theta_{r \theta}=\Gamma^\theta_{\theta r}=\cfrac{1}{2}$:
\begin{align*}
    \left(\frac{dv^r}{dt}-rv^\theta v^\theta\right)\mathbf{E}_r+\left(\frac{dv\theta}{dt}+\frac{2}{r}v^\theta v^r\right)\mathbf{E}_\theta=\\
    =(\ddot{r}-r\ddot{\theta}^2)\mathbf{E}_r+(\ddot{\theta}+\frac{2}{r}\dot{r}\dot{\theta})\mathbf{E}_\theta=\\
    =a^r\mathbf{E}_r+a^\theta\mathbf{E}_\theta
\end{align*}
Si osservi che ponendo $\mathbf{E}_r=\mathbf{u}$ e $\mathbf{E}_\theta=r\mathbf{v}$ con $\mathbf{u}$ e $\mathbf{v}$ versori nella rappresentazione radiale del moto si ottiene la classica scomposizione:
\begin{align*}
    \mathbf{a}=\mathbf{a}_{\text{radiale}}+\mathbf{a}_{\text{trasversale}}
\end{align*}}{}{}
\paragrafo{Dalla traiettoria agli enti fondamentali della cinematica}{ Sia $\mathbf{r}=r(\theta)$ . Da esso si possono scrivere tutti gli enti fonamentali della cinematica.\\
Sia $\mathbf{r}=\mathbf{r}(\theta(t))\mathbf{u}$ la nostra traiettoria del moto. Studiamo la velocità:
\begin{align*}
    \mathbf{v}=\frac{d\mathbf{r}}{dt}=\frac{dr}{d\theta}\cdot \dot{\theta}\mathbf{u}+r\frac{d\mathbf{u}}{d\theta}\cdot\dot{\theta}=\dot{\theta}\left(\frac{dr}{d\theta}\mathbf{u}+r\frac{d\mathbf{u}}{d\theta}\right)
\end{align*}
Ora utilizzando la costante delle aree $c=r^2\dot{\theta}$ otteniamo:
\begin{align*}
    \frac{c}{r^2}\left(\frac{dr}{d\theta}\mathbf{u}+r\frac{d\mathbf{u}}{d\theta}\right)=c\left(\frac{1}{r^2}\frac{dr}{d\theta}\mathbf{u}+\frac{1}{r}\frac{d\mathbf{u}}{d\theta}\right)
\end{align*}
e riconoscendo in $\cfrac{d\mathbf{u}}{d\theta}=\theta$, abbiamo che:
\begin{align*}
    \mathbf{v}=c\left(-\frac{d}{d\theta}\left(\frac{1}{r}\right)\mathbf{u}+\frac{1}{r}\tau\right)
\end{align*}
Prendiamo adesso in esame l'accelerazione, sfruttando l'espressione appena ottenuta per la velocità:
\begin{align*}
    \frac{d\mathbf{v}}{dt}=c\left(\frac{d}{d\theta}\left(\frac{d}{d\theta}\frac{1}{r}\right)\right)\mathbf{u}
\end{align*}}{}{}
%%SPAZI AFFINI EUCLIDEI
%%
%
%%
%
%
\definizione{ Uno spazio affine euclideo è una quaterna $(A,E,\delta,\mathbf{g})$, dove $(A,E,\delta)$ è uno spazio affine e $\mathbf{g}$ è un tensore metrico su $E$ (considerato come spazio vettoriale euclideo). Ovvero $\mathbf{g}$ è una forma bilineare simmetrica:
\begin{align*}
    \mathbf{g}\colon \mathcal{X}(A)\times \mathcal{X}(A)\to \mathcal{F}(A)\\
    (\mathbf{X},\mathbf{Y})\mapsto \mathbf{X}\cdot\mathbf{Y}=\mathbf{g}(\mathbf{X},\mathbf{Y})
\end{align*}
dove, considerata $g$ come il prodotto scalare su $E$, $\forall\, P \in A$ l'immagine è definita come:
\begin{align*}
    (\mathbf{X}\cdot\mathbf{Y})(P)=g(\mathbf{X}(P),\mathbf{Y}(P))
\end{align*}}
\definizione{ Definiamo l'\underline{ascissa euclidea} (o \underline{ascissa curvilinea}) una funzione monotona crescente $g$ tale che presa:
\begin{align*}
    g\colon I \to \mathbb{R}\\
    t\mapsto s(t)
\end{align*}
allora 
\begin{align*}
    \frac{ds}{dt}=|\mathbf{v}|=\sqrt{gij\frac{dq^i}{dt}\cdot \frac{dq^j}{dt}}
\end{align*}}
%%
%%
%% da fare meglio, lei spiega di merda, se si fa geo3 si capisce davvero la teoria semplice delle curve differenziabili(triedro etc...)
%%
%%
%%
\section{Moti geodetici su una superficie}
Sia $Q$ una superficie regolare nello spazio tridimensionale con rappresentazione $OP=\mathbf{r}(q^1,q^2)$. Sia assegnata sulla superficie una curva $\gamma$ di equazioni parametriche $q^i(t)=\gamma^i(t)$.\\
Il vettore $\mathbf{v}(t)$ tangente alla curva è banalmente tangente anche alla superficie. Se consideriamo invece il suo vettore derivata:
\begin{align*}
    \frac{d\mathbf{v}}{dt}
\end{align*}
questo in generale non è tangente alla superficie. Infatti:
\begin{align*}
   \frac{d\mathbf{v}}{dt} =\frac{dv^i}{dt}\mathbf{E}_i+v^i\frac{d\mathbf{E}_i}{dt}=\frac{dv^i}{dt}\mathbf{E}_i+v^i\frac{\partial \mathbf{E}_i}{\partial q^j}\frac{dq^j}{dt}=\frac{dv^i}{dt}\mathbf{E}_i+v^i\frac{dq^j}{dt}(\Gamma_{ji}^k\mathbf{E}_k+B_{ji}\mathbf{N})
\end{align*}
con $\mathbf{N}$ il vettore normale alla superficie. Quindi:
\begin{align*}
    \frac{d\mathbf{v}}{dt}=\underbrace{\left(\frac{dv^i}{dt}+v^k\frac{dq^i}{dt}\Gamma^i_{jk}\right)\mathbf{E}_i}_{\text{tangente a }Q}+v^i\frac{dq^j}{dt}B_{ji}\mathbf{N}
\end{align*}
La componente tangente della derivata della velocità è chiamata \ul{derivata intrinseca della velocità/accelerazione intrinseca}. Se volessimo scriverla però rispetto alle coordinate non affini $(q^i)$, essendo $v^i=\cfrac{dq^i}{dt}$:
\begin{align*}
    \mathbf{a}^k_{\text{intrinseca}}=\frac{d^2q^k}{dt^2}+\Gamma^k_{ij}\frac{dq^i}{dt}\cdot \frac{dq^j}{dt}
\end{align*}
Abbiamo cosi scomposto l'accelerazione come:
\begin{align*}
    \mathbf{a}=\mathbf{a}^k_{\text{intrinseca}}+\mathbf{a}_{\mathbf{N}}
\end{align*}
Una volta scissa l'accelerazione possiamo finalmente definire cos'è 
\definizione{ Il \ul{moto geodetico} o \ul{moto inerziale} su una superficie è il moto in cui:
\begin{align*}
    \mathbf{a}_{\text{intrinseca}}=0, \quad\forall \, t && [\mathbf{a} = \mathbf{a}_{\mathbf{N}}]
\end{align*}}
Questa condizione può essere anche rappresentata sotto forma di sistema di equazioni differenziali infatti:\\
\begin{minipage}{4cm}
    \begin{align*}
        \frac{d^2q^k}{dt^2}+\Gamma^k_{ij}\frac{dq^i}{dt}\cdot \frac{dq^j}{dt}=0
    \end{align*}
\end{minipage}$\longleftrightarrow$
\begin{minipage}{4cm}
    \begin{align*}
        \begin{cases}
            \cfrac{dq^i}{dt}=v^i\\
            \cfrac{dv^k}{dt}=-\Gamma^k_{ij}v^iv^j
        \end{cases}
    \end{align*}
\end{minipage}
%da aggiustare freccia ahha
\\
Cosi facendo i moti geodetici diventano le curve integrali di questo sistema dinamico associato al campo vettoriale $\mathbf{X}$ definito sui vettori dello spazio tangente alla superficie $Q$. Questo scritto come derivazione sarebbe:
\begin{align*}
    \mathbf{X}=v^i\frac{\partial}{\partial q^i}-\Gamma^k_{ij}v^iv^j\frac{\partial}{\partial v^k}
\end{align*}
$\mathbf{X}$ con un abuso di notazione prende il nome di \underline{flusso geodetico}.
Cosi facendo assegnato un $P_0\in Q$ e $\mathbf{v}_0$ tangente a $Q$ in $P_0$, $\exists!$ una curva geodetica massimale basata in $P_0$ e avente come vettore tangente in $t=0$ il vettore $\mathbf{v}_0$.
\paragraph*{Energia cinetica come integrale primo delle geodetiche}
\begin{align*}
    \mathbf{F}=\mathbf{F}(\mathbf{r},\mathbf{v}) && \mathbf{r}=\mathbf{r}(t)=\gamma(t)
\end{align*}
\begin{align*}
    \frac{d}{dt}\mathbf{F}(\mathbf{r}(t),\mathbf{v}(t))=0
\end{align*}
\begin{align*}
    \frac{d}{dv}(\mathbf{v}\cdot \mathbf{v})=2\mathbf{v}\cdot\mathbf{a}=0
\end{align*}
Poichè $\mathbf{a}=\mathbf{a}_{\mathbf{N}}\perp \mathbf{v}$\\
Quindi l'energia cinetica è un integrale primo delle geodetiche. In particolare il moto è uniforme, ovvero il modulo della velocità è costante lungo le geodetiche.
\chapter{Il modello della visione}
\section{Introduzione} Il modello della visione di Jean Petitot è un tentativo di rappresentare tramite la modellizzazione matematica come gli oggetti ed enti del mondo esterno vengano recepiti, codificati  e rappresentati dalla corteccia visuale del nostro cervello, in particolare si sofferma su una modellizzazione del primo stadio di rappresentazione degli oggetti esterni, il cosiddetto $V1$.
Ci si chiede appunto come enti geometrici esterni semplici come punti o anche più complessi come linee e forme possano essere interpretate e codificate dal nostro apparato neuro-visivo.
\section{Il modello}
\paragraph*{Def}La struttura geometrica più importante definita sulla mappa delle fibre che modella il funzionamento ottico di $V1$ è chiamata \underline{struttura di contatto}, denotata con $\mathcal{C}$.\\
Il modello geometrico della visione di Petito rappresenta le connessioni neuronali retina-corteccia visiva nel seguente modo:
\begin{align*}
    I\xlongrightarrow[]{\gamma}A \xlongrightarrow[]{\hat{\mathbf{X}}}A\times E\xlongrightarrow[]{\dot{\gamma}}A'\xlongrightarrow[]{\mathbf{X}}A'\times E'
\end{align*}
Dove:
\begin{align*}
    \gamma \colon I \to A\\
    t\mapsto (x,y)
\end{align*}
che poi viene inviato tramite $\hat{\mathbf{X}}$:
\begin{align*}
    \hat{\mathbf{X}}\colon A \to A\times E\\
    (x,y)\mapsto (x,y,\dot{x},\dot{y})
\end{align*}
Successivamente:
\begin{align*}
    \dot{\gamma}\colon A\times E\to A'\\
    (x,y,\dot{x},\dot{y})\mapsto (x,y,p=\dot{y})
\end{align*}
E infine tramite $\mathbf{X}\in ker\omega$, con $\omega=dy-pdx$:
\begin{align*}
    \mathbf{X}\colon A'\to A'\times E'\\
    (x,y,p=\dot{y})\mapsto (x,y,p=\dot{y},\dot{x}=1,\dot{y}=p,\dot{p}=\ddot{y})
\end{align*}
Dove $\Gamma$ è la curva geodetica per $g_{\mathcal{C}}$ ed è definita come:
\begin{align*}
    \Gamma= \dot{\gamma}_{|(\dot{x}=1,\dot{y}=p)} && \textit{lift di Legendre}
\end{align*}
e invece:
\begin{align*}
    g_{\mathcal{C}}(\mathbf{t}_i,\mathbf{t}_j)=\delta_{ij}&& i=1,2
\end{align*}
e $\{\mathbf{t}_1,\mathbf{t}_2\}$ che generano il $\ker\omega$.
%% non si capisce nulla, lo so


% \part{Introduzione alle tecniche geometriche per lo studio dei modelli differenziali in fisica matematica}

% \chapter{Spazi Affini}\days{13 aprile 2023}
\definizione{
    Uno \emph{spazio affine} $ \mathcal{A} $ è una terna $ (\mathcal{A}, E, \bm{\delta}) $, dove $ \mathcal{A} $ è un insieme di punti, $ E $ è uno spazio vettoriale \emph{soggiacente}, $ \bm{\delta} $ è una funzione \[
        \bm{\delta}: \mathcal{A}\times \mathcal{A} \longrightarrow E
    \]tali che \begin{enumerate}
        \item $ \forall\, (P,\bm{v}) \in \mathcal{A}\times E $, $ \exists!\, Q \in \mathcal{A} $ tale che $ \bm{\delta}(P, Q)= \bm{v} $;
        \item $ \forall\, (P,Q,R) $, $ \bm{\delta}(P,Q)+\bm{\delta}(Q,R)=\bm{\delta}(P,R) $
    \end{enumerate}
}
\notazione{
    In tutta questa sezione, indicheremo con le lettere in grasseto ($\bm{v}$) i \emph{vettori}, mentre con lettere maiuscole ($P,Q$) i \emph{punti} appartenenti ad $ \mathcal{A} $.
}
\definizione{
    La \emph{dimensione} dello spazio affine $ \mathcal{A}=(\mathcal{A}, E,\bm{\delta}) $ è la dimensione dello spazio vettoriale $ E $.
}
\definizione{
    $ \bm{v} \in E $ si chiama \emph{vettore libero}. 

    Una coppia $ (P,\bm{v}) \in \mathcal{A}\times E $ è un \emph{vettore applicato} in $ P $, ed è in corrispondenza biunivoca con la coppia di punti \[
        (P,Q) \in \mathcal{A}\times\mathcal{A}
    \]tale che $ \bm{\delta}(P,Q)=\bm{v} $
}
\definizione{
    $ B \subseteq \mathcal{A} $ è un \emph{sottospazio affine} se \[
        \bm{\delta}(B\times B) \subseteq E
    \]
}
%TODO manca un pezzo sull'origine fissata.
\esempio{
    Un esempio di spazio affine è, fissato $ E $ spazio vettoriale: \[
        (E,E,\bm{\delta})
    \]dove $ \bm{\delta} $ è la funzione che ad ogni coppia di vettori ne associa la differenza. 
}
\paragrafo{Riferimento cartesiano}{%
    Un \emph{riferimento cartesiano} è una coppia $ (O,\bm{c}_{\alpha}  ) $, dove $ O $ è l'origine fissata, e $ (\bm{c}_{\alpha} ) = \{\bm{c}_1,\dots,\bm{c}_{n} \} $ è una base dello spazio soggiacente di dimensione $ n $.
}{}{}
\osservazione{
    Vi è una corrispondenza biunivoca, fissato un riferimento cartesiano $ (O,\bm{c}_{\alpha}) $: \begin{align*}
        \bm{\Phi}:\mathcal{A}\times\mathcal{A}&\longrightarrow \R^{n} \\ 
        P&\longrightarrow (x^{\alpha})
    \end{align*}dove, se indico $ \bm{\delta}(O,P) = \bm{OP}$: \[
        \bm{OP}\underset{\footnotemark}{=} x^{\alpha}\,\bm{c}_{\alpha} = \bm{x},\qquad \bm{x} \in \R^{n}
    \]\footnotetext{Si è assunta la \emph{convenzione di Einstein sugli indici ripetuti}, per cui, quando compare un indice in alto e uno in basso, si intende la sommatoria sull'indice: \[
        a^{\alpha}b_{\alpha} \coloneqq \sum_{\alpha}   a_{\alpha}b_{\alpha}
    \]e l'indice si dice \emph{indice sommato} o \emph{indice muto}.}Si ha quindi che le \[
        x^{\alpha}: \mathcal{A}\longrightarrow \R
    \]sono le \emph{coordinate cartesiane} o \emph{affini}.
}
\paragrafo{Trasformazioni affini}{%
    Le trasformazioni affini mandano un riferimento cartesiano un altro: \[
        (O, \bm{c}_{\alpha})\longmapsto (O', \bm{c}_{\alpha}')
    \]%TODO manca un pezzo
}{}{}
%TODO manca un pezzo 
\paragrafo{Campi scalari}{%
    Un \emph{campo scalare} è \[
        f:\mathcal{A}\longrightarrow \R
    \]e si ha quindi la funzione \[
        f \circ \bm{\Phi}^{-1} : \R^{n}\longrightarrow \R
    \]chiamata \emph{rappresentazione del campo $ f $}.
}{}{}
%TODO manca un pezzo
\paragrafo{Campo Vettoriale}{%
    Un \emph{campo vettoriale} $ \bm{X} $ su $ \mathcal{A} $ è una mappa \[\begin{aligned}
        \bm{X}:\mathcal{A}&\longrightarrow \mathcal{A}\times E\\ 
        P&\longmapsto \bm{X}(P)
    \end{aligned}\hspace{3em}\bm{X}(P) = X^{\alpha}\bm{c}_{\alpha}.\]

    Vale che \begin{align*}
        (\bm{X}+\bm{Y})(P)= \bm{X}(P)+\bm{Y}(P)\\ 
        (a\bm{X})(P)= a\bm{X}(P)
    \end{align*}e dunque indichiamo con $ \chi(\mathcal{A}) $ l'anello dei campi vettoriali su $ \mathcal{A} $.
}{}{}
%TODO manca un pezzo % Lezione 13
% \days{18 aprile 2023}

\paragrafo{Topologia}{%
    Sugli spazi affini possiamo definire gli aperti, imponendo che $ \bm{\Phi} $ sia continua, ovvero considerando \[
        U = \bm{\Phi}^{-1}(\mathds{U})
    \]aperto, con $ \mathds{U} $ aperto di $ \R^{n} $.
}{}{} % Lezione 14


\backmatter

\cleardoublepage\normalem
\printbibliography

\end{document}