\days{2 marzo 2023}
\section{Equazioni autonome in due dimensioni}
%% BEGIN Riduzione di un problema di Cauchy autonomo in due dimensioni 
\paragrafo{Orbite non singolari}{%
    Consideriamo l'equazione \[
        \begin{cases}
            x'=f_{1}(x,y)\\ 
            y' = f_{2}(x,y)  
        \end{cases}
    \]Supponiamo $ \bm{p}:\bm{f}(\bm{p})\neq \bm{0} $, $ \bm{p}=(p_1,p_2) $. Com è fatta l'orbita per $ \bm{p} $?

    Se $ \bm{f}(\bm{p})\neq \bm{0} \neq \bm{0} $, allora almeno una delle sue componenti è non nulla. Supponiamo che $ f_{1}(\bm{p})\neq 0  $. 
    
    $\implies$ il vettore tangente all'orbita per $ \bm{p} $ non è verticale. 
    
    Consideriamo allora il sistema: \[
        \begin{cases}
            x'=f_{1}(x,y)\\ 
            y' = f_{2}(x,y)\\ 
            \left(x(0),y(0)\right)=\bm{p}
        \end{cases}
    \]e sia $ \bm{u} $ una soluzione massimale, tale che $ \bm{u}(0)=\bm{p} $. $ \bm{u}'(0) $ non è un vettore verticale 
    
    $\implies$ localmente posso esprimere la seconda componente dell'orbita per $ \bm{p} $ in funzione della prima. \[
        \exists!\, \left\{ \begin{aligned}
        \varphi: I_{p_1}  &\longrightarrow I_{p_2}  \\
        x &\longmapsto \varphi(x)
        \end{aligned}\right.
    \]e questa funzione descrive l'orbita \[
        \begin{cases}
            x=u_1(t)\\ 
            \begin{aligned}
                y=\varphi(x)&= u_2(t)\\ 
                & = \varphi\left(u_1(t)\right)
            \end{aligned}
        \end{cases}
    \]Si ha che \[
        f_2\left(\bm{u}(t)\right) = u_2'(t) = \varphi'\left(u_1(t)\right) \cdot u_1'(t)
    \]da cui $ \displaystyle f_2(x,y)=\varphi'(x)\,f_1(x,y) $ \[
        \varphi'(x)=\frac{f_2(x,y)}{f_1(x,y)} = \frac{f_2\left(x,\varphi(x)\right)}{f_1\left(x,\varphi(x)\right)}.
    \]

    Stiamo dicendo che se $ f_1(p)\neq 0 $ l'orbita che passa per $ \bm{p} $ la posso esprimere mediante una funzione $ y=\varphi(x) $ che soddisfa \[
        \begin{cases}
            \displaystyle\varphi'(x)=\frac{f_2\left(x,\varphi(x)\right)}{f_1\left(x,\varphi(x)\right)}\\[2ex]
            \varphi(p_1)=p_2
        \end{cases}
    \]ovvero un problema di Cauchy monodimensionale \emph{non autonomo}.
}{dafkjbnadlfjknasdlkjncdajklscnadlskjn}{}
\teorema{dlkjanbflkasjdbfnlakjnflkasjdnlkasdjnflaksdfjnlkjnlkjn}{
    Dato il problema di Cauchy \[
        \begin{cases}
            x'=f_1(x,y)\\ 
            y'=f_2(x,y)\\
            \left(x(0),y(0)\right)=\bm{p}
        \end{cases}\tag*{(PC)}
    \]con $ \bm{f}:\Omega \subseteq \R^{2}\to \R^{2}$ di classe $ C^{1} $. Supponiamo che $ f_1(\bm{p})\neq 0 $ (campo non verticale in $ \bm{p} $). Allora per degli opportuni intorni $ I_{p_1}, I_{p_2}$: 

    $ \varphi: I_{p_1}\longrightarrow I_{p_2}   $ è una rappresentazione locale dell'orbita della soluzione di (PC)

    $ \iff $ $ \varphi $ è soluzione di \[
        \begin{cases}
            \displaystyle\varphi'(x)=\frac{f_2\left(x,\varphi(x)\right)}{f_1\left(x,\varphi(x)\right)}\\[2ex]
            \varphi(p_1)=p_2
        \end{cases}\tag*{(PC)\textsubscript{$\varphi$}}
    \]
}
\dimostrazione{dlkjanbflkasjdbfnlakjnflkasjdnlkasdjnflaksdfjnlkjnlkjn}{
    \begin{itemize}
        \item[($\implies$)] Vedi: \framref{dafkjbnadlfjknasdlkjncdajklscnadlskjn}
        \item[($\impliedby$)] $ \varphi $ è soluzione di (PC)\textsubscript{$\varphi$}. Per ipotesi sappiamo che $ f_1\left(x,\varphi(x)\right) \neq 0$ in un opportuno intorno di $ p_1 $, $ I_{p_1}  $ (poiché $ f $ è $ C^{1} $). 
        
        Allora, $ \forall\, x \in I_{p_1}  $ sia\[
            \omega(x)=\int_{p_1}^{x}\frac{d\,\xi}{f_1\left(\xi, \varphi(\xi)\right)}.
        \]Poiché $ f_1 $ è continua e non nulla 
        
        $\implies$ $ \omega $ è derivabile e \[
            \begin{cases}
                \omega'(x)=\displaystyle\frac{1}{f_1\left(x,\varphi(x)\right)}\\[2ex]
                \omega(p_1)=0
            \end{cases}
        \]Poiché $ \omega' $ ha segno costante 
        
        $\implies$ $ \omega $ è strettamente monotona 
        
        $\implies$ $ \omega $ è invertibile. \[
             \begin{aligned}
                \omega: I_{p_1}\to I \ni 0 \quad\leadsto\quad\omega^{-1}=: v:I&\longrightarrow I_{p_1}\\ 
                v(0) &= p_1\\ 
                v'(t) &=\displaystyle \frac{1}{\omega'\left(v(t)\right)}  
            \end{aligned}
        \] 
        
        $\implies$ $ \displaystyle v'(t) = f_1\left(v(t), \varphi\left(v(t)\right)\right) $. 

        Se chiamo \[
            \bm{u}(t)\coloneqq\begin{pmatrix}
                v(t)\\\varphi\left(v(t)\right)
            \end{pmatrix}
        \]ho che \[
            \begin{cases}
                u_1'(t)=v'(t)=  f_1\left(v(t), \varphi\left(v(t)\right)\right) = f_1\left(u_1(t),u_2(t)\right)\\[2ex]
                u_2'(t)=\varphi'\left(v(t)\right)\,v'(t) = \displaystyle\frac{f_2}{f_1} f_1 = f_2
            \end{cases}
        \]e inoltre $ \bm{u}(0)=\bm{p}. $\qed
    \end{itemize}
}
%% END
\esempio{
    \[
        \begin{cases}
            x'=-y^{2}\\ 
            y'=x^{2}
        \end{cases}
    \]
    \begin{enumerate}
        \item \emph{Equilibri}. \[
            \bm{f}(x,y)=(-y^{2},x^{2}) = (0,0)
        \] 
        
        $\implies$ ho un solo punto di equilibrio, $ \bm{0} $. 
        \item \emph{Punti con tangente orizzontale o verticale}.
        
        Il vettore tangente ad una soluzione in $ (x,y) $ è $ \bm{f}(x,y) $, dunque: \begin{itemize}
            \item tangente verticale: $ x'=0 $ $ \iff $ $ y^{2} =0$ 
            
            $\implies$ l'asse $ x $ viene intersecato verticalmente dalle orbite;
            \item tangente verticale: $ y'=0 $ $ \iff $ $ x^{2}=0 $ 
            
            $\implies$ l'asse $ y $ viene intersecato orizzontalmente dalle orbite.
        \end{itemize}

        In $ (x,0) $ il vettore tangente alla soluzione che passa per quel punto è $ (0,x^{2}) $, dunque tutte le orbite che passano per l'asse delle $ x $ sono percorse verso l'alto.

        In $ (0,y) $ il vettore tangente alla soluzione che passa per quel punto è $ (-y^{2}, 0 ) $, dunque tutte le orbite che passano per l'asse delle $ y $ sono percorse verso sinistra.
        \item \emph{Orbite non singolari}
        
        Consideriamo i punti in cui $ f_1(x,y)\neq 0 $, ovvero i punti con $ y\neq 0 $. Cerchiamo l'equazione delle orbite dei punti che non sono sull'asse $ x $. \[
            \begin{cases}
                \varphi'(x)=-\frac{x^{2}}{\varphi^{2}(x)}\\
                \varphi(p_1)=p_2\tag*{(PC)\textsubscript{$\varphi$}}
            \end{cases}
        \]che è a variabili separabili: \begin{align*}
            \varphi^{2}(x)\,\varphi'(x) & = -x^{2}\\[2ex] 
            \int \varphi^{2}(x)\,\varphi'(x)\,dx & = \int -x^{2}\,dx\\[2ex] 
            \int \varphi\,d\varphi & = -\int x^{2}\,dx\\[2ex]
            \frac{1}{3} \varphi^{3}(x) &= -\frac{1}{3}x^{3} + c
        \end{align*}

        Imponendo il passaggio per $ \bm{p} $, otteniamo che $ \displaystyle c= \frac{1}{3}\, p_2^{3}$. 
        
        Dunque \[
            \varphi(x)=\sqrt[3]{p_2^{3}-x^{3}}
        \]e le orbite sono quelle mostrate in figura \ref{fig:orbiteradiceterza}
        \begin{figure}
            \begin{center}
                % This file was created by matlab2tikz.
%
%The latest updates can be retrieved from
%  http://www.mathworks.com/matlabcentral/fileexchange/22022-matlab2tikz-matlab2tikz
%where you can also make suggestions and rate matlab2tikz.
%
\begin{tikzpicture}

    \begin{axis}[%
      width=0.8*\textwidth,
      axis equal,
      axis lines=middle,
      xmin=-5,
      xmax=5,
      ymin=-5,
      ymax=5,
      axis background/.style={fill=white}
      ]
    \addplot [black] {-x};
    \fill [black] (0,0) circle (0.1);
    \draw [-{Latex[length=2.5mm]}] (1,-1) -- (0.15,-0.15);
    \draw [-{Latex[length=2.5mm]}] (0,0) -- (-0.45,0.45);
    \addplot [color=black, forget plot]
      table[row sep=crcr]{%
    -5	2.43333124830413\\
    -4.95	2.20325791411787\\
    -4.9	1.91810939473754\\
    -4.85	1.51715490996357\\
    -4.8	0\\
    };
    \addplot [color=black, forget plot]
      table[row sep=crcr]{%
    -4.8	-0\\
    -4.75	-1.50665561007714\\
    -4.7	-1.89165313589352\\
    -4.65	-2.15783154291702\\
    -4.6	-2.36666887081282\\
    -4.55	-2.54044224221422\\
    -4.5	-2.69008710671805\\
    -4.45	-2.82188944180945\\
    -4.4	-2.939838638621\\
    -4.35	-3.04664595911424\\
    -4.3	-3.14424918656787\\
    -4.25	-3.23408790150043\\
    -4.2	-3.31726488820732\\
    -4.15	-3.39464627442907\\
    -4.1	-3.46692655020797\\
    -4.05	-3.53467239404558\\
    -4	-3.59835315622033\\
    -3.95	-3.65836263567708\\
    -3.9	-3.71503499958271\\
    -3.85	-3.76865665681006\\
    -3.8	-3.8194752712541\\
    -3.75	-3.86770671173796\\
    -3.7	-3.91354048616405\\
    -3.65	-3.95714404406568\\
    -3.6	-3.99866622197514\\
    -3.55	-4.0382400308764\\
    -3.5	-4.07598493260902\\
    -3.45	-4.11200871494441\\
    -3.4	-4.14640904832589\\
    -3.35	-4.17927478776402\\
    -3.3	-4.21068706897219\\
    -3.25	-4.24072023705863\\
    -3.2	-4.26944263795511\\
    -3.15	-4.29691729655335\\
    -3.1	-4.32320250073683\\
    -3.05	-4.34835230677976\\
    -3	-4.3724169786731\\
    -2.95	-4.39544337163964\\
    -2.9	-4.41747526827306\\
    -2.85	-4.43855367427352\\
    -2.8	-4.45871707957417\\
    -2.75	-4.47800168969859\\
    -2.7	-4.49644163141121\\
    -2.65	-4.51406913608526\\
    -2.6	-4.5309147036877\\
    -2.55	-4.54700724984663\\
    -2.5	-4.56237423810478\\
    -2.45	-4.57704179916166\\
    -2.4	-4.59103483865373\\
    -2.35	-4.60437713480893\\
    -2.3	-4.61709142713204\\
    -2.25	-4.62919949712464\\
    -2.2	-4.64072224191325\\
    -2.15	-4.65167974154844\\
    -2.1	-4.66209132064226\\
    -2.05	-4.67197560492966\\
    -2	-4.68135057326899\\
    -1.95	-4.69023360553561\\
    -1.9	-4.69864152680957\\
    -1.85	-4.70659064821255\\
    -1.8	-4.7140968047088\\
    -1.75	-4.72117539014982\\
    -1.7	-4.72784138981183\\
    -1.65	-4.73410941064774\\
    -1.6	-4.73999370945179\\
    -1.55	-4.7455082191138\\
    -1.5	-4.7506665731213\\
    -1.45	-4.75548212845142\\
    -1.4	-4.75996798697958\\
    -1.35	-4.76413701551886\\
    -1.3	-4.76800186459246\\
    -1.25	-4.77157498603078\\
    -1.2	-4.77486864947567\\
    -1.15	-4.77789495786555\\
    -1.1	-4.78066586196769\\
    -1.05	-4.78319317401706\\
    -1	-4.78548858051474\\
    -0.95	-4.78756365423345\\
    -0.899999999999999	-4.78942986547258\\
    -0.85	-4.79109859260026\\
    -0.8	-4.79258113191607\\
    -0.75	-4.79388870686385\\
    -0.699999999999999	-4.79503247662074\\
    -0.649999999999999	-4.79602354408525\\
    -0.6	-4.79687296328433\\
    -0.55	-4.7975917462165\\
    -0.5	-4.79819086914608\\
    -0.449999999999999	-4.79868127836094\\
    -0.399999999999999	-4.79907389540437\\
    -0.35	-4.79937962178998\\
    -0.3	-4.79960934320654\\
    -0.25	-4.79977393321878\\
    -0.199999999999999	-4.79988425646833\\
    -0.149999999999999	-4.79995117137829\\
    -0.0999999999999996	-4.7999855323638\\
    -0.0499999999999998	-4.79999819155024\\
    8.88178419700125e-16	-4.8\\
    0.0500000000000007	-4.80000180844839\\
    0.1	-4.80001446754899\\
    0.149999999999999	-4.8000488276283\\
    0.199999999999999	-4.80011573795004\\
    0.25	-4.80022604548897\\
    0.3	-4.80039059321517\\
    0.35	-4.80062021788932\\
    0.399999999999999	-4.80092574737107\\
    0.45	-4.80131799744245\\
    0.5	-4.80180776814992\\
    0.55	-4.80240583966935\\
    0.6	-4.8031229676995\\
    0.649999999999999	-4.8039698783909\\
    0.7	-4.80495726281873\\
    0.75	-4.80609577100958\\
    0.8	-4.80739600553389\\
    0.85	-4.80886851467779\\
    0.899999999999999	-4.81052378520972\\
    0.95	-4.8123722347595\\
    1	-4.81442420382937\\
    1.05	-4.81668994745888\\
    1.1	-4.81917962656736\\
    1.15	-4.82190329900017\\
    1.2	-4.82487091030687\\
    1.25	-4.82809228428169\\
    1.3	-4.83157711329864\\
    1.35	-4.8353349484757\\
    1.4	-4.8393751897043\\
    1.45	-4.84370707558215\\
    1.5	-4.84833967328888\\
    1.55	-4.8532818684457\\
    1.6	-4.85854235500106\\
    1.65	-4.86412962518571\\
    1.7	-4.87005195958094\\
    1.75	-4.87631741734448\\
    1.8	-4.88293382663862\\
    1.85	-4.88990877530508\\
    1.9	-4.89724960183065\\
    1.95	-4.90496338664699\\
    2	-4.91305694380694\\
    2.05	-4.92153681307814\\
    2.1	-4.93040925249353\\
    2.15	-4.93968023139562\\
    2.2	-4.94935542401002\\
    2.25	-4.95944020358025\\
    2.3	-4.96993963709383\\
    2.35	-4.98085848062601\\
    2.4	-4.99220117532457\\
    2.45	-5.00397184405552\\
    2.5	-5.01617428872595\\
    2.55	-5.02881198829654\\
    2.6	-5.04188809749251\\
    2.65	-5.05540544621769\\
    2.7	-5.06936653967282\\
    2.75	-5.08377355917515\\
    2.8	-5.09862836367258\\
    2.85	-5.11393249194192\\
    2.9	-5.12968716545745\\
    2.95	-5.14589329191211\\
    3	-5.16255146937099\\
    3.05	-5.17966199103314\\
    3.1	-5.19722485057551\\
    3.15	-5.2152397480499\\
    3.2	-5.23370609630174\\
    3.25	-5.25262302787746\\
    3.3	-5.27198940238559\\
    3.35	-5.29180381427528\\
    3.4	-5.31206460099498\\
    3.45	-5.33276985149294\\
    3.5	-5.35391741502115\\
    3.55	-5.37550491020363\\
    3.6	-5.3975297343305\\
    3.65	-5.41998907283918\\
    3.7	-5.44287990894502\\
    3.75	-5.46619903338409\\
    3.8	-5.48994305423219\\
    3.85	-5.51410840676505\\
    3.9	-5.53869136332627\\
    3.95	-5.56368804317094\\
    4	-5.58909442225448\\
    4.05	-5.61490634293814\\
    4.1	-5.64111952358409\\
    4.15	-5.6677295680154\\
    4.2	-5.6947319748176\\
    4.25	-5.72212214646078\\
    4.3	-5.74989539822316\\
    4.35	-5.77804696689878\\
    4.4	-5.80657201927403\\
    4.45	-5.83546566035971\\
    4.5	-5.86472294136691\\
    4.55	-5.89433886741701\\
    4.6	-5.92430840497767\\
    4.65	-5.9546264890184\\
    4.7	-5.98528802988077\\
    4.75	-6.01628791985994\\
    4.8	-6.04762103949539\\
    4.85	-6.07928226357011\\
    4.9	-6.11126646681883\\
    4.95	-6.14356852934671\\
    5	-6.17618334176118\\
    };
    
    \addplot[area legend, draw=black, fill=black, forget plot]
    table[row sep=crcr] {%
    x	y\\
    -3.72083693095564	-3.89868811849171\\
    -3.70976252472368	-3.90976252472368\\
    -3.70976252472368	-3.90976252472368\\
    -3.70976252472368	-3.90976252472368\\
    -3.70976252472368	-3.90976252472368\\
    -3.70976252472368	-3.90976252472368\\
    -3.73191133718761	-3.88761371225975\\
    -3.7829093445653	-3.93861171963744\\
    -3.90976252472368	-3.70976252472368\\
    -3.68091332980991	-3.83661570488206\\
    -3.73191133718761	-3.88761371225975\\
    }--cycle;
    \addplot [color=black, forget plot]
      table[row sep=crcr]{%
    -5	3.41469990581837\\
    -4.95	3.30508475380455\\
    -4.9	3.19010616025606\\
    -4.85	3.06878578458167\\
    -4.8	2.939838638621\\
    -4.75	2.80152446746804\\
    -4.7	2.65139359475723\\
    -4.65	2.48581971993629\\
    -4.6	2.2990544317303\\
    -4.55	2.08103675817557\\
    -4.5	1.81114484514709\\
    -4.45	1.43210263327108\\
    -4.4	0\\
    };
    \addplot [color=black, forget plot]
      table[row sep=crcr]{%
    -4.4	-0\\
    -4.35	-1.42129436301186\\
    -4.3	-1.7839101015784\\
    -4.25	-2.03427387882488\\
    -4.2	-2.23043112043839\\
    -4.15	-2.39342092816138\\
    -4.1	-2.53357376911577\\
    -4.05	-2.65682942377063\\
    -4	-2.76695856664369\\
    -3.95	-2.86652450319574\\
    -3.9	-2.95735977133685\\
    -3.85	-3.04082614752338\\
    -3.8	-3.11796711754744\\
    -3.75	-3.18960248319805\\
    -3.7	-3.25638979571086\\
    -3.65	-3.3188657699515\\
    -3.6	-3.37747509342703\\
    -3.55	-3.4325910094841\\
    -3.5	-3.48453036602697\\
    -3.45	-3.53356484084942\\
    -3.4	-3.57992946398274\\
    -3.35	-3.62382918986462\\
    -3.3	-3.66544403681055\\
    -3.25	-3.70493315680415\\
    -3.2	-3.74243809494084\\
    -3.15	-3.77808542685497\\
    -3.1	-3.81198891294557\\
    -3.05	-3.84425127311386\\
    -3	-3.87496566046572\\
    -2.95	-3.90421689400247\\
    -2.9	-3.93208249670732\\
    -2.85	-3.95863357525606\\
    -2.8	-3.98393556988968\\
    -2.75	-4.00804889711701\\
    -2.7	-4.03102950339355\\
    -2.65	-4.05292934440889\\
    -2.6	-4.07379680186249\\
    -2.55	-4.09367704743449\\
    -2.5	-4.11261236193034\\
    -2.45	-4.1306424161952\\
    -2.4	-4.1478045192799\\
    -2.35	-4.16413383843723\\
    -2.3	-4.17966359479151\\
    -2.25	-4.19442523792143\\
    -2.2	-4.20844860209926\\
    -2.15	-4.22176204651851\\
    -2.1	-4.23439258150042\\
    -2.05	-4.24636598238375\\
    -2	-4.2577068925634\\
    -1.95	-4.2684389169411\\
    -1.9	-4.27858470688149\\
    -1.85	-4.28816603762166\\
    -1.8	-4.29720387895949\\
    -1.75	-4.30571845994028\\
    -1.7	-4.31372932817138\\
    -1.65	-4.3212554043164\\
    -1.6	-4.32831503225368\\
    -1.55	-4.33492602532579\\
    -1.5	-4.34110570905614\\
    -1.45	-4.34687096066514\\
    -1.4	-4.35223824567986\\
    -1.35	-4.35722365189759\\
    -1.3	-4.3618429209344\\
    -1.25	-4.36611147756342\\
    -1.2	-4.37004445702512\\
    -1.15	-4.37365673047123\\
    -1.1	-4.37696292868603\\
    -1.05	-4.3799774642127\\
    -0.999999999999999	-4.3827145519982\\
    -0.949999999999999	-4.38518822865711\\
    -0.899999999999999	-4.38741237044377\\
    -0.849999999999999	-4.38940071001136\\
    -0.799999999999999	-4.39116685202757\\
    -0.749999999999999	-4.39272428770812\\
    -0.699999999999999	-4.39408640832168\\
    -0.649999999999999	-4.39526651771311\\
    -0.599999999999999	-4.39627784388566\\
    -0.549999999999999	-4.3971335496773\\
    -0.499999999999999	-4.39784674256114\\
    -0.449999999999999	-4.39843048359551\\
    -0.399999999999999	-4.39889779554507\\
    -0.35	-4.39926167019068\\
    -0.299999999999999	-4.39953507484244\\
    -0.249999999999999	-4.39973095806743\\
    -0.199999999999999	-4.39986225464099\\
    -0.149999999999999	-4.39994188972841\\
    -0.0999999999999996	-4.39998278230177\\
    -0.0499999999999989	-4.39999784779509\\
    8.88178419700125e-16	-4.4\\
    0.0500000000000007	-4.4000021522028\\
    0.100000000000001	-4.40001721756348\\
    0.15	-4.40005810873671\\
    0.200000000000001	-4.4001377367351\\
    0.250000000000001	-4.40026900903497\\
    0.3	-4.40046482692586\\
    0.35	-4.40073808210567\\
    0.399999999999999	-4.40110165252475\\
    0.45	-4.40156839748363\\
    0.5	-4.40215115199106\\
    0.55	-4.40286272039121\\
    0.6	-4.40371586927097\\
    0.649999999999999	-4.40472331966154\\
    0.7	-4.40589773855084\\
    0.75	-4.40725172972692\\
    0.8	-4.40879782397563\\
    0.85	-4.41054846865957\\
    0.899999999999999	-4.41251601670897\\
    0.95	-4.41471271505911\\
    1	-4.41715069257253\\
    1.05	-4.41984194748881\\
    1.1	-4.42279833444796\\
    1.15	-4.426031551138\\
    1.2	-4.42955312462073\\
    1.25	-4.43337439739352\\
    1.3	-4.43750651324839\\
    1.35	-4.44196040299268\\
    1.4	-4.44674677009875\\
    1.45	-4.45187607635236\\
    1.5	-4.45735852757163\\
    1.55	-4.4632040594701\\
    1.6	-4.46942232373823\\
    1.65	-4.47602267441873\\
    1.7	-4.4830141546505\\
    1.75	-4.49040548385564\\
    1.8	-4.49820504544276\\
    1.85	-4.50642087509766\\
    1.9	-4.51506064972999\\
    1.95	-4.52413167714154\\
    2	-4.53364088647765\\
    2.05	-4.54359481951908\\
    2.1	-4.55399962286679\\
    2.15	-4.56486104106651\\
    2.2	-4.57618441071419\\
    2.25	-4.58797465557711\\
    2.3	-4.60023628275896\\
    2.35	-4.61297337993022\\
    2.4	-4.62618961363815\\
    2.45	-4.63988822870376\\
    2.5	-4.65407204870588\\
    2.55	-4.66874347754531\\
    2.6	-4.68390450207526\\
    2.65	-4.69955669577749\\
    2.7	-4.71570122345718\\
    2.75	-4.73233884692341\\
    2.8	-4.74946993161649\\
    2.85	-4.76709445413835\\
    2.9	-4.78521201063719\\
    2.95	-4.80382182599364\\
    3	-4.82292276375218\\
    3.05	-4.84251333673835\\
    3.1	-4.86259171830001\\
    3.15	-4.8831557541092\\
    3.2	-4.90420297445978\\
    3.25	-4.92573060699547\\
    3.3	-4.94773558980296\\
    3.35	-4.97021458480507\\
    3.4	-4.99316399138997\\
    3.45	-5.01657996021387\\
    3.5	-5.04045840711649\\
    3.55	-5.06479502709083\\
    3.6	-5.0895853082511\\
    3.65	-5.11482454574594\\
    3.7	-5.14050785556663\\
    3.75	-5.16663018820354\\
    3.8	-5.19318634210752\\
    3.85	-5.22017097691613\\
    3.9	-5.24757862640856\\
    3.95	-5.27540371115633\\
    4	-5.30364055084068\\
    4.05	-5.332283376211\\
    4.1	-5.36132634066212\\
    4.15	-5.39076353141155\\
    4.2	-5.42058898026123\\
    4.25	-5.45079667393119\\
    4.3	-5.48138056395562\\
    4.35	-5.51233457613462\\
    4.4	-5.54365261953744\\
    4.45	-5.57532859505547\\
    4.5	-5.6073564035055\\
    4.55	-5.63972995328585\\
    4.6	-5.67244316758968\\
    4.65	-5.70548999118167\\
    4.7	-5.73886439674557\\
    4.75	-5.77256039081165\\
    4.8	-5.80657201927403\\
    4.85	-5.84089337250893\\
    4.9	-5.87551859010587\\
    4.95	-5.91044186522429\\
    5	-5.94565744858888\\
    };
    
    \addplot[area legend, draw=black, fill=black, forget plot]
    table[row sep=crcr] {%
    x	y\\
    -3.39322040106781	-3.59134422759226\\
    -3.39228231433004	-3.59228231433004\\
    -3.39228231433004	-3.59228231433004\\
    -3.39228231433004	-3.59228231433004\\
    -3.39228231433004	-3.59228231433004\\
    -3.39228231433004	-3.59228231433004\\
    -3.39415848780559	-3.59040614085449\\
    -3.45096958087035	-3.64721723391925\\
    -3.59228231433004	-3.39228231433004\\
    -3.33734739474082	-3.53359504778972\\
    -3.39415848780559	-3.59040614085449\\
    }--cycle;
    \addplot [color=black, forget plot]
      table[row sep=crcr]{%
    -5	3.93649718310217\\
    -4.95	3.85495791238075\\
    -4.9	3.77155585461804\\
    -4.85	3.68609646863451\\
    -4.8	3.59835315622033\\
    -4.75	3.50805965759495\\
    -4.7	3.41490000543241\\
    -4.65	3.31849501754868\\
    -4.6	3.2183837676756\\
    -4.55	3.11399757505469\\
    -4.5	3.00462250345868\\
    -4.45	2.88934356885749\\
    -4.4	2.76695856664369\\
    -4.35	2.63583878230465\\
    -4.3	2.49369075744464\\
    -4.25	2.3371182901163\\
    -4.2	2.16073591750959\\
    -4.15	1.95511477752707\\
    -4.1	1.70092222164965\\
    -4.05	1.34444447606544\\
    -4	0\\
    };
    \addplot [color=black, forget plot]
      table[row sep=crcr]{%
    -4	-0\\
    -3.95	-1.33328732480132\\
    -3.9	-1.67280845040534\\
    -3.85	-1.90684282377084\\
    -3.8	-2.08989857798625\\
    -3.75	-2.24173925559838\\
    -3.7	-2.37207210759166\\
    -3.65	-2.48648035278273\\
    -3.6	-2.58850945078415\\
    -3.55	-2.68057039355508\\
    -3.5	-2.76438740683944\\
    -3.45	-2.84124222420702\\
    -3.4	-2.91211735227267\\
    -3.35	-2.97778497771623\\
    -3.3	-3.03886470765518\\
    -3.25	-3.09586249965226\\
    -3.2	-3.14919774648174\\
    -3.15	-3.19922263018076\\
    -3.1	-3.24623627419026\\
    -3.05	-3.2904953014888\\
    -3	-3.33222185164595\\
    -2.95	-3.37160976432446\\
    -2.9	-3.40882941563296\\
    -2.85	-3.4440315485722\\
    -2.8	-3.47735034137574\\
    -2.75	-3.50890589081016\\
    -2.7	-3.53880624095771\\
    -2.65	-3.56714905500759\\
    -2.6	-3.59402300383506\\
    -2.55	-3.61950892782091\\
    -2.5	-3.64368081556092\\
    -2.45	-3.66660663354419\\
    -2.4	-3.68834903364676\\
    -2.35	-3.70896595976667\\
    -2.3	-3.72851117067382\\
    -2.25	-3.74703469284268\\
    -2.2	-3.7645832144467\\
    -2.15	-3.78120042964864\\
    -2.1	-3.79692734069548\\
    -2.05	-3.81180252402532\\
    -2	-3.82586236554478\\
    -1.95	-3.839141269386\\
    -1.9	-3.85167184375929\\
    -1.85	-3.8634850669495\\
    -1.8	-3.87461043603686\\
    -1.75	-3.88507610053522\\
    -1.7	-3.89490898281869\\
    -1.65	-3.90413488693867\\
    -1.6	-3.912778597207\\
    -1.55	-3.92086396773085\\
    -1.5	-3.928414003924\\
    -1.45	-3.93545093688214\\
    -1.4	-3.94199629139342\\
    -1.35	-3.94807094825566\\
    -1.3	-3.95369520148593\\
    -1.25	-3.95888881093441\\
    -1.2	-3.96367105075071\\
    -1.15	-3.96806075409521\\
    -1.1	-3.97207635444015\\
    -1.05	-3.97573592376282\\
    -1	-3.97905720789639\\
    -0.95	-3.98205765927177\\
    -0.9	-3.98475446725496\\
    -0.85	-3.9871645862595\\
    -0.8	-3.98930476179092\\
    -0.75	-3.99119155456048\\
    -0.7	-3.99284136278774\\
    -0.65	-3.99427044279541\\
    -0.6	-3.9954949279864\\
    -0.55	-3.9965308462796\\
    -0.5	-3.99739413607028\\
    -0.45	-3.99810066077037\\
    -0.4	-3.99866622197514\\
    -0.35	-3.99910657129448\\
    -0.3	-3.99943742087989\\
    -0.25	-3.99967445267212\\
    -0.2	-3.99983332638841\\
    -0.15	-3.999929686264\\
    -0.0999999999999996	-3.99997916655816\\
    -0.0499999999999998	-3.99999739583164\\
    0	-4\\
    0.0499999999999998	-4.00000260416497\\
    0.100000000000001	-4.00002083322483\\
    0.15	-4.00007031126407\\
    0.2	-4.0001666597227\\
    0.25	-4.00032549434597\\
    0.3	-4.00056242091697\\
    0.350000000000001	-4.00089302977628\\
    0.4	-4.00133288913564\\
    0.45	-4.00189753719581\\
    0.5	-4.00260247308292\\
    0.55	-4.00346314662189\\
    0.6	-4.00449494697088\\
    0.649999999999999	-4.00571319014622\\
    0.7	-4.00713310547364\\
    0.75	-4.0087698210081\\
    0.8	-4.01063834797167\\
    0.85	-4.01275356426607\\
    0.899999999999999	-4.01513019712441\\
    0.95	-4.01778280497382\\
    1	-4.02072575858906\\
    1.05	-4.02397322162419\\
    1.1	-4.02753913061748\\
    1.15	-4.0314371745714\\
    1.2	-4.03568077421651\\
    1.25	-4.04028306107407\\
    1.3	-4.04525685643787\\
    1.35	-4.05061465039998\\
    1.4	-4.05636858104921\\
    1.45	-4.06253041397349\\
    1.5	-4.06911152219885\\
    1.55	-4.07612286669812\\
    1.6	-4.08357497760117\\
    1.65	-4.09147793623635\\
    1.7	-4.09984135812903\\
    1.75	-4.10867437707794\\
    1.8	-4.11798563042365\\
    1.85	-4.12778324561581\\
    1.9	-4.13807482817678\\
    1.95	-4.14886745114917\\
    2	-4.16016764610381\\
    2.05	-4.17198139577253\\
    2.1	-4.18431412835758\\
    2.15	-4.19717071355611\\
    2.2	-4.21055546032458\\
    2.25	-4.22447211639401\\
    2.3	-4.23892386953327\\
    2.35	-4.25391335054368\\
    2.4	-4.26944263795511\\
    2.45	-4.28551326438077\\
    2.5	-4.30212622447582\\
    2.55	-4.31928198443375\\
    2.6	-4.33698049294394\\
    2.65	-4.3552211935248\\
    2.7	-4.3740030381387\\
    2.75	-4.39332450198799\\
    2.8	-4.41318359938584\\
    2.85	-4.43357790059152\\
    2.9	-4.45450454949628\\
    2.95	-4.47596028204469\\
    3	-4.49794144527541\\
    3.05	-4.52044401686613\\
    3.1	-4.54346362506894\\
    3.15	-4.5669955689254\\
    3.2	-4.59103483865373\\
    3.25	-4.61557613610522\\
    3.3	-4.64061389519198\\
    3.35	-4.66614230219372\\
    3.4	-4.69215531585751\\
    3.45	-4.71864668721076\\
    3.5	-4.74560997901466\\
    3.55	-4.77303858479212\\
    3.6	-4.80092574737107\\
    3.65	-4.82926457689136\\
    3.7	-4.85804806822989\\
    3.75	-4.88726911780576\\
    3.8	-4.9169205397334\\
    3.85	-4.94699508129806\\
    3.9	-4.97748543773394\\
    3.95	-5.00838426629069\\
    4	-5.03968419957949\\
    4.05	-5.07137785819438\\
    4.1	-5.1034578626093\\
    4.15	-5.13591684435504\\
    4.2	-5.16874745648424\\
    4.25	-5.20194238333536\\
    4.3	-5.23549434960994\\
    4.35	-5.26939612877947\\
    4.4	-5.30364055084068\\
    4.45	-5.33822050943947\\
    4.5	-5.37312896838572\\
    4.55	-5.4083589675817\\
    4.6	-5.44390362838836\\
    4.65	-5.47975615845403\\
    4.7	-5.51590985603062\\
    4.75	-5.55235811380267\\
    4.8	-5.58909442225448\\
    4.85	-5.6261123726007\\
    4.9	-5.66340565930516\\
    4.95	-5.70096808221258\\
    5	-5.73879354831717\\
    };
    
    \addplot[area legend, draw=black, fill=black, forget plot]
    table[row sep=crcr] {%
    x	y\\
    -3.06589924167803	-3.28370496619477\\
    -3.06589924167803	-3.28370496619477\\
    -3.06589924167803	-3.28370496619477\\
    -3.06589924167803	-3.28370496619477\\
    -3.06589924167803	-3.28370496619477\\
    -3.06589924167803	-3.28370496619477\\
    -3.06589924167803	-3.28370496619477\\
    -3.12580117350242	-3.34360689801916\\
    -3.2748021039364	-3.0748021039364\\
    -3.00599730985363	-3.22380303437037\\
    -3.06589924167803	-3.28370496619477\\
    }--cycle;
    \addplot [color=black, forget plot]
      table[row sep=crcr]{%
    -5	4.27893064384067\\
    -4.95	4.21024285680844\\
    -4.9	4.14068166173797\\
    -4.85	4.07018073559454\\
    -4.8	3.99866622197514\\
    -4.75	3.92605553987667\\
    -4.7	3.85225594439255\\
    -4.65	3.77716277394065\\
    -4.6	3.70065729739648\\
    -4.55	3.62260404495532\\
    -4.5	3.54284746479196\\
    -4.45	3.46120768760821\\
    -4.4	3.37747509342703\\
    -4.35	3.29140324406587\\
    -4.3	3.20269954491915\\
    -4.25	3.11101268703176\\
    -4.2	3.01591541718809\\
    -4.15	2.91688034715677\\
    -4.1	2.81324507009292\\
    -4.05	2.70416025311282\\
    -4	2.58850945078415\\
    -3.95	2.46477946918048\\
    -3.9	2.33083860350175\\
    -3.85	2.18352884962524\\
    -3.8	2.01784038710825\\
    -3.75	1.8249984362281\\
    -3.7	1.58700410245128\\
    -3.65	1.25382826354938\\
    -3.6	0\\
    };
    \addplot [color=black, forget plot]
      table[row sep=crcr]{%
    -3.6	-0\\
    -3.55	-1.24227232487739\\
    -3.5	-1.5578855860124\\
    -3.45	-1.77500165310962\\
    -3.4	-1.94447262241856\\
    -3.35	-2.08474393357267\\
    -3.3	-2.20487897896575\\
    -3.25	-2.31009269819094\\
    -3.2	-2.40369800276695\\
    -3.15	-2.48794866615549\\
    -3.1	-2.56445716036143\\
    -3.05	-2.63442267416152\\
    -3	-2.69876486716525\\
    -2.95	-2.75820688858728\\
    -2.9	-2.81332930277019\\
    -2.85	-2.86460645344058\\
    -2.8	-2.91243176788213\\
    -2.75	-2.95713584190443\\
    -2.7	-2.99899966648136\\
    -2.65	-3.03826449734771\\
    -2.6	-3.07513935076953\\
    -2.55	-3.10980678624441\\
    -2.5	-3.14242743042701\\
    -2.45	-3.17314356103136\\
    -2.4	-3.20208197846633\\
    -2.35	-3.22935633063355\\
    -2.3	-3.25506901284616\\
    -2.25	-3.27931273400483\\
    -2.2	-3.30217181798291\\
    -2.15	-3.32372329298502\\
    -2.1	-3.34403780968063\\
    -2.05	-3.36318041997095\\
    -2	-3.38121124148806\\
    -1.95	-3.39818602776578\\
    -1.9	-3.41415666004636\\
    -1.85	-3.42917157359721\\
    -1.8	-3.4432761289903\\
    -1.75	-3.45651293688558\\
    -1.7	-3.46892214333853\\
    -1.65	-3.48054168143494\\
    -1.6	-3.49140749407442\\
    -1.55	-3.50155373192867\\
    -1.5	-3.51101292995174\\
    -1.45	-3.51981616528734\\
    -1.4	-3.52799319897976\\
    -1.35	-3.5355726035316\\
    -1.3	-3.54258187804877\\
    -1.25	-3.54904755246047\\
    -1.2	-3.55499528208884\\
    -1.15	-3.56044993366385\\
    -1.1	-3.56543566372656\\
    -1.05	-3.56997599023468\\
    -1	-3.57409385807344\\
    -0.949999999999999	-3.57781169907984\\
    -0.899999999999999	-3.58115148710675\\
    -0.85	-3.5841347885821\\
    -0.799999999999999	-3.58678280895741\\
    -0.75	-3.58911643538605\\
    -0.699999999999999	-3.59115627592492\\
    -0.649999999999999	-3.5929226955121\\
    -0.6	-3.59443584893705\\
    -0.549999999999999	-3.5957157109879\\
    -0.5	-3.59678210393225\\
    -0.449999999999999	-3.59765472246325\\
    -0.399999999999999	-3.59835315622033\\
    -0.35	-3.5988969099748\\
    -0.299999999999999	-3.59930542155328\\
    -0.25	-3.59959807755697\\
    -0.199999999999999	-3.59979422692153\\
    -0.149999999999999	-3.59991319235125\\
    -0.0999999999999996	-3.59997427965163\\
    -0.0499999999999994	-3.59999678497655\\
    4.44089209850063e-16	-3.6\\
    0.0500000000000007	-3.6000032150177\\
    0.100000000000001	-3.60002571998085\\
    0.15	-3.60008680346253\\
    0.200000000000001	-3.60020574955751\\
    0.25	-3.6004018327177\\
    0.300000000000001	-3.6006943105283\\
    0.350000000000001	-3.60110241443493\\
    0.4	-3.60164533843748\\
    0.45	-3.60234222577462\\
    0.500000000000001	-3.60321215363176\\
    0.550000000000001	-3.60427411591559\\
    0.600000000000001	-3.60554700415042\\
    0.65	-3.60704958656418\\
    0.7	-3.60880048544596\\
    0.75	-3.61081815287203\\
    0.8	-3.61312084491233\\
    0.85	-3.61572659444601\\
    0.899999999999999	-3.61865318273015\\
    0.95	-3.62191810988224\\
    1	-3.62553856445265\\
    1.05	-3.62953139227823\\
    1.1	-3.63391306482233\\
    1.15	-3.63869964721907\\
    1.2	-3.64390676625079\\
    1.25	-3.64954957849654\\
    1.3	-3.65564273889617\\
    1.35	-3.66220036997897\\
    1.4	-3.66923603200685\\
    1.45	-3.67676269428092\\
    1.5	-3.6847927078552\\
    1.55	-3.69333777989389\\
    1.6	-3.70240894989728\\
    1.65	-3.71201656800751\\
    1.7	-3.72217027558834\\
    1.75	-3.73287898825314\\
    1.8	-3.74415088149343\\
    1.85	-3.75599337903529\\
    1.9	-3.76841314402527\\
    1.95	-3.78141607311938\\
    2	-3.79500729352051\\
    2.05	-3.80919116298053\\
    2.1	-3.82397127275443\\
    2.15	-3.8393504534655\\
    2.2	-3.85533078381304\\
    2.25	-3.87191360202824\\
    2.3	-3.88909951995964\\
    2.35	-3.90688843964739\\
    2.4	-3.92527957222631\\
    2.45	-3.9442714589805\\
    2.5	-3.96386199435843\\
    2.55	-3.98404845074623\\
    2.6	-4.00482750478888\\
    2.65	-4.02619526504359\\
    2.7	-4.04814730074787\\
    2.75	-4.07067867148454\\
    2.8	-4.09378395752964\\
    2.85	-4.11745729067414\\
    2.9	-4.14169238531793\\
    2.95	-4.16648256964389\\
    3	-4.19182081669086\\
    3.05	-4.21769977515622\\
    3.1	-4.24411179977208\\
    3.15	-4.2710489811132\\
    3.2	-4.29850317470858\\
    3.25	-4.32646602934382\\
    3.3	-4.35492901445552\\
    3.35	-4.38388344653348\\
    3.4	-4.41332051446052\\
    3.45	-4.44323130373325\\
    3.5	-4.47360681951958\\
    3.55	-4.50443800852139\\
    3.6	-4.53571577962154\\
    3.65	-4.56743102330508\\
    3.7	-4.59957462985363\\
    3.75	-4.63213750632089\\
    3.8	-4.66511059230433\\
    3.85	-4.69848487453516\\
    3.9	-4.73225140031419\\
    3.95	-4.76640128982643\\
    4	-4.80092574737107\\
    4.05	-4.83581607154715\\
    4.1	-4.87106366443767\\
    4.15	-4.9066600398371\\
    4.2	-4.94259683056856\\
    4.25	-4.97886579493782\\
    4.3	-5.01545882237184\\
    4.35	-5.05236793828941\\
    4.4	-5.0895853082511\\
    4.45	-5.12710324143526\\
    4.5	-5.16491419348545\\
    4.55	-5.20301076877394\\
    4.6	-5.24138572212416\\
    4.65	-5.28003196003382\\
    4.7	-5.31894254143834\\
    4.75	-5.35811067805309\\
    4.8	-5.3975297343305\\
    4.85	-5.43719322706679\\
    4.9	-5.47709482469102\\
    4.95	-5.51722834626726\\
    5	-5.557587760239\\
    };
    
    \addplot[area legend, draw=black, fill=black, forget plot]
    table[row sep=crcr] {%
    x	y\\
    -2.74134520650515	-2.97329858058037\\
    -2.74134520650515	-2.97329858058037\\
    -2.74134520650515	-2.97329858058037\\
    -2.74134520650515	-2.97329858058037\\
    -2.74134520650515	-2.97329858058037\\
    -2.74134520650515	-2.97329858058037\\
    -2.74134520650515	-2.97329858058037\\
    -2.80327552494466	-3.03522889901988\\
    -2.95732189354276	-2.75732189354276\\
    -2.67941488806564	-2.91136826214086\\
    -2.74134520650515	-2.97329858058037\\
    }--cycle;
    \addplot [color=black, forget plot]
      table[row sep=crcr]{%
    -5	4.51814892747109\\
    -4.95	4.45669362552509\\
    -4.9	4.39477686003393\\
    -4.85	4.33237018370575\\
    -4.8	4.26944263795511\\
    -4.75	4.20596045253427\\
    -4.7	4.14188669869105\\
    -4.65	4.07718088688333\\
    -4.6	4.01179849797794\\
    -4.55	3.94569043417403\\
    -4.5	3.87880237242552\\
    -4.45	3.81107399863058\\
    -4.4	3.74243809494084\\
    -4.35	3.67281944469957\\
    -4.3	3.60213350900703\\
    -4.25	3.53028481465819\\
    -4.2	3.45716497362754\\
    -4.15	3.38265022702969\\
    -4.1	3.30659836798173\\
    -4.05	3.2288448424795\\
    -4	3.14919774648174\\
    -3.95	3.0674313166234\\
    -3.9	2.983277327649\\
    -3.85	2.89641352118544\\
    -3.8	2.80644772607596\\
    -3.75	2.71289555841655\\
    -3.7	2.61514825740898\\
    -3.65	2.51242481396752\\
    -3.6	2.40369800276695\\
    -3.55	2.28757477335282\\
    -3.5	2.16209159538837\\
    -3.45	2.02433804508015\\
    -3.4	1.86969463209304\\
    -3.35	1.69006605486246\\
    -3.3	1.46883860941191\\
    -3.25	1.15980897567894\\
    -3.2	0\\
    };
    \addplot [color=black, forget plot]
      table[row sep=crcr]{%
    -3.2	-0\\
    -3.15	-1.14779033794501\\
    -3.1	-1.43855436194468\\
    -3.05	-1.63806819432795\\
    -3	-1.7933914044787\\
    -2.95	-1.92160248521068\\
    -2.9	-2.03109730607616\\
    -2.85	-2.12670912096187\\
    -2.8	-2.21150992547155\\
    -2.75	-2.28759069782819\\
    -2.7	-2.35644826269705\\
    -2.65	-2.41919645247475\\
    -2.6	-2.47668999972181\\
    -2.55	-2.52960145159073\\
    -2.5	-2.57847114115864\\
    -2.45	-2.62374089275194\\
    -2.4	-2.66577748131676\\
    -2.35	-2.70488940294455\\
    -2.3	-2.74133914329627\\
    -2.25	-2.77535233466951\\
    -2.2	-2.80712471264813\\
    -2.15	-2.83682748460024\\
    -2.1	-2.86461153103556\\
    -2.05	-2.89061073525953\\
    -2	-2.91494465244874\\
    -1.95	-2.93772067151641\\
    -1.9	-2.95903578284901\\
    -1.85	-2.97897803642651\\
    -1.8	-2.99762775427414\\
    -1.75	-3.01505854618548\\
    -1.7	-3.03133816656442\\
    -1.65	-3.04652924193967\\
    -1.6	-3.06068989243582\\
    -1.55	-3.07387426569862\\
    -1.5	-3.08613299808309\\
    -1.45	-3.09751361504498\\
    -1.4	-3.10806088042817\\
    -1.35	-3.11781710256464\\
    -1.3	-3.12682240369041\\
    -1.25	-3.13511495804882\\
    -1.2	-3.1427312031392\\
    -1.15	-3.14970602782766\\
    -1.1	-3.15607294043183\\
    -1.05	-3.16186421939409\\
    -1	-3.16711104874753\\
    -0.95	-3.17184364023773\\
    -0.899999999999999	-3.17609134367924\\
    -0.85	-3.17988274688643\\
    -0.799999999999999	-3.18324576631711\\
    -0.75	-3.18620772939668\\
    -0.7	-3.18879544934471\\
    -0.649999999999999	-3.19103529320198\\
    -0.6	-3.19295324364839\\
    -0.549999999999999	-3.19457495511033\\
    -0.5	-3.1959258045759\\
    -0.45	-3.19703093746689\\
    -0.399999999999999	-3.19791530885622\\
    -0.35	-3.1986037202666\\
    -0.299999999999999	-3.19912085224062\\
    -0.25	-3.19949129283252\\
    -0.2	-3.1997395621377\\
    -0.149999999999999	-3.19989013294668\\
    -0.0999999999999996	-3.19996744758552\\
    -0.0499999999999994	-3.19999593098441\\
    4.44089209850063e-16	-3.2\\
    0.0500000000000003	-3.20000406900524\\
    0.100000000000001	-3.2000325517522\\
    0.15	-3.20010985950961\\
    0.200000000000001	-3.20026039547678\\
    0.25	-3.20050854547952\\
    0.3	-3.20087866496163\\
    0.350000000000001	-3.20139506229741\\
    0.4	-3.20208197846633\\
    0.450000000000001	-3.20296356315068\\
    0.5	-3.20406384733972\\
    0.55	-3.20540671255072\\
    0.600000000000001	-3.20701585680648\\
    0.65	-3.20891475754098\\
    0.700000000000001	-3.21112663163925\\
    0.75	-3.21367439285311\\
    0.8	-3.21658060687125\\
    0.85	-3.21986744435871\\
    0.9	-3.22355663231713\\
    0.95	-3.22766940415125\\
    1	-3.23222644885926\\
    1.05	-3.23724785979289\\
    1.1	-3.24275308345714\\
    1.15	-3.24876086883799\\
    1.2	-3.25528921775908\\
    1.25	-3.2623553367739\\
    1.3	-3.269975591098\\
    1.35	-3.27816546107627\\
    1.4	-3.28693950166235\\
    1.45	-3.29631130536136\\
    1.5	-3.3062934690535\\
    1.55	-3.31689756507478\\
    1.6	-3.32813411688305\\
    1.65	-3.34001257958378\\
    1.7	-3.35254132553103\\
    1.75	-3.36572763515658\\
    1.8	-3.37957769311521\\
    1.85	-3.3940965897682\\
    1.9	-3.40928832796128\\
    1.95	-3.42515583498954\\
    2	-3.44170097958066\\
    2.05	-3.45892459367131\\
    2.1	-3.47682649869993\\
    2.15	-3.49540553609387\\
    2.2	-3.51465960159039\\
    2.25	-3.53458568299978\\
    2.3	-3.55517990099529\\
    2.35	-3.5764375524985\\
    2.4	-3.59835315622033\\
    2.45	-3.62092049991666\\
    2.5	-3.64413268892273\\
    2.55	-3.66798219554225\\
    2.6	-3.69246090888418\\
    2.65	-3.71756018476147\\
    2.7	-3.74327089529209\\
    2.75	-3.76958347787087\\
    2.8	-3.79648798321173\\
    2.85	-3.82397412219231\\
    2.9	-3.85203131126585\\
    2.95	-3.88064871623874\\
    3	-3.90981529424461\\
    3.05	-3.93951983377774\\
    3.1	-3.96975099267893\\
    3.15	-4.00049733399549\\
    3.2	-4.03174735966359\\
    3.25	-4.0634895419857\\
    3.3	-4.09571235289781\\
    3.35	-4.12840429104089\\
    3.4	-4.16155390666829\\
    3.45	-4.19514982443573\\
    3.5	-4.22918076413343\\
    3.55	-4.26363555943029\\
    3.6	-4.29850317470858\\
    3.65	-4.33377272007456\\
    3.7	-4.36943346463502\\
    3.75	-4.40547484813357\\
    3.8	-4.44188649104214\\
    3.85	-4.47865820320454\\
    3.9	-4.51577999112813\\
    3.95	-4.55324206401897\\
    4	-4.59103483865373\\
    4.05	-4.6291489431793\\
    4.1	-4.6675752199277\\
    4.15	-4.70630472733087\\
    4.2	-4.74532874101596\\
    4.25	-4.78463875415766\\
    4.3	-4.82422647716039\\
    4.35	-4.86408383673865\\
    4.4	-4.90420297445978\\
    4.45	-4.94457624480922\\
    4.5	-4.98519621283419\\
    4.55	-5.0260556514178\\
    4.6	-5.06714753823154\\
    4.65	-5.10846505241062\\
    4.7	-5.1500015709927\\
    4.75	-5.1917506651576\\
    4.8	-5.23370609630174\\
    4.85	-5.27586181197868\\
    4.9	-5.31821194173362\\
    4.95	-5.36075079285757\\
    5	-5.40347284608396\\
    };
    
    \addplot[area legend, draw=black, fill=black, forget plot]
    table[row sep=crcr] {%
    x	y\\
    -2.41892334053047	-2.66076002576777\\
    -2.41892334053047	-2.66076002576777\\
    -2.41892334053047	-2.66076002576777\\
    -2.41892334053047	-2.66076002576777\\
    -2.41892334053047	-2.66076002576777\\
    -2.41892334053047	-2.66076002576777\\
    -2.41892334053047	-2.66076002576777\\
    -2.48227065590585	-2.72410734114315\\
    -2.63984168314912	-2.43984168314912\\
    -2.35557602515509	-2.59741271039238\\
    -2.41892334053047	-2.66076002576777\\
    }--cycle;
    \addplot [color=black, forget plot]
      table[row sep=crcr]{%
    -5	4.68827619738926\\
    -4.95	4.63128291461227\\
    -4.9	4.5740345539613\\
    -4.85	4.51651748681601\\
    -4.8	4.45871707957417\\
    -4.75	4.40061759581542\\
    -4.7	4.34220208634965\\
    -4.65	4.28345226529961\\
    -4.6	4.22434837002808\\
    -4.55	4.164869002306\\
    -4.5	4.1049909476128\\
    -4.45	4.04468896883863\\
    -4.4	3.98393556988968\\
    -4.35	3.92270072374216\\
    -4.3	3.86095155829406\\
    -4.25	3.79865199185627\\
    -4.2	3.73576230821072\\
    -4.15	3.67223865871609\\
    -4.1	3.60803247578532\\
    -4.05	3.54308977795475\\
    -4	3.47735034137574\\
    -3.95	3.41074670541341\\
    -3.9	3.34320297045916\\
    -3.85	3.27463333307323\\
    -3.8	3.20494028573448\\
    -3.75	3.13401238363513\\
    -3.7	3.06172144585091\\
    -3.65	2.98791900777284\\
    -3.6	2.91243176788213\\
    -3.55	2.83505566177586\\
    -3.5	2.75554802817152\\
    -3.45	2.6736170683825\\
    -3.4	2.58890737691659\\
    -3.35	2.50097961609079\\
    -3.3	2.40928119148166\\
    -3.25	2.31310259436156\\
    -3.2	2.21150992547155\\
    -3.15	2.10323575242108\\
    -3.1	1.98649230986882\\
    -3.05	1.85862782751761\\
    -3	1.71543045234635\\
    -2.95	1.54951424251114\\
    -2.9	1.34571112600068\\
    -2.85	1.06180923959634\\
    -2.8	0\\
    };
    \addplot [color=black, forget plot]
      table[row sep=crcr]{%
    -2.8	-0\\
    -2.75	-1.04924359733213\\
    -2.7	-1.31404881408406\\
    -2.65	-1.49515100841119\\
    -2.6	-1.63565775765116\\
    -2.55	-1.75122363400783\\
    -2.5	-1.84954943982631\\
    -2.45	-1.93507118478761\\
    -2.4	-2.01061027812539\\
    -2.35	-2.07808849597626\\
    -2.3	-2.13888257964319\\
    -2.25	-2.19401783917641\\
    -2.2	-2.2442817761255\\
    -2.15	-2.29029464245921\\
    -2.1	-2.33255529615217\\
    -2.05	-2.37147213991386\\
    -2	-2.40738466192258\\
    -1.95	-2.44057884009659\\
    -1.9	-2.47129841528428\\
    -1.85	-2.49975330899155\\
    -1.8	-2.52612602132067\\
    -1.75	-2.55057657089264\\
    -1.7	-2.57324636310784\\
    -1.65	-2.59426125790634\\
    -1.6	-2.61373403083503\\
    -1.55	-2.63176636823053\\
    -1.5	-2.64845050035241\\
    -1.45	-2.66387055007701\\
    -1.4	-2.67810365588134\\
    -1.35	-2.69122091406451\\
    -1.3	-2.70328817496487\\
    -1.25	-2.71436672031028\\
    -1.2	-2.72451384307563\\
    -1.15	-2.7337833468196\\
    -1.1	-2.74222597807767\\
    -1.05	-2.7498898027468\\
    -1	-2.75682053532422\\
    -0.95	-2.76306182822313\\
    -0.9	-2.7686555270807\\
    -0.85	-2.77364189692434\\
    -0.8	-2.77805982321302\\
    -0.75	-2.78194699107888\\
    -0.7	-2.78534004552747\\
    -0.65	-2.78827473488684\\
    -0.6	-2.79078603940697\\
    -0.55	-2.79290828658616\\
    -0.5	-2.7946752545277\\
    -0.45	-2.79612026439937\\
    -0.399999999999999	-2.79727626287235\\
    -0.35	-2.79817589524919\\
    -0.3	-2.79885156984793\\
    -0.25	-2.79933551408778\\
    -0.2	-2.79965982261845\\
    -0.149999999999999	-2.79985649774756\\
    -0.0999999999999996	-2.79995748234757\\
    -0.0499999999999998	-2.79999468536406\\
    4.44089209850063e-16	-2.8\\
    0.0500000000000003	-2.80000531461576\\
    0.100000000000001	-2.80004251636121\\
    0.15	-2.80014348754473\\
    0.2	-2.80034009474402\\
    0.25	-2.80066417067502\\
    0.3	-2.80114748885848\\
    0.350000000000001	-2.80182173115804\\
    0.4	-2.80271844830938\\
    0.45	-2.80386901361591\\
    0.5	-2.80530457005292\\
    0.55	-2.80705597109652\\
    0.600000000000001	-2.80915371567669\\
    0.65	-2.81162787774153\\
    0.7	-2.81450803101234\\
    0.75	-2.81782316960242\\
    0.8	-2.82160162526491\\
    0.850000000000001	-2.82587098212319\\
    0.9	-2.83065798981818\\
    0.95	-2.83598847607753\\
    1	-2.84188725976877\\
    1.05	-2.8483780655392\\
    1.1	-2.85548344116701\\
    1.15	-2.86322467874888\\
    1.2	-2.87162174082733\\
    1.25	-2.88069319251583\\
    1.3	-2.89045614061086\\
    1.35	-2.90092618058876\\
    1.4	-2.91211735227267\\
    1.45	-2.9240421048243\\
    1.5	-2.93671127156856\\
    1.55	-2.95013405500169\\
    1.6	-2.96431802216825\\
    1.65	-2.97926911042469\\
    1.7	-2.99499164344107\\
    1.75	-3.01148835713308\\
    1.8	-3.02876043506725\\
    1.85	-3.04680755274759\\
    1.9	-3.06562793007403\\
    1.95	-3.08521839116532\\
    2	-3.10557443066252\\
    2.05	-3.12669028557426\\
    2.1	-3.14855901169279\\
    2.15	-3.17117256359859\\
    2.2	-3.19452187728006\\
    2.25	-3.21859695442227\\
    2.3	-3.24338694746137\\
    2.35	-3.26888024455833\\
    2.4	-3.29506455371237\\
    2.45	-3.32192698531026\\
    2.5	-3.34945413248832\\
    2.55	-3.37763214876757\\
    2.6	-3.40644682250726\\
    2.65	-3.43588364780516\\
    2.7	-3.46592789155308\\
    2.75	-3.49656465643253\\
    2.8	-3.52777893970564\\
    2.85	-3.55955568772126\\
    2.9	-3.59187984611387\\
    2.95	-3.62473640572403\\
    3	-3.65811044431324\\
    3.05	-3.69198716418307\\
    3.1	-3.7263519258398\\
    3.15	-3.76119027787001\\
    3.2	-3.79648798321173\\
    3.25	-3.83223104202003\\
    3.3	-3.86840571133459\\
    3.35	-3.90499852176285\\
    3.4	-3.94199629139342\\
    3.45	-3.97938613715373\\
    3.5	-4.01715548382202\\
    3.55	-4.05529207089804\\
    3.6	-4.09378395752964\\
    3.65	-4.13261952568365\\
    3.7	-4.17178748174028\\
    3.75	-4.21127685667977\\
    3.8	-4.25107700501994\\
    3.85	-4.29117760265231\\
    3.9	-4.33156864371396\\
    3.95	-4.37224043662185\\
    4	-4.41318359938584\\
    4.05	-4.45438905430713\\
    4.1	-4.49584802215894\\
    4.15	-4.53755201593769\\
    4.2	-4.57949283426403\\
    4.25	-4.62166255450535\\
    4.3	-4.66405352568401\\
    4.35	-4.7066583612282\\
    4.4	-4.74946993161649\\
    4.45	-4.79248135696076\\
    4.5	-4.83568599956711\\
    4.55	-4.87907745650921\\
    4.6	-4.92264955224409\\
    4.65	-4.96639633129637\\
    4.7	-5.01031205103314\\
    4.75	-5.05439117454833\\
    4.8	-5.09862836367258\\
    4.85	-5.14301847212166\\
    4.9	-5.18755653879443\\
    4.95	-5.23223778122896\\
    5	-5.27705758922345\\
    };
    
    \addplot[area legend, draw=black, fill=black, forget plot]
    table[row sep=crcr] {%
    x	y\\
    -2.09792872329418	-2.34679422221677\\
    -2.09792872329418	-2.34679422221677\\
    -2.09792872329418	-2.34679422221677\\
    -2.09792872329418	-2.34679422221677\\
    -2.09792872329418	-2.34679422221677\\
    -2.09792872329418	-2.34679422221677\\
    -2.09792872329418	-2.34679422221677\\
    -2.16228377861537	-2.41114927753796\\
    -2.32236147275548	-2.12236147275548\\
    -2.03357366797299	-2.28243916689559\\
    -2.09792872329418	-2.34679422221677\\
    }--cycle;
    \addplot [color=black, forget plot]
      table[row sep=crcr]{%
    -5	4.80843424529093\\
    -4.95	4.75430265724247\\
    -4.9	4.70003017937465\\
    -4.85	4.64560999326295\\
    -4.8	4.59103483865373\\
    -4.75	4.53629697659133\\
    -4.7	4.48138814870033\\
    -4.65	4.42629953213446\\
    -4.6	4.37102168962962\\
    -4.55	4.3155445140118\\
    -4.5	4.25985716640718\\
    -4.45	4.20394800728021\\
    -4.4	4.1478045192799\\
    -4.35	4.09141322070095\\
    -4.3	4.0347595681581\\
    -4.25	3.97782784682101\\
    -4.2	3.92060104625254\\
    -4.15	3.86306071952359\\
    -4.1	3.80518682282467\\
    -4.05	3.74695753223821\\
    -4	3.68834903364676\\
    -3.95	3.62933528089639\\
    -3.9	3.56988771626229\\
    -3.85	3.50997494591197\\
    -3.8	3.44956236134631\\
    -3.75	3.38861169560332\\
    -3.7	3.32708050017967\\
    -3.65	3.26492152494216\\
    -3.6	3.20208197846633\\
    -3.55	3.13850263982626\\
    -3.5	3.07411678426132\\
    -3.45	3.00884887348345\\
    -3.4	2.94261294536828\\
    -3.35	2.87531061546316\\
    -3.3	2.80682857120563\\
    -3.25	2.73703539441857\\
    -3.2	2.66577748131676\\
    -3.15	2.59287373022066\\
    -3.1	2.51810851596043\\
    -3.05	2.44122223322373\\
    -3	2.3618983098613\\
    -2.95	2.27974495710834\\
    -2.9	2.19426882937724\\
    -2.85	2.10483579455697\\
    -2.8	2.01061027812539\\
    -2.75	1.9104571148998\\
    -2.7	1.80277350207521\\
    -2.65	1.68517971004287\\
    -2.6	1.5538924023345\\
    -2.55	1.40227097406978\\
    -2.5	1.21666562415206\\
    -2.45	0.959054697368768\\
    -2.4	0\\
    };
    \addplot [color=black, forget plot]
      table[row sep=crcr]{%
    -2.4	-0\\
    -2.35	-0.94582656794676\\
    -2.3	-1.18333443540641\\
    -2.25	-1.34504355335903\\
    -2.2	-1.4699193193105\\
    -2.15	-1.57212459328393\\
    -2.1	-1.65863244410366\\
    -2.05	-1.73346327510399\\
    -2	-1.79917657811017\\
    -1.95	-1.85751749979135\\
    -1.9	-1.90973763562705\\
    -1.85	-1.95677024308203\\
    -1.8	-1.99933311098757\\
    -1.75	-2.03799246630451\\
    -1.7	-2.07320452416294\\
    -1.65	-2.1053435344861\\
    -1.6	-2.13472131897756\\
    -1.55	-2.16160125036841\\
    -1.5	-2.18620848933655\\
    -1.45	-2.20873763413653\\
    -1.4	-2.22935853978708\\
    -1.35	-2.24822081570561\\
    -1.3	-2.26545735184385\\
    -1.25	-2.28118711905239\\
    -1.2	-2.29551741932687\\
    -1.15	-2.30854571356602\\
    -1.1	-2.32036112095662\\
    -1.05	-2.33104566032113\\
    -0.999999999999999	-2.3406752866345\\
    -0.949999999999999	-2.34932076340479\\
    -0.899999999999999	-2.3570484023544\\
    -0.849999999999999	-2.36392069490592\\
    -0.799999999999999	-2.3699968547259\\
    -0.749999999999999	-2.37533328656065\\
    -0.699999999999999	-2.37998399348979\\
    -0.649999999999999	-2.38400093229623\\
    -0.599999999999999	-2.38743432473783\\
    -0.549999999999999	-2.39033293098385\\
    -0.499999999999999	-2.39274429025737\\
    -0.449999999999999	-2.39471493273629\\
    -0.399999999999999	-2.39629056595804\\
    -0.349999999999999	-2.39751623831037\\
    -0.299999999999999	-2.39843648164217\\
    -0.25	-2.39909543457304\\
    -0.199999999999999	-2.39953694770219\\
    -0.149999999999999	-2.39980467160327\\
    -0.0999999999999992	-2.39994212823416\\
    -0.0499999999999994	-2.3999927661819\\
    8.88178419700125e-16	-2.4\\
    0.0500000000000007	-2.40000723377449\\
    0.100000000000001	-2.40005786897502\\
    0.150000000000001	-2.40019529660758\\
    0.200000000000001	-2.40046287368554\\
    0.250000000000001	-2.40090388407496\\
    0.300000000000001	-2.40156148384975\\
    0.350000000000001	-2.40247863140937\\
    0.400000000000001	-2.40369800276695\\
    0.450000000000001	-2.40526189260486\\
    0.500000000000001	-2.40721210191468\\
    0.550000000000001	-2.40958981328368\\
    0.600000000000001	-2.41243545515343\\
    0.650000000000001	-2.41578855664932\\
    0.700000000000001	-2.41968759485215\\
    0.750000000000001	-2.42416983664444\\
    0.800000000000001	-2.42927117750053\\
    0.850000000000001	-2.43502597979047\\
    0.900000000000001	-2.44146691331931\\
    0.950000000000001	-2.44862480091532\\
    1	-2.45652847190347\\
    1.05	-2.46520462624676\\
    1.1	-2.47467771200501\\
    1.15	-2.48496981854691\\
    1.2	-2.49610058766228\\
    1.25	-2.50808714436297\\
    1.3	-2.52094404874626\\
    1.35	-2.53468326983641\\
    1.4	-2.54931418183629\\
    1.45	-2.56484358272872\\
    1.5	-2.58127573468549\\
    1.55	-2.59861242528775\\
    1.6	-2.61685304815087\\
    1.65	-2.63599470119279\\
    1.7	-2.65603230049749\\
    1.75	-2.67695870751057\\
    1.8	-2.69876486716525\\
    1.85	-2.72143995447251\\
    1.9	-2.74497152711609\\
    1.95	-2.76934568166313\\
    2	-2.79454721112724\\
    2.05	-2.82055976179205\\
    2.1	-2.8473659874088\\
    2.15	-2.87494769911158\\
    2.2	-2.90328600963702\\
    2.25	-2.93236147068346\\
    2.3	-2.96215420248884\\
    2.35	-2.99264401494038\\
    2.4	-3.02381051974769\\
    2.45	-3.05563323340941\\
    2.5	-3.08809167088059\\
    2.55	-3.12116543000122\\
    2.6	-3.15483426687613\\
    2.65	-3.18907816250322\\
    2.7	-3.22387738103143\\
    2.75	-3.25921252009453\\
    2.8	-3.29506455371237\\
    2.85	-3.3314148682816\\
    2.9	-3.36824529219294\\
    2.95	-3.40553811961646\\
    3	-3.4432761289903\\
    3.05	-3.48144259673413\\
    3.1	-3.52002130668921\\
    3.15	-3.55899655576197\\
    3.2	-3.59835315622033\\
    3.25	-3.63807643506226\\
    3.3	-3.67815223084484\\
    3.35	-3.71856688833088\\
    3.4	-3.75930725127913\\
    3.45	-3.80036065367366\\
    3.5	-3.84171490965925\\
    3.55	-3.88335830242172\\
    3.6	-3.92527957222631\\
    3.65	-3.9674679038028\\
    3.7	-4.00991291324399\\
    3.75	-4.05260463456297\\
    3.8	-4.09553350603637\\
    3.85	-4.1386903564432\\
    3.9	-4.18206639129361\\
    3.95	-4.22565317912812\\
    4	-4.26944263795511\\
    4.05	-4.31342702188364\\
    4.1	-4.35759890799854\\
    4.15	-4.40195118351628\\
    4.2	-4.44647703325237\\
    4.25	-4.49116992742435\\
    4.3	-4.53602360980881\\
    4.35	-4.58103208626559\\
    4.4	-4.62618961363815\\
    4.45	-4.6714906890354\\
    4.5	-4.7169300394969\\
    4.55	-4.76250261204083\\
    4.6	-4.80820356409192\\
    4.65	-4.85402825428448\\
    4.7	-4.89997223363413\\
    4.75	-4.94603123707086\\
    4.8	-4.99220117532457\\
    4.85	-5.03847812715392\\
    4.9	-5.08485833190836\\
    4.95	-5.13133818241302\\
    5	-5.17791421816565\\
    };
    
    \addplot[area legend, draw=black, fill=black, forget plot]
    table[row sep=crcr] {%
    x	y\\
    -1.77796634607363	-2.03179617865005\\
    -1.77796634607363	-2.03179617865005\\
    -1.77796634607363	-2.03179617865005\\
    -1.77796634607363	-2.03179617865005\\
    -1.77796634607363	-2.03179617865005\\
    -1.77796634607363	-2.03179617865005\\
    -1.77796634607363	-2.03179617865005\\
    -1.84303315127912	-2.09686298385554\\
    -2.00488126236184	-1.80488126236184\\
    -1.71289954086813	-1.96672937344456\\
    -1.77796634607363	-2.03179617865005\\
    }--cycle;
    \addplot [color=black, forget plot]
      table[row sep=crcr]{%
    -5	4.89097324650875\\
    -4.95	4.83868300516958\\
    -4.9	4.78631809839936\\
    -4.85	4.73387514737148\\
    -4.8	4.68135057326899\\
    -4.75	4.62874058236986\\
    -4.7	4.5760411497677\\
    -4.65	4.52324800157807\\
    -4.6	4.47035659546091\\
    -4.55	4.4173620992673\\
    -4.5	4.36425936759302\\
    -4.45	4.31104291599141\\
    -4.4	4.2577068925634\\
    -4.35	4.20424504660246\\
    -4.3	4.15065069392485\\
    -4.25	4.09691667846105\\
    -4.2	4.04303532961917\\
    -4.15	3.98899841485557\\
    -4.1	3.93479708679752\\
    -4.05	3.88042182415671\\
    -4	3.82586236554478\\
    -3.95	3.77110763515079\\
    -3.9	3.71614565905798\\
    -3.85	3.66096347075747\\
    -3.8	3.60554700415042\\
    -3.75	3.54988097200598\\
    -3.7	3.49394872744598\\
    -3.65	3.43773210553963\\
    -3.6	3.38121124148806\\
    -3.55	3.32436436112808\\
    -3.5	3.26716753854379\\
    -3.45	3.20959441438957\\
    -3.4	3.15161586702219\\
    -3.35	3.09319962661313\\
    -3.3	3.03430981992715\\
    -3.25	2.97490643021857\\
    -3.2	2.91494465244874\\
    -3.15	2.85437411839193\\
    -3.1	2.79313795863858\\
    -3.05	2.73117165824929\\
    -3	2.66840164872194\\
    -2.95	2.60474355930451\\
    -2.9	2.54010002292785\\
    -2.85	2.47435789212902\\
    -2.8	2.40738466192258\\
    -2.75	2.33902380933802\\
    -2.7	2.26908862610406\\
    -2.65	2.1973539123126\\
    -2.6	2.12354456278193\\
    -2.55	2.04731951866994\\
    -2.5	1.96824859155109\\
    -2.45	1.88577792730902\\
    -2.4	1.79917657811017\\
    -2.35	1.70745000271621\\
    -2.3	1.6091918838378\\
    -2.25	1.50231125172934\\
    -2.2	1.38347928332185\\
    -2.15	1.24684537872232\\
    -2.1	1.08036795875479\\
    -2.05	0.850461110824827\\
    -2	0\\
    };
    \addplot [color=black, forget plot]
      table[row sep=crcr]{%
    -2	-0\\
    -1.95	-0.836404225202668\\
    -1.9	-1.04494928899312\\
    -1.85	-1.18603605379583\\
    -1.8	-1.29425472539207\\
    -1.75	-1.38219370341972\\
    -1.7	-1.45605867613633\\
    -1.65	-1.51943235382759\\
    -1.6	-1.57459887324087\\
    -1.55	-1.62311813709513\\
    -1.5	-1.66611092582298\\
    -1.45	-1.70441470781648\\
    -1.4	-1.73867517068787\\
    -1.35	-1.76940312047886\\
    -1.3	-1.79701150191753\\
    -1.25	-1.82184040778046\\
    -1.2	-1.84417451682338\\
    -1.15	-1.86425558533691\\
    -1.1	-1.88229160722335\\
    -1.05	-1.89846367034774\\
    -1	-1.91293118277239\\
    -0.95	-1.92583592187964\\
    -0.899999999999999	-1.93730521801843\\
    -0.849999999999999	-1.94745449140934\\
    -0.799999999999999	-1.9563892986035\\
    -0.75	-1.964207001962\\
    -0.7	-1.97099814569671\\
    -0.649999999999999	-1.97684760074296\\
    -0.599999999999999	-1.98183552537535\\
    -0.549999999999999	-1.98603817722008\\
    -0.5	-1.9895286039482\\
    -0.45	-1.99237723362748\\
    -0.399999999999999	-1.99465238089546\\
    -0.349999999999999	-1.99642068139387\\
    -0.299999999999999	-1.9977474639932\\
    -0.25	-1.99869706803514\\
    -0.2	-1.99933311098757\\
    -0.149999999999999	-1.99971871043995\\
    -0.0999999999999994	-1.9999166631942\\
    -0.0499999999999994	-1.99998958327908\\
    4.44089209850063e-16	-2\\
    0.0500000000000007	-2.00001041661241\\
    0.100000000000001	-2.00008332986135\\
    0.15	-2.00028121045849\\
    0.200000000000001	-2.00066644456782\\
    0.25	-2.00130123654146\\
    0.300000000000001	-2.00224747348544\\
    0.350000000000001	-2.00356655273682\\
    0.400000000000001	-2.00531917398583\\
    0.450000000000001	-2.0075650985622\\
    0.5	-2.01036287929453\\
    0.550000000000001	-2.01376956530874\\
    0.600000000000001	-2.01784038710825\\
    0.650000000000001	-2.02262842821894\\
    0.700000000000001	-2.02818429052461\\
    0.75	-2.03455576109942\\
    0.800000000000001	-2.04178748880059\\
    0.850000000000001	-2.04992067906451\\
    0.900000000000001	-2.05899281521183\\
    0.950000000000001	-2.06903741408839\\
    1	-2.0800838230519\\
    1.05	-2.09215706417879\\
    1.1	-2.10527773016229\\
    1.15	-2.11946193476664\\
    1.2	-2.13472131897756\\
    1.25	-2.15106311223791\\
    1.3	-2.16849024647197\\
    1.35	-2.18700151906935\\
    1.4	-2.20659179969292\\
    1.45	-2.22725227474814\\
    1.5	-2.24897072263771\\
    1.55	-2.27173181253447\\
    1.6	-2.29551741932687\\
    1.65	-2.32030694759599\\
    1.7	-2.34607765792876\\
    1.75	-2.37280498950733\\
    1.8	-2.40046287368554\\
    1.85	-2.42902403411495\\
    1.9	-2.4584602698667\\
    1.95	-2.48874271886697\\
    2	-2.51984209978975\\
    2.05	-2.55172893130465\\
    2.1	-2.58437372824212\\
    2.15	-2.61774717480497\\
    2.2	-2.65182027542034\\
    2.25	-2.68656448419286\\
    2.3	-2.72195181419418\\
    2.35	-2.75795492801531\\
    2.4	-2.79454721112724\\
    2.45	-2.83170282965258\\
    2.5	-2.86939677415858\\
    2.55	-2.90760489104936\\
    2.6	-2.94630390307269\\
    2.65	-2.98547142037234\\
    2.7	-3.02508594341803\\
    2.75	-3.06512685903736\\
    2.8	-3.10557443066252\\
    2.85	-3.14640978379311\\
    2.9	-3.18761488756736\\
    2.95	-3.22917253323018\\
    3	-3.27106631018859\\
    3.05	-3.31328058025453\\
    3.1	-3.35580045059213\\
    3.15	-3.39861174581109\\
    3.2	-3.44170097958066\\
    3.25	-3.48505532607801\\
    3.3	-3.52866259153212\\
    3.35	-3.5725111860773\\
    3.4	-3.61659009608984\\
    3.45	-3.66088885714617\\
    3.5	-3.70539752771031\\
    3.55	-3.75010666363274\\
    3.6	-3.79500729352051\\
    3.65	-3.84009089502003\\
    3.7	-3.88534937203808\\
    3.75	-3.93077503291408\\
    3.8	-3.97636056954548\\
    3.85	-4.02209903745978\\
    3.9	-4.06798383681954\\
    3.95	-4.11400869434114\\
    4	-4.16016764610381\\
    4.05	-4.20645502122203\\
    4.1	-4.25286542635209\\
    4.15	-4.29939373100183\\
    4.2	-4.34603505361155\\
    4.25	-4.39278474837337\\
    4.3	-4.43963839275638\\
    4.35	-4.48659177570492\\
    4.4	-4.53364088647765\\
    4.45	-4.58078190409608\\
    4.5	-4.62801118737159\\
    4.55	-4.67532526548129\\
    4.6	-4.72272082906374\\
    4.65	-4.77019472180698\\
    4.7	-4.81774393250215\\
    4.75	-4.86536558753726\\
    4.8	-4.91305694380694\\
    4.85	-4.96081538201475\\
    4.9	-5.00863840034637\\
    4.95	-5.05652360849244\\
    5	-5.10446872200146\\
    };
    
    \addplot[area legend, draw=black, fill=black, forget plot]
    table[row sep=crcr] {%
    x	y\\
    -1.45878108492119	-1.7160210190152\\
    -1.45878108492119	-1.7160210190152\\
    -1.45878108492119	-1.7160210190152\\
    -1.45878108492119	-1.7160210190152\\
    -1.45878108492119	-1.7160210190152\\
    -1.45878108492119	-1.7160210190152\\
    -1.45878108492119	-1.7160210190152\\
    -1.52433680556425	-1.78157673965827\\
    -1.6874010519682	-1.4874010519682\\
    -1.39322536427813	-1.65046529837215\\
    -1.45878108492119	-1.7160210190152\\
    }--cycle;
    \addplot [color=black, forget plot]
      table[row sep=crcr]{%
    -5	4.94477904098059\\
    -4.95	4.89363848904296\\
    -4.9	4.84246180797421\\
    -4.85	4.79124743234037\\
    -4.8	4.73999370945179\\
    -4.75	4.68869889334021\\
    -4.7	4.63736113823424\\
    -4.65	4.5859784914838\\
    -4.6	4.53454888587835\\
    -4.55	4.48307013129735\\
    -4.5	4.4315399056241\\
    -4.45	4.3799557448459\\
    -4.4	4.32831503225368\\
    -4.35	4.27661498664397\\
    -4.3	4.22485264941332\\
    -4.25	4.17302487042138\\
    -4.2	4.12112829248242\\
    -4.15	4.06915933432656\\
    -4.1	4.01711417185019\\
    -4.05	3.96498871745015\\
    -4	3.912778597207\\
    -3.95	3.86047912564922\\
    -3.9	3.80808527779036\\
    -3.85	3.75559165808512\\
    -3.8	3.70299246589589\\
    -3.75	3.65028145699731\\
    -3.7	3.5974519005707\\
    -3.65	3.54449653105004\\
    -3.6	3.49140749407442\\
    -3.55	3.4381762856732\\
    -3.5	3.38479368365655\\
    -3.45	3.33124966999775\\
    -3.4	3.27753334276884\\
    -3.35	3.22363281591635\\
    -3.3	3.16953510482831\\
    -3.25	3.11522599523022\\
    -3.2	3.06068989243582\\
    -3.15	3.00590964734172\\
    -3.1	2.95086635475631\\
    -3.05	2.89553911864629\\
    -3	2.83990477760478\\
    -2.95	2.78393758220642\\
    -2.9	2.72760881380063\\
    -2.85	2.67088633154152\\
    -2.8	2.61373403083503\\
    -2.75	2.55611119158266\\
    -2.7	2.49797168815881\\
    -2.65	2.43926302431389\\
    -2.6	2.37992514417486\\
    -2.55	2.31988895376138\\
    -2.5	2.25907446373555\\
    -2.45	2.19738843001697\\
    -2.4	2.13472131897756\\
    -2.35	2.07094334934552\\
    -2.3	2.00589924898897\\
    -2.25	1.93940118621276\\
    -2.2	1.87121904747042\\
    -2.15	1.80106675450351\\
    -2.1	1.72858248681377\\
    -2.05	1.65329918399086\\
    -2	1.57459887324087\\
    -1.95	1.4916386638245\\
    -1.9	1.40322386303798\\
    -1.85	1.30757412870449\\
    -1.8	1.20184900138347\\
    -1.75	1.08104579769418\\
    -1.7	0.93484731604652\\
    -1.65	0.734419304705956\\
    -1.6	0\\
    };
    \addplot [color=black, forget plot]
      table[row sep=crcr]{%
    -1.6	-0\\
    -1.55	-0.719277180972339\\
    -1.5	-0.896695702239352\\
    -1.45	-1.01554865303808\\
    -1.4	-1.10575496273577\\
    -1.35	-1.17822413134853\\
    -1.3	-1.2383449998609\\
    -1.25	-1.28923557057932\\
    -1.2	-1.33288874065838\\
    -1.15	-1.37066957164814\\
    -1.1	-1.40356235632406\\
    -1.05	-1.43230576551778\\
    -1	-1.45747232622437\\
    -0.95	-1.47951789142451\\
    -0.9	-1.49881387713707\\
    -0.85	-1.51566908328221\\
    -0.8	-1.53034494621791\\
    -0.75	-1.54306649904155\\
    -0.7	-1.55403044021409\\
    -0.65	-1.5634112018452\\
    -0.6	-1.5713656015696\\
    -0.55	-1.57803647021591\\
    -0.5	-1.58355552437377\\
    -0.45	-1.58804567183962\\
    -0.399999999999999	-1.59162288315856\\
    -0.35	-1.59439772467235\\
    -0.3	-1.59647662182419\\
    -0.25	-1.59796290228795\\
    -0.2	-1.59895765442811\\
    -0.149999999999999	-1.59956042612031\\
    -0.0999999999999996	-1.59986978106885\\
    -0.0499999999999996	-1.59998372379276\\
    4.44089209850063e-16	-1.6\\
    0.0500000000000005	-1.6000162758761\\
    0.100000000000001	-1.60013019773839\\
    0.15	-1.60043933248082\\
    0.2	-1.60104098923317\\
    0.25	-1.60203192366986\\
    0.3	-1.60350792840324\\
    0.350000000000001	-1.60556331581963\\
    0.4	-1.60829030343562\\
    0.450000000000001	-1.61177831615857\\
    0.5	-1.61611322442963\\
    0.55	-1.62137654172857\\
    0.600000000000001	-1.62764460887954\\
    0.65	-1.634987795549\\
    0.700000000000001	-1.64346975083118\\
    0.75	-1.65314673452675\\
    0.800000000000001	-1.66406705844152\\
    0.850000000000001	-1.67627066276551\\
    0.9	-1.6897888465576\\
    0.950000000000001	-1.70464416398064\\
    1	-1.72085048979033\\
    1.05	-1.73841324934997\\
    1.1	-1.7573298007952\\
    1.15	-1.77758995049765\\
    1.2	-1.79917657811017\\
    1.25	-1.82206634446136\\
    1.3	-1.84623045444209\\
    1.35	-1.87163544764605\\
    1.4	-1.89824399160587\\
    1.45	-1.92601565563293\\
    1.5	-1.95490764712231\\
    1.55	-1.98487549633947\\
    1.6	-2.0158736798318\\
    1.65	-2.0478561764489\\
    1.7	-2.08077695333414\\
    1.75	-2.11459038206672\\
    1.8	-2.14925158735429\\
    1.85	-2.18471673231751\\
    1.9	-2.22094324552107\\
    1.95	-2.25788999556407\\
    2	-2.29551741932687\\
    2.05	-2.33378760996385\\
    2.1	-2.37266437050798\\
    2.15	-2.41211323858019\\
    2.2	-2.45210148722989\\
    2.25	-2.4925981064171\\
    2.3	-2.53357376911577\\
    2.35	-2.57500078549635\\
    2.4	-2.61685304815087\\
    2.45	-2.65910597086681\\
    2.5	-2.70173642304198\\
    2.55	-2.74472266146485\\
    2.6	-2.78804426086241\\
    2.65	-2.83168204433867\\
    2.7	-2.87561801458909\\
    2.75	-2.91983528657429\\
    2.8	-2.96431802216825\\
    2.85	-3.00905136715629\\
    2.9	-3.05402139084372\\
    2.95	-3.09921502844282\\
    3	-3.14462002633127\\
    3.05	-3.19022489021596\\
    3.1	-3.23601883618965\\
    3.15	-3.28199174463239\\
    3.2	-3.32813411688305\\
    3.25	-3.37443703458653\\
    3.3	-3.42089212160868\\
    3.35	-3.46749150840182\\
    3.4	-3.5142277986987\\
    3.45	-3.56109403841015\\
    3.5	-3.60808368660189\\
    3.55	-3.65519058842745\\
    3.6	-3.70240894989728\\
    3.65	-3.74973331436812\\
    3.7	-3.7971585406413\\
    3.75	-3.84467978256357\\
    3.8	-3.89229247002981\\
    3.85	-3.93999229129192\\
    3.9	-3.98777517648406\\
    3.95	-4.03563728227971\\
    4	-4.08357497760117\\
    4.05	-4.1315848303075\\
    4.1	-4.17966359479151\\
    4.15	-4.2278082004213\\
    4.2	-4.27601574076625\\
    4.25	-4.32428346355148\\
    4.3	-4.37260876128874\\
    4.35	-4.42098916253565\\
    4.4	-4.46942232373823\\
    4.45	-4.51790602161539\\
    4.5	-4.56643814604669\\
    4.55	-4.61501669342777\\
    4.6	-4.66363976046024\\
    4.65	-4.71230553834548\\
    4.7	-4.76101230735385\\
    4.75	-4.80975843174301\\
    4.8	-4.85854235500106\\
    4.85	-4.90736259539177\\
    4.9	-4.9562177417811\\
    4.95	-5.00510644972551\\
    5	-5.05402743780422\\
    };
    
    \addplot[area legend, draw=black, fill=black, forget plot]
    table[row sep=crcr] {%
    x	y\\
    -1.14018938045399	-1.39965230269513\\
    -1.14018938045399	-1.39965230269513\\
    -1.14018938045399	-1.39965230269513\\
    -1.14018938045399	-1.39965230269513\\
    -1.14018938045399	-1.39965230269513\\
    -1.14018938045399	-1.39965230269513\\
    -1.14018938045399	-1.39965230269513\\
    -1.20606381689394	-1.46552673913508\\
    -1.36992084157456	-1.16992084157456\\
    -1.07431494401403	-1.33377786625518\\
    -1.14018938045399	-1.39965230269513\\
    }--cycle;
    \addplot [color=black, forget plot]
      table[row sep=crcr]{%
    -5	4.97685300871582\\
    -4.95	4.92637966277395\\
    -4.9	4.87589157482422\\
    -4.85	4.82538812237561\\
    -4.8	4.77486864947567\\
    -4.75	4.72433246450907\\
    -4.7	4.67377883782354\\
    -4.65	4.62320699916736\\
    -4.6	4.572616134921\\
    -4.55	4.52200538510347\\
    -4.5	4.471373840132\\
    -4.45	4.42072053731118\\
    -4.4	4.37004445702512\\
    -4.35	4.31934451860321\\
    -4.3	4.26861957582661\\
    -4.25	4.21786841203885\\
    -4.2	4.16708973481953\\
    -4.15	4.11628217017533\\
    -4.1	4.06544425619686\\
    -4.05	4.0145744361234\\
    -4	3.96367105075071\\
    -3.95	3.91273233010832\\
    -3.9	3.86175638432359\\
    -3.85	3.81074119357857\\
    -3.8	3.75968459705344\\
    -3.75	3.70858428073544\\
    -3.7	3.65743776395566\\
    -3.65	3.60624238449633\\
    -3.6	3.55499528208884\\
    -3.55	3.50369338009591\\
    -3.5	3.45233336514064\\
    -3.45	3.40091166440876\\
    -3.4	3.3494244203076\\
    -3.35	3.29786746211483\\
    -3.3	3.24623627419026\\
    -3.25	3.19452596025271\\
    -3.2	3.1427312031392\\
    -3.15	3.09084621936181\\
    -3.1	3.03886470765518\\
    -3.05	2.9867797905596\\
    -3	2.93458394790525\\
    -2.95	2.88226894084361\\
    -2.9	2.829825724804\\
    -2.85	2.77724434942192\\
    -2.8	2.72451384307563\\
    -2.75	2.67162207915641\\
    -2.7	2.61855562055581\\
    -2.65	2.56529953804358\\
    -2.6	2.51183719717883\\
    -2.55	2.45815000707663\\
    -2.5	2.40421712264547\\
    -2.45	2.35001508968732\\
    -2.4	2.29551741932687\\
    -2.35	2.24069407435017\\
    -2.3	2.18551084481481\\
    -2.25	2.12992858320359\\
    -2.2	2.07390225963995\\
    -2.15	2.01737978407905\\
    -2.1	1.96030052312627\\
    -2.05	1.90259341141233\\
    -2	1.84417451682338\\
    -1.95	1.78494385813114\\
    -1.9	1.72478118067315\\
    -1.85	1.66354025008984\\
    -1.8	1.60104098923317\\
    -1.75	1.53705839213066\\
    -1.7	1.47130647268414\\
    -1.65	1.40341428560281\\
    -1.6	1.33288874065838\\
    -1.55	1.25905425798022\\
    -1.5	1.18094915493065\\
    -1.45	1.09713441468862\\
    -1.4	1.0053051390627\\
    -1.35	0.901386751037605\\
    -1.3	0.776946201167251\\
    -1.25	0.608332812076032\\
    -1.2	0\\
    };
    \addplot [color=black, forget plot]
      table[row sep=crcr]{%
    -1.2	-0\\
    -1.15	-0.591667217703205\\
    -1.1	-0.734959659655251\\
    -1.05	-0.829316222051831\\
    -1	-0.899588289055083\\
    -0.95	-0.954868817813526\\
    -0.9	-0.999666555493786\\
    -0.85	-1.03660226208147\\
    -0.8	-1.06736065948878\\
    -0.75	-1.09310424466828\\
    -0.7	-1.11467926989354\\
    -0.65	-1.13272867592193\\
    -0.6	-1.14775870966343\\
    -0.55	-1.16018056047831\\
    -0.5	-1.17033764331725\\
    -0.45	-1.1785242011772\\
    -0.4	-1.18499842736295\\
    -0.35	-1.18999199674489\\
    -0.3	-1.19371716236892\\
    -0.25	-1.19637214512868\\
    -0.2	-1.19814528297902\\
    -0.15	-1.19921824082108\\
    -0.0999999999999996	-1.19976847385109\\
    -0.0499999999999996	-1.19997106411708\\
    4.44089209850063e-16	-1.2\\
    0.0500000000000003	-1.20002893448751\\
    0.1	-1.20023143684277\\
    0.15	-1.20078074192487\\
    0.2	-1.20184900138347\\
    0.25	-1.20360605095734\\
    0.3	-1.20621772757672\\
    0.35	-1.20984379742608\\
    0.4	-1.21463558875026\\
    0.45	-1.22073345665966\\
    0.5	-1.22826423595173\\
    0.55	-1.2373388560025\\
    0.6	-1.24805029383114\\
    0.65	-1.26047202437313\\
    0.7	-1.27465709091814\\
    0.75	-1.29063786734275\\
    0.8	-1.30842652407544\\
    0.850000000000001	-1.32801615024874\\
    0.9	-1.34938243358262\\
    0.95	-1.37248576355805\\
    1	-1.39727360556362\\
    1.05	-1.4236829937044\\
    1.1	-1.45164300481851\\
    1.15	-1.48107710124442\\
    1.2	-1.51190525987385\\
    1.25	-1.5440458354403\\
    1.3	-1.57741713343806\\
    1.35	-1.61193869051572\\
    1.4	-1.64753227685619\\
    1.45	-1.68412264609647\\
    1.5	-1.72163806449515\\
    1.55	-1.76001065334461\\
    1.6	-1.79917657811017\\
    1.65	-1.83907611542242\\
    1.7	-1.87965362563957\\
    1.75	-1.92085745482962\\
    1.8	-1.96263978611315\\
    1.85	-2.00495645662199\\
    1.9	-2.04776675301819\\
    1.95	-2.0910331956468\\
    2	-2.13472131897756\\
    2.05	-2.17879945399927\\
    2.1	-2.22323851662619\\
    2.15	-2.26801180490441\\
    2.2	-2.31309480681907\\
    2.25	-2.35846501974845\\
    2.3	-2.40410178204596\\
    2.35	-2.44998611681707\\
    2.4	-2.49610058766228\\
    2.45	-2.54242916595418\\
    2.5	-2.58895710908282\\
    2.55	-2.63567084902402\\
    2.6	-2.68255789054354\\
    2.65	-2.72960671833826\\
    2.7	-2.77680671242296\\
    2.75	-2.82414807109337\\
    2.8	-2.87162174082733\\
    2.85	-2.91921935252236\\
    2.9	-2.96693316350804\\
    2.95	-3.01475600481223\\
    3	-3.06268123320088\\
    3.05	-3.11070268755082\\
    3.1	-3.1588146491524\\
    3.15	-3.20701180557469\\
    3.2	-3.25528921775908\\
    3.25	-3.30364229003798\\
    3.3	-3.35206674280367\\
    3.35	-3.40055858757846\\
    3.4	-3.44911410426097\\
    3.45	-3.49772982034518\\
    3.5	-3.54640249192816\\
    3.55	-3.59512908634073\\
    3.6	-3.64390676625079\\
    3.65	-3.69273287510416\\
    3.7	-3.74160492378036\\
    3.75	-3.79052057835317\\
    3.8	-3.83947764885575\\
    3.85	-3.88847407896054\\
    3.9	-3.93750793649194\\
    3.95	-3.98657740469835\\
    4	-4.03568077421651\\
    4.05	-4.08481643566787\\
    4.1	-4.13398287283209\\
    4.15	-4.18317865634805\\
    4.2	-4.23240243789736\\
    4.25	-4.2816529448295\\
    4.3	-4.33092897519133\\
    4.35	-4.38022939312749\\
    4.4	-4.42955312462073\\
    4.45	-4.47889915354446\\
    4.5	-4.52826651800205\\
    4.55	-4.57765430692961\\
    4.6	-4.6270616569413\\
    4.65	-4.67648774939779\\
    4.7	-4.72593180768029\\
    4.75	-4.77539309465408\\
    4.8	-4.82487091030687\\
    4.85	-4.87436458954849\\
    4.9	-4.92387350015961\\
    4.95	-4.97339704087823\\
    5	-5.02293463961359\\
    };
    
    \addplot[area legend, draw=black, fill=black, forget plot]
    table[row sep=crcr] {%
    x	y\\
    -0.822046601550703	-1.08283466081114\\
    -0.822046601550703	-1.08283466081114\\
    -0.822046601550703	-1.08283466081114\\
    -0.822046601550703	-1.08283466081114\\
    -0.822046601550703	-1.08283466081114\\
    -0.822046601550703	-1.08283466081114\\
    -0.822046601550703	-1.08283466081114\\
    -0.888111026453545	-1.14889908571398\\
    -1.05244063118092	-0.85244063118092\\
    -0.75598217664786	-1.01677023590829\\
    -0.822046601550703	-1.08283466081114\\
    }--cycle;
    \addplot [color=black, forget plot]
      table[row sep=crcr]{%
    -5	4.99316399138997\\
    -4.95	4.94302490027429\\
    -4.9	4.89288151422978\\
    -4.85	4.84273365425806\\
    -4.8	4.79258113191607\\
    -4.75	4.74242374870999\\
    -4.7	4.69226129544322\\
    -4.65	4.64209355151434\\
    -4.6	4.59192028416048\\
    -4.55	4.54174124764133\\
    -4.5	4.4915561823583\\
    -4.45	4.44136481390288\\
    -4.4	4.39116685202757\\
    -4.35	4.34096198953208\\
    -4.3	4.29074990105671\\
    -4.25	4.24053024177388\\
    -4.2	4.19030264596785\\
    -4.15	4.1400667254915\\
    -4.1	4.08982206808783\\
    -4.05	4.03956823556236\\
    -4	3.98930476179092\\
    -3.95	3.93903115054581\\
    -3.9	3.88874687312076\\
    -3.85	3.83845136573318\\
    -3.8	3.78814402667926\\
    -3.75	3.73782421321457\\
    -3.7	3.68749123812914\\
    -3.65	3.63714436598227\\
    -3.6	3.58678280895741\\
    -3.55	3.53640572229232\\
    -3.5	3.48601219923388\\
    -3.45	3.43560126545931\\
    -3.4	3.38517187289809\\
    -3.35	3.33472289287901\\
    -3.3	3.28425310851596\\
    -3.25	3.23376120623338\\
    -3.2	3.18324576631711\\
    -3.15	3.13270525235931\\
    -3.1	3.08213799944491\\
    -3.05	3.0315422009035\\
    -3	2.98091589342134\\
    -2.95	2.93025694027429\\
    -2.9	2.87956301240214\\
    -2.85	2.82883156699606\\
    -2.8	2.77805982321302\\
    -2.75	2.72724473456117\\
    -2.7	2.6763829574156\\
    -2.65	2.62547081502136\\
    -2.6	2.57450425621572\\
    -2.55	2.52347880794844\\
    -2.5	2.4723895204903\\
    -2.45	2.42123090398711\\
    -2.4	2.3699968547259\\
    -2.35	2.31868056911712\\
    -2.3	2.26727444293918\\
    -2.25	2.21576995281205\\
    -2.2	2.16415751612684\\
    -2.15	2.11242632470666\\
    -2.1	2.06056414624121\\
    -2.05	2.0085570859251\\
    -2	1.9563892986035\\
    -1.95	1.90404263889518\\
    -1.9	1.85149623294794\\
    -1.85	1.79872595028535\\
    -1.8	1.74570374703721\\
    -1.75	1.69239684182827\\
    -1.7	1.63876667138442\\
    -1.65	1.58476755241416\\
    -1.6	1.53034494621791\\
    -1.55	1.47543317737815\\
    -1.5	1.41995238880239\\
    -1.45	1.36380440690032\\
    -1.4	1.30686701541751\\
    -1.35	1.2489858440794\\
    -1.3	1.18996257208743\\
    -1.25	1.12953723186777\\
    -1.2	1.06736065948878\\
    -1.15	1.00294962449448\\
    -1.1	0.935609523735209\\
    -1.05	0.864291243406886\\
    -1	0.787299436620434\\
    -0.95	0.70161193151899\\
    -0.9	0.600924500691737\\
    -0.85	0.46742365802326\\
    -0.8	0\\
    };
    \addplot [color=black, forget plot]
      table[row sep=crcr]{%
    -0.8	-0\\
    -0.75	-0.448347851119676\\
    -0.7	-0.552877481367887\\
    -0.65	-0.619172499930452\\
    -0.6	-0.66644437032919\\
    -0.55	-0.701781178162032\\
    -0.5	-0.728736163112184\\
    -0.45	-0.749406938568536\\
    -0.4	-0.765172473108956\\
    -0.35	-0.777015220107044\\
    -0.3	-0.7856828007848\\
    -0.25	-0.791777762186883\\
    -0.2	-0.795811441579278\\
    -0.15	-0.798238310912097\\
    -0.0999999999999998	-0.799478827214055\\
    -0.0499999999999998	-0.799934890534424\\
    2.22044604925031e-16	-0.8\\
    0.0500000000000003	-0.800065098869194\\
    0.1	-0.800520494616583\\
    0.15	-0.801753964201621\\
    0.2	-0.804145151717811\\
    0.25	-0.808056612214814\\
    0.3	-0.81382230443977\\
    0.35	-0.821734875415588\\
    0.4	-0.832033529220762\\
    0.45	-0.844894423278802\\
    0.5	-0.860425244895165\\
    0.55	-0.878664900397598\\
    0.6	-0.899588289055083\\
    0.65	-0.923115227221045\\
    0.7	-0.949121995802933\\
    0.75	-0.977453823561153\\
    0.8	-1.0079368399159\\
    0.85	-1.04038847666707\\
    0.9	-1.07462579367714\\
    0.95	-1.11047162276053\\
    1	-1.14775870966343\\
    1.05	-1.18633218525399\\
    1.1	-1.22605074361495\\
    1.15	-1.26678688455789\\
    1.2	-1.30842652407544\\
    1.25	-1.35086821152099\\
    1.3	-1.3940221304312\\
    1.35	-1.43780900729455\\
    1.4	-1.48215901108412\\
    1.45	-1.52701069542186\\
    1.5	-1.57231001316563\\
    1.55	-1.61800941809483\\
    1.6	-1.66406705844152\\
    1.65	-1.71044606080434\\
    1.7	-1.75711389934935\\
    1.75	-1.80404184330095\\
    1.8	-1.85120447494864\\
    1.85	-1.89857927032065\\
    1.9	-1.94614623501491\\
    1.95	-1.99388758824203\\
    2	-2.04178748880059\\
    2.05	-2.08983179739575\\
    2.1	-2.13800787038313\\
    2.15	-2.18630438064437\\
    2.2	-2.23471116186911\\
    2.25	-2.28321907302334\\
    2.3	-2.33181988023012\\
    2.35	-2.38050615367692\\
    2.4	-2.42927117750053\\
    2.45	-2.47810887089055\\
    2.5	-2.52701371890211\\
    2.55	-2.57598071168189\\
    2.6	-2.62500529099463\\
    2.65	-2.67408330309355\\
    2.7	-2.72321095711191\\
    2.75	-2.77238478826693\\
    2.8	-2.82160162526491\\
    2.85	-2.87085856138022\\
    2.9	-2.92015292875166\\
    2.95	-2.96948227550153\\
    3	-3.0188443453347\\
    3.05	-3.06823705932049\\
    3.1	-3.11765849959853\\
    3.15	-3.16710689478337\\
    3.2	-3.21658060687125\\
    3.25	-3.26607811947718\\
    3.3	-3.31559802725215\\
    3.35	-3.36513902634858\\
    3.4	-3.41469990581837\\
    3.45	-3.46427953984192\\
    3.5	-3.51387688069854\\
    3.55	-3.56349095239924\\
    3.6	-3.61312084491233\\
    3.65	-3.66276570891993\\
    3.7	-3.71242475105105\\
    3.75	-3.76209722954255\\
    3.8	-3.81178245028514\\
    3.85	-3.86147976321609\\
    3.9	-3.9111885590246\\
    3.95	-3.96090826613942\\
    4	-4.01063834797167\\
    4.05	-4.0603783003885\\
    4.1	-4.11012764939604\\
    4.15	-4.159885949012\\
    4.2	-4.20965277931068\\
    4.25	-4.25942774462447\\
    4.3	-4.30921047188801\\
    4.35	-4.35900060911202\\
    4.4	-4.40879782397564\\
    4.45	-4.45860180252669\\
    4.5	-4.50841224798078\\
    4.55	-4.55822887961054\\
    4.6	-4.60805143171761\\
    4.65	-4.65787965268028\\
    4.7	-4.70771330407054\\
    4.75	-4.75755215983494\\
    4.8	-4.80739600553389\\
    4.85	-4.85724463763487\\
    4.9	-4.9070978628551\\
    4.95	-4.95695549754991\\
    5	-5.00681736714308\\
    };
    
    \addplot[area legend, draw=black, fill=black, forget plot]
    table[row sep=crcr] {%
    x	y\\
    -0.504229446767442	-0.765691394807118\\
    -0.504229446767441	-0.765691394807118\\
    -0.504229446767441	-0.765691394807118\\
    -0.504229446767441	-0.765691394807118\\
    -0.504229446767441	-0.765691394807118\\
    -0.504229446767441	-0.765691394807118\\
    -0.504229446767442	-0.765691394807118\\
    -0.570390488919266	-0.831852436958942\\
    -0.73496042078728	-0.53496042078728\\
    -0.438068404615619	-0.699530352655294\\
    -0.504229446767442	-0.765691394807118\\
    }--cycle;
    \addplot [color=black, forget plot]
      table[row sep=crcr]{%
    -5	4.99914652098967\\
    -4.95	4.94912918735758\\
    -4.9	4.89911132016312\\
    -4.85	4.84909289727554\\
    -4.8	4.79907389540437\\
    -4.75	4.74905429002565\\
    -4.7	4.69903405530267\\
    -4.65	4.64901316400076\\
    -4.6	4.59899158739541\\
    -4.55	4.54896929517344\\
    -4.5	4.49894625532624\\
    -4.45	4.44892243403476\\
    -4.4	4.39889779554507\\
    -4.35	4.34887230203398\\
    -4.3	4.29884591346357\\
    -4.25	4.24881858742365\\
    -4.2	4.19879027896099\\
    -4.15	4.148760940394\\
    -4.1	4.0987305211115\\
    -4.05	4.04869896735392\\
    -4	3.99866622197514\\
    -3.95	3.94863222418314\\
    -3.9	3.89859690925702\\
    -3.85	3.84856020823817\\
    -3.8	3.79852204759267\\
    -3.75	3.74848234884181\\
    -3.7	3.6984410281574\\
    -3.65	3.64839799591779\\
    -3.6	3.59835315622033\\
    -3.55	3.54830640634515\\
    -3.5	3.4982576361648\\
    -3.45	3.44820672749324\\
    -3.4	3.39815355336708\\
    -3.35	3.34809797725079\\
    -3.3	3.29803985215652\\
    -3.25	3.247979019668\\
    -3.2	3.19791530885622\\
    -3.15	3.14784853507302\\
    -3.1	3.09777849860655\\
    -3.05	3.04770498318025\\
    -3	2.99762775427414\\
    -2.95	2.94754655724383\\
    -2.9	2.89746111520897\\
    -2.85	2.84737112667825\\
    -2.8	2.79727626287235\\
    -2.75	2.74717616470037\\
    -2.7	2.69707043933719\\
    -2.65	2.64695865634035\\
    -2.6	2.59684034323389\\
    -2.55	2.54671498047403\\
    -2.5	2.49658199569499\\
    -2.45	2.44644075711489\\
    -2.4	2.39629056595804\\
    -2.35	2.34613064772161\\
    -2.3	2.29596014208024\\
    -2.25	2.24577809117915\\
    -2.2	2.19558342601378\\
    -2.15	2.14537495052835\\
    -2.1	2.09515132298392\\
    -2.05	2.04491103404392\\
    -2	1.99465238089546\\
    -1.95	1.94437343656038\\
    -1.9	1.89407201333963\\
    -1.85	1.84374561906457\\
    -1.8	1.7933914044787\\
    -1.75	1.74300609961694\\
    -1.7	1.69258593644905\\
    -1.65	1.64212655425798\\
    -1.6	1.59162288315856\\
    -1.55	1.54106899972245\\
    -1.5	1.49045794671067\\
    -1.45	1.43978150620107\\
    -1.4	1.38902991160651\\
    -1.35	1.3381914787078\\
    -1.3	1.28725212810786\\
    -1.25	1.23619476024515\\
    -1.2	1.18499842736295\\
    -1.15	1.13363722146959\\
    -1.1	1.08207875806342\\
    -1.05	1.0302820731206\\
    -1	0.978194649301749\\
    -0.95	0.925748116473972\\
    -0.899999999999999	0.872851873518604\\
    -0.849999999999999	0.819383335692209\\
    -0.799999999999999	0.765172473108955\\
    -0.75	0.709976194401196\\
    -0.7	0.653433507708757\\
    -0.649999999999999	0.594981286043714\\
    -0.599999999999999	0.533680329744389\\
    -0.549999999999999	0.467804761867604\\
    -0.499999999999999	0.393649718310217\\
    -0.449999999999999	0.300462250345868\\
    -0.399999999999999	0\\
    };
    \addplot [color=black, forget plot]
      table[row sep=crcr]{%
    -0.399999999999999	-0\\
    -0.349999999999999	-0.276438740683943\\
    -0.299999999999999	-0.333222185164595\\
    -0.249999999999999	-0.364368081556092\\
    -0.199999999999999	-0.382586236554477\\
    -0.149999999999999	-0.392841400392399\\
    -0.0999999999999994	-0.397905720789639\\
    -0.0499999999999994	-0.399739413607027\\
    5.55111512312578e-16	-0.399999999999999\\
    0.0500000000000005	-0.400260247308291\\
    0.100000000000001	-0.402072575858905\\
    0.150000000000001	-0.406911152219885\\
    0.200000000000001	-0.416016764610381\\
    0.250000000000001	-0.430212622447582\\
    0.300000000000001	-0.449794144527541\\
    0.350000000000001	-0.474560997901466\\
    0.400000000000001	-0.503968419957949\\
    0.450000000000001	-0.537312896838572\\
    0.500000000000001	-0.573879354831717\\
    0.550000000000001	-0.613025371807473\\
    0.600000000000001	-0.654213262037718\\
    0.650000000000001	-0.697011065215602\\
    0.700000000000001	-0.741079505542062\\
    0.750000000000001	-0.786155006582817\\
    0.800000000000001	-0.832033529220762\\
    0.850000000000001	-0.878556949674674\\
    0.900000000000001	-0.925602237474319\\
    0.950000000000001	-0.973073117507453\\
    1	-1.02089374440029\\
    1.05	-1.06900393519156\\
    1.1	-1.11735558093456\\
    1.15	-1.16590994011506\\
    1.2	-1.21463558875027\\
    1.25	-1.26350685945106\\
    1.3	-1.31250264549731\\
    1.35	-1.36160547855596\\
    1.4	-1.41080081263245\\
    1.45	-1.46007646437583\\
    1.5	-1.50942217266735\\
    1.55	-1.55882924979926\\
    1.6	-1.60829030343562\\
    1.65	-1.65779901362608\\
    1.7	-1.70734995290919\\
    1.75	-1.75693844034927\\
    1.8	-1.80656042245616\\
    1.85	-1.85621237552553\\
    1.9	-1.90589122514257\\
    1.95	-1.9555942795123\\
    2	-2.00531917398583\\
    2.05	-2.05506382469802\\
    2.1	-2.10482638965534\\
    2.15	-2.15460523594401\\
    2.2	-2.20439891198782\\
    2.25	-2.25420612399039\\
    2.3	-2.30402571585881\\
    2.35	-2.35385665203527\\
    2.4	-2.40369800276695\\
    2.45	-2.45354893142755\\
    2.5	-2.50340868357154\\
    2.55	-2.55327657745637\\
    2.6	-2.60315199581259\\
    2.65	-2.65303437867816\\
    2.7	-2.7029232171428\\
    2.75	-2.75281804787314\\
    2.8	-2.80271844830938\\
    2.85	-2.85262403244123\\
    2.9	-2.90253444708476\\
    2.95	-2.95244936859361\\
    3	-3.0023684999476\\
    3.05	-3.05229156817026\\
    3.1	-3.10221832203344\\
    3.15	-3.15214853001325\\
    3.2	-3.20208197846633\\
    3.25	-3.25201846999985\\
    3.3	-3.301957822012\\
    3.35	-3.35189986538303\\
    3.4	-3.40184444329931\\
    3.45	-3.45179141019516\\
    3.5	-3.50174063079926\\
    3.55	-3.55169197927399\\
    3.6	-3.60164533843748\\
    3.65	-3.65160059905935\\
    3.7	-3.70155765922244\\
    3.75	-3.75151642374346\\
    3.8	-3.80147680364636\\
    3.85	-3.85143871568324\\
    3.9	-3.90140208189766\\
    3.95	-3.95136682922639\\
    4	-4.00133288913564\\
    4.05	-4.0513001972884\\
    4.1	-4.10126869323999\\
    4.15	-4.15123832015895\\
    4.2	-4.20120902457112\\
    4.25	-4.25118075612447\\
    4.3	-4.30115346737305\\
    4.35	-4.35112711357808\\
    4.4	-4.40110165252475\\
    4.45	-4.45107704435337\\
    4.5	-4.50105325140348\\
    4.55	-4.5510302380699\\
    4.6	-4.60100797066964\\
    4.65	-4.65098641731874\\
    4.7	-4.70096554781825\\
    4.75	-4.7509453335485\\
    4.8	-4.80092574737107\\
    4.85	-4.85090676353777\\
    4.9	-4.90088835760606\\
    4.95	-4.95087050636041\\
    5	-5.00085318773919\\
    };
    
    \addplot[area legend, draw=black, fill=black, forget plot]
    table[row sep=crcr] {%
    x	y\\
    -0.186625652171389	-0.44833476861589\\
    -0.186625652171388	-0.44833476861589\\
    -0.186625652171388	-0.44833476861589\\
    -0.186625652171388	-0.44833476861589\\
    -0.186625652171388	-0.44833476861589\\
    -0.186625652171388	-0.44833476861589\\
    -0.186625652171389	-0.44833476861589\\
    -0.252822131523008	-0.514531247967509\\
    -0.41748021039364	-0.217480210393639\\
    -0.120429172819772	-0.382138289264271\\
    -0.186625652171389	-0.44833476861589\\
    }--cycle;
    \addplot [color=black, forget plot]
      table[row sep=crcr]{%
    -5	5.00085318773919\\
    -4.95	4.95087050636041\\
    -4.9	4.90088835760606\\
    -4.85	4.85090676353777\\
    -4.8	4.80092574737107\\
    -4.75	4.7509453335485\\
    -4.7	4.70096554781825\\
    -4.65	4.65098641731874\\
    -4.6	4.60100797066964\\
    -4.55	4.5510302380699\\
    -4.5	4.50105325140348\\
    -4.45	4.45107704435337\\
    -4.4	4.40110165252475\\
    -4.35	4.35112711357808\\
    -4.3	4.30115346737305\\
    -4.25	4.25118075612447\\
    -4.2	4.20120902457112\\
    -4.15	4.15123832015895\\
    -4.1	4.10126869323999\\
    -4.05	4.0513001972884\\
    -4	4.00133288913564\\
    -3.95	3.95136682922639\\
    -3.9	3.90140208189766\\
    -3.85	3.85143871568324\\
    -3.8	3.80147680364636\\
    -3.75	3.75151642374346\\
    -3.7	3.70155765922244\\
    -3.65	3.65160059905935\\
    -3.6	3.60164533843748\\
    -3.55	3.55169197927399\\
    -3.5	3.50174063079926\\
    -3.45	3.45179141019516\\
    -3.4	3.40184444329931\\
    -3.35	3.35189986538303\\
    -3.3	3.301957822012\\
    -3.25	3.25201846999985\\
    -3.2	3.20208197846633\\
    -3.15	3.15214853001325\\
    -3.1	3.10221832203344\\
    -3.05	3.05229156817026\\
    -3	3.0023684999476\\
    -2.95	2.95244936859361\\
    -2.9	2.90253444708476\\
    -2.85	2.85262403244123\\
    -2.8	2.80271844830938\\
    -2.75	2.75281804787314\\
    -2.7	2.7029232171428\\
    -2.65	2.65303437867816\\
    -2.6	2.60315199581259\\
    -2.55	2.55327657745637\\
    -2.5	2.50340868357154\\
    -2.45	2.45354893142755\\
    -2.4	2.40369800276695\\
    -2.35	2.35385665203527\\
    -2.3	2.30402571585881\\
    -2.25	2.25420612399039\\
    -2.2	2.20439891198782\\
    -2.15	2.15460523594401\\
    -2.1	2.10482638965534\\
    -2.05	2.05506382469802\\
    -2	2.00531917398583\\
    -1.95	1.9555942795123\\
    -1.9	1.90589122514257\\
    -1.85	1.85621237552553\\
    -1.8	1.80656042245616\\
    -1.75	1.75693844034927\\
    -1.7	1.70734995290919\\
    -1.65	1.65779901362608\\
    -1.6	1.60829030343562\\
    -1.55	1.55882924979926\\
    -1.5	1.50942217266735\\
    -1.45	1.46007646437583\\
    -1.4	1.41080081263245\\
    -1.35	1.36160547855596\\
    -1.3	1.31250264549731\\
    -1.25	1.26350685945106\\
    -1.2	1.21463558875027\\
    -1.15	1.16590994011506\\
    -1.1	1.11735558093456\\
    -1.05	1.06900393519156\\
    -1	1.02089374440029\\
    -0.95	0.973073117507453\\
    -0.9	0.925602237474319\\
    -0.85	0.878556949674674\\
    -0.8	0.832033529220762\\
    -0.75	0.786155006582817\\
    -0.7	0.741079505542062\\
    -0.65	0.697011065215602\\
    -0.6	0.654213262037718\\
    -0.55	0.613025371807473\\
    -0.5	0.573879354831717\\
    -0.45	0.537312896838572\\
    -0.4	0.503968419957949\\
    -0.35	0.474560997901467\\
    -0.3	0.449794144527542\\
    -0.25	0.430212622447582\\
    -0.2	0.416016764610381\\
    -0.15	0.406911152219885\\
    -0.1	0.402072575858906\\
    -0.05	0.400260247308292\\
    0	0.4\\
    0.05	0.399739413607028\\
    0.1	0.397905720789639\\
    0.15	0.3928414003924\\
    0.2	0.382586236554478\\
    0.25	0.364368081556092\\
    0.3	0.333222185164595\\
    0.35	0.276438740683944\\
    0.4	0\\
    };
    \addplot [color=black, forget plot]
      table[row sep=crcr]{%
    0.4	-0\\
    0.45	-0.300462250345868\\
    0.5	-0.393649718310217\\
    0.55	-0.467804761867605\\
    0.6	-0.533680329744389\\
    0.65	-0.594981286043714\\
    0.7	-0.653433507708757\\
    0.75	-0.709976194401196\\
    0.8	-0.765172473108956\\
    0.85	-0.81938333569221\\
    0.9	-0.872851873518604\\
    0.95	-0.925748116473972\\
    1	-0.97819464930175\\
    1.05	-1.0302820731206\\
    1.1	-1.08207875806342\\
    1.15	-1.13363722146959\\
    1.2	-1.18499842736295\\
    1.25	-1.23619476024515\\
    1.3	-1.28725212810786\\
    1.35	-1.3381914787078\\
    1.4	-1.38902991160651\\
    1.45	-1.43978150620107\\
    1.5	-1.49045794671067\\
    1.55	-1.54106899972245\\
    1.6	-1.59162288315856\\
    1.65	-1.64212655425798\\
    1.7	-1.69258593644905\\
    1.75	-1.74300609961694\\
    1.8	-1.7933914044787\\
    1.85	-1.84374561906457\\
    1.9	-1.89407201333963\\
    1.95	-1.94437343656038\\
    2	-1.99465238089546\\
    2.05	-2.04491103404392\\
    2.1	-2.09515132298392\\
    2.15	-2.14537495052835\\
    2.2	-2.19558342601378\\
    2.25	-2.24577809117915\\
    2.3	-2.29596014208024\\
    2.35	-2.34613064772161\\
    2.4	-2.39629056595804\\
    2.45	-2.44644075711489\\
    2.5	-2.49658199569499\\
    2.55	-2.54671498047403\\
    2.6	-2.59684034323389\\
    2.65	-2.64695865634035\\
    2.7	-2.69707043933719\\
    2.75	-2.74717616470037\\
    2.8	-2.79727626287235\\
    2.85	-2.84737112667825\\
    2.9	-2.89746111520897\\
    2.95	-2.94754655724383\\
    3	-2.99762775427414\\
    3.05	-3.04770498318025\\
    3.1	-3.09777849860655\\
    3.15	-3.14784853507302\\
    3.2	-3.19791530885622\\
    3.25	-3.247979019668\\
    3.3	-3.29803985215652\\
    3.35	-3.34809797725079\\
    3.4	-3.39815355336708\\
    3.45	-3.44820672749324\\
    3.5	-3.4982576361648\\
    3.55	-3.54830640634515\\
    3.6	-3.59835315622033\\
    3.65	-3.64839799591779\\
    3.7	-3.6984410281574\\
    3.75	-3.74848234884181\\
    3.8	-3.79852204759267\\
    3.85	-3.84856020823817\\
    3.9	-3.89859690925702\\
    3.95	-3.94863222418314\\
    4	-3.99866622197514\\
    4.05	-4.04869896735392\\
    4.1	-4.0987305211115\\
    4.15	-4.148760940394\\
    4.2	-4.19879027896099\\
    4.25	-4.24881858742365\\
    4.3	-4.29884591346357\\
    4.35	-4.34887230203398\\
    4.4	-4.39889779554507\\
    4.45	-4.44892243403476\\
    4.5	-4.49894625532624\\
    4.55	-4.54896929517344\\
    4.6	-4.59899158739541\\
    4.65	-4.64901316400076\\
    4.7	-4.69903405530267\\
    4.75	-4.74905429002565\\
    4.8	-4.79907389540437\\
    4.85	-4.84909289727554\\
    4.9	-4.89911132016312\\
    4.95	-4.94912918735758\\
    5	-4.99914652098967\\
    };
    
    \addplot[area legend, draw=black, fill=black, forget plot]
    table[row sep=crcr] {%
    x	y\\
    0.448370024112	0.186590396675279\\
    0.448370024111999	0.186590396675279\\
    0.448370024111999	0.186590396675279\\
    0.448370024111999	0.186590396675279\\
    0.448370024111999	0.186590396675279\\
    0.448370024111999	0.186590396675279\\
    0.448370024112001	0.186590396675279\\
    0.382163435409551	0.120383807972829\\
    0.21748021039364	0.41748021039364\\
    0.51457661281445	0.252796985377729\\
    0.448370024112001	0.186590396675279\\
    }--cycle;
    \addplot [color=black, forget plot]
      table[row sep=crcr]{%
    -5	5.00681736714308\\
    -4.95	4.95695549754991\\
    -4.9	4.9070978628551\\
    -4.85	4.85724463763487\\
    -4.8	4.80739600553389\\
    -4.75	4.75755215983494\\
    -4.7	4.70771330407054\\
    -4.65	4.65787965268028\\
    -4.6	4.60805143171761\\
    -4.55	4.55822887961054\\
    -4.5	4.50841224798078\\
    -4.45	4.45860180252669\\
    -4.4	4.40879782397564\\
    -4.35	4.35900060911202\\
    -4.3	4.30921047188801\\
    -4.25	4.25942774462447\\
    -4.2	4.20965277931068\\
    -4.15	4.159885949012\\
    -4.1	4.11012764939604\\
    -4.05	4.0603783003885\\
    -4	4.01063834797167\\
    -3.95	3.96090826613942\\
    -3.9	3.9111885590246\\
    -3.85	3.86147976321609\\
    -3.8	3.81178245028514\\
    -3.75	3.76209722954255\\
    -3.7	3.71242475105105\\
    -3.65	3.66276570891993\\
    -3.6	3.61312084491233\\
    -3.55	3.56349095239924\\
    -3.5	3.51387688069854\\
    -3.45	3.46427953984192\\
    -3.4	3.41469990581837\\
    -3.35	3.36513902634858\\
    -3.3	3.31559802725215\\
    -3.25	3.26607811947718\\
    -3.2	3.21658060687125\\
    -3.15	3.16710689478337\\
    -3.1	3.11765849959853\\
    -3.05	3.06823705932049\\
    -3	3.0188443453347\\
    -2.95	2.96948227550153\\
    -2.9	2.92015292875166\\
    -2.85	2.87085856138022\\
    -2.8	2.82160162526491\\
    -2.75	2.77238478826693\\
    -2.7	2.72321095711191\\
    -2.65	2.67408330309355\\
    -2.6	2.62500529099463\\
    -2.55	2.57598071168189\\
    -2.5	2.52701371890211\\
    -2.45	2.47810887089055\\
    -2.4	2.42927117750053\\
    -2.35	2.38050615367692\\
    -2.3	2.33181988023012\\
    -2.25	2.28321907302334\\
    -2.2	2.23471116186911\\
    -2.15	2.18630438064437\\
    -2.1	2.13800787038313\\
    -2.05	2.08983179739575\\
    -2	2.04178748880059\\
    -1.95	1.99388758824203\\
    -1.9	1.94614623501491\\
    -1.85	1.89857927032065\\
    -1.8	1.85120447494864\\
    -1.75	1.80404184330094\\
    -1.7	1.75711389934935\\
    -1.65	1.71044606080434\\
    -1.6	1.66406705844152\\
    -1.55	1.61800941809483\\
    -1.5	1.57231001316563\\
    -1.45	1.52701069542186\\
    -1.4	1.48215901108412\\
    -1.35	1.43780900729455\\
    -1.3	1.3940221304312\\
    -1.25	1.35086821152099\\
    -1.2	1.30842652407544\\
    -1.15	1.26678688455789\\
    -1.1	1.22605074361495\\
    -1.05	1.18633218525399\\
    -1	1.14775870966343\\
    -0.95	1.11047162276053\\
    -0.9	1.07462579367714\\
    -0.85	1.04038847666707\\
    -0.8	1.0079368399159\\
    -0.75	0.977453823561153\\
    -0.7	0.949121995802933\\
    -0.65	0.923115227221045\\
    -0.6	0.899588289055083\\
    -0.55	0.878664900397598\\
    -0.5	0.860425244895165\\
    -0.45	0.844894423278803\\
    -0.4	0.832033529220762\\
    -0.35	0.821734875415588\\
    -0.3	0.81382230443977\\
    -0.25	0.808056612214814\\
    -0.2	0.804145151717812\\
    -0.15	0.801753964201621\\
    -0.1	0.800520494616584\\
    -0.05	0.800065098869195\\
    0	0.8\\
    0.05	0.799934890534424\\
    0.1	0.799478827214056\\
    0.15	0.798238310912097\\
    0.2	0.795811441579278\\
    0.25	0.791777762186883\\
    0.3	0.7856828007848\\
    0.35	0.777015220107044\\
    0.4	0.765172473108956\\
    0.45	0.749406938568536\\
    0.5	0.728736163112184\\
    0.55	0.701781178162032\\
    0.6	0.666444370329191\\
    0.65	0.619172499930452\\
    0.7	0.552877481367887\\
    0.75	0.448347851119676\\
    0.8	0\\
    };
    \addplot [color=black, forget plot]
      table[row sep=crcr]{%
    0.8	-0\\
    0.85	-0.46742365802326\\
    0.9	-0.600924500691737\\
    0.95	-0.70161193151899\\
    1	-0.787299436620435\\
    1.05	-0.864291243406886\\
    1.1	-0.935609523735209\\
    1.15	-1.00294962449448\\
    1.2	-1.06736065948878\\
    1.25	-1.12953723186777\\
    1.3	-1.18996257208743\\
    1.35	-1.2489858440794\\
    1.4	-1.30686701541751\\
    1.45	-1.36380440690032\\
    1.5	-1.41995238880239\\
    1.55	-1.47543317737815\\
    1.6	-1.53034494621791\\
    1.65	-1.58476755241416\\
    1.7	-1.63876667138442\\
    1.75	-1.69239684182827\\
    1.8	-1.74570374703721\\
    1.85	-1.79872595028535\\
    1.9	-1.85149623294794\\
    1.95	-1.90404263889518\\
    2	-1.9563892986035\\
    2.05	-2.0085570859251\\
    2.1	-2.06056414624121\\
    2.15	-2.11242632470666\\
    2.2	-2.16415751612684\\
    2.25	-2.21576995281205\\
    2.3	-2.26727444293918\\
    2.35	-2.31868056911712\\
    2.4	-2.3699968547259\\
    2.45	-2.42123090398711\\
    2.5	-2.4723895204903\\
    2.55	-2.52347880794844\\
    2.6	-2.57450425621572\\
    2.65	-2.62547081502136\\
    2.7	-2.6763829574156\\
    2.75	-2.72724473456117\\
    2.8	-2.77805982321302\\
    2.85	-2.82883156699606\\
    2.9	-2.87956301240214\\
    2.95	-2.93025694027429\\
    3	-2.98091589342134\\
    3.05	-3.0315422009035\\
    3.1	-3.08213799944491\\
    3.15	-3.13270525235931\\
    3.2	-3.18324576631711\\
    3.25	-3.23376120623338\\
    3.3	-3.28425310851596\\
    3.35	-3.33472289287901\\
    3.4	-3.38517187289809\\
    3.45	-3.43560126545931\\
    3.5	-3.48601219923388\\
    3.55	-3.53640572229232\\
    3.6	-3.58678280895741\\
    3.65	-3.63714436598227\\
    3.7	-3.68749123812914\\
    3.75	-3.73782421321457\\
    3.8	-3.78814402667926\\
    3.85	-3.83845136573318\\
    3.9	-3.88874687312076\\
    3.95	-3.93903115054581\\
    4	-3.98930476179092\\
    4.05	-4.03956823556236\\
    4.1	-4.08982206808783\\
    4.15	-4.1400667254915\\
    4.2	-4.19030264596785\\
    4.25	-4.24053024177388\\
    4.3	-4.29074990105671\\
    4.35	-4.34096198953208\\
    4.4	-4.39116685202757\\
    4.45	-4.44136481390288\\
    4.5	-4.4915561823583\\
    4.55	-4.54174124764133\\
    4.6	-4.59192028416048\\
    4.65	-4.64209355151434\\
    4.7	-4.69226129544322\\
    4.75	-4.74242374870999\\
    4.8	-4.79258113191607\\
    4.85	-4.84273365425806\\
    4.9	-4.89288151422978\\
    4.95	-4.94302490027429\\
    5	-4.99316399138997\\
    };
    
    \addplot[area legend, draw=black, fill=black, forget plot]
    table[row sep=crcr] {%
    x	y\\
    0.765973439638583	0.503947401935975\\
    0.765973439638583	0.503947401935975\\
    0.765973439638583	0.503947401935975\\
    0.765973439638583	0.503947401935975\\
    0.765973439638583	0.503947401935975\\
    0.765973439638583	0.503947401935975\\
    0.765973439638583	0.503947401935974\\
    0.69973152243276	0.43770548473015\\
    0.53496042078728	0.734960420787281\\
    0.832215356844406	0.570189319141798\\
    0.765973439638583	0.503947401935974\\
    }--cycle;
    \addplot [color=black, forget plot]
      table[row sep=crcr]{%
    -5	5.02293463961359\\
    -4.95	4.97339704087823\\
    -4.9	4.92387350015961\\
    -4.85	4.87436458954849\\
    -4.8	4.82487091030687\\
    -4.75	4.77539309465408\\
    -4.7	4.72593180768029\\
    -4.65	4.67648774939779\\
    -4.6	4.6270616569413\\
    -4.55	4.57765430692961\\
    -4.5	4.52826651800205\\
    -4.45	4.47889915354446\\
    -4.4	4.42955312462073\\
    -4.35	4.38022939312749\\
    -4.3	4.33092897519133\\
    -4.25	4.2816529448295\\
    -4.2	4.23240243789736\\
    -4.15	4.18317865634805\\
    -4.1	4.13398287283209\\
    -4.05	4.08481643566787\\
    -4	4.03568077421651\\
    -3.95	3.98657740469835\\
    -3.9	3.93750793649194\\
    -3.85	3.88847407896054\\
    -3.8	3.83947764885575\\
    -3.75	3.79052057835317\\
    -3.7	3.74160492378036\\
    -3.65	3.69273287510416\\
    -3.6	3.64390676625079\\
    -3.55	3.59512908634073\\
    -3.5	3.54640249192816\\
    -3.45	3.49772982034518\\
    -3.4	3.44911410426097\\
    -3.35	3.40055858757846\\
    -3.3	3.35206674280367\\
    -3.25	3.30364229003798\\
    -3.2	3.25528921775908\\
    -3.15	3.20701180557469\\
    -3.1	3.1588146491524\\
    -3.05	3.11070268755082\\
    -3	3.06268123320088\\
    -2.95	3.01475600481223\\
    -2.9	2.96693316350804\\
    -2.85	2.91921935252236\\
    -2.8	2.87162174082733\\
    -2.75	2.82414807109337\\
    -2.7	2.77680671242296\\
    -2.65	2.72960671833826\\
    -2.6	2.68255789054354\\
    -2.55	2.63567084902402\\
    -2.5	2.58895710908282\\
    -2.45	2.54242916595418\\
    -2.4	2.49610058766228\\
    -2.35	2.44998611681707\\
    -2.3	2.40410178204596\\
    -2.25	2.35846501974845\\
    -2.2	2.31309480681907\\
    -2.15	2.26801180490441\\
    -2.1	2.22323851662619\\
    -2.05	2.17879945399927\\
    -2	2.13472131897756\\
    -1.95	2.0910331956468\\
    -1.9	2.04776675301819\\
    -1.85	2.00495645662199\\
    -1.8	1.96263978611315\\
    -1.75	1.92085745482962\\
    -1.7	1.87965362563957\\
    -1.65	1.83907611542242\\
    -1.6	1.79917657811017\\
    -1.55	1.76001065334461\\
    -1.5	1.72163806449515\\
    -1.45	1.68412264609647\\
    -1.4	1.64753227685619\\
    -1.35	1.61193869051572\\
    -1.3	1.57741713343806\\
    -1.25	1.5440458354403\\
    -1.2	1.51190525987385\\
    -1.15	1.48107710124442\\
    -1.1	1.45164300481851\\
    -1.05	1.4236829937044\\
    -1	1.39727360556362\\
    -0.95	1.37248576355805\\
    -0.9	1.34938243358262\\
    -0.85	1.32801615024874\\
    -0.8	1.30842652407544\\
    -0.75	1.29063786734275\\
    -0.7	1.27465709091815\\
    -0.65	1.26047202437313\\
    -0.6	1.24805029383114\\
    -0.55	1.2373388560025\\
    -0.5	1.22826423595173\\
    -0.45	1.22073345665966\\
    -0.4	1.21463558875027\\
    -0.35	1.20984379742608\\
    -0.3	1.20621772757672\\
    -0.25	1.20360605095734\\
    -0.2	1.20184900138347\\
    -0.15	1.20078074192488\\
    -0.0999999999999999	1.20023143684277\\
    -0.0499999999999998	1.20002893448751\\
    0	1.2\\
    0.05	1.19997106411708\\
    0.1	1.19976847385109\\
    0.15	1.19921824082108\\
    0.2	1.19814528297902\\
    0.25	1.19637214512868\\
    0.3	1.19371716236892\\
    0.35	1.18999199674489\\
    0.4	1.18499842736295\\
    0.45	1.1785242011772\\
    0.5	1.17033764331725\\
    0.55	1.16018056047831\\
    0.6	1.14775870966343\\
    0.65	1.13272867592193\\
    0.7	1.11467926989354\\
    0.75	1.09310424466828\\
    0.8	1.06736065948878\\
    0.85	1.03660226208147\\
    0.9	0.999666555493786\\
    0.95	0.954868817813526\\
    1	0.899588289055083\\
    1.05	0.829316222051831\\
    1.1	0.734959659655251\\
    1.15	0.591667217703205\\
    1.2	0\\
    };
    \addplot [color=black, forget plot]
      table[row sep=crcr]{%
    1.2	-0\\
    1.25	-0.608332812076033\\
    1.3	-0.776946201167251\\
    1.35	-0.901386751037605\\
    1.4	-1.0053051390627\\
    1.45	-1.09713441468862\\
    1.5	-1.18094915493065\\
    1.55	-1.25905425798022\\
    1.6	-1.33288874065838\\
    1.65	-1.40341428560281\\
    1.7	-1.47130647268414\\
    1.75	-1.53705839213066\\
    1.8	-1.60104098923317\\
    1.85	-1.66354025008984\\
    1.9	-1.72478118067315\\
    1.95	-1.78494385813114\\
    2	-1.84417451682338\\
    2.05	-1.90259341141233\\
    2.1	-1.96030052312627\\
    2.15	-2.01737978407905\\
    2.2	-2.07390225963995\\
    2.25	-2.12992858320359\\
    2.3	-2.18551084481481\\
    2.35	-2.24069407435017\\
    2.4	-2.29551741932687\\
    2.45	-2.35001508968732\\
    2.5	-2.40421712264546\\
    2.55	-2.45815000707663\\
    2.6	-2.51183719717883\\
    2.65	-2.56529953804358\\
    2.7	-2.61855562055581\\
    2.75	-2.67162207915641\\
    2.8	-2.72451384307563\\
    2.85	-2.77724434942192\\
    2.9	-2.829825724804\\
    2.95	-2.88226894084361\\
    3	-2.93458394790525\\
    3.05	-2.9867797905596\\
    3.1	-3.03886470765518\\
    3.15	-3.09084621936181\\
    3.2	-3.1427312031392\\
    3.25	-3.19452596025271\\
    3.3	-3.24623627419026\\
    3.35	-3.29786746211483\\
    3.4	-3.3494244203076\\
    3.45	-3.40091166440876\\
    3.5	-3.45233336514064\\
    3.55	-3.50369338009591\\
    3.6	-3.55499528208884\\
    3.65	-3.60624238449633\\
    3.7	-3.65743776395566\\
    3.75	-3.70858428073544\\
    3.8	-3.75968459705344\\
    3.85	-3.81074119357857\\
    3.9	-3.86175638432359\\
    3.95	-3.91273233010832\\
    4	-3.96367105075071\\
    4.05	-4.0145744361234\\
    4.1	-4.06544425619686\\
    4.15	-4.11628217017533\\
    4.2	-4.16708973481953\\
    4.25	-4.21786841203885\\
    4.3	-4.26861957582661\\
    4.35	-4.31934451860321\\
    4.4	-4.37004445702512\\
    4.45	-4.42072053731118\\
    4.5	-4.471373840132\\
    4.55	-4.52200538510347\\
    4.6	-4.572616134921\\
    4.65	-4.62320699916736\\
    4.7	-4.67377883782354\\
    4.75	-4.72433246450907\\
    4.8	-4.77486864947567\\
    4.85	-4.82538812237561\\
    4.9	-4.87589157482422\\
    4.95	-4.92637966277395\\
    5	-4.97685300871582\\
    };
    
    \addplot[area legend, draw=black, fill=black, forget plot]
    table[row sep=crcr] {%
    x	y\\
    1.08378659285017	0.821094669511674\\
    1.08378659285017	0.821094669511674\\
    1.08378659285017	0.821094669511674\\
    1.08378659285017	0.821094669511674\\
    1.08378659285017	0.821094669511674\\
    1.08378659285017	0.821094669511674\\
    1.08378659285017	0.821094669511674\\
    1.01744920582758	0.754757282489084\\
    0.85244063118092	1.05244063118092\\
    1.15012397987276	0.887432056534264\\
    1.08378659285017	0.821094669511674\\
    }--cycle;
    \addplot [color=black, forget plot]
      table[row sep=crcr]{%
    -5	5.05402743780422\\
    -4.95	5.00510644972551\\
    -4.9	4.9562177417811\\
    -4.85	4.90736259539177\\
    -4.8	4.85854235500106\\
    -4.75	4.80975843174301\\
    -4.7	4.76101230735385\\
    -4.65	4.71230553834548\\
    -4.6	4.66363976046024\\
    -4.55	4.61501669342777\\
    -4.5	4.56643814604669\\
    -4.45	4.51790602161539\\
    -4.4	4.46942232373823\\
    -4.35	4.42098916253565\\
    -4.3	4.37260876128874\\
    -4.25	4.32428346355148\\
    -4.2	4.27601574076625\\
    -4.15	4.2278082004213\\
    -4.1	4.17966359479151\\
    -4.05	4.1315848303075\\
    -4	4.08357497760117\\
    -3.95	4.03563728227971\\
    -3.9	3.98777517648406\\
    -3.85	3.93999229129192\\
    -3.8	3.89229247002981\\
    -3.75	3.84467978256357\\
    -3.7	3.7971585406413\\
    -3.65	3.74973331436812\\
    -3.6	3.70240894989728\\
    -3.55	3.65519058842745\\
    -3.5	3.60808368660189\\
    -3.45	3.56109403841015\\
    -3.4	3.5142277986987\\
    -3.35	3.46749150840182\\
    -3.3	3.42089212160868\\
    -3.25	3.37443703458653\\
    -3.2	3.32813411688305\\
    -3.15	3.28199174463239\\
    -3.1	3.23601883618965\\
    -3.05	3.19022489021596\\
    -3	3.14462002633127\\
    -2.95	3.09921502844282\\
    -2.9	3.05402139084372\\
    -2.85	3.00905136715629\\
    -2.8	2.96431802216825\\
    -2.75	2.91983528657429\\
    -2.7	2.87561801458909\\
    -2.65	2.83168204433867\\
    -2.6	2.78804426086241\\
    -2.55	2.74472266146485\\
    -2.5	2.70173642304198\\
    -2.45	2.65910597086681\\
    -2.4	2.61685304815087\\
    -2.35	2.57500078549635\\
    -2.3	2.53357376911577\\
    -2.25	2.4925981064171\\
    -2.2	2.45210148722989\\
    -2.15	2.41211323858019\\
    -2.1	2.37266437050798\\
    -2.05	2.33378760996385\\
    -2	2.29551741932687\\
    -1.95	2.25788999556407\\
    -1.9	2.22094324552107\\
    -1.85	2.18471673231751\\
    -1.8	2.14925158735429\\
    -1.75	2.11459038206672\\
    -1.7	2.08077695333414\\
    -1.65	2.0478561764489\\
    -1.6	2.0158736798318\\
    -1.55	1.98487549633947\\
    -1.5	1.95490764712231\\
    -1.45	1.92601565563293\\
    -1.4	1.89824399160587\\
    -1.35	1.87163544764605\\
    -1.3	1.84623045444209\\
    -1.25	1.82206634446136\\
    -1.2	1.79917657811017\\
    -1.15	1.77758995049765\\
    -1.1	1.7573298007952\\
    -1.05	1.73841324934997\\
    -1	1.72085048979033\\
    -0.95	1.70464416398064\\
    -0.9	1.68978884655761\\
    -0.85	1.67627066276552\\
    -0.8	1.66406705844152\\
    -0.75	1.65314673452675\\
    -0.7	1.64346975083118\\
    -0.65	1.634987795549\\
    -0.6	1.62764460887954\\
    -0.55	1.62137654172857\\
    -0.5	1.61611322442963\\
    -0.45	1.61177831615857\\
    -0.4	1.60829030343562\\
    -0.35	1.60556331581963\\
    -0.3	1.60350792840324\\
    -0.25	1.60203192366986\\
    -0.2	1.60104098923317\\
    -0.15	1.60043933248082\\
    -0.1	1.60013019773839\\
    -0.05	1.6000162758761\\
    0	1.6\\
    0.05	1.59998372379276\\
    0.1	1.59986978106885\\
    0.15	1.59956042612031\\
    0.2	1.59895765442811\\
    0.25	1.59796290228795\\
    0.3	1.59647662182419\\
    0.35	1.59439772467235\\
    0.4	1.59162288315856\\
    0.45	1.58804567183962\\
    0.5	1.58355552437377\\
    0.55	1.57803647021591\\
    0.6	1.5713656015696\\
    0.65	1.5634112018452\\
    0.7	1.55403044021409\\
    0.75	1.54306649904155\\
    0.8	1.53034494621791\\
    0.85	1.51566908328221\\
    0.9	1.49881387713707\\
    0.95	1.47951789142451\\
    1	1.45747232622437\\
    1.05	1.43230576551778\\
    1.1	1.40356235632406\\
    1.15	1.37066957164814\\
    1.2	1.33288874065838\\
    1.25	1.28923557057932\\
    1.3	1.2383449998609\\
    1.35	1.17822413134853\\
    1.4	1.10575496273577\\
    1.45	1.01554865303808\\
    1.5	0.896695702239352\\
    1.55	0.719277180972339\\
    1.6	0\\
    };
    \addplot [color=black, forget plot]
      table[row sep=crcr]{%
    1.6	-0\\
    1.65	-0.734419304705956\\
    1.7	-0.93484731604652\\
    1.75	-1.08104579769418\\
    1.8	-1.20184900138347\\
    1.85	-1.30757412870449\\
    1.9	-1.40322386303798\\
    1.95	-1.4916386638245\\
    2	-1.57459887324087\\
    2.05	-1.65329918399086\\
    2.1	-1.72858248681377\\
    2.15	-1.80106675450351\\
    2.2	-1.87121904747042\\
    2.25	-1.93940118621276\\
    2.3	-2.00589924898897\\
    2.35	-2.07094334934552\\
    2.4	-2.13472131897756\\
    2.45	-2.19738843001697\\
    2.5	-2.25907446373555\\
    2.55	-2.31988895376138\\
    2.6	-2.37992514417486\\
    2.65	-2.43926302431389\\
    2.7	-2.49797168815881\\
    2.75	-2.55611119158266\\
    2.8	-2.61373403083503\\
    2.85	-2.67088633154152\\
    2.9	-2.72760881380063\\
    2.95	-2.78393758220642\\
    3	-2.83990477760478\\
    3.05	-2.89553911864629\\
    3.1	-2.95086635475631\\
    3.15	-3.00590964734172\\
    3.2	-3.06068989243582\\
    3.25	-3.11522599523022\\
    3.3	-3.16953510482831\\
    3.35	-3.22363281591635\\
    3.4	-3.27753334276884\\
    3.45	-3.33124966999775\\
    3.5	-3.38479368365655\\
    3.55	-3.4381762856732\\
    3.6	-3.49140749407442\\
    3.65	-3.54449653105004\\
    3.7	-3.5974519005707\\
    3.75	-3.65028145699731\\
    3.8	-3.70299246589589\\
    3.85	-3.75559165808512\\
    3.9	-3.80808527779036\\
    3.95	-3.86047912564921\\
    4	-3.912778597207\\
    4.05	-3.96498871745015\\
    4.1	-4.01711417185019\\
    4.15	-4.06915933432656\\
    4.2	-4.12112829248242\\
    4.25	-4.17302487042138\\
    4.3	-4.22485264941332\\
    4.35	-4.27661498664397\\
    4.4	-4.32831503225368\\
    4.45	-4.3799557448459\\
    4.5	-4.4315399056241\\
    4.55	-4.48307013129735\\
    4.6	-4.53454888587835\\
    4.65	-4.5859784914838\\
    4.7	-4.63736113823424\\
    4.75	-4.68869889334021\\
    4.8	-4.73999370945179\\
    4.85	-4.79124743234037\\
    4.9	-4.84246180797421\\
    4.95	-4.89363848904296\\
    5	-4.94477904098059\\
    };
    
    \addplot[area legend, draw=black, fill=black, forget plot]
    table[row sep=crcr] {%
    x	y\\
    1.40190910322764	1.13793257992148\\
    1.40190910322764	1.13793257992148\\
    1.40190910322764	1.13793257992148\\
    1.40190910322764	1.13793257992148\\
    1.40190910322764	1.13793257992148\\
    1.40190910322764	1.13793257992148\\
    1.40190910322764	1.13793257992148\\
    1.33538753964841	1.07141101634225\\
    1.16992084157456	1.36992084157456\\
    1.46843066680687	1.20445414350071\\
    1.40190910322764	1.13793257992148\\
    }--cycle;
    \addplot [color=black, forget plot]
      table[row sep=crcr]{%
    -5	5.10446872200146\\
    -4.95	5.05652360849244\\
    -4.9	5.00863840034637\\
    -4.85	4.96081538201475\\
    -4.8	4.91305694380694\\
    -4.75	4.86536558753726\\
    -4.7	4.81774393250215\\
    -4.65	4.77019472180698\\
    -4.6	4.72272082906374\\
    -4.55	4.67532526548129\\
    -4.5	4.62801118737159\\
    -4.45	4.58078190409608\\
    -4.4	4.53364088647765\\
    -4.35	4.48659177570492\\
    -4.3	4.43963839275638\\
    -4.25	4.39278474837337\\
    -4.2	4.34603505361155\\
    -4.15	4.29939373100183\\
    -4.1	4.25286542635209\\
    -4.05	4.20645502122203\\
    -4	4.16016764610381\\
    -3.95	4.11400869434114\\
    -3.9	4.06798383681954\\
    -3.85	4.02209903745978\\
    -3.8	3.97636056954548\\
    -3.75	3.93077503291408\\
    -3.7	3.88534937203808\\
    -3.65	3.84009089502003\\
    -3.6	3.79500729352051\\
    -3.55	3.75010666363274\\
    -3.5	3.70539752771031\\
    -3.45	3.66088885714617\\
    -3.4	3.61659009608984\\
    -3.35	3.5725111860773\\
    -3.3	3.52866259153212\\
    -3.25	3.48505532607801\\
    -3.2	3.44170097958066\\
    -3.15	3.39861174581109\\
    -3.1	3.35580045059213\\
    -3.05	3.31328058025453\\
    -3	3.27106631018859\\
    -2.95	3.22917253323018\\
    -2.9	3.18761488756736\\
    -2.85	3.14640978379311\\
    -2.8	3.10557443066252\\
    -2.75	3.06512685903736\\
    -2.7	3.02508594341803\\
    -2.65	2.98547142037234\\
    -2.6	2.94630390307269\\
    -2.55	2.90760489104936\\
    -2.5	2.86939677415858\\
    -2.45	2.83170282965258\\
    -2.4	2.79454721112724\\
    -2.35	2.75795492801531\\
    -2.3	2.72195181419418\\
    -2.25	2.68656448419286\\
    -2.2	2.65182027542034\\
    -2.15	2.61774717480497\\
    -2.1	2.58437372824212\\
    -2.05	2.55172893130465\\
    -2	2.51984209978975\\
    -1.95	2.48874271886697\\
    -1.9	2.4584602698667\\
    -1.85	2.42902403411495\\
    -1.8	2.40046287368554\\
    -1.75	2.37280498950733\\
    -1.7	2.34607765792876\\
    -1.65	2.32030694759599\\
    -1.6	2.29551741932687\\
    -1.55	2.27173181253447\\
    -1.5	2.24897072263771\\
    -1.45	2.22725227474814\\
    -1.4	2.20659179969292\\
    -1.35	2.18700151906935\\
    -1.3	2.16849024647197\\
    -1.25	2.15106311223791\\
    -1.2	2.13472131897756\\
    -1.15	2.11946193476664\\
    -1.1	2.10527773016229\\
    -1.05	2.09215706417879\\
    -1	2.0800838230519\\
    -0.95	2.06903741408839\\
    -0.9	2.05899281521183\\
    -0.85	2.04992067906451\\
    -0.8	2.04178748880059\\
    -0.75	2.03455576109943\\
    -0.7	2.02818429052461\\
    -0.65	2.02262842821894\\
    -0.6	2.01784038710825\\
    -0.55	2.01376956530874\\
    -0.5	2.01036287929453\\
    -0.45	2.0075650985622\\
    -0.4	2.00531917398583\\
    -0.35	2.00356655273682\\
    -0.3	2.00224747348544\\
    -0.25	2.00130123654146\\
    -0.2	2.00066644456782\\
    -0.15	2.00028121045849\\
    -0.1	2.00008332986135\\
    -0.0500000000000003	2.00001041661241\\
    0	2\\
    0.0499999999999998	1.99998958327908\\
    0.0999999999999999	1.9999166631942\\
    0.15	1.99971871043995\\
    0.2	1.99933311098757\\
    0.25	1.99869706803514\\
    0.3	1.9977474639932\\
    0.35	1.99642068139387\\
    0.4	1.99465238089546\\
    0.45	1.99237723362748\\
    0.5	1.9895286039482\\
    0.55	1.98603817722008\\
    0.6	1.98183552537535\\
    0.65	1.97684760074296\\
    0.7	1.97099814569671\\
    0.75	1.964207001962\\
    0.8	1.9563892986035\\
    0.85	1.94745449140934\\
    0.9	1.93730521801843\\
    0.95	1.92583592187964\\
    1	1.91293118277239\\
    1.05	1.89846367034774\\
    1.1	1.88229160722335\\
    1.15	1.86425558533691\\
    1.2	1.84417451682338\\
    1.25	1.82184040778046\\
    1.3	1.79701150191753\\
    1.35	1.76940312047886\\
    1.4	1.73867517068787\\
    1.45	1.70441470781648\\
    1.5	1.66611092582298\\
    1.55	1.62311813709513\\
    1.6	1.57459887324087\\
    1.65	1.51943235382759\\
    1.7	1.45605867613633\\
    1.75	1.38219370341972\\
    1.8	1.29425472539207\\
    1.85	1.18603605379583\\
    1.9	1.04494928899312\\
    1.95	0.836404225202668\\
    2	0\\
    };
    \addplot [color=black, forget plot]
      table[row sep=crcr]{%
    2	-0\\
    2.05	-0.850461110824827\\
    2.1	-1.08036795875479\\
    2.15	-1.24684537872232\\
    2.2	-1.38347928332185\\
    2.25	-1.50231125172934\\
    2.3	-1.6091918838378\\
    2.35	-1.70745000271621\\
    2.4	-1.79917657811017\\
    2.45	-1.88577792730902\\
    2.5	-1.96824859155109\\
    2.55	-2.04731951866994\\
    2.6	-2.12354456278193\\
    2.65	-2.1973539123126\\
    2.7	-2.26908862610406\\
    2.75	-2.33902380933802\\
    2.8	-2.40738466192258\\
    2.85	-2.47435789212902\\
    2.9	-2.54010002292785\\
    2.95	-2.60474355930451\\
    3	-2.66840164872194\\
    3.05	-2.73117165824929\\
    3.1	-2.79313795863858\\
    3.15	-2.85437411839193\\
    3.2	-2.91494465244874\\
    3.25	-2.97490643021857\\
    3.3	-3.03430981992715\\
    3.35	-3.09319962661313\\
    3.4	-3.15161586702219\\
    3.45	-3.20959441438957\\
    3.5	-3.26716753854379\\
    3.55	-3.32436436112808\\
    3.6	-3.38121124148806\\
    3.65	-3.43773210553963\\
    3.7	-3.49394872744598\\
    3.75	-3.54988097200598\\
    3.8	-3.60554700415042\\
    3.85	-3.66096347075747\\
    3.9	-3.71614565905798\\
    3.95	-3.77110763515079\\
    4	-3.82586236554478\\
    4.05	-3.88042182415671\\
    4.1	-3.93479708679752\\
    4.15	-3.98899841485557\\
    4.2	-4.04303532961917\\
    4.25	-4.09691667846105\\
    4.3	-4.15065069392485\\
    4.35	-4.20424504660246\\
    4.4	-4.2577068925634\\
    4.45	-4.31104291599141\\
    4.5	-4.36425936759302\\
    4.55	-4.4173620992673\\
    4.6	-4.47035659546091\\
    4.65	-4.52324800157807\\
    4.7	-4.5760411497677\\
    4.75	-4.62874058236986\\
    4.8	-4.68135057326899\\
    4.85	-4.73387514737148\\
    4.9	-4.78631809839936\\
    4.95	-4.83868300516958\\
    5	-4.89097324650875\\
    };
    
    \addplot[area legend, draw=black, fill=black, forget plot]
    table[row sep=crcr] {%
    x	y\\
    1.72043130526918	1.45437079866722\\
    1.72043130526918	1.45437079866722\\
    1.72043130526918	1.45437079866722\\
    1.72043130526918	1.45437079866722\\
    1.72043130526918	1.45437079866722\\
    1.72043130526918	1.45437079866722\\
    1.72043130526918	1.45437079866722\\
    1.65361095539292	1.38755044879096\\
    1.4874010519682	1.6874010519682\\
    1.78725165514544	1.52119114854348\\
    1.72043130526918	1.45437079866722\\
    }--cycle;
    \addplot [color=black, forget plot]
      table[row sep=crcr]{%
    -5	5.17791421816565\\
    -4.95	5.13133818241302\\
    -4.9	5.08485833190836\\
    -4.85	5.03847812715392\\
    -4.8	4.99220117532457\\
    -4.75	4.94603123707086\\
    -4.7	4.89997223363413\\
    -4.65	4.85402825428448\\
    -4.6	4.80820356409192\\
    -4.55	4.76250261204083\\
    -4.5	4.7169300394969\\
    -4.45	4.6714906890354\\
    -4.4	4.62618961363815\\
    -4.35	4.58103208626559\\
    -4.3	4.53602360980881\\
    -4.25	4.49116992742435\\
    -4.2	4.44647703325237\\
    -4.15	4.40195118351628\\
    -4.1	4.35759890799854\\
    -4.05	4.31342702188364\\
    -4	4.26944263795511\\
    -3.95	4.22565317912812\\
    -3.9	4.18206639129361\\
    -3.85	4.1386903564432\\
    -3.8	4.09553350603637\\
    -3.75	4.05260463456297\\
    -3.7	4.00991291324399\\
    -3.65	3.9674679038028\\
    -3.6	3.92527957222631\\
    -3.55	3.88335830242172\\
    -3.5	3.84171490965925\\
    -3.45	3.80036065367366\\
    -3.4	3.75930725127913\\
    -3.35	3.71856688833088\\
    -3.3	3.67815223084484\\
    -3.25	3.63807643506226\\
    -3.2	3.59835315622033\\
    -3.15	3.55899655576197\\
    -3.1	3.52002130668921\\
    -3.05	3.48144259673413\\
    -3	3.4432761289903\\
    -2.95	3.40553811961646\\
    -2.9	3.36824529219294\\
    -2.85	3.3314148682816\\
    -2.8	3.29506455371237\\
    -2.75	3.25921252009453\\
    -2.7	3.22387738103143\\
    -2.65	3.18907816250322\\
    -2.6	3.15483426687613\\
    -2.55	3.12116543000122\\
    -2.5	3.08809167088059\\
    -2.45	3.05563323340941\\
    -2.4	3.0238105197477\\
    -2.35	2.99264401494038\\
    -2.3	2.96215420248884\\
    -2.25	2.93236147068346\\
    -2.2	2.90328600963702\\
    -2.15	2.87494769911158\\
    -2.1	2.8473659874088\\
    -2.05	2.82055976179205\\
    -2	2.79454721112724\\
    -1.95	2.76934568166313\\
    -1.9	2.74497152711609\\
    -1.85	2.72143995447251\\
    -1.8	2.69876486716525\\
    -1.75	2.67695870751057\\
    -1.7	2.65603230049749\\
    -1.65	2.63599470119279\\
    -1.6	2.61685304815087\\
    -1.55	2.59861242528775\\
    -1.5	2.58127573468549\\
    -1.45	2.56484358272872\\
    -1.4	2.54931418183629\\
    -1.35	2.53468326983641\\
    -1.3	2.52094404874626\\
    -1.25	2.50808714436297\\
    -1.2	2.49610058766228\\
    -1.15	2.48496981854691\\
    -1.1	2.47467771200501\\
    -1.05	2.46520462624676\\
    -1	2.45652847190347\\
    -0.95	2.44862480091532\\
    -0.9	2.44146691331931\\
    -0.85	2.43502597979047\\
    -0.8	2.42927117750053\\
    -0.75	2.42416983664444\\
    -0.7	2.41968759485215\\
    -0.65	2.41578855664932\\
    -0.6	2.41243545515343\\
    -0.55	2.40958981328368\\
    -0.5	2.40721210191468\\
    -0.45	2.40526189260486\\
    -0.4	2.40369800276695\\
    -0.35	2.40247863140937\\
    -0.3	2.40156148384975\\
    -0.25	2.40090388407496\\
    -0.2	2.40046287368554\\
    -0.15	2.40019529660758\\
    -0.1	2.40005786897502\\
    -0.0500000000000003	2.40000723377449\\
    -4.44089209850063e-16	2.4\\
    0.0499999999999998	2.3999927661819\\
    0.0999999999999996	2.39994212823416\\
    0.15	2.39980467160327\\
    0.2	2.39953694770219\\
    0.25	2.39909543457304\\
    0.3	2.39843648164217\\
    0.35	2.39751623831037\\
    0.4	2.39629056595804\\
    0.45	2.39471493273629\\
    0.5	2.39274429025737\\
    0.55	2.39033293098385\\
    0.6	2.38743432473783\\
    0.65	2.38400093229623\\
    0.7	2.37998399348979\\
    0.75	2.37533328656065\\
    0.8	2.3699968547259\\
    0.85	2.36392069490592\\
    0.9	2.3570484023544\\
    0.95	2.34932076340479\\
    1	2.3406752866345\\
    1.05	2.33104566032113\\
    1.1	2.32036112095663\\
    1.15	2.30854571356602\\
    1.2	2.29551741932687\\
    1.25	2.28118711905239\\
    1.3	2.26545735184385\\
    1.35	2.24822081570561\\
    1.4	2.22935853978708\\
    1.45	2.20873763413653\\
    1.5	2.18620848933655\\
    1.55	2.16160125036841\\
    1.6	2.13472131897756\\
    1.65	2.1053435344861\\
    1.7	2.07320452416294\\
    1.75	2.03799246630451\\
    1.8	1.99933311098757\\
    1.85	1.95677024308203\\
    1.9	1.90973763562705\\
    1.95	1.85751749979135\\
    2	1.79917657811017\\
    2.05	1.73346327510399\\
    2.1	1.65863244410366\\
    2.15	1.57212459328393\\
    2.2	1.4699193193105\\
    2.25	1.34504355335903\\
    2.3	1.18333443540641\\
    2.35	0.94582656794676\\
    2.4	0\\
    };
    \addplot [color=black, forget plot]
      table[row sep=crcr]{%
    2.4	-0\\
    2.45	-0.959054697368768\\
    2.5	-1.21666562415206\\
    2.55	-1.40227097406978\\
    2.6	-1.5538924023345\\
    2.65	-1.68517971004287\\
    2.7	-1.80277350207521\\
    2.75	-1.9104571148998\\
    2.8	-2.01061027812539\\
    2.85	-2.10483579455697\\
    2.9	-2.19426882937724\\
    2.95	-2.27974495710834\\
    3	-2.3618983098613\\
    3.05	-2.44122223322374\\
    3.1	-2.51810851596044\\
    3.15	-2.59287373022066\\
    3.2	-2.66577748131676\\
    3.25	-2.73703539441857\\
    3.3	-2.80682857120563\\
    3.35	-2.87531061546316\\
    3.4	-2.94261294536828\\
    3.45	-3.00884887348345\\
    3.5	-3.07411678426132\\
    3.55	-3.13850263982626\\
    3.6	-3.20208197846633\\
    3.65	-3.26492152494216\\
    3.7	-3.32708050017967\\
    3.75	-3.38861169560332\\
    3.8	-3.44956236134631\\
    3.85	-3.50997494591197\\
    3.9	-3.56988771626229\\
    3.95	-3.62933528089639\\
    4	-3.68834903364676\\
    4.05	-3.74695753223821\\
    4.1	-3.80518682282467\\
    4.15	-3.86306071952359\\
    4.2	-3.92060104625254\\
    4.25	-3.97782784682101\\
    4.3	-4.0347595681581\\
    4.35	-4.09141322070095\\
    4.4	-4.1478045192799\\
    4.45	-4.20394800728021\\
    4.5	-4.25985716640718\\
    4.55	-4.3155445140118\\
    4.6	-4.37102168962962\\
    4.65	-4.42629953213446\\
    4.7	-4.48138814870033\\
    4.75	-4.53629697659133\\
    4.8	-4.59103483865373\\
    4.85	-4.64560999326295\\
    4.9	-4.70003017937465\\
    4.95	-4.75430265724247\\
    5	-4.80843424529093\\
    };
    
    \addplot[area legend, draw=black, fill=black, forget plot]
    table[row sep=crcr] {%
    x	y\\
    2.03942871719321	1.77033380753047\\
    2.03942871719321	1.77033380753047\\
    2.03942871719321	1.77033380753047\\
    2.03942871719321	1.77033380753047\\
    2.03942871719321	1.77033380753047\\
    2.03942871719321	1.77033380753047\\
    2.03942871719321	1.77033380753046\\
    1.97217331677884	1.7030784071161\\
    1.80488126236184	2.00488126236184\\
    2.10668411760758	1.83758920794483\\
    2.03942871719321	1.77033380753046\\
    }--cycle;
    \addplot [color=black, forget plot]
      table[row sep=crcr]{%
    -5	5.27705758922345\\
    -4.95	5.23223778122896\\
    -4.9	5.18755653879443\\
    -4.85	5.14301847212166\\
    -4.8	5.09862836367258\\
    -4.75	5.05439117454833\\
    -4.7	5.01031205103314\\
    -4.65	4.96639633129637\\
    -4.6	4.92264955224409\\
    -4.55	4.87907745650921\\
    -4.5	4.83568599956711\\
    -4.45	4.79248135696076\\
    -4.4	4.74946993161649\\
    -4.35	4.7066583612282\\
    -4.3	4.66405352568401\\
    -4.25	4.62166255450535\\
    -4.2	4.57949283426403\\
    -4.15	4.53755201593769\\
    -4.1	4.49584802215894\\
    -4.05	4.45438905430713\\
    -4	4.41318359938584\\
    -3.95	4.37224043662185\\
    -3.9	4.33156864371396\\
    -3.85	4.29117760265231\\
    -3.8	4.25107700501994\\
    -3.75	4.21127685667977\\
    -3.7	4.17178748174028\\
    -3.65	4.13261952568365\\
    -3.6	4.09378395752964\\
    -3.55	4.05529207089804\\
    -3.5	4.01715548382202\\
    -3.45	3.97938613715373\\
    -3.4	3.94199629139342\\
    -3.35	3.90499852176285\\
    -3.3	3.86840571133459\\
    -3.25	3.83223104202003\\
    -3.2	3.79648798321173\\
    -3.15	3.76119027787001\\
    -3.1	3.7263519258398\\
    -3.05	3.69198716418307\\
    -3	3.65811044431324\\
    -2.95	3.62473640572403\\
    -2.9	3.59187984611387\\
    -2.85	3.55955568772126\\
    -2.8	3.52777893970564\\
    -2.75	3.49656465643253\\
    -2.7	3.46592789155308\\
    -2.65	3.43588364780516\\
    -2.6	3.40644682250726\\
    -2.55	3.37763214876757\\
    -2.5	3.34945413248832\\
    -2.45	3.32192698531027\\
    -2.4	3.29506455371237\\
    -2.35	3.26888024455833\\
    -2.3	3.24338694746137\\
    -2.25	3.21859695442227\\
    -2.2	3.19452187728006\\
    -2.15	3.17117256359859\\
    -2.1	3.14855901169279\\
    -2.05	3.12669028557426\\
    -2	3.10557443066252\\
    -1.95	3.08521839116532\\
    -1.9	3.06562793007403\\
    -1.85	3.04680755274759\\
    -1.8	3.02876043506725\\
    -1.75	3.01148835713308\\
    -1.7	2.99499164344107\\
    -1.65	2.97926911042469\\
    -1.6	2.96431802216825\\
    -1.55	2.95013405500169\\
    -1.5	2.93671127156856\\
    -1.45	2.9240421048243\\
    -1.4	2.91211735227267\\
    -1.35	2.90092618058876\\
    -1.3	2.89045614061086\\
    -1.25	2.88069319251583\\
    -1.2	2.87162174082733\\
    -1.15	2.86322467874888\\
    -1.1	2.85548344116701\\
    -1.05	2.8483780655392\\
    -1	2.84188725976877\\
    -0.95	2.83598847607753\\
    -0.9	2.83065798981818\\
    -0.85	2.82587098212319\\
    -0.8	2.82160162526491\\
    -0.75	2.81782316960242\\
    -0.7	2.81450803101234\\
    -0.65	2.81162787774153\\
    -0.6	2.80915371567669\\
    -0.55	2.80705597109652\\
    -0.5	2.80530457005292\\
    -0.45	2.80386901361591\\
    -0.4	2.80271844830938\\
    -0.35	2.80182173115804\\
    -0.3	2.80114748885848\\
    -0.25	2.80066417067502\\
    -0.2	2.80034009474402\\
    -0.15	2.80014348754474\\
    -0.1	2.80004251636121\\
    -0.0499999999999998	2.80000531461576\\
    0	2.8\\
    0.0500000000000003	2.79999468536406\\
    0.1	2.79995748234758\\
    0.15	2.79985649774756\\
    0.2	2.79965982261845\\
    0.25	2.79933551408778\\
    0.3	2.79885156984793\\
    0.35	2.79817589524919\\
    0.4	2.79727626287235\\
    0.45	2.79612026439937\\
    0.5	2.7946752545277\\
    0.55	2.79290828658616\\
    0.6	2.79078603940697\\
    0.65	2.78827473488684\\
    0.7	2.78534004552747\\
    0.75	2.78194699107888\\
    0.8	2.77805982321302\\
    0.85	2.77364189692434\\
    0.9	2.7686555270807\\
    0.95	2.76306182822314\\
    1	2.75682053532422\\
    1.05	2.7498898027468\\
    1.1	2.74222597807768\\
    1.15	2.7337833468196\\
    1.2	2.72451384307563\\
    1.25	2.71436672031028\\
    1.3	2.70328817496487\\
    1.35	2.69122091406451\\
    1.4	2.67810365588134\\
    1.45	2.66387055007701\\
    1.5	2.64845050035241\\
    1.55	2.63176636823053\\
    1.6	2.61373403083503\\
    1.65	2.59426125790634\\
    1.7	2.57324636310784\\
    1.75	2.55057657089265\\
    1.8	2.52612602132067\\
    1.85	2.49975330899155\\
    1.9	2.47129841528428\\
    1.95	2.44057884009659\\
    2	2.40738466192258\\
    2.05	2.37147213991386\\
    2.1	2.33255529615217\\
    2.15	2.29029464245921\\
    2.2	2.2442817761255\\
    2.25	2.19401783917641\\
    2.3	2.13888257964319\\
    2.35	2.07808849597626\\
    2.4	2.01061027812539\\
    2.45	1.93507118478761\\
    2.5	1.84954943982631\\
    2.55	1.75122363400783\\
    2.6	1.63565775765116\\
    2.65	1.49515100841119\\
    2.7	1.31404881408406\\
    2.75	1.04924359733213\\
    2.8	0\\
    };
    \addplot [color=black, forget plot]
      table[row sep=crcr]{%
    2.8	-0\\
    2.85	-1.06180923959634\\
    2.9	-1.34571112600068\\
    2.95	-1.54951424251114\\
    3	-1.71543045234635\\
    3.05	-1.85862782751761\\
    3.1	-1.98649230986883\\
    3.15	-2.10323575242108\\
    3.2	-2.21150992547155\\
    3.25	-2.31310259436156\\
    3.3	-2.40928119148166\\
    3.35	-2.50097961609079\\
    3.4	-2.58890737691659\\
    3.45	-2.6736170683825\\
    3.5	-2.75554802817152\\
    3.55	-2.83505566177586\\
    3.6	-2.91243176788214\\
    3.65	-2.98791900777284\\
    3.7	-3.06172144585091\\
    3.75	-3.13401238363513\\
    3.8	-3.20494028573448\\
    3.85	-3.27463333307323\\
    3.9	-3.34320297045916\\
    3.95	-3.41074670541341\\
    4	-3.47735034137574\\
    4.05	-3.54308977795475\\
    4.1	-3.60803247578532\\
    4.15	-3.67223865871609\\
    4.2	-3.73576230821072\\
    4.25	-3.79865199185627\\
    4.3	-3.86095155829406\\
    4.35	-3.92270072374216\\
    4.4	-3.98393556988968\\
    4.45	-4.04468896883863\\
    4.5	-4.1049909476128\\
    4.55	-4.164869002306\\
    4.6	-4.22434837002808\\
    4.65	-4.28345226529961\\
    4.7	-4.34220208634965\\
    4.75	-4.40061759581542\\
    4.8	-4.45871707957417\\
    4.85	-4.51651748681601\\
    4.9	-4.5740345539613\\
    4.95	-4.63128291461227\\
    5	-4.68827619738926\\
    };
    
    \addplot[area legend, draw=black, fill=black, forget plot]
    table[row sep=crcr] {%
    x	y\\
    2.35895698338541	2.08576596212555\\
    2.35895698338541	2.08576596212555\\
    2.35895698338541	2.08576596212555\\
    2.35895698338541	2.08576596212555\\
    2.35895698338541	2.08576596212555\\
    2.35895698338541	2.08576596212555\\
    2.35895698338541	2.08576596212555\\
    2.29111431242103	2.01792329116116\\
    2.12236147275548	2.32236147275548\\
    2.42679965434979	2.15360863308993\\
    2.35895698338541	2.08576596212555\\
    }--cycle;
    \addplot [color=black, forget plot]
      table[row sep=crcr]{%
    -5	5.40347284608396\\
    -4.95	5.36075079285757\\
    -4.9	5.31821194173362\\
    -4.85	5.27586181197868\\
    -4.8	5.23370609630174\\
    -4.75	5.1917506651576\\
    -4.7	5.1500015709927\\
    -4.65	5.10846505241062\\
    -4.6	5.06714753823154\\
    -4.55	5.0260556514178\\
    -4.5	4.98519621283419\\
    -4.45	4.94457624480922\\
    -4.4	4.90420297445978\\
    -4.35	4.86408383673865\\
    -4.3	4.82422647716039\\
    -4.25	4.78463875415766\\
    -4.2	4.74532874101596\\
    -4.15	4.70630472733087\\
    -4.1	4.6675752199277\\
    -4.05	4.6291489431793\\
    -4	4.59103483865373\\
    -3.95	4.55324206401897\\
    -3.9	4.51577999112813\\
    -3.85	4.47865820320454\\
    -3.8	4.44188649104214\\
    -3.75	4.40547484813357\\
    -3.7	4.36943346463502\\
    -3.65	4.33377272007456\\
    -3.6	4.29850317470858\\
    -3.55	4.26363555943029\\
    -3.5	4.22918076413343\\
    -3.45	4.19514982443573\\
    -3.4	4.16155390666829\\
    -3.35	4.12840429104089\\
    -3.3	4.09571235289781\\
    -3.25	4.0634895419857\\
    -3.2	4.03174735966359\\
    -3.15	4.00049733399549\\
    -3.1	3.96975099267893\\
    -3.05	3.93951983377774\\
    -3	3.90981529424461\\
    -2.95	3.88064871623874\\
    -2.9	3.85203131126585\\
    -2.85	3.82397412219231\\
    -2.8	3.79648798321173\\
    -2.75	3.76958347787087\\
    -2.7	3.74327089529209\\
    -2.65	3.71756018476147\\
    -2.6	3.69246090888418\\
    -2.55	3.66798219554225\\
    -2.5	3.64413268892273\\
    -2.45	3.62092049991666\\
    -2.4	3.59835315622033\\
    -2.35	3.5764375524985\\
    -2.3	3.55517990099529\\
    -2.25	3.53458568299978\\
    -2.2	3.51465960159039\\
    -2.15	3.49540553609387\\
    -2.1	3.47682649869993\\
    -2.05	3.45892459367131\\
    -2	3.44170097958066\\
    -1.95	3.42515583498954\\
    -1.9	3.40928832796128\\
    -1.85	3.3940965897682\\
    -1.8	3.37957769311521\\
    -1.75	3.36572763515658\\
    -1.7	3.35254132553103\\
    -1.65	3.34001257958378\\
    -1.6	3.32813411688305\\
    -1.55	3.31689756507478\\
    -1.5	3.3062934690535\\
    -1.45	3.29631130536136\\
    -1.4	3.28693950166235\\
    -1.35	3.27816546107627\\
    -1.3	3.269975591098\\
    -1.25	3.2623553367739\\
    -1.2	3.25528921775908\\
    -1.15	3.24876086883799\\
    -1.1	3.24275308345714\\
    -1.05	3.23724785979289\\
    -1	3.23222644885926\\
    -0.95	3.22766940415125\\
    -0.9	3.22355663231713\\
    -0.85	3.21986744435871\\
    -0.8	3.21658060687125\\
    -0.75	3.21367439285311\\
    -0.7	3.21112663163925\\
    -0.65	3.20891475754098\\
    -0.6	3.20701585680648\\
    -0.55	3.20540671255072\\
    -0.5	3.20406384733972\\
    -0.45	3.20296356315068\\
    -0.4	3.20208197846633\\
    -0.35	3.20139506229741\\
    -0.3	3.20087866496163\\
    -0.25	3.20050854547952\\
    -0.2	3.20026039547678\\
    -0.15	3.20010985950961\\
    -0.1	3.2000325517522\\
    -0.0499999999999998	3.20000406900524\\
    0	3.2\\
    0.0499999999999998	3.19999593098441\\
    0.1	3.19996744758552\\
    0.15	3.19989013294668\\
    0.2	3.1997395621377\\
    0.25	3.19949129283252\\
    0.3	3.19912085224062\\
    0.35	3.1986037202666\\
    0.4	3.19791530885622\\
    0.45	3.19703093746689\\
    0.5	3.1959258045759\\
    0.55	3.19457495511033\\
    0.6	3.19295324364839\\
    0.65	3.19103529320198\\
    0.7	3.18879544934471\\
    0.75	3.18620772939668\\
    0.8	3.18324576631711\\
    0.85	3.17988274688643\\
    0.9	3.17609134367924\\
    0.95	3.17184364023773\\
    1	3.16711104874753\\
    1.05	3.16186421939409\\
    1.1	3.15607294043183\\
    1.15	3.14970602782766\\
    1.2	3.1427312031392\\
    1.25	3.13511495804882\\
    1.3	3.12682240369041\\
    1.35	3.11781710256465\\
    1.4	3.10806088042818\\
    1.45	3.09751361504498\\
    1.5	3.08613299808309\\
    1.55	3.07387426569862\\
    1.6	3.06068989243582\\
    1.65	3.04652924193967\\
    1.7	3.03133816656442\\
    1.75	3.01505854618548\\
    1.8	2.99762775427414\\
    1.85	2.97897803642651\\
    1.9	2.95903578284902\\
    1.95	2.93772067151641\\
    2	2.91494465244874\\
    2.05	2.89061073525953\\
    2.1	2.86461153103556\\
    2.15	2.83682748460024\\
    2.2	2.80712471264813\\
    2.25	2.77535233466951\\
    2.3	2.74133914329627\\
    2.35	2.70488940294455\\
    2.4	2.66577748131676\\
    2.45	2.62374089275194\\
    2.5	2.57847114115864\\
    2.55	2.52960145159073\\
    2.6	2.47668999972181\\
    2.65	2.41919645247475\\
    2.7	2.35644826269705\\
    2.75	2.28759069782819\\
    2.8	2.21150992547155\\
    2.85	2.12670912096187\\
    2.9	2.03109730607616\\
    2.95	1.92160248521068\\
    3	1.7933914044787\\
    3.05	1.63806819432795\\
    3.1	1.43855436194468\\
    3.15	1.14779033794501\\
    3.2	0\\
    };
    \addplot [color=black, forget plot]
      table[row sep=crcr]{%
    3.2	-0\\
    3.25	-1.15980897567894\\
    3.3	-1.46883860941191\\
    3.35	-1.69006605486246\\
    3.4	-1.86969463209304\\
    3.45	-2.02433804508015\\
    3.5	-2.16209159538837\\
    3.55	-2.28757477335282\\
    3.6	-2.40369800276695\\
    3.65	-2.51242481396752\\
    3.7	-2.61514825740898\\
    3.75	-2.71289555841655\\
    3.8	-2.80644772607596\\
    3.85	-2.89641352118544\\
    3.9	-2.983277327649\\
    3.95	-3.0674313166234\\
    4	-3.14919774648174\\
    4.05	-3.2288448424795\\
    4.1	-3.30659836798173\\
    4.15	-3.38265022702969\\
    4.2	-3.45716497362754\\
    4.25	-3.53028481465819\\
    4.3	-3.60213350900703\\
    4.35	-3.67281944469957\\
    4.4	-3.74243809494084\\
    4.45	-3.81107399863058\\
    4.5	-3.87880237242552\\
    4.55	-3.94569043417403\\
    4.6	-4.01179849797794\\
    4.65	-4.07718088688333\\
    4.7	-4.14188669869105\\
    4.75	-4.20596045253427\\
    4.8	-4.26944263795511\\
    4.85	-4.33237018370575\\
    4.9	-4.39477686003393\\
    4.95	-4.45669362552509\\
    5	-4.51814892747109\\
    };
    
    \addplot[area legend, draw=black, fill=black, forget plot]
    table[row sep=crcr] {%
    x	y\\
    2.67904861900171	2.40063474729653\\
    2.67904861900171	2.40063474729653\\
    2.67904861900171	2.40063474729653\\
    2.67904861900171	2.40063474729653\\
    2.67904861900171	2.40063474729653\\
    2.67904861900171	2.40063474729653\\
    2.67904861900171	2.40063474729653\\
    2.61045713390448	2.3320432621993\\
    2.43984168314912	2.63984168314912\\
    2.74764010409894	2.46922623239376\\
    2.67904861900171	2.40063474729653\\
    }--cycle;
    \addplot [color=black, forget plot]
      table[row sep=crcr]{%
    -5	5.557587760239\\
    -4.95	5.51722834626726\\
    -4.9	5.47709482469102\\
    -4.85	5.43719322706679\\
    -4.8	5.3975297343305\\
    -4.75	5.35811067805309\\
    -4.7	5.31894254143834\\
    -4.65	5.28003196003382\\
    -4.6	5.24138572212416\\
    -4.55	5.20301076877394\\
    -4.5	5.16491419348545\\
    -4.45	5.12710324143526\\
    -4.4	5.0895853082511\\
    -4.35	5.05236793828941\\
    -4.3	5.01545882237184\\
    -4.25	4.97886579493782\\
    -4.2	4.94259683056856\\
    -4.15	4.9066600398371\\
    -4.1	4.87106366443767\\
    -4.05	4.83581607154715\\
    -4	4.80092574737107\\
    -3.95	4.76640128982643\\
    -3.9	4.73225140031419\\
    -3.85	4.69848487453516\\
    -3.8	4.66511059230433\\
    -3.75	4.63213750632089\\
    -3.7	4.59957462985363\\
    -3.65	4.56743102330508\\
    -3.6	4.53571577962154\\
    -3.55	4.50443800852139\\
    -3.5	4.47360681951958\\
    -3.45	4.44323130373325\\
    -3.4	4.41332051446052\\
    -3.35	4.38388344653348\\
    -3.3	4.35492901445552\\
    -3.25	4.32646602934382\\
    -3.2	4.29850317470858\\
    -3.15	4.2710489811132\\
    -3.1	4.24411179977208\\
    -3.05	4.21769977515622\\
    -3	4.19182081669086\\
    -2.95	4.16648256964389\\
    -2.9	4.14169238531793\\
    -2.85	4.11745729067414\\
    -2.8	4.09378395752964\\
    -2.75	4.07067867148454\\
    -2.7	4.04814730074787\\
    -2.65	4.0261952650436\\
    -2.6	4.00482750478888\\
    -2.55	3.98404845074623\\
    -2.5	3.96386199435843\\
    -2.45	3.9442714589805\\
    -2.4	3.92527957222631\\
    -2.35	3.90688843964739\\
    -2.3	3.88909951995964\\
    -2.25	3.87191360202824\\
    -2.2	3.85533078381304\\
    -2.15	3.8393504534655\\
    -2.1	3.82397127275444\\
    -2.05	3.80919116298053\\
    -2	3.79500729352051\\
    -1.95	3.78141607311939\\
    -1.9	3.76841314402527\\
    -1.85	3.75599337903529\\
    -1.8	3.74415088149343\\
    -1.75	3.73287898825314\\
    -1.7	3.72217027558834\\
    -1.65	3.71201656800751\\
    -1.6	3.70240894989728\\
    -1.55	3.69333777989389\\
    -1.5	3.6847927078552\\
    -1.45	3.67676269428093\\
    -1.4	3.66923603200685\\
    -1.35	3.66220036997897\\
    -1.3	3.65564273889617\\
    -1.25	3.64954957849654\\
    -1.2	3.6439067662508\\
    -1.15	3.63869964721907\\
    -1.1	3.63391306482233\\
    -1.05	3.62953139227823\\
    -1	3.62553856445265\\
    -0.95	3.62191810988224\\
    -0.899999999999999	3.61865318273015\\
    -0.85	3.61572659444601\\
    -0.8	3.61312084491233\\
    -0.75	3.61081815287203\\
    -0.7	3.60880048544596\\
    -0.65	3.60704958656418\\
    -0.6	3.60554700415042\\
    -0.55	3.60427411591559\\
    -0.5	3.60321215363176\\
    -0.45	3.60234222577463\\
    -0.4	3.60164533843748\\
    -0.35	3.60110241443494\\
    -0.3	3.60069431052831\\
    -0.25	3.6004018327177\\
    -0.2	3.60020574955752\\
    -0.15	3.60008680346253\\
    -0.1	3.60002571998085\\
    -0.0500000000000003	3.6000032150177\\
    0	3.6\\
    0.0499999999999998	3.59999678497655\\
    0.1	3.59997427965163\\
    0.15	3.59991319235125\\
    0.2	3.59979422692153\\
    0.25	3.59959807755697\\
    0.3	3.59930542155328\\
    0.35	3.5988969099748\\
    0.4	3.59835315622033\\
    0.45	3.59765472246325\\
    0.5	3.59678210393225\\
    0.55	3.5957157109879\\
    0.6	3.59443584893706\\
    0.65	3.5929226955121\\
    0.7	3.59115627592492\\
    0.75	3.58911643538605\\
    0.8	3.58678280895741\\
    0.85	3.5841347885821\\
    0.9	3.58115148710675\\
    0.95	3.57781169907985\\
    1	3.57409385807344\\
    1.05	3.56997599023468\\
    1.1	3.56543566372656\\
    1.15	3.56044993366385\\
    1.2	3.55499528208884\\
    1.25	3.54904755246047\\
    1.3	3.54258187804877\\
    1.35	3.5355726035316\\
    1.4	3.52799319897976\\
    1.45	3.51981616528734\\
    1.5	3.51101292995174\\
    1.55	3.50155373192867\\
    1.6	3.49140749407442\\
    1.65	3.48054168143494\\
    1.7	3.46892214333853\\
    1.75	3.45651293688558\\
    1.8	3.4432761289903\\
    1.85	3.42917157359721\\
    1.9	3.41415666004636\\
    1.95	3.39818602776578\\
    2	3.38121124148806\\
    2.05	3.36318041997095\\
    2.1	3.34403780968063\\
    2.15	3.32372329298502\\
    2.2	3.30217181798291\\
    2.25	3.27931273400483\\
    2.3	3.25506901284616\\
    2.35	3.22935633063355\\
    2.4	3.20208197846633\\
    2.45	3.17314356103136\\
    2.5	3.14242743042701\\
    2.55	3.10980678624441\\
    2.6	3.07513935076953\\
    2.65	3.03826449734771\\
    2.7	2.99899966648136\\
    2.75	2.95713584190443\\
    2.8	2.91243176788214\\
    2.85	2.86460645344058\\
    2.9	2.81332930277019\\
    2.95	2.75820688858728\\
    3	2.69876486716525\\
    3.05	2.63442267416152\\
    3.1	2.56445716036143\\
    3.15	2.48794866615549\\
    3.2	2.40369800276695\\
    3.25	2.31009269819094\\
    3.3	2.20487897896575\\
    3.35	2.08474393357267\\
    3.4	1.94447262241857\\
    3.45	1.77500165310962\\
    3.5	1.5578855860124\\
    3.55	1.24227232487739\\
    3.6	0\\
    };
    \addplot [color=black, forget plot]
      table[row sep=crcr]{%
    3.6	-0\\
    3.65	-1.25382826354938\\
    3.7	-1.58700410245128\\
    3.75	-1.8249984362281\\
    3.8	-2.01784038710825\\
    3.85	-2.18352884962524\\
    3.9	-2.33083860350175\\
    3.95	-2.46477946918048\\
    4	-2.58850945078415\\
    4.05	-2.70416025311281\\
    4.1	-2.81324507009292\\
    4.15	-2.91688034715677\\
    4.2	-3.01591541718809\\
    4.25	-3.11101268703176\\
    4.3	-3.20269954491915\\
    4.35	-3.29140324406586\\
    4.4	-3.37747509342703\\
    4.45	-3.46120768760821\\
    4.5	-3.54284746479196\\
    4.55	-3.62260404495532\\
    4.6	-3.70065729739648\\
    4.65	-3.77716277394065\\
    4.7	-3.85225594439255\\
    4.75	-3.92605553987667\\
    4.8	-3.99866622197514\\
    4.85	-4.07018073559454\\
    4.9	-4.14068166173797\\
    4.95	-4.21024285680844\\
    5	-4.27893064384067\\
    };
    
    \addplot[area legend, draw=black, fill=black, forget plot]
    table[row sep=crcr] {%
    x	y\\
    2.99971246073864	2.71493132634688\\
    2.99971246073864	2.71493132634688\\
    2.99971246073864	2.71493132634688\\
    2.99971246073864	2.71493132634688\\
    2.99971246073864	2.71493132634688\\
    2.99971246073864	2.71493132634688\\
    2.99971246073864	2.71493132634688\\
    2.93020808404376	2.645426949652\\
    2.75732189354276	2.95732189354276\\
    3.06921683743352	2.78443570304176\\
    2.99971246073864	2.71493132634688\\
    }--cycle;
    \addplot [color=black, forget plot]
      table[row sep=crcr]{%
    -5	5.73879354831717\\
    -4.95	5.70096808221258\\
    -4.9	5.66340565930516\\
    -4.85	5.6261123726007\\
    -4.8	5.58909442225448\\
    -4.75	5.55235811380267\\
    -4.7	5.51590985603062\\
    -4.65	5.47975615845403\\
    -4.6	5.44390362838836\\
    -4.55	5.4083589675817\\
    -4.5	5.37312896838572\\
    -4.45	5.33822050943947\\
    -4.4	5.30364055084068\\
    -4.35	5.26939612877947\\
    -4.3	5.23549434960994\\
    -4.25	5.20194238333536\\
    -4.2	5.16874745648424\\
    -4.15	5.13591684435504\\
    -4.1	5.1034578626093\\
    -4.05	5.07137785819438\\
    -4	5.03968419957949\\
    -3.95	5.00838426629069\\
    -3.9	4.97748543773394\\
    -3.85	4.94699508129806\\
    -3.8	4.9169205397334\\
    -3.75	4.88726911780576\\
    -3.7	4.85804806822989\\
    -3.65	4.82926457689136\\
    -3.6	4.80092574737107\\
    -3.55	4.77303858479212\\
    -3.5	4.74560997901466\\
    -3.45	4.71864668721076\\
    -3.4	4.69215531585751\\
    -3.35	4.66614230219372\\
    -3.3	4.64061389519198\\
    -3.25	4.61557613610522\\
    -3.2	4.59103483865373\\
    -3.15	4.5669955689254\\
    -3.1	4.54346362506894\\
    -3.05	4.52044401686613\\
    -3	4.49794144527541\\
    -2.95	4.47596028204469\\
    -2.9	4.45450454949628\\
    -2.85	4.43357790059152\\
    -2.8	4.41318359938584\\
    -2.75	4.39332450198799\\
    -2.7	4.3740030381387\\
    -2.65	4.3552211935248\\
    -2.6	4.33698049294394\\
    -2.55	4.31928198443375\\
    -2.5	4.30212622447582\\
    -2.45	4.28551326438077\\
    -2.4	4.26944263795511\\
    -2.35	4.25391335054368\\
    -2.3	4.23892386953327\\
    -2.25	4.22447211639401\\
    -2.2	4.21055546032458\\
    -2.15	4.19717071355611\\
    -2.1	4.18431412835758\\
    -2.05	4.17198139577253\\
    -2	4.16016764610381\\
    -1.95	4.14886745114917\\
    -1.9	4.13807482817678\\
    -1.85	4.12778324561581\\
    -1.8	4.11798563042365\\
    -1.75	4.10867437707794\\
    -1.7	4.09984135812903\\
    -1.65	4.09147793623635\\
    -1.6	4.08357497760117\\
    -1.55	4.07612286669812\\
    -1.5	4.06911152219885\\
    -1.45	4.06253041397349\\
    -1.4	4.05636858104921\\
    -1.35	4.05061465039998\\
    -1.3	4.04525685643787\\
    -1.25	4.04028306107407\\
    -1.2	4.03568077421651\\
    -1.15	4.0314371745714\\
    -1.1	4.02753913061748\\
    -1.05	4.02397322162419\\
    -1	4.02072575858906\\
    -0.95	4.01778280497382\\
    -0.899999999999999	4.01513019712441\\
    -0.85	4.01275356426607\\
    -0.8	4.01063834797167\\
    -0.75	4.0087698210081\\
    -0.7	4.00713310547364\\
    -0.649999999999999	4.00571319014622\\
    -0.6	4.00449494697088\\
    -0.55	4.00346314662189\\
    -0.5	4.00260247308292\\
    -0.45	4.00189753719581\\
    -0.4	4.00133288913564\\
    -0.350000000000001	4.00089302977628\\
    -0.3	4.00056242091697\\
    -0.25	4.00032549434597\\
    -0.2	4.0001666597227\\
    -0.15	4.00007031126407\\
    -0.100000000000001	4.00002083322483\\
    -0.0499999999999998	4.00000260416497\\
    0	4\\
    0.0499999999999998	3.99999739583164\\
    0.0999999999999996	3.99997916655816\\
    0.15	3.999929686264\\
    0.2	3.99983332638841\\
    0.25	3.99967445267212\\
    0.3	3.99943742087989\\
    0.35	3.99910657129448\\
    0.4	3.99866622197514\\
    0.45	3.99810066077037\\
    0.5	3.99739413607028\\
    0.55	3.9965308462796\\
    0.6	3.9954949279864\\
    0.65	3.99427044279541\\
    0.7	3.99284136278774\\
    0.75	3.99119155456048\\
    0.8	3.98930476179092\\
    0.85	3.9871645862595\\
    0.9	3.98475446725496\\
    0.95	3.98205765927177\\
    1	3.97905720789639\\
    1.05	3.97573592376282\\
    1.1	3.97207635444015\\
    1.15	3.96806075409521\\
    1.2	3.96367105075071\\
    1.25	3.95888881093441\\
    1.3	3.95369520148593\\
    1.35	3.94807094825566\\
    1.4	3.94199629139342\\
    1.45	3.93545093688214\\
    1.5	3.928414003924\\
    1.55	3.92086396773085\\
    1.6	3.912778597207\\
    1.65	3.90413488693867\\
    1.7	3.89490898281869\\
    1.75	3.88507610053522\\
    1.8	3.87461043603686\\
    1.85	3.8634850669495\\
    1.9	3.85167184375929\\
    1.95	3.839141269386\\
    2	3.82586236554478\\
    2.05	3.81180252402532\\
    2.1	3.79692734069548\\
    2.15	3.78120042964864\\
    2.2	3.7645832144467\\
    2.25	3.74703469284268\\
    2.3	3.72851117067382\\
    2.35	3.70896595976667\\
    2.4	3.68834903364676\\
    2.45	3.66660663354419\\
    2.5	3.64368081556092\\
    2.55	3.61950892782091\\
    2.6	3.59402300383506\\
    2.65	3.56714905500759\\
    2.7	3.53880624095771\\
    2.75	3.50890589081016\\
    2.8	3.47735034137574\\
    2.85	3.4440315485722\\
    2.9	3.40882941563296\\
    2.95	3.37160976432446\\
    3	3.33222185164595\\
    3.05	3.2904953014888\\
    3.1	3.24623627419026\\
    3.15	3.19922263018076\\
    3.2	3.14919774648174\\
    3.25	3.09586249965226\\
    3.3	3.03886470765518\\
    3.35	2.97778497771623\\
    3.4	2.91211735227267\\
    3.45	2.84124222420702\\
    3.5	2.76438740683944\\
    3.55	2.68057039355508\\
    3.6	2.58850945078415\\
    3.65	2.48648035278273\\
    3.7	2.37207210759166\\
    3.75	2.24173925559838\\
    3.8	2.08989857798625\\
    3.85	1.90684282377084\\
    3.9	1.67280845040534\\
    3.95	1.33328732480132\\
    4	0\\
    };
    \addplot [color=black, forget plot]
      table[row sep=crcr]{%
    4	-0\\
    4.05	-1.34444447606544\\
    4.1	-1.70092222164965\\
    4.15	-1.95511477752707\\
    4.2	-2.16073591750959\\
    4.25	-2.3371182901163\\
    4.3	-2.49369075744464\\
    4.35	-2.63583878230465\\
    4.4	-2.76695856664369\\
    4.45	-2.88934356885749\\
    4.5	-3.00462250345868\\
    4.55	-3.11399757505469\\
    4.6	-3.2183837676756\\
    4.65	-3.31849501754868\\
    4.7	-3.41490000543241\\
    4.75	-3.50805965759495\\
    4.8	-3.59835315622033\\
    4.85	-3.68609646863451\\
    4.9	-3.77155585461804\\
    4.95	-3.85495791238075\\
    5	-3.93649718310217\\
    };
    
    \addplot[area legend, draw=black, fill=black, forget plot]
    table[row sep=crcr] {%
    x	y\\
    3.32093593264383	3.02866827522897\\
    3.32093593264383	3.02866827522896\\
    3.32093593264383	3.02866827522896\\
    3.32093593264383	3.02866827522896\\
    3.32093593264383	3.02866827522896\\
    3.32093593264383	3.02866827522896\\
    3.32093593264383	3.02866827522897\\
    3.25035819298282	2.95809053556796\\
    3.0748021039364	3.2748021039364\\
    3.39151367230484	3.09924601488997\\
    3.32093593264383	3.02866827522897\\
    }--cycle;
    \addplot [color=black, forget plot]
      table[row sep=crcr]{%
    -5	5.94565744858888\\
    -4.95	5.91044186522429\\
    -4.9	5.87551859010587\\
    -4.85	5.84089337250893\\
    -4.8	5.80657201927403\\
    -4.75	5.77256039081165\\
    -4.7	5.73886439674557\\
    -4.65	5.70548999118167\\
    -4.6	5.67244316758968\\
    -4.55	5.63972995328585\\
    -4.5	5.6073564035055\\
    -4.45	5.57532859505547\\
    -4.4	5.54365261953744\\
    -4.35	5.51233457613462\\
    -4.3	5.48138056395562\\
    -4.25	5.45079667393119\\
    -4.2	5.42058898026123\\
    -4.15	5.39076353141155\\
    -4.1	5.36132634066212\\
    -4.05	5.332283376211\\
    -4	5.30364055084068\\
    -3.95	5.27540371115633\\
    -3.9	5.24757862640856\\
    -3.85	5.22017097691613\\
    -3.8	5.19318634210752\\
    -3.75	5.16663018820354\\
    -3.7	5.14050785556663\\
    -3.65	5.11482454574594\\
    -3.6	5.0895853082511\\
    -3.55	5.06479502709083\\
    -3.5	5.04045840711649\\
    -3.45	5.01657996021387\\
    -3.4	4.99316399138997\\
    -3.35	4.97021458480507\\
    -3.3	4.94773558980296\\
    -3.25	4.92573060699547\\
    -3.2	4.90420297445978\\
    -3.15	4.88315575410921\\
    -3.1	4.86259171830001\\
    -3.05	4.84251333673835\\
    -3	4.82292276375219\\
    -2.95	4.80382182599364\\
    -2.9	4.78521201063719\\
    -2.85	4.76709445413835\\
    -2.8	4.74946993161649\\
    -2.75	4.73233884692341\\
    -2.7	4.71570122345719\\
    -2.65	4.69955669577749\\
    -2.6	4.68390450207526\\
    -2.55	4.66874347754531\\
    -2.5	4.65407204870588\\
    -2.45	4.63988822870376\\
    -2.4	4.62618961363815\\
    -2.35	4.61297337993023\\
    -2.3	4.60023628275897\\
    -2.25	4.58797465557711\\
    -2.2	4.57618441071419\\
    -2.15	4.56486104106651\\
    -2.1	4.55399962286679\\
    -2.05	4.54359481951908\\
    -2	4.53364088647765\\
    -1.95	4.52413167714154\\
    -1.9	4.51506064972999\\
    -1.85	4.50642087509766\\
    -1.8	4.49820504544276\\
    -1.75	4.49040548385564\\
    -1.7	4.4830141546505\\
    -1.65	4.47602267441874\\
    -1.6	4.46942232373823\\
    -1.55	4.4632040594701\\
    -1.5	4.45735852757163\\
    -1.45	4.45187607635236\\
    -1.4	4.44674677009875\\
    -1.35	4.44196040299268\\
    -1.3	4.43750651324839\\
    -1.25	4.43337439739352\\
    -1.2	4.42955312462073\\
    -1.15	4.426031551138\\
    -1.1	4.42279833444796\\
    -1.05	4.41984194748881\\
    -1	4.41715069257253\\
    -0.95	4.41471271505911\\
    -0.899999999999999	4.41251601670898\\
    -0.85	4.41054846865957\\
    -0.8	4.40879782397564\\
    -0.75	4.40725172972693\\
    -0.7	4.40589773855084\\
    -0.649999999999999	4.40472331966154\\
    -0.6	4.40371586927097\\
    -0.55	4.40286272039121\\
    -0.5	4.40215115199107\\
    -0.45	4.40156839748363\\
    -0.399999999999999	4.40110165252475\\
    -0.35	4.40073808210567\\
    -0.3	4.40046482692586\\
    -0.25	4.40026900903497\\
    -0.2	4.4001377367351\\
    -0.149999999999999	4.40005810873671\\
    -0.0999999999999996	4.40001721756348\\
    -0.0499999999999998	4.4000021522028\\
    0	4.4\\
    0.0499999999999998	4.39999784779509\\
    0.100000000000001	4.39998278230177\\
    0.15	4.39994188972842\\
    0.2	4.39986225464099\\
    0.25	4.39973095806743\\
    0.3	4.39953507484244\\
    0.350000000000001	4.39926167019068\\
    0.4	4.39889779554507\\
    0.45	4.39843048359551\\
    0.5	4.39784674256115\\
    0.55	4.39713354967731\\
    0.6	4.39627784388566\\
    0.65	4.39526651771311\\
    0.7	4.39408640832168\\
    0.75	4.39272428770812\\
    0.8	4.39116685202757\\
    0.85	4.38940071001136\\
    0.9	4.38741237044377\\
    0.95	4.38518822865711\\
    1	4.3827145519982\\
    1.05	4.37997746421271\\
    1.1	4.37696292868603\\
    1.15	4.37365673047123\\
    1.2	4.37004445702512\\
    1.25	4.36611147756342\\
    1.3	4.3618429209344\\
    1.35	4.35722365189759\\
    1.4	4.35223824567986\\
    1.45	4.34687096066515\\
    1.5	4.34110570905614\\
    1.55	4.33492602532579\\
    1.6	4.32831503225368\\
    1.65	4.3212554043164\\
    1.7	4.31372932817138\\
    1.75	4.30571845994028\\
    1.8	4.29720387895949\\
    1.85	4.28816603762167\\
    1.9	4.27858470688149\\
    1.95	4.26843891694111\\
    2	4.2577068925634\\
    2.05	4.24636598238375\\
    2.1	4.23439258150042\\
    2.15	4.22176204651851\\
    2.2	4.20844860209926\\
    2.25	4.19442523792143\\
    2.3	4.17966359479151\\
    2.35	4.16413383843723\\
    2.4	4.1478045192799\\
    2.45	4.1306424161952\\
    2.5	4.11261236193034\\
    2.55	4.09367704743449\\
    2.6	4.07379680186249\\
    2.65	4.05292934440889\\
    2.7	4.03102950339355\\
    2.75	4.00804889711701\\
    2.8	3.98393556988968\\
    2.85	3.95863357525606\\
    2.9	3.93208249670732\\
    2.95	3.90421689400247\\
    3	3.87496566046572\\
    3.05	3.84425127311386\\
    3.1	3.81198891294557\\
    3.15	3.77808542685497\\
    3.2	3.74243809494084\\
    3.25	3.70493315680415\\
    3.3	3.66544403681055\\
    3.35	3.62382918986462\\
    3.4	3.57992946398274\\
    3.45	3.53356484084942\\
    3.5	3.48453036602697\\
    3.55	3.4325910094841\\
    3.6	3.37747509342703\\
    3.65	3.3188657699515\\
    3.7	3.25638979571086\\
    3.75	3.18960248319805\\
    3.8	3.11796711754744\\
    3.85	3.04082614752338\\
    3.9	2.95735977133685\\
    3.95	2.86652450319574\\
    4	2.76695856664369\\
    4.05	2.65682942377063\\
    4.1	2.53357376911577\\
    4.15	2.39342092816138\\
    4.2	2.23043112043839\\
    4.25	2.03427387882488\\
    4.3	1.7839101015784\\
    4.35	1.42129436301186\\
    4.4	0\\
    };
    \addplot [color=black, forget plot]
      table[row sep=crcr]{%
    4.4	-0\\
    4.45	-1.43210263327108\\
    4.5	-1.81114484514709\\
    4.55	-2.08103675817557\\
    4.6	-2.2990544317303\\
    4.65	-2.48581971993629\\
    4.7	-2.65139359475723\\
    4.75	-2.80152446746804\\
    4.8	-2.939838638621\\
    4.85	-3.06878578458167\\
    4.9	-3.19010616025605\\
    4.95	-3.30508475380455\\
    5	-3.41469990581837\\
    };
    
    \addplot[area legend, draw=black, fill=black, forget plot]
    table[row sep=crcr] {%
    x	y\\
    3.64268943741856	3.34187519124151\\
    3.64268943741856	3.34187519124151\\
    3.64268943741856	3.34187519124151\\
    3.64268943741856	3.34187519124151\\
    3.64268943741856	3.34187519124151\\
    3.64268943741856	3.34187519124151\\
    3.64268943741856	3.34187519124151\\
    3.57088635031179	3.27007210413474\\
    3.39228231433004	3.59228231433004\\
    3.71449252452534	3.41367827834829\\
    3.64268943741856	3.34187519124151\\
    }--cycle;
    \addplot [color=black, forget plot]
      table[row sep=crcr]{%
    -5	6.17618334176118\\
    -4.95	6.14356852934671\\
    -4.9	6.11126646681883\\
    -4.85	6.07928226357011\\
    -4.8	6.04762103949539\\
    -4.75	6.01628791985994\\
    -4.7	5.98528802988077\\
    -4.65	5.9546264890184\\
    -4.6	5.92430840497767\\
    -4.55	5.89433886741701\\
    -4.5	5.86472294136692\\
    -4.45	5.83546566035971\\
    -4.4	5.80657201927403\\
    -4.35	5.77804696689878\\
    -4.3	5.74989539822316\\
    -4.25	5.72212214646078\\
    -4.2	5.6947319748176\\
    -4.15	5.66772956801541\\
    -4.1	5.64111952358409\\
    -4.05	5.61490634293814\\
    -4	5.58909442225448\\
    -3.95	5.56368804317094\\
    -3.9	5.53869136332627\\
    -3.85	5.51410840676505\\
    -3.8	5.48994305423219\\
    -3.75	5.46619903338409\\
    -3.7	5.44287990894502\\
    -3.65	5.41998907283918\\
    -3.6	5.3975297343305\\
    -3.55	5.37550491020363\\
    -3.5	5.35391741502115\\
    -3.45	5.33276985149294\\
    -3.4	5.31206460099498\\
    -3.35	5.29180381427528\\
    -3.3	5.27198940238559\\
    -3.25	5.25262302787746\\
    -3.2	5.23370609630174\\
    -3.15	5.2152397480499\\
    -3.1	5.19722485057551\\
    -3.05	5.17966199103314\\
    -3	5.16255146937099\\
    -2.95	5.14589329191211\\
    -2.9	5.12968716545745\\
    -2.85	5.11393249194193\\
    -2.8	5.09862836367258\\
    -2.75	5.08377355917515\\
    -2.7	5.06936653967282\\
    -2.65	5.05540544621769\\
    -2.6	5.04188809749251\\
    -2.55	5.02881198829654\\
    -2.5	5.01617428872595\\
    -2.45	5.00397184405552\\
    -2.4	4.99220117532457\\
    -2.35	4.98085848062601\\
    -2.3	4.96993963709383\\
    -2.25	4.95944020358025\\
    -2.2	4.94935542401002\\
    -2.15	4.93968023139562\\
    -2.1	4.93040925249353\\
    -2.05	4.92153681307815\\
    -2	4.91305694380694\\
    -1.95	4.90496338664699\\
    -1.9	4.89724960183065\\
    -1.85	4.88990877530508\\
    -1.8	4.88293382663862\\
    -1.75	4.87631741734448\\
    -1.7	4.87005195958094\\
    -1.65	4.86412962518571\\
    -1.6	4.85854235500106\\
    -1.55	4.8532818684457\\
    -1.5	4.84833967328888\\
    -1.45	4.84370707558215\\
    -1.4	4.83937518970431\\
    -1.35	4.8353349484757\\
    -1.3	4.83157711329864\\
    -1.25	4.82809228428169\\
    -1.2	4.82487091030687\\
    -1.15	4.82190329900017\\
    -1.1	4.81917962656736\\
    -1.05	4.81668994745888\\
    -1	4.81442420382937\\
    -0.95	4.8123722347595\\
    -0.899999999999999	4.81052378520972\\
    -0.85	4.80886851467779\\
    -0.8	4.80739600553389\\
    -0.75	4.80609577100958\\
    -0.7	4.80495726281873\\
    -0.649999999999999	4.8039698783909\\
    -0.6	4.8031229676995\\
    -0.55	4.80240583966936\\
    -0.5	4.80180776814992\\
    -0.45	4.80131799744245\\
    -0.399999999999999	4.80092574737107\\
    -0.35	4.80062021788932\\
    -0.3	4.80039059321517\\
    -0.25	4.80022604548897\\
    -0.199999999999999	4.80011573795004\\
    -0.149999999999999	4.8000488276283\\
    -0.0999999999999996	4.80001446754899\\
    -0.0499999999999998	4.80000180844839\\
    0	4.8\\
    0.0500000000000007	4.79999819155024\\
    0.100000000000001	4.7999855323638\\
    0.15	4.79995117137829\\
    0.2	4.79988425646833\\
    0.250000000000001	4.79977393321878\\
    0.300000000000001	4.79960934320654\\
    0.350000000000001	4.79937962178998\\
    0.4	4.79907389540437\\
    0.45	4.79868127836094\\
    0.500000000000001	4.79819086914608\\
    0.550000000000001	4.7975917462165\\
    0.600000000000001	4.79687296328433\\
    0.65	4.79602354408526\\
    0.7	4.79503247662074\\
    0.750000000000001	4.79388870686385\\
    0.800000000000001	4.79258113191607\\
    0.850000000000001	4.79109859260026\\
    0.9	4.78942986547258\\
    0.950000000000001	4.78756365423345\\
    1	4.78548858051474\\
    1.05	4.78319317401706\\
    1.1	4.78066586196769\\
    1.15	4.77789495786555\\
    1.2	4.77486864947567\\
    1.25	4.77157498603078\\
    1.3	4.76800186459246\\
    1.35	4.76413701551886\\
    1.4	4.75996798697958\\
    1.45	4.75548212845142\\
    1.5	4.7506665731213\\
    1.55	4.7455082191138\\
    1.6	4.73999370945179\\
    1.65	4.73410941064774\\
    1.7	4.72784138981183\\
    1.75	4.72117539014982\\
    1.8	4.7140968047088\\
    1.85	4.70659064821255\\
    1.9	4.69864152680957\\
    1.95	4.69023360553561\\
    2	4.68135057326899\\
    2.05	4.67197560492966\\
    2.1	4.66209132064226\\
    2.15	4.65167974154845\\
    2.2	4.64072224191325\\
    2.25	4.62919949712464\\
    2.3	4.61709142713204\\
    2.35	4.60437713480893\\
    2.4	4.59103483865373\\
    2.45	4.57704179916166\\
    2.5	4.56237423810478\\
    2.55	4.54700724984663\\
    2.6	4.5309147036877\\
    2.65	4.51406913608526\\
    2.7	4.49644163141122\\
    2.75	4.47800168969859\\
    2.8	4.45871707957417\\
    2.85	4.43855367427352\\
    2.9	4.41747526827306\\
    2.95	4.39544337163964\\
    3	4.37241697867311\\
    3.05	4.34835230677977\\
    3.1	4.32320250073683\\
    3.15	4.29691729655335\\
    3.2	4.26944263795511\\
    3.25	4.24072023705863\\
    3.3	4.21068706897219\\
    3.35	4.17927478776402\\
    3.4	4.14640904832589\\
    3.45	4.11200871494441\\
    3.5	4.07598493260902\\
    3.55	4.0382400308764\\
    3.6	3.99866622197514\\
    3.65	3.95714404406568\\
    3.7	3.91354048616405\\
    3.75	3.86770671173796\\
    3.8	3.8194752712541\\
    3.85	3.76865665681006\\
    3.9	3.71503499958271\\
    3.95	3.65836263567708\\
    4	3.59835315622033\\
    4.05	3.53467239404558\\
    4.1	3.46692655020797\\
    4.15	3.39464627442907\\
    4.2	3.31726488820732\\
    4.25	3.23408790150043\\
    4.3	3.14424918656787\\
    4.35	3.04664595911424\\
    4.4	2.939838638621\\
    4.45	2.82188944180945\\
    4.5	2.69008710671805\\
    4.55	2.54044224221422\\
    4.6	2.36666887081282\\
    4.65	2.15783154291702\\
    4.7	1.89165313589352\\
    4.75	1.50665561007714\\
    4.8	0\\
    };
    \addplot [color=black, forget plot]
      table[row sep=crcr]{%
    4.8	-0\\
    4.85	-1.51715490996357\\
    4.9	-1.91810939473754\\
    4.95	-2.20325791411787\\
    5	-2.43333124830413\\
    };
    
    \addplot[area legend, draw=black, fill=black, forget plot]
    table[row sep=crcr] {%
    x	y\\
    3.96493174019612	3.65459330925124\\
    3.96493174019612	3.65459330925124\\
    3.96493174019612	3.65459330925124\\
    3.96493174019612	3.65459330925124\\
    3.96493174019612	3.65459330925124\\
    3.96493174019612	3.65459330925124\\
    3.96493174019612	3.65459330925124\\
    3.8917631450717	3.58142471412683\\
    3.70976252472368	3.90976252472368\\
    4.03810033532053	3.72776190437566\\
    3.96493174019612	3.65459330925124\\
    }--cycle;
    \end{axis}
    \end{tikzpicture}%
            \end{center}
            \caption{Orbite per il problema di Cauchy}\label{fig:orbiteradiceterza}
        \end{figure}
    \end{enumerate}
}
\paragrafo{Esempio}{%
\[
    \begin{cases}
        x'=y^{2}\\ 
        y'=-xy
    \end{cases}    
\]
\begin{enumerate}
    \item \emph{Equilibri}
    
    Tutti i punti $ (x,0) $ sono equilibri.
    \item \emph{Tangente orizzontale e verticale}
    
    Tutta l'asse delle $ y $ è composta da punti a tangente orizzontale. 
    \item \emph{Orbite non singolari}
    
    Consideriamo $ f_1(x,y)\neq 0 $, ovvero quelli per i quali $ y\neq 0 $: troviamo tutte le orbite, infatti i punti $ (x,0) $ sono equilibri. Calcolando: \[
        \varphi'(x)=-\frac{x\,\varphi(x)}{\varphi^{2}(x)}=-\frac{x}{\varphi(x)}
    \]Da qui, possiamo dire \[
        \varphi(x)\,\varphi'(x)=-x \quad \implies \quad \varphi^{2}(x)=-x^{2}+c
    \]
    
    $c>0$, e $ \displaystyle \varphi(x)=\pm (c-x^{2})^{1/2}$. 

    Inoltre $ x'=y^{2}>0 $ lungo le orbite. Dunque il diagramma di fase è quello illustrato in figure \ref{fig:eqwsdcs}. 
    \begin{figure}
        \begin{center}
            \begin{tikzpicture}
                \draw [-Latex] (-3,0) -- (3,0);
                \draw [-Latex] (0,-3) -- (0,3);
                \draw [ultra thick] (-1.8,0) -- (1.8,0);
                \draw [ultra thick, dashed] (-1.8,0) -- (-2.9,0);
                \draw [ultra thick, dashed] (1.8,0) -- (2.9,0);
                \foreach \r in{0.5,1,...,2.5}{
                    \draw (0,0) circle (\r);
                    \fill [white] (\r,0) circle (0.05);
                    \draw (\r,0) circle (0.055);
                    \fill [white] (-\r,0) circle (0.05);
                    \draw (-\r,0) circle (0.055);
                    \draw [-Latex] (0,\r) -- (0.1, \r);
                    \draw [-Latex] (0,-\r) -- (0.1, -\r);
                };
            \end{tikzpicture}
        \end{center}
        \caption{Diagramma di fase per l'esempio \framref{jjdjdjdkjnsjdkjncskjndsckjn}}\label{fig:eqwsdcs}
    \end{figure}
\end{enumerate}
}{jjdjdjdkjnsjdkjncskjndsckjn}{}