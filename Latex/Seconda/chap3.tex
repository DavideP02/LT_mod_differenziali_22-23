\chapter{Il modello della visione}
\section{Introduzione} Il modello della visione di Jean Petitot è un tentativo di rappresentare tramite la modellizzazione matematica come gli oggetti ed enti del mondo esterno vengano recepiti, codificati  e rappresentati dalla corteccia visuale del nostro cervello, in particolare si sofferma su una modellizzazione del primo stadio di rappresentazione degli oggetti esterni, il cosiddetto $V1$.
Ci si chiede appunto come enti geometrici esterni semplici come punti o anche più complessi come linee e forme possano essere interpretate e codificate dal nostro apparato neuro-visivo.
\section{Il modello}
\paragraph*{Def}La struttura geometrica più importante definita sulla mappa delle fibre che modella il funzionamento ottico di $V1$ è chiamata \emph{struttura di contatto}, denotata con $\mathcal{C}$.\\
Il modello geometrico della visione di Petito rappresenta le connessioni neuronali retina-corteccia visiva nel seguente modo:
\begin{align*}
    I\xrightarrow[]{\gamma}A \xrightarrow[]{\mathbf{X}}A\times E\xrightarrow[]{\dot{\gamma}}A'\xrightarrow[]{\mathbf{X}}A'\times E'
\end{align*}
Dove:
\begin{align*}
    \gamma \colon I \to A\\
    t\mapsto (x,y)
\end{align*}
che poi viene inviato tramite $\mathbf{X}$:
\begin{align*}
    \mathbf{X}\colon A \to A\times E\\
    (x,y)\mapsto (x,y,\dot{x},\dot{y})
\end{align*}
Successivamente:
\begin{align*}
    \dot{\gamma}\colon A\times E\to A'\\
    (x,y,\dot{x},\dot{y})\mapsto (x,y,p=\dot{y})
\end{align*}
E infine tramite $\mathbf{X}\in ker\omega$, con $\omega=dy-pdx$:
\begin{align*}
    \mathbf{X}\colon A'\to A'\times E'\\
    (x,y,p=\dot{y})\mapsto (x,y,p=\dot{y},\dot{x}=1,\dot{y}=p,\dot{p}=\ddot{y})
\end{align*}
Dove $\Gamma$ è la curva geodetica per $g_{\mathcal{C}}$ ed è definita come:
\begin{align*}
    \Gamma= \dot{\gamma}_{|(\dot{x}=1,\dot{y}=p)} && \textit{lift di Legendre}
\end{align*}
e invece:
\begin{align*}
    g_{\mathcal{C}}(\mathbf{t}_i,\mathbf{t}_j)=\delta_{ij}&& i=1,2
\end{align*}
e $\{\mathbf{t}_1,\mathbf{t}_2\}$ che generano il $\ker\omega$.


\paragraph{andrea donati}
Sia $\bm{X}$ un campo vettoriale e $\Gamma$ una curva integrale di $\bm{X}$, dove:
\begin{align*}
    \bm{X}\colon A' \to TA'\\
    (x,y,p)\mapsto (\xi, \eta, \pi)
\end{align*}
dove con $TA'$ indichiamo il fibrato tangente di $A'$.
Defiamo inoltre:
\begin{align*}
    \dot{\Gamma}=(\bm{X}\circ \Gamma)=\dot{x}\partial_x+\dot{y}\partial_y+\dot{p}\partial_p
\end{align*}
Supponiamo che:
\begin{align*}
    \begin{cases*}
        \dot{x}=1\\
        \dot{y}=p
    \end{cases*}
\end{align*}
lungo $\Gamma$, si ha dunque che:
\begin{align*}
    \dot{\Gamma}=\partial_x+p\partial_y+\dot{p}\partial_p=\partial_x+\dot{y}\partial_y+\ddot{y}\partial_y
\end{align*}
Ora, considerata la forma $w=-pdx+dy$, si noti che lungo $\Gamma$ $w(\bm{x})=0$ e in generale il $\ker w$ definisce una struttura di contatto su $A'$:
\begin{align*}
    \bm{x}\in \ker w \iff \eta-p\xi=0
\end{align*}
dove il membro a destra definisce un piano di contatto.

Un tale campo vettoriale può essere scritto nel seguente modo:
\begin{align*}
    \xi (\partial_x +p\partial_y)+\pi\partial_p = 0 = \xi\partial_x+\overbrace{\xi p}^{\eta}\partial_y+\pi\partial_p
\end{align*}
Ora definiamo i vettori:
\begin{align*}
    \bm{t}_1=\partial_x+p\partial_y \quad \bm{t}_2=\partial_p \quad \bm{t}_1\perp\bm{t}_2
\end{align*}
Creando cosi una coppia $\left\{\bm{t}_1,\bm{t}_2\right\}$ di vettori ortonormali (ortogonali sicuramente, ci basta normalizzarli successivamente).

Lungo $\Gamma$ $\dot{y}=p\dot{x}$, ma $\dot{x}=1\leadsto x=s+c$.

Possiamo così riparametrizzare $\Gamma$ per il parametro $x$ e vedere $\Gamma$ come l'indicatrice delle tangenti di una curva $\gamma$, dove quindi:
\begin{align*}
    \gamma&=(x,y(x)) \\
    \Gamma\colon A'&\longrightarrow A'\times E \\
     (x,y,p)&\longleftrightarrow (x,y,\dot{x},\dot{y})
\end{align*}

Esistono solo curve integrali di dimensione 1 dal teorema di Frobenius che dice che $dw \wedge w=0\Rightarrow$ $\nexists$ superfici integrali della distribuzione di contatto.

Le proprietà di $w$ permettono di definire una metrica sulla struttura di contatto detta \ul{metrica di Carnot-Caratheodory} denotata con $g_\mathcal{C}$.

Le curve geodetiche rispetto a $g_\mathcal{C}$ sono curve $\Gamma$ tangenti a $\mathcal{C}$ (struttura di contatto), ovvero curve integrali del campo di partenza e tali che:
\begin{align*}
    \begin{cases*}
        \dot{x}=1\\
        \dot{y}=p
    \end{cases*}\text{sollevamento di Legendre (non lui eh, qualcosa qui vicino)}
\end{align*}
Definiamo con$A'=$corteccia visiva primaria ($V_1$).

Stiamo studiando il modello di Petitot. Riprendendo il discorso fatto, definiamo la metrica di Carnot-Caratheodory come:
\begin{align*}
    g_{\mathcal{C}}(\bm{t}_1,\bm{t}_j)=\delta_{ij}\quad i,j\in \left\{1,2\right\}
\end{align*}
$A'$ è un gruppo di Lie con le operazioni:
\begin{enumerate}
    \item $(x,y,p)\cdot (x',y',p')=(x+x',y+y',p+p')$
    \item $[(\xi,\eta,\pi),(\xi',\eta',\pi')]=(0,\xi'\pi-\xi\pi',0)$
\end{enumerate}
Tramite questa metrica definiamo la distanza:
\begin{align*}
    d_{\mathcal{C}}({P'}_1,{P'}_2)=inf\left\{\int_I \norma{\dot{\Gamma}(s)}\dif s\right\}
\end{align*}
Estendiamo la nostra base ortonormale $\left\{\bm{t}_1,\bm{t}_2\right\}$ a una base $\left\{\bm{t}_1,\bm{t}_2,\bm{t}_3\right\}$ con un opportuno $\bm{t}_3$ tale che si mantenga l'ortonormalità della base.
Inoltre $\left\{\bm{t}_1,\bm{t}_2\right\}$ è una metrica sub-Riemaniana.

Ci poniamo l'obiettivo di minimizzare:
\begin{align*}
    \int_{{P'}_A}^{{P'}_{B}} \dif s \quad \quad \dif s^2=\dif x^2+\dif y^2+\dif p^2
\end{align*}
con norma del vettore tangente lungo $\Gamma$ che è:
\begin{align*}
    \sqrt{\xi^2+\eta^2+\pi^2}= \sqrt{1+\dot{y}^2(x)+\ddot{y}^2(x)}
\end{align*}


Definiamo:
\begin{align*}
    L(x,y,p,\dot{x},\dot{y},\dot{p})=\sqrt{\xi^2+\eta^2+\pi^2}=\sqrt{1+\dot{y}^2(x)+\ddot{y}^2(x)}
\end{align*}
e usiamo le equazioni di Eulero-Lagrange:
\begin{align*}
    \begin{cases*}
        \cfrac{\partial}{\partial y}L-\cfrac{d}{dx}\left(\cfrac{\partial}{\partial \eta}\right)L=0 \quad \leadsto \cfrac{\partial}{\partial y}L=0\\
        \cfrac{\partial}{\partial p}L-\cfrac{d}{dx}\left(\cfrac{\partial}{\partial \pi}\right)L=0 \quad \leadsto \cfrac{\partial}{\partial y}L=0
    \end{cases*}
\end{align*}
I due annullamenti sono perchè $L$ non dipende da $y$ e da $p$.

Poniamo $\Sigma=\xi p-\eta$, ma $\xi=1$ lungo $\Gamma$ quindi $\Sigma=p-\eta$.
Questo nuovo oggetto ci serve per imporre il vincolo del piano di contatto:
\begin{align*}
    \begin{cases*}
        \left(\cfrac{\partial}{\partial y}-\cfrac{d}{dx}\cfrac{\partial}{\partial \eta}\right)(L+\overbrace{\lambda(x)}^{\text{moltiplicatore di Lagrange}}\Sigma)=0 \\
        \left(\cfrac{\partial}{\partial p}-\cfrac{d}{dx}\cfrac{\partial}{\partial \pi}\right)(L+\lambda(x)\Sigma)=0 
    \end{cases*}\\
    \begin{cases*}
        \cfrac{d}{dx}\left(\cfrac{\partial L}{\partial \eta}-\lambda(x)\right)=0\longrightarrow \cfrac{\partial L}{\partial \eta}=\lambda(x)+A \quad A\in \R\\
        \lambda(x)-\cfrac{d}{dx}\left(\cfrac{\partial L}{\partial \pi}\right)=0\longrightarrow\lambda(x)+A=\cfrac{d}{dx}\left(\cfrac{\partial L}{\partial \pi}\right)+A
    \end{cases*}
\end{align*}
Ottenendo cosi:
\begin{align*}
    \dpd{L}{\eta}=\lambda(x)+A=\dod{}{x}\left(\dpd{L}{\pi}\right)
\end{align*}
Ovvero:
\begin{align*}
    \frac{\dot{y}(x)}{\sqrt{1+\dot{y}^2(x)+\ddot{y}^2(x)}}=A+\dod{}{x}\frac{\ddot{y}}{\sqrt{1+\dot{y}^2(x)+\ddot{y}^2(x)}}
\end{align*}
Svolgendo calcoli otteniamo:
\begin{align*}
    \ddot{y}^2=\frac{(1+\dot{y}^2)^2-(1+\dot{y}^2)(A\dot{y}+B)^2}{(A\dot{y}+B)^2} \quad A,B\in \R \quad *
\end{align*}
Le soluzioni di questa complessa espressione si ottengono da integrali ellittici.

Se $y$ è pari, $\gamma$ è simmetrica ($\gamma(x)=\gamma(-x)$), avendo $A=0$ e $K=\cfrac{1}{B}$.

Sostituendo otteniamo:
\begin{align*}
    \ddot{y}^2=(1+\dot{y}^2)^2[k^2(1+\dot{y}^2)-1]
\end{align*}
Concludendo:
\begin{align*}
    x(y)=c \int^{\dot{y}}_0 \frac{1}{\sqrt{(1+t^2)(1+\frac{k^2}{k^2-1}t^2)}} 
\end{align*}