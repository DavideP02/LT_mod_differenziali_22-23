%%
%%
%%
%%CHAPTER 2 CURVE, SUPERFICI ETC
%%
%%
%%
\chapter{Curve negli spazi affini, rappresentazione in coordinate non affini, e sistemi dinamici}
%%
%%
%%
%%
%%CURVA PARAMETRIZZATA
%%
%%
\definizione{%
Chiamiamo \underline{curva parametrizzata} in uno spazio affine $A$ un'applicazione $\gamma \colon I \to A$ da un intervallo aperto $I\subseteq \mathbb{R}$ nello spazio affine.
}
\definizione{%
Considerata un'origine $O\in A$, per la curva $\gamma$ vi è una \underline{rappresentazione vettoriale}:
\begin{align*}
    \mathbf{x}=\gamma(t)&& \text{con }\mathbf{x}=OP
\end{align*}
che dunque identifica i punti $P\in A$ con il loro vettore posizione rispetto al punto $O$.}


Siano $(x^\alpha)$ delle coordinate cartesiane aventi origine in $O$. Si possono allora considerare le equazioni parametriche:
\begin{align*}
    x^\alpha= \gamma^\alpha(t) && \alpha=1,\dots, n
\end{align*}
\paragrafo{Interpretazione cinematica}{Una curva può essere interpretata come moto di un punto $P$ nello spazio affine, se il parametro $t$ viene inteso come tempo.\\
Nel caso in cui la curva rappresenti il moto di un punto nello spazio affine tridimensionale euclideo, il generico vettore $OP=\mathbf{x}$ è chiamato \underline{vettore posizione}.}{}{}

\definizione{%
L'immagine della curva, cioè l'insieme 
\begin{align*}
    \gamma(I)=\{P\in A | \exists\,t \in I: \gamma(t)=P\}
\end{align*}
 è detta \underline{traiettoria} o \underline{orbita}\footnote{In geometria è questa in realtà la vera e propria curva, ovvero il luogo dei punti definito da $n-1$ equazioni.}.
 }
\definizione{Il \underline{vettore tangente} alla curva $\gamma$ nel punto $\gamma(t)$ è il vettore denotato con $\dot{\gamma}(t)$ definito dal limite:
\begin{align*}
    \dot{\gamma}(t)=\lim_{h\to 0}\frac{\gamma(t+h)-\gamma(t)}{h}
\end{align*}
Questo nel contesto cinematico prende il nome di \underline{velocità istantanea} e lo si denota con $\mathbf{v}(t)$.}
\paragrafo{Campo tangente come curva}{ Conviene interpretare il vettore tangente $\dot{\gamma}(t)=\mathbf{v}(t)$ come vettore applicato nel punto $\gamma(t)$. Ovvero come un'applicazione:
\begin{align*}
    \hat{\gamma}(t)\colon I \to A\times E\\
    t\mapsto (\gamma(t),\dot{\gamma}(t))
\end{align*}
che viene detta \underline{curva tangente} della curva $\gamma\colon I \to A$.}{}{}
\definizione{ Siano $\gamma \colon I \to \mathbb{R}$ una curva e $F\colon A \to \mathbb{R}$ un campo scalare, entrambi almeno di classe $C^1$.
Definiamo la \ul{derivata del campo scalare $F$ lungo la curva $\gamma$} come:
\begin{align*}
    \frac{d}{dt}(F\circ \gamma)(t)=\langle \mathbf{v}(t),dF\rangle \quad \forall t \in I
\end{align*}
Questa è la definizione naturale, infatti:
\begin{align*}
    \frac{d}{dt}(F\circ \gamma)(t)=\frac{\partial F}{\partial x^\alpha}\frac{d\gamma^\alpha}{dt}(t)=\frac{\partial F}{\partial x^\alpha}v^\alpha(t)=\langle\mathbf{v}(t),dF\rangle
\end{align*}}
% %%
% %%
% %%CURVA INTEGRALE
% %%
% %%
\definizione{Una \underline{curva integrale di un campo vettoriale $\mathbf{X}$} è una curva:
\begin{align*}
    \gamma \colon I \to A
\end{align*}
tale che:
\begin{itemize}
    \item $0\in I\subseteq \mathbb{R}$;
    \item $\forall \,\gamma(t)\in A$, il vettore tangente $\dot{\gamma}(t)$ coincide con il valore del campo $\mathbf{X}$ in quel punto ovvero:
    \begin{align*}
        \dot{\gamma}=\mathbf{X}\circ \gamma=\mathbf{X}(\gamma(t))\\
        I\xrightarrow{\gamma}A\xrightarrow[]{\mathbf{X}}A\times E\\
        \dot{\gamma}(t)\colon I \to A\times E
    \end{align*}
\end{itemize}}
\definizione{Diciamo inoltre che la curva integrale è \underline{basata nel punto} $P_0$ se $\gamma(0)=P_0$.}
\notazione{In rappresentazione vettoriale sarebbe:
\begin{align*}
    \mathbf{x}=\gamma(t) \text{ curva integrale} \iff \frac{d\mathbf{x}}{dt}=\mathbf{X}(\mathbf{x})
\end{align*}}


Le curve integrali rappresentano i moti delle particelle del fluido secondo l'interpretazione del campo $\mathbf{X}$ come campo di velocità.

\definizione{ Un campo vettoriale interpretato come campo di velocità viene detto \underline{sistema dinamico}.}
\notazione[]{ Un sistema dinamico si può rappresentare dunque come un'equazione differenziale:
\begin{itemize}
        \item vettoriale:
        \begin{align*}
            \frac{d\mathbf{x}}{dt}=\mathbf{X}(\mathbf{x})
        \end{align*}
        \item in coordinate affini:
        \begin{align*}
            \frac{dx^\alpha}{dt}=X^\alpha (x^\beta)
        \end{align*}
        \item in coordinate generiche:
        \begin{align*}
            \frac{dq^i}{dt}=X^i (x^j)
        \end{align*}
\end{itemize}
In particolare sono $n$ equazioni differenziali ordinarie in forma normale autonome.}
\section{Risoluzione sistemi dinamici} 
Integrare significa trovare tutte le soluzioni del sistema dinamico e queste costituiscono lo spazio delle soluzioni/spazio dei moti.
\definizione{Dato un sistema dinamico e un punto $P_0\in A$, parliamo di \ul{curva integrale massimale} $\gamma_{P_0}\colon I_{P_0}\to A$, quando presa un'altra curva integrale $\gamma\colon I\to A$ basata in $P_0$, allora:
\begin{align*}
    I \subseteq I_{P_0}&& \gamma_{P_0|_I}=\gamma
\end{align*}}
Una volta fissate le condizioni iniziali/dati iniziali riusciamo ad individuare un'unica soluzione del sistema dinamico grazie al
\teorema[Teorema di Cauchy]{fgdslkjlsdaljkfsd}{ Sia $\mathbf{X}$ un campo vettoriale di classe $C^k$($k\ge1$) su un dominio $M$.
Fissato un punto $P_0\in M$ esiste una e una sola curva integrale massimale basata in $P_0$
\begin{align*}
    \gamma_{P_0}\colon I_{P_0}\to M
\end{align*}
Se $I_{P_0}=\mathbb{R}$, $\mathbf{X}$ si dice \ul{completo}.}
\definizione{Il \underline{flusso del campo di vettori $\mathbf{X}$} descrive lo spazio delle soluzioni (o insieme di tutte le curve integrali del campo $\mathbf{X}$).
Si definisce come la funzione:
\begin{align*}
    \varphi\colon D\subseteq \mathbb{R}\times M \to M\\
    (t,P_0)\mapsto \gamma_{P_0}(t)
\end{align*}}
\teorema[]{sdljfkdslkgjldgdd}{Sia $\mathbf{X}$ campo vettoriale di classe $C^k$($k\ge1$) su un dominio $M$, allora:
\begin{enumerate}
    \item Il dominio $D$ del flusso $\varphi$ è un aperto di $\mathbb{R}\times M$ e $\varphi\in C^k(D)$
    \item Sia $V\subseteq M$ aperto, $\delta >0$ e si consideri $(-\delta,\delta)\times V\subseteq D$. Allora $\forall \, t \in (-\delta,\delta)$:
    \begin{align*}
       \varphi_t\colon V\to V_t\\
       P\mapsto \varphi(t,P) 
    \end{align*}
    è un omemomorfismo $C^k$ di $V$ su $V_t\subseteq M$ con $\varphi_t\colon P\mapsto\varphi(-t,P)$ omeomorfismo inverso
    \item Vale:
    \begin{align*}
      \varphi(t,\varphi(s,P))=\varphi(t+s,P)  
    \end{align*}
    $\forall\, t,s,P$ per cui i due membri hanno significato.
\end{enumerate}}
\osservazione[]{Se $\mathbf{X}$ è completo, allora:
\begin{itemize}
    \item $D=\mathbb{R}\times M$
    \item $\varphi_t\colon M\to M,P\mapsto\varphi_t(P)=\gamma_P(t)$ è una trasformazione $C^k$ di M.
\end{itemize}}
\definizione{ Al variare del parametro $t\in \mathbb{R}$ i diversi flussi $\varphi_t$ costituiscono un \underline{gruppo ad un parametro}, ovvero un'insieme $\{\varphi_t|t\in \mathbb{R}\}$ tale che valgono:
\begin{itemize}
    \item $\varphi_t\circ \varphi_s=\varphi_{t+s}$
    \item $\varphi_t\circ\varphi_s=\varphi_s\circ \varphi_t$
    \item $\varphi_0=id_M$
    \item $(\varphi_t)^{-1}=\varphi_{-t}$
\end{itemize}}
\osservazione[]{ Ad ogni gruppo ad un parametro si può associare il campo vettoriale $\mathbf{X}$ corrispondente e viceversa. Ovvero questo sono condizioni necessarie e sufficienti affinché un insieme di curve $\varphi(t,P)$ sia lo spazio delle soluzioni di un certo campo vettoriale.}
\definizione{ Sia $\mathbf{X}\in \mathcal{X}(A)$. Definiamo l'\underline{integrale primo di $\mathbf{X}$} come il campo scalare $F\colon A\to \mathbb{R}$ tale che $\forall\, \gamma \colon I \to A$ curva integrale, vale:

\parbox{10em}{\centering
   $F\circ \gamma (t_1)=F\circ \gamma (t_2)$
   $\forall\, t_1,t_2\in I$
    }$\iff$
\parbox{10em}{
    $\cfrac{d}{dt}(F\circ\gamma)(t)=0, \forall \, t\in I$
  }
    $\iff$
\parbox{10em}{
         $\langle \mathbf{X},dF\rangle =0$}
}
% %%
% %%
% %%
% %%
% %%
% %%%VELOCITA' E ACCELERAZIONE
% %%
% %
% %
% %%
\section{Velocità e accelerazione in coordinate non affini}
\paragrafo{Velocità in coordinate non affini}{ Consideriamo una curva $\gamma$ e delle coordinate non affini $(q^i)$. La curva è rappresentata da:
\begin{align*}
    q^i(t)=\gamma^i(t)
\end{align*}
Ricordando che $\mathbf{x}=x^i\mathbf{E}_i$, abbiamo che:
\begin{align*}
    \mathbf{v}=\frac{d\mathbf{x}(t)}{dt}=\underbrace{\frac{\partial \mathbf{x}}{\partial q^i}}_{\mathbf{E}_i}\cdot \frac{dq^i}{dt}=v^i\mathbf{E}_i
\end{align*}
Quindi la velocità in componenti rispetto alle cordinate $(q^i)$ è data da 
\begin{align*}
    v^i(t)=\frac{dq^i(t)}{dt}
\end{align*}}{}{}
\osservazione[]{ Come già visto l'espressione del campo di vettori $\mathbf{v}$ non cambia nei due sistemi $(\mathbf{c}_\alpha)$ e $(\mathbf{E}_i)$, ovvero:
\begin{align*}
    \mathbf{v}=v^\alpha \mathbf{c}_\alpha = v^i\mathbf{E}_i
\end{align*}}
\paragrafo{Accelerazione in coordinate non affini}{ Nel riferimento affine $(\mathbf{c}_\alpha)$, l'accelerazione è definita come:
\begin{align*}
    \mathbf{a}=\frac{d\mathbf{v}}{dt}=\frac{dv^\alpha}{dt}(t)\mathbf{c}_\alpha=a^\alpha(t)\mathbf{c}_\alpha
\end{align*}
Ora allora analizziamo $\cfrac{d\mathbf{v}}{dt}$ passando per le coordinate non affini:
\begin{align*}
    \frac{d\mathbf{v}}{dt}=\frac{dv^i}{dt}\mathbf{E}_i+v^i\frac{d}{dt}\mathbf{E}_i=\frac{dv^i}{dt}\mathbf{E}_i+v^i\frac{\partial}{\partial q^j}(\mathbf{E}_i)\cdot\overbrace{\frac{dq^j}{dt}}^{v^j}=\frac{dv^i}{dt}\mathbf{E}_i+\boxed{v^iv^j\Gamma^k_{ji}\mathbf{E}_k}
\end{align*}
Ora, scambiando $i\leftrightarrow k$ in $\square$ e raccogliendo, otteniamo:
\begin{align*}
    \left(\frac{dv^i}{dt}+v^kv^j\Gamma_{jk}^i\right)\mathbf{E}_i=a^i\mathbf{E}_i
\end{align*}
Quindi si noti che \underline{a meno che} $\Gamma_{jk}^i=0$, abbiamo:
\begin{align*}
    a^i\ne \frac{dv^i}{dt}
\end{align*}}{}{}
\paragrafo{Coordinate polari}{%
Consideriamo le coordinate polari. Per quanto riguarda la velocità abbiamo:
\begin{align*}
    \mathbf{v}=\dot{r}\mathbf{E}_r+\dot{\theta}\mathbf{E}_\theta=v^r\mathbf{E}_r+v^\theta\mathbf{E}_\theta
\end{align*}
Ora analizziamo l'accelerazione sfruttando l'espressione ottenuta precedentemente:
\begin{align*}
    \mathbf{a}=\left(\frac{dv^r}{dt}+\cancel{v^rv^\theta\Gamma^r_{r\theta}}+v^\theta v^\theta\Gamma^r_{\theta \theta}+\cancel{v^\theta v^r\Gamma^r_{\theta r}}+\cancel{v^rv^r\Gamma^r_{rr}}\right)\mathbf{E}_r+\\
    +\left(\frac{dv^\theta}{dt}+v^rv^\theta\Gamma^\theta_{r\theta}+\cancel{v^r v^r\Gamma^\theta_{rr}}+v^\theta v^r\Gamma^\theta_{\theta r}+\cancel{v^\theta v^\theta \Gamma^\theta_{\theta \theta}}\right)\mathbf{E}_\theta
\end{align*}
Ricordando inoltre che $\Gamma^r_{\theta \theta}=-r$ e $\Gamma^\theta_{r \theta}=\Gamma^\theta_{\theta r}=\cfrac{1}{r}$:
\begin{align*}
    \left(\frac{dv^r}{dt}-rv^\theta v^\theta\right)\mathbf{E}_r+\left(\frac{dv^\theta}{dt}+\frac{2}{r}v^\theta v^r\right)\mathbf{E}_\theta=\\
    =(\ddot{r}-r\dot{\theta}^2)\mathbf{E}_r+(\ddot{\theta}+\frac{2}{r}\dot{r}\dot{\theta})\mathbf{E}_\theta=\\
    =a^r\mathbf{E}_r+a^\theta\mathbf{E}_\theta
\end{align*}
Si osservi che ponendo $\mathbf{E}_r=\mathbf{u}$ e $\mathbf{E}_\theta=r\mathbf{\tau}$ con $\mathbf{u}$ e $\mathbf{\tau}$ versori nella rappresentazione radiale del moto si ottiene la classica scomposizione:
\begin{align*}
    \mathbf{a}=\mathbf{a}_{\text{radiale}}+\mathbf{a}_{\text{trasversale}}
\end{align*}%
}{}{}
\definizione{
Definiamo la \underline{velocità areolare} come:
\[
    \bm{v}_{ar}=\frac{1}{2}\bm{r}\wedge \bm{v}=\frac{1}{2}r^2\dot{\theta}\bm{k}
\]
}
\definizione{%
Definiamo \ul{moto centrale} un moto tale per cui:
\begin{align*}
    \exists \,O\in A \text{ origine }\colon \forall\, t\in I \quad \bm{a}(t)\wedge \bm{r}(t)=0
\end{align*}
}
Studiando adesso che la derivata della velocità areolare è:
\begin{align*}
    \frac{d}{dt}(\bm{v}_{ar})=\frac{1}{2}\frac{d}{dt}(\bm{r} \wedge \bm{v})=\frac{1}{2}(\overbrace{\bm{v}\wedge \bm{v}}^{0}+\bm{r}\wedge \bm{a})=\frac{1}{2}\bm{r}\wedge \bm{a}
\end{align*}
e che quindi è uguale a $0$, ovvero la velocità areolare è costante$\iff$ il moto è centrale.
\paragrafo{Dalla traiettoria agli enti fondamentali della cinematica}{ Sia $\mathbf{r}=r(\theta)$ . Da esso si possono scrivere tutti gli enti fondamentali della cinematica.\\
Sia $\mathbf{r}=\mathbf{r}(\theta(t))\mathbf{u}$ la nostra traiettoria del moto. Studiamo la velocità:
\begin{align*}
    \mathbf{v}=\frac{d\mathbf{r}}{dt}=\frac{dr}{d\theta}\cdot \dot{\theta}\mathbf{u}+r\frac{d\mathbf{u}}{d\theta}\cdot\dot{\theta}=\dot{\theta}\left(\frac{dr}{d\theta}\mathbf{u}+r\frac{d\mathbf{u}}{d\theta}\right)
\end{align*}
Ora utilizzando la costante delle aree $c=r^2\dot{\theta}$ otteniamo:
\begin{align*}
    \frac{c}{r^2}\left(\frac{dr}{d\theta}\mathbf{u}+r\frac{d\mathbf{u}}{d\theta}\right)=c\left(\frac{1}{r^2}\frac{dr}{d\theta}\mathbf{u}+\frac{1}{r}\frac{d\mathbf{u}}{d\theta}\right)
\end{align*}
e riconoscendo in $\cfrac{d\mathbf{u}}{d\theta}=\tau$, abbiamo che:
\begin{align*}
    \mathbf{v}=c\left(-\frac{d}{d\theta}\left(\frac{1}{r}\right)\mathbf{u}+\frac{1}{r}\tau\right)
\end{align*}
Prendiamo adesso in esame l'accelerazione, sfruttando l'espressione appena ottenuta per la velocità:
\begin{align*}
    \frac{d\mathbf{v}}{dt}=c\left(-\frac{d}{d\theta}\frac{d}{d\theta}\left(\frac{1}{r}\right)\cdot \bm{u}-\cancel{\frac{d}{d\theta}\frac{1}{r}\frac{d\bm{u}}{dt}}+\cancel{\frac{d}{dt}\frac{1}{r}\cdot \bm{\tau}}+\frac{1}{r}\frac{d\bm{\tau}}{dt}\right)=\\
    =c\dot{\theta}\left(-\frac{d^2}{d\theta^2}\left(\frac{1}{r}\right)\cdot \bm{u}+\frac{1}{r}\frac{d\bm{\tau}}{d\theta}\right)
\end{align*}
Ricordando che $\cfrac{d\bm{\tau}}{d\theta}=-\bm{u}$ da $\cfrac{d\bm{\tau}}{dt}=-\cfrac{d\theta}{dt}\bm{u}$, abbiamo:
\begin{align*}
    \bm{a}= -\frac{c^2}{r^2}\left(\frac{d^2}{d\theta^2}\left(\frac{1}{r}\right)+\frac{1}{r}\right)\bm{u}
\end{align*} 
}{}{}
% %%SPAZI AFFINI EUCLIDEI
% %%
% %
% %%
% %
% %
\definizione{ Uno spazio affine euclideo è una quaterna $(A,E,\delta,\mathbf{g})$, dove $(A,E,\delta)$ è uno spazio affine e $\mathbf{g}$ è un tensore metrico su $E$ (considerato come spazio vettoriale euclideo). Ovvero $\mathbf{g}$ è una forma bilineare simmetrica:
\begin{align*}
    \mathbf{g}\colon \mathcal{X}(A)\times \mathcal{X}(A)\to \mathcal{F}(A)\\
    (\mathbf{X},\mathbf{Y})\mapsto \mathbf{X}\cdot\mathbf{Y}=\mathbf{g}(\mathbf{X},\mathbf{Y})
\end{align*}
dove, considerata $g$ come il prodotto scalare su $E$, $\forall\, P \in A$ l'immagine è definita come:
\begin{align*}
    (\mathbf{X}\cdot\mathbf{Y})(P)=g(\mathbf{X}(P),\mathbf{Y}(P))
\end{align*}}
\definizione{ Definiamo l'\underline{ascissa euclidea} (o \underline{ascissa curvilinea}) una funzione monotona crescente $g$ tale che presa:
\begin{align*}
    g\colon I \to \mathbb{R}\\
    t\mapsto s(t)
\end{align*}
allora 
\begin{align*}
    \frac{ds}{dt}=|\mathbf{v}|=\sqrt{g_{ij}\frac{dq^i}{dt}\cdot \frac{dq^j}{dt}}
\end{align*}}
% %%
% %%
% %% da fare meglio, lei spiega di merda, se si fa geo3 si capisce davvero la teoria semplice delle curve differenziabili(triedro etc...)
% %%
% %%
% %%
\section{Moti geodetici su una superficie}
Sia $Q$ una superficie regolare nello spazio tridimensionale con rappresentazione $OP=\mathbf{r}(q^1,q^2)$. Sia assegnata sulla superficie una curva $\gamma$ di equazioni parametriche $q^i(t)=\gamma^i(t)$.

Il vettore $\mathbf{v}(t)$ tangente alla curva è banalmente tangente anche alla superficie. Se consideriamo invece il suo vettore derivata:
\begin{align*}
    \frac{d\mathbf{v}}{dt}
\end{align*}
questo in generale non è tangente alla superficie. Infatti:
\begin{align*}
   \frac{d\mathbf{v}}{dt} =\frac{dv^i}{dt}\mathbf{E}_i+v^i\frac{d\mathbf{E}_i}{dt}=\frac{dv^i}{dt}\mathbf{E}_i+v^i\frac{\partial \mathbf{E}_i}{\partial q^j}\frac{dq^j}{dt}=\frac{dv^i}{dt}\mathbf{E}_i+v^i\frac{dq^j}{dt}(\Gamma_{ji}^k\mathbf{E}_k+B_{ji}\mathbf{N})
\end{align*}
con $\mathbf{N}$ il vettore normale alla superficie. Quindi:
\begin{align*}
    \frac{d\mathbf{v}}{dt}=\underbrace{\left(\frac{dv^i}{dt}+v^k\frac{dq^i}{dt}\Gamma^i_{jk}\right)\mathbf{E}_i}_{\text{tangente a }Q}+v^i\frac{dq^j}{dt}B_{ji}\mathbf{N}
\end{align*}
La componente tangente della derivata della velocità è chiamata \underline{derivata intrinseca della velocità/accelerazione intrinseca}. Se volessimo scriverla però rispetto alle coordinate non affini $(q^i)$, essendo $v^i=\cfrac{dq^i}{dt}$: %TODO
\begin{align*}
    \mathbf{a}^k_{\text{intrinseca}}=\frac{d^2q^k}{dt^2}+\Gamma^k_{ij}\frac{dq^i}{dt}\cdot \frac{dq^j}{dt}
\end{align*}
Abbiamo cosi scomposto l'accelerazione come:
\begin{align*}
    \mathbf{a}=\mathbf{a}^k_{\text{intrinseca}}+\mathbf{a}_{\mathbf{N}}
\end{align*}
Una volta scissa l'accelerazione possiamo finalmente definire cos'è 
\definizione{ Il \ul{moto geodetico} o \ul{moto inerziale} su una superficie è il moto in cui:
\begin{align*}
    \mathbf{a}_{\text{intrinseca}}=0, \quad\forall \, t && [\mathbf{a} = \mathbf{a}_{\mathbf{N}}]
\end{align*}}

Questa condizione può essere anche rappresentata sotto forma di sistema di equazioni differenziali infatti:\\
\begin{minipage}{4cm}
    \begin{align*}
        \frac{d^2q^k}{dt^2}+\Gamma^k_{ij}\frac{dq^i}{dt}\cdot \frac{dq^j}{dt}=0
    \end{align*}
\end{minipage}$\longleftrightarrow$
\begin{minipage}{4cm}
    \begin{align*}
        \begin{cases}
            \cfrac{dq^i}{dt}=v^i\\
            \cfrac{dv^k}{dt}=-\Gamma^k_{ij}v^iv^j
        \end{cases}
    \end{align*}
\end{minipage}
%da aggiustare freccia ahha
\\
Cosi facendo i moti geodetici diventano le curve integrali di questo sistema dinamico associato al campo vettoriale $\mathbf{X}$ definito sui vettori dello spazio tangente alla superficie $Q$. Questo scritto come derivazione sarebbe:
\begin{align*}
    \mathbf{X}=v^i\frac{\partial}{\partial q^i}-\Gamma^k_{ij}v^iv^j\frac{\partial}{\partial v^k}
\end{align*}
$\mathbf{X}$ con un abuso di notazione prende il nome di \underline{flusso geodetico}.
Cosi facendo assegnato un $P_0\in Q$ e $\mathbf{v}_0$ tangente a $Q$ in $P_0$, $\exists!$ una curva geodetica massimale basata in $P_0$ e avente come vettore tangente in $t=0$ il vettore $\mathbf{v}_0$.
\paragraph*{Energia cinetica come integrale primo delle geodetiche}
\begin{align*}
    \mathbf{F}=\mathbf{F}(\mathbf{r},\mathbf{v}) && \mathbf{r}=\mathbf{r}(t)=\gamma(t)
\end{align*}
\begin{align*}
    \frac{d}{dt}\mathbf{F}(\mathbf{r}(t),\mathbf{v}(t))=0
\end{align*}
\begin{align*}
    \frac{d}{dv}(\mathbf{v}\cdot \mathbf{v})=2\mathbf{v}\cdot\mathbf{a}=0
\end{align*}
Poichè $\mathbf{a}=\mathbf{a}_{\mathbf{N}}\perp \mathbf{v}$\\
Quindi l'energia cinetica è un integrale primo delle geodetiche. In particolare il moto è uniforme, ovvero il modulo della velocità è costante lungo le geodetiche.