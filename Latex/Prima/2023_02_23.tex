\stepcounter{capitoloeccolo}\chapter{Ripasso E.D.O.}
\days{23 febbraio}
%% BEGIN Problema di Cauchy
\paragrafo{Problema di Cauchy}{%
    Data una funzione a valori in $ \R^{n} $, $ \bm{f}=\bm{f}(t,\bm{x}) $, con \begin{itemize}
        \item $ t \in \R  $, $ \bm{x} \in \R^{n} $ o, più precisamente
        \item $ (t,\bm{x}) \in \Omega \subseteq \R\times \R^{n}$, $ \Omega $ aperto
    \end{itemize}ci si chiede sotto quali condizioni su $ \bm{f} $ il Problema di Cauchy \[
        \begin{cases}
            \bm{u}'(t)= \bm{f}\left(t, \bm{u}(t)\right)\\ 
            \bm{u}(t_0) = \bm{x}_0
        \end{cases}
    \]ammetta \emph{almeno} una soluzione o ammetta \emph{esattamente} una soluzione, al variare della condizione iniziale $ (t_0, \bm{x}_0) \in \Omega $.
}{}{}
%% END
%% BEGIN Teorema di Peano
\teorema[Teorema di Peano]{ljkndasflkjnasdlfkjndsalkfjndafs}{
    Se $ \bm{f} $ è continua su $ \Omega $, allora per ogni punto $ (t_0,\bm{x}_0) \in \Omega$ esiste un intorno di $ t_0 $ nel quale è definita \emph{almeno} una soluzione del Problema di Cauchy \[
        \begin{cases}
            \bm{u}'(t)= \bm{f}\left(t, \bm{u}(t)\right)\\ 
            \bm{u}(t_0) = \bm{x}_0
        \end{cases}
    \]
}
%% END
%% BEGIN Pennello di Peano
\paragrafo{Pennello di Peano}{%
    La sola ipotesi di continuità non basta ad affermare anche l'unicità della soluzione del PdC. 
    
    Si scopre infatti che se si dimostra che il problema di Cauchy ammette due soluzioni distinte, allora in realtà ne ha infinite.
    Questo fenomeno, denominato \ul{pennello di Peano}, è reso evidente dal seguente esempio.
}{}{}
\esempio{
    Consideriamo il problema di Cauchy \[
        \begin{cases}
            u'(t)=\sqrt[3]{u(t)}\\ 
            u(0)=0
        \end{cases}
    \]Questo problema ammette certamente la soluzione $ u \equiv 0 $, ma anche le soluzioni \[
        u_{0}^{\pm} =\begin{cases}
            \pm \displaystyle \left(\frac{2}{3}\, t\right)^{\frac{3}{2}} & t\ge 0\\ 
            0 & t<0
        \end{cases} 
    \]
}
%% END
%% BEGIN Teorema di Cauchy-Lipschitz
Per garantire l'unicità della soluzione, risulta cosi necessario il cosiddetto 
\teorema[Teorema di Cauchy-Lipschitz]{ldksnaflkjnadflkjadnsflajknkljndasf}{
    Se $ \bm{f} $ è \begin{itemize}
        \item continua,
        \item localmente lipschitziana rispetto alla seconda variabile e uniformemente nella prima\footnote{Ovvero \[
            \forall\, K \subset \Omega,\quad \exists\,L>0:\quad \norma{\bm{f}(t,\bm{x})-f(t,\bm{y})}\le L\,\norma{\bm{x}-\bm{y}}\qquad \forall\,(t,\bm{x}), (t,\bm{y}) \in K.
        \]}
    \end{itemize}allora per ogni punto $ (t_0,\bm{x}_0) \in \Omega $ esiste un intorno di $ t_0 $, $ [t_0-\delta,t_0+\delta] $ nel quale è definita un'unica soluzione del problema di Cauchy \[
        \begin{cases}
            \bm{u}'(t)= \bm{f}\left(t, \bm{u}(t)\right)\\ 
            \bm{u}(t_0) = \bm{x}_0
        \end{cases}
    \]
}
\dimostrazione{ldksnaflkjnadflkjadnsflajknkljndasf}{
    La dimostrazione si articola nei seguenti passaggi: \begin{itemize}
        \item si considera l'\emph{equazione di Volterra} \[
            \bm{u}(t)= \bm{x}_0 + \int_{t_0}^{t} \bm{f}\left(s,\bm{u}(s)\right)\dif s
        \]e si mostra che quest’ultima ammette un'unica soluzione continua in un intorno di $ t_0 $;
        \item esistenza e unicità dell’equazione di Volterra si dimostrano applicando il Teorema delle contrazioni di Banach-Caccioppoli;
        \item la seguente successione \[
            \bm{u}_0(t) \equiv \bm{x}_0,\qquad \bm{u}_n(t) = \bm{x}_0 + \int_{t_0}^{t} \bm{f}\left(s,\bm{u}_{n-1}(s)\right)\dif s,\qquad \forall\, n \in \N
        \]risulta convergere uniformemente alla soluzione dell’equazione di Volterra e dunque all’unica soluzione del Problema di Cauchy.
    \end{itemize}
}
\osservazione{
    L'intervallo di definizione della soluzione del problema di Cauchy è certamente più ampio di $ [t_0-\delta,t_0+\delta] $: possiamo infatti applicare lo stesso teorema di esistenza ed unicità locale ai problemi di Cauchy con condizioni iniziali \[
        \left(t_0\pm\delta, \bm{u}(t_0 \pm \delta)\right) \in \Omega
    \]ed iterare questo procedimento. 

    In generale, quindi, esiste un intervallo $ (T_{\min}, T_{\max}  ) $ per la soluzione $ \bm{u}_{(t_0,\bm{x}_0)}(t) $, che per costruzione non può che essere aperto e connesso.
}
%% END
%% BEGIN Teorema di Esistenza Globale
\teorema[Teorema di Esistenza Globale]{sasdfkjnasdflkjnasdlkfjnfasd}{
    Sia $ f $ tale che le ipotesi del teorema di Cauchy-Lipschitz siano soddisfatte. 

    Sia inoltre $ S=(a,b)\times \R^{n} $ una striscia tale che $ \overline{S} \subseteq \Omega $. Se esiste una coppia di costanti positive $ k_1,k_2 $ tali per cui \[
        \norma{\bm{f}(t,\bm{x})} \le k_1+k_2\,\norma{\bm{x}},\quad \forall\, (t,\bm{x}) \in \overline{S}
    \]allora, per ogni $ (t_0,\bm{x}_0) \in S $, l'intervallo massimale della soluzione $ \bm{u}_{(t_0,\bm{x}_0)}(t) $ contiene l'intervallo $ [a,b] $.
}
\paragrafo{Osservazione}{%
    Questo teorema è ciò che garantisce l'esistenza globale per i sistemi lineari del tipo \[
        \bm{x}'(t)= A(t)\, \bm{x}(t)+ \bm{b}(t)
    \]con $ A(t) $ matrice $ n\times n $
}{}{}
%% END
%% BEGIN Generalizzazione esplosione tempo finito
\teorema{fasdkjnfalskjnfalkjdnflaj}{
    Sia $ \bm{f} $ tale che le ipotesi del teorema di Cauchy-Lipschitz siano soddisfatte. Sia $ K \subset \subset \Omega $\footnote{Ovvero $ K $ contenuto in $ \Omega $ e $ K $ compatto.}, $ (t_0,\bm{x}_0) \in K $ e $ (T_{\min}, T_{\max}  ) $ l'intervallo massimale di definizione di $ \bm{u}_{(t_0,\bm{x}_0)}(t) $.

    Allora il grafico di $ \bm{u} $ esce definitivamente da $ K $ quando $ t \to T_{\min}^{+}  $ o $ t \to T_{\max}^{-}  $
}
%% END
%% BEGIN Esplosione in tempo finito
\paragrafo{Corollario - Esplosione in tempo finito}{%
    Sia $ \bm{f} $ tale che soddisfi le condizioni del teorema di Cauchy-Lipschitz, e sia $ (T_{\min},T_{\max})$ l'intervallo massimale di $ \bm{u}_{(t_0,\bm{x}_0)}(t) $.

    Se $ T_{\max} < + \infty   $ allora \[
        \lim_{t\to T_{\max}^{-} } \norma{\bm{u}_{(t_0,\bm{x}_0)}(t)} = + \infty
    \]se tale limite esiste. Analogamente se $ T_{\min}> - \infty  $.
}{}{}
%% END
\paragrafo{Attenzione}{%
    Per i prossimi risultati si consideri $ \Omega=\R\times \R^{n} $
}{}{}
%% BEGIN Limitatezza a priori
\paragrafo{Corollario - Limitatezza a priori}{%
    Sia $ \bm{f} $ tale che le ipotesi del teorema di Cauchy-Lipschitz siano soddisfatte. Sia $ (T_{\min},T_{\max}) $ l'intervallo massimale di $ \bm{u}_{(t_0,\bm{x}_0)}(t)$.

    Se esiste $ C>0 $ tale per cui \[
        \norma{\bm{u}_{(t_0,\bm{x}_0)}(t)}\le C,\qquad \forall\, t \in [t_0,T_{\max})
    \]allora $ T_{\max} = + \infty  $.
}{dakfjhbaksjdhfbaksjdhfbakjhbkjhb}{}
%% END
