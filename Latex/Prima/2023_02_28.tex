\days{28 febbraio}
%TODO leggere teorema dell'asintoto e scrivere una nota a riguardo

\section{Risultati e definizioni sui sistemi autonomi}

\paragrafo{Ipotesi}{%
    Dato il problema di Cauchy\[
        \begin{cases}
            \bm{x}'=\bm{f}(\bm{x})\\ 
            \bm{x}(t_0)=\bm{x}_0 \in \Omega
        \end{cases}
    \]
    $ \bm{f}:\Omega \subseteq \R^{n}\to \R^{n} $,$n\ge 1$, continua e localmente lipschitziana\footnote{Si ricordi che: $C^{1}\Rightarrow$ localmente lipschitziana}, vale il teorema di esistenza e unicità locale della soluzione, $ \forall\, t_0 \in\R  $.
}{daflkjnasdlkfjnasdlkfjnasdkfjnaskdjnfkjnkj}{}
\definizione{
    $ \Omega $ si dice \emph{spazio delle fasi} o \emph{degli stati}.
}
\definizione{
    Se $ \bm{u} $ è una soluzione massimale di $ \bm{x}'=\bm{f}(\bm{x}) $, l'insieme \[
        \gamma\coloneqq\left\{u(t): t \in (T_{\min}, T_{\max}  )\right\} 
    \]è detta \emph{orbita} per $ \bm{x}'=\bm{f}(\bm{x}) $ e la soluzione $ \bm{u} $ è una parametrizzazione di $ \gamma  $.
}
\osservazione{
    \[
        \left\{\begin{aligned}
            (T_{\min}, T_{\max}  ) &\to \R^{n}\\ 
            t \mapsto \bm{u}(t)
        \end{aligned}\right.
    \]è una curva in $ \R^{n} $ con sostegno $ \gamma  $.
}
\definizione{%
        Lo spazio delle fasi in cui vengono disegnate le orbite con il loro verso di percorrenza (indotto dalle soluzioni) si dice \emph{ritratto di fase}.
}
\definizione{%
    I punti $ \bm{p} \in \Omega \colon \bm{f}(\bm{p})=\bm{0} $ si chiamano \emph{equilibri} o \emph{punti singolari}
}
\paragrafo{Esempio}{%
\[
    \begin{cases}
        x'=-y^{2}\\ 
        y'=x^{2}
    \end{cases}\qquad \begin{aligned}
        \bm{f}(x,y) &= (-y^{2},x^{2})\\ 
        \bm{f}:\R^{2} &\to \R^{2}
    \end{aligned}
\]Chi sono gli equilibri? Cerco $ (x_0,y_0): \bm{f}(x_0,y_0)=\bm{0} $ \[
    (-y^{2},x^{2})=\bm{0}\,\iff\, (x,y)=\bm{0}
\]
\begin{figure}[H]
    \label{fig:puntoequilibrioA}
        \begin{center}
            \begin{tikzpicture}
                \draw [-Stealth] (-2,0) -- (2,0);
                \draw [-Stealth] (0,-1.4) -- (0,2);
                \node at (2,-0.3) {$x$};
                \node at (.3,1.9) {$y$};
                \fill (0,0) circle (0.1);
            \end{tikzpicture}
        \end{center}    
        \caption{Punti di equilibrio per l'esempio \framref{daflkjansdflkjnasdflkjadsnchiduaskjbfcgukjbh}}
\end{figure}
Dunque l'unico punto di equilibrio è l'origine.
}{daflkjansdflkjnasdflkjadsnchiduaskjbfcgukjbh}{}
\paragrafo{Esempio}{%
\[
    \begin{cases}
        x'=-y^{2}\,x\\ 
        y'=x^{2}
    \end{cases}\qquad
        \bm{f}(x,y) = (-y^{2}\,x,x^{2})
\]Gli equilibri sono tutti punti nella forma $ (0,y_0) $.
\begin{figure}[H]
\label{puntoequilibrioB}
    \begin{center}
        \begin{tikzpicture}
            \draw [-Stealth] (-2,0) -- (2,0);
            \draw [-Stealth] (0,-1.5) -- (0,2.1);
            \node at (2,-0.3) {$x$};
            \node at (.3,1.9) {$y$};
            \draw [ultra thick] (0,-0.8) -- (0,1.3);
            \draw [ultra thick, dashed] (0,1.3) -- (0,1.8);
            \draw [ultra thick, dashed] (0,-0.8) -- (0,-1.3);
        \end{tikzpicture}
    \end{center}
    \caption{Punti di equilibrio per l'esempio \framref{jjdjjduudjjdekkjnsdikjnskjnsdikjsnbfik}}
\end{figure}
}{jjdjjduudjjdekkjnsdikjnskjnsdikjsnbfik}{}
\teorema{daflkjndasflkjndaslkjnfdalkjnaslkjfnlkjn}{
    Sotto le ipotesi di \framref{daflkjnasdlkfjnasdlkfjnasdkfjnaskdjnfkjnkj}, per ogni punto dello spazio delle fasi, $ \Omega $, passa una e una sola orbita.
}
\dimostrazione{daflkjndasflkjndaslkjnfdalkjnaslkjfnlkjn}{
    Sia $ \bm{p} \in \Omega $, e consideriamo il Problema di Cauchy \[
        \begin{cases}
            \bm{x}'=\bm{f}(\bm{x})\\ 
            \bm{x}(t_0)=\bm{p} 
        \end{cases}\tag{$PC_{\bm{p}}$}
    \]
    \begin{itemize}
        \item[($\exists$)] Il problema di Cauchy ammette un'unica soluzione $ \bm{u} $ (per il teorema di Cauchy-Lipschitz) e l'orbita associata a $ \bm{u} $ passa per $ \bm{p} $: \[
            \bm{p} \in \gamma=\left\{\bm{u}(t): t \in (T_{\min}, T_{\max}  )\right\} 
        \]
        \item[($!$)] Intuitivamente, l'unicità è giustificata dal fatto che il sistema è autonomo, e quindi dal fatto che le traslate in $ t $ delle soluzioni sono ancora soluzioni.
        
        Sappiamo che $ \bm{p} \in \gamma  $. Supponiamo che esista un'altra orbita $ \tilde{\gamma } $ tale che $ \bm{p} \in \tilde{\gamma } $. Allora $ \tilde{\gamma} $ è orbita di $ \tilde{\bm{u}} $, soluzione di \[
            \begin{cases}
                \bm{x}'=\bm{f}(\bm{x})\\ 
                \bm{x}(\tilde{t}_0)= \bm{p}
            \end{cases}
        \]Siano $ I $ e $ \tilde{I} $ gli intervalli massimali, rispettivamente, di $ \bm{u} $ e $ \tilde{\bm{u}} $. Siano\begin{align*}
            T&\coloneqq \tilde{t}_0-{t_0}\\
            \bm{v}(t) &\coloneqq \tilde{\bm{u}}(t+\tilde{t}_0-t_0)=\tilde{\bm{u}}(t+T).
        \end{align*}Si ha che \begin{itemize}
            \item l'orbita di $ \bm{v} $ è $ \tilde{\gamma} $;
            \item $ \bm{v} $ è massimale, in quanto lo è $ \tilde{\bm{u}} $;
            \item $ \bm{v} $ soddisfa \[
                \begin{cases}
                    \bm{v}'(t)= \tilde{\bm{u}}'(t+T)= \bm{f}\left(\tilde{\bm{u}}(t+T)\right)=\bm{f}(\bm{v}(t))\\ 
                   \bm{v}(t_0)=\tilde{\bm{u}}(t_0+T)=\tilde{\bm{u}}(\tilde{t}_0)=\bm{p}
                \end{cases}
            \]e quindi $ \bm{v} $ è soluzione massimale di \[
                \begin{cases}
                    \bm{v}'=\bm{f}(\bm{v})\\ 
                    \bm{v}(t_0)=\bm{p}
                \end{cases}
            \] 
            
            $\implies$ $ \bm{u} $ e $ \bm{v} $ sono la stessa soluzione (massimale) di \[
                \begin{cases}
                    \bm{x}'=\bm{f}(\bm{x})\\ 
                    \bm{x}(t_0)=\bm{p}
                \end{cases}
            \] 
            
            $\implies$ $ I $ e $ \tilde{I} $ sono uno traslato dell'altro, e $ \gamma $ e $ \tilde{\gamma} $ coincidono.\qed
        \end{itemize}
    \end{itemize}
}
\osservazione{
    La mappa: \[
        \parbox{5.5em}{\centering soluzione di $ \bm{x}'=\bm{f}(\bm{x}) $}\quad\longmapsto\quad\text{orbita}
    \]è ben definita, ma non è iniettiva (è suriettiva!).

    Infatti, se $ \gamma $ è orbita \[
        \gamma=\left\{u(t):t \in (a,b)\right\}\underset{\footnotemark}{=}\left\{u_{\tau}(t)=u(t+\tau): t \in (a-\tau, b-\tau) \right\}.
    \]\footnotetext{$\forall \tau \in \R$}Dunque ogni orbita di $ \bm{x}'=\bm{f}(\bm{x}) $ ha infinite parametrizzazioni.
}
\teorema{lkjndalkjndfalkjndasklfjnasdkljfnasdkljfnasdjkndaslkjnfaldskjnaslfkjnsaldkjn}{%
    Sotto le ipotesi di \framref{daflkjnasdlkfjnasdlkfjnasdkfjnaskdjnfkjnkj}, sia $ \gamma^{\star} $ un'orbita di $ \bm{x}'=\bm{f}(\bm{x}) $. Allora: \[
        \gamma^{\star} =\{\bm{p}\}\,\iff\, \bm{f}(\bm{p})=\bm{0}.
    \]
}
\dimostrazione{lkjndalkjndfalkjndasklfjnasdkljfnasdkljfnasdjkndaslkjnfaldskjnaslfkjnsaldkjn}{%
    \begin{itemize}
        \item[($\impliedby$)] Se $ \bm{f}(\bm{p})=\bm{0} $, allora $ \bm{u}(t)\equiv\bm{p} $ $ \forall\, t \in \R $ è soluzione, e \[
            \gamma^{\star} = \left\{\bm{u}(t): t \in \R\right\}=\{\bm{p}\}.
        \]
        \item[($\implies$)] Se $ \gamma^{\star} =\{\bm{p}\} $, allora esiste una soluzione $ \bm{u} $ di $ \bm{x}'=\bm{f}(\bm{x}) $ tale che $ \bm{u}(t)=\bm{p} $ per ogni $ t \in (T_{\min},T_{\max}  ) \leadsto \bm{u}(t) $ è costante, e dunque \[
            \bm{0}=\bm{u}'(t)=\bm{f}\left(\bm{u}(t)\right)=\bm{f}(\bm{p}).\qedd
        \]
    \end{itemize}
}
\definizione{%
    Una soluzione $ \bm{u} $ di $ \bm{x}'=\bm{f}(\bm{x}) $ si dice \emph{periodica} di periodo $ T>0 $ se \begin{enumerate}
        \item $ \bm{u} $ è definita su $ \R $;
        \item $ \bm{u}(t+T)=\bm{u}(t) $, $ \forall\, t \in \R $
        \item $ \displaystyle T=\inf\left\{\tau >0: u(t+\tau)=u(t)\vspace{1em},\forall\,t \in \R\right\} $
    \end{enumerate}
}
\definizione{
    L'orbita corrispondente ad una soluzione periodica si chiama \emph{orbita periodica} e i suoi punti si chiamano \emph{punti periodici}.
}
\teorema{daflkjnadfkljansdfklasjdnfcaskdjncdas}{
    Sotto le ipotesi di \framref{daflkjnasdlkfjnasdlkfjnasdkfjnaskdjnfkjnkj}, se $ \bm{u} $ è una soluzione \emph{non} costante di $ \bm{x}'=\bm{f}(\bm{x}) $ con intervallo massimale $ J $ e  \[
        \exists\, t_1,t_2 \in J:\quad t_1\neq t_2,\quad \bm{u}(t_1)=\bm{u}(t_2)
    \]allora $ \bm{u} $ è una soluzione \emph{periodica}.
}
\osservazione{
    Le orbite periodiche si chiamano anche \emph{orbite chiuse}. Infatti, il teorema \teoref{daflkjnadfkljansdfklasjdnfcaskdjncdas} ci dice che se un'orbita si autointerseca, allora è periodica.
}
\corollario{adkjnaldfkjnadslfkjnadsflkjnadslkjn}{
    Le orbite di $ \bm{x}'=\bm{f}(\bm{x}) $ possono essere: \begin{enumerate}
        \item punti di equilibrio, $ \{\bm{p}\} $;
        \item periodiche/chiuse;
        \item orbite senza autointersezioni e contenenti più di un solo punto.
    \end{enumerate}
}
\paragrafo{Caso particolare}{%
    Per $ n=1 $ non esistono orbite periodiche non costanti, perché dovrebbero cambiare la monotonia e non è possibile perché $ x'=f(x) $, e la cambierebbero su punti di equilibrio.
}{}{}
\dimostrazione{daflkjnadfkljansdfklasjdnfcaskdjncdas}{
    È analoga al teorema \teoref{daflkjndasflkjndaslkjnfdalkjnaslkjfnlkjn}. 

    Sia $ \bm{p}\coloneqq \bm{u}(t_1)= \bm{u}(t_2) $. Allora $ \bm{u} $ risolve due problemi di Cauchy: \[
        \begin{cases}
            \bm{x}'=\bm{f}(\bm{x})\\ 
            \bm{x}(t_1)=\bm{p}
        \end{cases}\qquad \begin{cases}
            \bm{x}'=\bm{f}(\bm{x})\\ 
            \bm{x}(t_2)=\bm{p}
        \end{cases}
    \]entrambi risolti su $ J $.

    Gli intervalli massimali di questi due problemi di Cauchy devono essere uno traslato dell'altro, dunque $ J=\R $. 

    Inoltre, supponendo $ t_2>t_1 $, si ha che \[
        \bm{u}(t),\qquad \bm{u}\left(t+(t_2-t_1)\right)
    \]risolvono entrambe \[
        \begin{cases}
            \bm{x}'=\bm{f}(\bm{x})\\ 
            \bm{x}(t_1)=\bm{p}
        \end{cases}
    \]
    Definito $T\coloneq t_2-t_1$, per esistenza e unicità della soluzione allora: \[
        \bm{u}(t)=\bm{u}(t+T),\quad \forall\, t \in\R \qedd
    \]
}
\definizione{
    Un’orbita si dice \emph{singolare} se si riduce ad un punto solo. Altrimenti si dice \emph{regolare}.
}
%% BEGIN Orbita di una equazione differenziale come curva tangente al campo
\teorema{dakfjnasldkfjnlasdkjfnadslkfjnaslkj}{
    Sia $ \gamma $ un'orbita non singolare di $ \bm{x}'=\bm{f}(\bm{x}) $, $ \bm{x} \in \Omega \subseteq \R^{n} $ aperto, $ \bm{f}:\Omega \to \R^{n} $ localmente lipschitziana. 

    Allora $ \gamma $ è una curva orientata in $ \Omega $ tangente in ogni punto al campo $ \bm{f} $.
}
\dimostrazione{dakfjnasldkfjnlasdkjfnadslkfjnaslkj}{
Sia $ \bm{u} $ soluzione di $ \bm{x}'=\bm{f}(\bm{x}) $ tale che $ \gamma= \gamma_{\bm{u}}  $. Sia $ \bm{p}_0  \in \gamma$ e $ t_0 \colon \bm{u}(t_0)=\bm{p}_0 $. 

Poichè $ \gamma $ non è singolare $\leadsto \bm{f}(\bm{p}_0) \neq \bm{0}$ (altrimenti $ \gamma=\{\bm{p}_0\} $ per il teorema precedente.)

La funzione $ t\mapsto \bm{u}(t) $ parametrizza $ \gamma $. Quindi il vettore: \[
    \bm{v}\coloneqq \lim_{h\to 0} \frac{\bm{u}(t_0+h)-\bm{u}(t_0)}{h}
\]è il vettore tangente a $\gamma$ nel punto $ \bm{p}_0 $, ammesso che esista. 

La funzione $ \bm{u} $ è di classe $ C^{1} $, quindi il limite esiste ed è $ \bm{u}'(t_0) $. Essendo \[
    \bm{u}'(t)=\bm{f}\left(\bm{u}(t)\right)
\]si ha che \[
    \bm{v}=\bm{u}'(t_0)=\bm{f}\left(\bm{u}(t_0)\right)= \bm{f}(\bm{p}_0)\qedd
\]}
%% END
\newpage
\section{Stabilità dei punti di equilibrio}
\paragrafo{Obiettivo}{%
    Vogliamo classificare i punti di equilibrio in base a come le altre soluzioni si comportano in loro prossimità.
}{}{}
\definizione{
    Sotto le ipotesi \framref{daflkjnasdlkfjnasdlkfjnasdkfjnaskdjnfkjnkj}, sia $ \bm{p} \colon \bm{f}(\bm{p})=\bm{0} $. $ \bm{p} $ si dice \emph{stabile} se $ \forall\,\varepsilon>0 $, $ \exists\, \delta>0 \colon$ se $ \hat{\bm{x}} \in \Omega $, $ \norma{\hat{\bm{x}}-\bm{p}}<\delta $ 
    
    $\implies$ la soluzione di \[
        \begin{cases}
            \bm{x}'=\bm{f}(\bm{x})\\ 
            \bm{x}(t_0)=\hat{\bm{x}}
        \end{cases}
    \]è definita su $ (t_0,+ \infty) $ e dista da $ \bm{p} $ al più $ \varepsilon $.
}\\
\definizione{
    Sotto le ipotesi \framref{daflkjnasdlkfjnasdlkfjnasdkfjnaskdjnfkjnkj}, sia $ \bm{p} \colon \bm{f}(\bm{p})=\bm{0} $. $ \bm{p} $ si dice \emph{instabile} se non è stabile. 
}

Intuitivamente, se $p$ è un equilibrio, pur di partire sufficientemente vicino, riusciamo a far rimanere quanto vogliamo la soluzione \textit{vicina} a $p$.\\
Al contrario, invece, essere instabile significa che $ \exists\, \varepsilon $ e una successione $ \{\bm{x}_{n} \}_{n \in \N} \subseteq \Omega $, $ \bm{x}_{n} \to \bm{p} $ tale che \[
        \exists\, t_{n}\colon \bm{u}_{(t_0,\bm{x}_n)}(t_{n} )> \varepsilon \quad \quad \forall\, n
    \]dove $ \bm{u}_{(t_0,\bm{x}_n)} $ è soluzione di \[
        \begin{cases}
            \bm{x}'=\bm{f}(\bm{x})\\ 
            \bm{x}(t_0)=\bm{x}_n
        \end{cases}
    \]

\definizione{
    Sotto le ipotesi \framref{daflkjnasdlkfjnasdlkfjnasdkfjnaskdjnfkjnkj}, sia $ \bm{p} \colon \bm{f}(\bm{p})=\bm{0} $.\\ $ \bm{p} $ si dice \emph{asintoticamente stabile} se:
    \begin{itemize}
        \item è stabile;
        \item $ \exists\,\delta>0 $ tale che $ \forall\, \overline{\bm{x}} \in B_{\delta}(\bm{p}) \subseteq \Omega $ si ha che \[
            \lim_{t\to \infty} \bm{u}_{(t_0,\overline{\bm{x}})} (t) = \bm{p}
        \]
    \end{itemize}
}