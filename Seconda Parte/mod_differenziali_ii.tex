\chapter{Campi di vettori e forme differenziali su spazi affini}
	\definizione{ Si chiama \underline{spazio affine} la terna $(A,E,\delta)$,dove:
    \begin{itemize}
        \item $A$ è un insieme di elementi che chiamiamo punti
        \item $E$ è uno spazio vettoriale
        \item $\delta \colon A \times A \to E$ è un'applicazione tale che:
        \begin{enumerate}
            \item $\forall\, (P,\bm{v}) \in A\times E, \exists! \, Q \in A \colon \delta(P,Q)=\mathbf{v}$
            \item $\forall\, P,Q,R \in A, \delta(P,Q)+\delta(Q,R)=\delta(P,R)$
        \end{enumerate}
    \end{itemize}
    La \underline{dimensione dello spazio affine} $A$ è la dimensione dello spazio vettoriale soggiaciente $E$.}
    \notazione[]{Uno spazio affine $(A,E,\delta)$ è indicato più brevemente con $A$, mentre il vettore $\delta(P,Q)$ è indicato semplicemente con $PQ$.}
    \definizione{Un vettore $\mathbf{v}\in E$ è chiamato \underline{vettore libero}, mentre con la coppia $(P,\mathbf{v})$ indicheremo il vettore $\mathbf{v}$ applicato nel punto $P$.}
    \definizione{ Sia $A$ uno spazio affine. Un \underline{riferimento cartesiano} è una coppia $(O,\mathbf{c}_\alpha)$, dove:
    \begin{itemize}
        \item $O$ è l'origine
        \item $(\mathbf{c}_\alpha)=(\mathbf{c}_1,\dots,\mathbf{c}_n)$ è una base di $E$
    \end{itemize}}

    Un riferimento cartesiano stabilisce una corrispondenza biunivoca
        \begin{align*}
            \Phi \colon A \leftrightarrow \mathbb{R}^n\\
            P\leftrightarrow (x^\alpha)
        \end{align*}
    dove ad ogni punto $P\in A$ si fa corrispondere l'ennupla reale $(x^\alpha)$ costituita dalle componenti secondo la base $(\mathbf{c}_\alpha)$ del vettore $OP$:
        \begin{align*}
            OP=\mathbf{x}=x^\alpha\mathbf{c}_\alpha
        \end{align*}
    Risultano così definite anche delle mappe $x^\alpha\colon A \to \mathbb{R}$ dette \underline{coordinate cartesiane} o \underline{affini}, tali che $x^\alpha(P)\mathbf{c}_\alpha=OP$
    \osservazione{Questa corrispondenza biunivoca istituisce anche una topologia di $A$ indotta dalla topologia di $\mathbb{R}^n$.}
    \paragrafo{Trasformazioni affini}{ Se si considerano due riferimenti affini $(O,\mathbf{c}_\alpha)$ e $(O',\mathbf{c}_{\alpha'})$, 
    allora vi è un legame tra i due sistemi di coordinate indotti $(x^\alpha)$ e $(x^{\alpha'})$, ovvero:
        \begin{align*}
            x^\alpha=a^\alpha_{\alpha'}x^{\alpha'}+b^\alpha && x^{\alpha'}=a^{\alpha'}_\alpha x^\alpha+b^{\alpha'}
        \end{align*}
    Dove $a^\alpha_{\alpha'}$ e $a^{\alpha'}_\alpha$ compongono le matrici dei cambiamenti di base:
    \begin{align*}
        \mathbf{c}_{\alpha'}=a^\alpha_{\alpha'}\mathbf{c}_\alpha && \mathbf{c}_\alpha=a^{\alpha'}_\alpha \mathbf{c}_{\alpha'}
    \end{align*}
    Mentre $(b^\alpha)$ $(b^{\alpha'})$ sono le componenti rispettivamente del vettore $OO'$ e $O'O$.}{}{}{}
    \definizione{ Sia $A$ uno spazio affine. Una funzione $f$ del tipo:
    \begin{align*}
        f:A\to \mathbb{R}\\
        P\mapsto f(P)
    \end{align*}
    è detta \underline{campo scalare}.}
    \definizione{Possiamo vedere $f=g\circ \Phi$ con:
    \begin{align*}
    A\xrightarrow[]{\Phi} \mathbb{R}^n\xrightarrow[]{g} \mathbb{R}\\P\xmapsto[]{\Phi}(x^\alpha)\xmapsto[]{g} g(x^\alpha)
    \end{align*}
    Dove $g\colon\mathbb{R}^n\to \mathbb{R}$ è detta \underline{rappresentazione del campo $f$}.}
    \proprieta{ L'insieme dei campi scalari su $A$ ha una struttura di anello commutativo ed algebra associativa e commutativa.\\
     Siano $f,g$ due campi scalari e $P\in A$, allora vale:
    \begin{enumerate}
        \item \textit{Somma di campi}: $(f+g)(P)=f(P)+g(P)$
        \item \textit{Prodotto numerico}: $(fg)(P)=f(P)g(P)$
        \item \textit{Prodotto per uno scalare}: $(af)(P)=af(P), a \in \mathbb{R}$
    \end{enumerate}
    Denoteremo con $\mathcal{F}(A)$ l'anello dei campi scalari sullo spazio affine $A$.}
    \definizione{ Un \underline{campo vettoriale} è una mappa:
    \begin{align*}
        \mathbf{X}\colon A \to A\times E\\
        P\mapsto (P,\mathbf{X}(P))
    \end{align*}
    ovvero ad un punto $P$ associa un vettore $\mathbf{X(P)}$ applicato in $P$.}
    \definizione{Fissato un riferimento cartesiano, ogni campo vettoriale $\mathbf{X}$ risulta rappresentato da un insieme di $n$ funzioni reali $X^\alpha\colon A \to \mathbb{R}$, dette \underline{componenti cartesiane}, tali che:
    \begin{align*}
        \mathbf{X}(P)=X^\alpha(P)\mathbf{c}_\alpha
    \end{align*}
    Queste componenti in quanto campi scalari hanno anche loro una funzione rappresentativa, tale per cui si possono anche denotare come $X^\alpha(x^{\beta})$.}
    \proprieta[]{Siano $\mathbf{X}$ e $\mathbf{Y}$ due campi vettoriali. Sono definite le operazioni di:
    \begin{enumerate}
        \item \textit{Somma}: $(\mathbf{X}+\mathbf{Y})(P)=\mathbf{X}(P)+\mathbf{Y}(P)$
        \item \textit{Prodotto per un numero reale}: $(a\mathbf{X})(P)=a(\mathbf{X}(P)), a \in \mathbb{R}$
    \end{enumerate}
    Con queste due operazioni l'insieme dei campi vettoriali su $A$, che denoteremo con $\mathcal{X}(A)$, ha una struttura di modulo.}
    \definizione{ Sia $f\in \mathcal{F}(A)$ e $X\in \mathcal{X}(A)$. La \underline{derivata di un campo scalare $f$} \underline{rispetto ad un campo vettoriale $\textbf{X}$} è il campo scalare:
    \begin{align*}
        \bm{X} (f)=X^\alpha\cfrac{\partial f}{\partial x^\alpha}
    \end{align*}
    Dove con: 
    \begin{align*}
            \frac{\partial f}{\partial x^\alpha}
    \end{align*}
    s'intende la derivata parziale rispetto alla $x^\alpha$ della funzione rappresentativa $f(x^1,\dots,x^n)$ in un qualunque sistema di coordinate cartesiane.}
   \osservazione{ Dunque è una mappa che, una volta fissato un $\mathbf{X}\in \mathcal{X}(A)$, lavora:
    \begin{align*}
        \bm{X} \colon \mathcal{F}(A)\to \mathcal{F}(A)\\
        f\mapsto \bm{X} (f)=X^\alpha\cfrac{\partial f}{\partial x^\alpha}
    \end{align*}}
    \proprieta{ Siano $f,g\in \mathcal{F}(A)$. Si verifica facilmente che ${\bm{X}}$ soddisfa:
    \begin{itemize}
        \item \textit{$\mathbb{R}$-lineare}: $\bm{X}(af+bg)=a\bm{X}(f)+b\bm{X}(g)$$\quad a,b\in \mathbb{R}$
        \item \textit{Regola di Leibnitz}: $\bm{X}(f\cdot g)=\bm{X}(f)\cdot g+f\cdot \textbf{X}(g)$
    \end{itemize} Una funzione che soddisfa queste due proprietà è per l'appunto chiamata \underline{derivazione}.}
    \paragrafo{Invarianza per cambiamenti di coordinate affini} {Un'altro fatto interessante in merito alla derivazione di un campo scalare $f$ rispetto ad un campo vettoriale $\mathbf{X}$ è la sua indipendenza dalle coordinate affini stabilite.\\
    Sia $X\in \mathcal{X}(A)$ e siano $(\mathbf{c}_\alpha)$ e $(\mathbf{c}_{\alpha'})$ due basi. Si consideri la rappresentazione del campo secondo le due basi:
    \begin{align*}
        \mathbf{X}=X^\alpha\mathbf{c}_\alpha =X^{\alpha'}\mathbf{c}_{\alpha'}
    \end{align*}
    Tenuto conto delle relazioni tra le basi, si ha la relazione:
    \begin{align*}
        X^{\alpha'}=a^{\alpha'}_\alpha X^\alpha
    \end{align*}
    D'altra parte, interpretando la $f$ come funzione delle $(x^\alpha)$ per il tramite delle $(x^{\alpha'})$ dalle relazioni precedenti tra le basi si ha:
    \begin{align*}
        \frac{\partial f}{\partial x^\alpha}=\frac{\partial f}{\partial x^{\alpha'}}\frac{\partial x^{\alpha'} }{\partial x^\alpha}=\frac{\partial f}{\partial x^{\alpha'}}a^{\alpha'}_\alpha
    \end{align*}
    Si ha quindi:
    \begin{align*}
        X^\alpha\frac{\partial f}{\partial x^\alpha}=X^\alpha\frac{\partial f}{\partial x^{\alpha'}}a^{\alpha'}_\alpha= X^{\alpha'}\frac{\partial f}{\partial x^{\alpha'}}
    \end{align*}
    Ciò mostra l'indipendenza della definizione dalla scelta delle coordinate cartesiane.}{}{}
    %%
    %%
    %%COMMUTATORE DI DUE CAMPI VETTORIALI
    %%
    %%
    %%
    \definizione{ Siano $\mathbf{X},\mathbf{Y}\in \mathcal{X}(A)$ .Definiamo il \underline{commutatore di $\mathbf{X}$ e $\mathbf{Y}$} come il campo vettoriale:
    \begin{align*}
    [\cdot,\cdot]\colon \mathcal{X}(A)\times \mathcal{X}(A)\to \mathcal{X}(A)\\
        (\mathbf{X},\mathbf{Y})\mapsto[\mathbf{X},\mathbf{Y}]_f=\mathbf{X}(\mathbf{Y}(f))-\mathbf{Y}(\mathbf{X}(f)) && f\in \mathcal{F}(A)
    \end{align*}}
    \proprieta{ Siano $\mathbf{X},\mathbf{Y},\mathbf{Z}\in \mathcal{X}(A)$. Il commutatore è un'operazione binaria interna $[\cdot,\cdot]$:
    \begin{itemize}
        \item \textit{Anticommutativa}: $[\mathbf{X},\mathbf{Y}]=-[\mathbf{Y},\mathbf{X}]$
        \item \textit{Bilineare}:
    $[a\mathbf{X}+b\mathbf{Y},\mathbf{Z}]=a[\mathbf{X},\mathbf{Z}]+b[\mathbf{Y},\mathbf{Z}]$
    \item Soddisfa l'\textit{Identità di Jacobi}:
    \begin{align*}
        [\mathbf{X},[\mathbf{Y},\mathbf{Z}]]+[\mathbf{Z},[\mathbf{X},\mathbf{Y}]]+[\mathbf{Y},[\mathbf{Z},\mathbf{X}]]=0
    \end{align*}
    \end{itemize}
    Ovvero $(\mathcal{X}(A),[\cdot,\cdot])$ è un'\underline{algebra di Lie}.}
\paragrafo{L'espressione del commutatore}{
Andiamo ad analizzare l'espressione del commutatore con l'obiettivo di renderla più semplice e compatta:
\begin{align*}
    [\bm{X},\bm{Y}]_f=X^\alpha\frac{\partial}{\partial x^\alpha}\left(Y^\beta \frac{\partial}{\partial x^\beta}(f)\right)-Y^\alpha \frac{\partial}{\partial x^\alpha}\left(X^\beta \frac{\partial}{\partial x^\beta}(f)\right)=\\
    =\left(X^\alpha \frac{\partial}{\partial x^\alpha}Y^\beta\right)\frac{\partial}{\partial x^\beta}(f)+\cancel{X^\alpha Y^\beta \frac{\partial}{\partial x^\alpha}\left(\frac{\partial }{\partial x^\beta}(f)\right)}+\\
    -\left(Y^\alpha \frac{\partial}{\partial x^\alpha}X^\beta\right)\frac{\partial}{\partial x^\beta}(f)+\cancel{Y^\alpha X^\beta \frac{\partial}{\partial x^\alpha}\left(\frac{\partial }{\partial x^\beta}(f)\right)}
\end{align*}
Scambiando $\alpha\leftrightarrow\beta$ e semplificando i termini opposti si ottiene:
\begin{align*}
    \left(X^\alpha \frac{\partial}{\partial x^\alpha}Y^\beta-Y^\alpha \frac{\partial}{\partial x^\alpha}X^\beta\right)\frac{\partial}{\partial x^\beta}(f)=[\bm{X},\bm{Y}]^\beta \frac{\partial}{\partial x^\beta}(f) \quad \forall\, f \in \mathcal{F}(A)
\end{align*}
Ottenendo cosi:
\begin{align*}
    [\bm{X},\bm{Y}]=[\bm{X},\bm{Y}]^\beta\bm{c}_\beta
\end{align*}
dove:
\begin{align*}
    [\bm{X},\bm{Y}]^\beta=X^\alpha \frac{\partial}{\partial x^\alpha}Y^\beta-Y^\alpha\frac{\partial}{\partial x^\alpha}X^\beta
\end{align*}
è la componente rispetto alla base $\bm{c}_\beta$.
}{ldflds}{}
\definizione{ Sia $\mathbf{X}\in \mathcal{X}(A)$. La \underline{divergenza} di $\mathbf{X}$ è una funzione:
\begin{align*}div\colon\mathcal{X}(A)\to \mathcal{F}(A)\\
     \bm{X}\mapsto div(\bm{X})
    \end{align*}
    Dove $div(\bm{X})=\cfrac{\partial X^1}{\partial x^1}+\dots +\cfrac{\partial X^n}{\partial x^n}=\cfrac{\partial X^\alpha}{\partial x^\alpha}$.}

\proprieta{Siano $\mathbf{X},\mathbf{Y}\in \mathcal{X}(A)$. La divergenza gode delle seguenti proprietà:
\begin{itemize}
    \item \textit{Somma}: $div(\bm{X}+\bm{Y})=div(\bm{X})+div(\bm{Y})$
    \item \textit{Prodotto per campo scalare}: $div(f\bm{X})=fdiv(\bm{X})+ \bm{X}(f),\: f\in \mathcal{F}(A)$
    \item $\bm{X}=$ costante\footnote{Considerato $\bm{X}=$ costante nelle coordinate cartesiane affini $(x^\alpha)$} $\Rightarrow \: div(\bm{X})=0$
\end{itemize}}

%%
%%
%%
%%CARTA E COORDINATE NON AFFINI
%%
%%
%%
\definizione{ Una \underline{carta} di dimensione $n$ su un insieme $A$ è una coppia $(U,\varphi)$, dove 
\begin{itemize}
    \item $U\subseteq A$  
    \item $\varphi$ è una mappa biettiva:
    \begin{align*}
        \varphi\colon U\to \varphi(U)\subseteq \mathbb{R}^n
    \end{align*}
    la cui immagine $\varphi(U)$ è un aperto di $\mathbb{R}^n$.
\end{itemize}}
\definizione{ Possiamo definire le \underline{coordinate associate} alla carta $(U,\varphi)$ come le $n$ funzioni:
\begin{align*}
    q^i\colon U\to \mathbb{R}&&
    q^i=pr_i\circ \varphi
\end{align*}
dove:\begin{minipage}{4cm}
\begin{align*}
pr_i\colon\mathbb{R}^n\to \mathbb{R}\\
    (r^1,...r^n)\mapsto r^i
\end{align*}
\end{minipage}
\begin{minipage}{7cm}
è la proiezione della i-esima coordinata.
\end{minipage}}
\osservazione[]{ Siano $A$ uno spazio affine, $(x^\alpha)$ delle coordinate affini su $A$ e $(U,\varphi)$ una carta di dimensione $n$.\\
Le coordinate $q^i$ si possono rappresentare come funzioni delle $n$ $(x^\alpha)$:
\begin{align*}
    q^i=q^i(x^\alpha)
\end{align*}
Essendo tutte applicazioni biettive si può invertire su $\varphi (U)$, ovvero:
\begin{align*}
    x^\alpha=x^\alpha(q^i)
\end{align*}
Quindi ricapitolando}
\definizione{ Si possono definire i \underline{cambiamenti} (o \underline{trasformazioni}) \underline{di coordinate}, come:
\begin{align*}
    q^i=q^i(x^\alpha) && x^\alpha=x^\alpha(q^i)
\end{align*}
Con le matrici Jacobiane delle trasformazioni, rispettivamente:
\begin{align*}
    E^i_\alpha= \frac{\partial q^i}{\partial x^\alpha} (x^\beta) && E^\alpha_i= \frac{\partial x^\alpha}{\partial q^i} (q^j)
\end{align*}
Queste sono regolari e una l'inversa dell'altra.}
\paragrafo{Coordinate non affini - Cerchio}{
Si considerino le coordinate del piano affine $(x,y)$ e le trasformazioni di queste in coordinate polari piane $(r,\theta)$:
\begin{align*}
x^\alpha=x^\alpha(q^i)\colon\begin{cases}
    x=r\,cos\theta\\
    y=r\,sen\theta
    \end{cases} &&r>0,\: -\pi<\theta<\pi
\end{align*}%% disegno disegno
Queste hanno come trasformazione inversa di coordinate:
\begin{align*}
q^i=q^i(x^\alpha)\colon \begin{cases}
r=\sqrt{x^2+y^2}\\
\theta=\begin{cases}
\arcsin\cfrac{y}{\sqrt{x^2+y^2}} \quad x\ge 0 \\
\pi-\arcsin\cfrac{y}{\sqrt{x^2+y^2}}\quad x<0,\, y>0\\
-\pi-\arcsin\cfrac{y}{\sqrt{x^2+y^2}}\quad x<0,\, y<0
\end{cases}
    \end{cases}
\end{align*}}{}{}
\paragrafo{ESEMPIO coordinate non affini - Sfera}{ Consideriamo le coordinate dello spazio affine $(O,(x,y,z))$ e la trasformazione in coordinate polari sferiche:
\begin{align*}
    x^\alpha=x^\alpha(q^i)\colon\begin{cases}
x=r\:sen\varphi\:cos\theta\\
y=r\:cos\varphi\:sin\theta\\
 z=r\:cos\varphi
    \end{cases}
\end{align*}
Dove $(q^1,q^2,q^3)=(r,\varphi,\theta)$ sono definite sul dominio aperto $U$ in $\mathbb{R}^3$, asportando il semiasse positivo delle $x$ e tutto l'asse $z$.\\
 La carta è a valori nell'aperto $\varphi(V)=\left\{(r,\varphi,\theta)\in \mathbb{R}^3\colon r>0, 0<\varphi<\pi, 0<\theta<2\pi\right\}$, dove:
\begin{align*}
    \begin{cases}
        r\equiv\text{raggio}\\
        \varphi\equiv\text{colatitudine}\\
        \theta\equiv \text{longitudine}
    \end{cases}
\end{align*}}{}{}
\notazione[]{In generale $x^\alpha=x^\alpha(q^i)$ si può anche scrivere come $\textbf{x}=\mathbf{x}(q^i)$ con $\mathbf{x}$ un vettore in $\mathbb{R}^3$ che rappresenta il vettore posizione $OP=x^\alpha \mathbf{e}_\alpha$.}
\definizione{ Si possono definire $n$ campi vettoriali su $\varphi(U)$, come segue:
\begin{align*}
    \mathbf{E}_i=\frac{\partial \mathbf{x}}{\partial q^i}
\end{align*}
Questi ovviamente hanno componenti rispetto alla base $(\mathbf{c}_\alpha)$:
\begin{align*}
    \mathbf{E}_i=E^\alpha_i\mathbf{c}_\alpha
\end{align*}
E vengono chiamati \underline{riferimento naturale associato alle coordinate non affini} $(q^i)$.}


Le loro caratteristiche principali sono:
\begin{enumerate}
   \item Non essere, in generale,  costanti, a causa della loro dipendenza $\mathbf{E}_i=\mathbf{E}_i(q^j)$.
   \item Essere tra loro indipendenti e costituire così una base dei campi di vettori su $U$.
\end{enumerate}
\paragrafo{Trasformazione di $\bm{X}$:coordinate affini$\leadsto$ coordinate non affini}{ Sia $\mathbf{X}\in \mathcal{X}(A)$. Sappiamo sia che:
\begin{align*}
    \mathbf{X}=X^i(q^j)\mathbf{E}_i=\boxed{X^i\mathbf{E}_i}=X^iE^\alpha_i\mathbf{c}_\alpha=\boxed{X^\alpha\mathbf{c}_\alpha}
\end{align*}
Uguagliando cosi i due termini evidenziati abbiamo le relazioni tra i coefficienti delle due rappresentazioni del campo vettoriale $\mathbf{X}$:
\begin{align*}
    X^i=E^i_\alpha X^\alpha\\
    X^\alpha=E^\alpha_iX^i
\end{align*}}{}{}
\paragrafo{Interpretazione degli $\mathbf{E}_i$ come derivazioni}{ Consideriamo i campi vettoriale $\mathbf{E}_i$. Questi possiamo interpretarli come delle derivazioni del tipo:
\begin{align*}
    \mathbf{E}_i(f)=\frac{\partial f}{\partial q^i}
\end{align*}
Ovvero come la derivata della funzione rappresentativa $f$ rispetto alle coordinate non affini $(q^i)$. Per fare ciò, quello che sta accadendo è:
\begin{align*}
    \varphi (U)\xrightarrow[]{\varphi^{-1}} U\xrightarrow[]{f} \mathbb{R}\\
    (q^i)\mapsto\varphi^{-1}(q^i)\mapsto f(\varphi^{-1}(q^i))\equiv f(q^i)
\end{align*}
Quindi procedendo con questa interpretazione:
\begin{align*}
    \bm{E}_i(f)=\frac{\partial}{\partial q^i}(f)=\frac{\partial x^\alpha}{\partial q^i}\frac{\partial}{\partial x^\alpha}(f)=E^\alpha_i\frac{\partial}{\partial x^\alpha}(f)
\end{align*}
Concludendo, dunque, per un generico campo vettoriale $\mathbf{X}$:
\begin{align*}
    \bm{X}(f)=X^i\frac{\partial}{\partial q^i}(f)=X^\alpha \frac{\partial}{\partial x^\alpha}(f)
\end{align*}}{}{}
%%
%%
%%SIMBOLI DI CHRISTOFFEL
%%
%%
\section{Simboli di Christoffel} Consideriamo i campi vettoriali $\mathbf{E_i}$. Visto che questi possono essere espressi in funzione delle coordinate $q^i$, ha senso considerarne le derivate parziali:
\begin{align*}
    \frac{\partial \mathbf{E}_i}{\partial q^j}=\frac{\partial^2\mathbf{x}}{\partial q^j\partial q^i}
\end{align*}
Queste derivate parziali sono a loro volta dei campi vettoriali e quindi possiamo considerarne la rappresentazione secondo il riferimento $(\mathbf{E}_i)$:
\begin{align*}
    \partial_j\mathbf{E}_i=\partial_j\partial_i\mathbf{x}=\Gamma^h_{ji}\mathbf{E}_h
\end{align*}
\definizione{Le componenti $\Gamma^h_{ji}$, definite come precede, sono delle funzioni sopra il dominio $U$ della carta, denominate \underline{simboli di Christoffel}.}
\proprieta{ I simboli di Christoffel hanno le seguenti proprietà:
\begin{itemize}
    \item $\boxed{\Gamma^h_{ji}=\Gamma^h_{ij}}$: Questo vale per definizione stessa dei simboli di Christoffel. Essendo definiti tramite le derivate seconde di funzioni regolari, sono simmetrici rispetto agli indici in basso.
    \item $\boxed{\Gamma^h_{ji}=0\iff \text{le coordinate sono cartesiane}}$: Dalla definizione si vede che sono identicamente nulli se e solo se i campi $\mathbf{E}_i$ sono costanti e ciò accade se e solo se le coordinate sono cartesiane.
\end{itemize}}
\osservazione[]{ Le componenti del commutatore sono le stesse in ogni sistema di coordinate:
\begin{align*}
    [\mathbf{X},\mathbf{Y}]^i=X^j\frac{\partial Y^i}{\partial q^j}-Y^i\frac{\partial X^j}{\partial q^j}
\end{align*}}
\definizione{ La \underline{divergenza in coordinate non affini} è il campo scalare:
\begin{align*}
    div\mathbf{X}=\frac{\partial}{\partial q^i}X^i+\Gamma^j_{ji}X^i
\end{align*}
Si verifica facilmente che soddisfa le condizioni precedentemente enunciate per la divergenza.}
\osservazione[]{ Questa, euristicamente, può essere vista come:
\begin{itemize}
    \item La traccia di un'opportuna matrice:
    \begin{align*}
        (a_{ij})=\left( \frac{\partial X^j}{\partial q^i}+\Gamma^j_{ik} X^k\right)
    \end{align*}
    \item Un prodotto che \textit{assomiglia} al prodotto scalare
\end{itemize}}
\newpage
\section{Forme differenziali}
\definizione{ Sia $A$ uno spazio affine. Una \underline{forma lineare} o \underline{1-forma} su $A$ è un'applicazione:
\begin{align*}
    \varphi \colon \mathcal{X}(A)\to \mathcal{F}(A)
\end{align*}
tale che $\varphi$ sia \textit{$\mathcal{F}(A)$-lineare}, ovvero:
\begin{align*}
    \varphi(f\mathbf{X}+g\mathbf{Y})=f\varphi(\mathbf{X})+g\varphi(\mathbf{Y})&& \forall\, f,g\in \mathcal{F}(A)\:\:\forall \, \mathbf{X},\mathbf{Y}\in \mathcal{X}(A)
\end{align*}}
\proprieta{L'insieme delle forme lineari su $A$, $\Phi^1(A)$, è un modulo sull'anello $\mathcal{F(A)}$. Le operazioni sono così definite:
\begin{itemize}
    \item \textit{Somma di 1-forme}: $(\varphi+ \psi)(\mathbf{X})=\varphi(\mathbf{X})+\psi(\mathbf{X})$, $\forall\, \varphi, \psi\in \Phi^1(A)$
    \item \textit{Prodotto per un campo scalare}: $(f\varphi)(\mathbf{X})=f\cdot \varphi(\mathbf{X})$, $\forall\, f \in \mathcal{F}(A)$
\end{itemize}}

\notazione[]{ Denotiamo con $\langle \mathbf{X},\varphi\rangle $ il valore della forma lineare $\varphi$ sul campo vettoriale $\mathbf{X}$. In tal modo}
\definizione{ Definiamo un'applicazione lineare:
\begin{align*}
    \langle \cdot,\cdot\rangle  \colon \mathcal{X}(A)\times \Phi^1(A)\to \mathcal{F}(A)
\end{align*}
che prende il nome di \underline{valutazione} tra una forma lineare e un campo vettoriale.}
\osservazione[]{Una forma lineare può anche essere interpretata come \underline{campo di covettori}, cioè come un'applicazione:
\begin{align*}
    \varphi \colon A \to A \times E^*
\end{align*}
che associa ad ogni punto $P\in A$ un covettore applicato in $P$.\\
Il collegamento tra questa e la definizione precedente è dato dalla formula:
\begin{align*}
    \langle \mathbf{X},\varphi(P)\rangle =\langle \mathbf{X}(P),\varphi(P)\rangle 
\end{align*} 
Assume cosi senso valutare una 1-forma $\varphi$ su di un vettore applicato $(P,\mathbf{v})$. Il risultato $\langle \mathbf{v},\varphi(P)\rangle $ è un numero reale.}
\paragrafo{Il differenziale}{Un esempio fondamentale di 1-forma è il \ul{differenziale $df$ di un campo scalare $f$}.\\
Questo è definito dall'uguaglianza:
\begin{align*}
    \langle \mathbf{X},df\rangle = \mathbf{X}(f)
\end{align*}
La linearità dell'applicazione:
\begin{align*}
    df\colon \mathcal{X}(A)\to \mathcal{F}(A)\\
    \mathbf{X}\mapsto \langle \mathbf{X},df\rangle 
\end{align*}
segue dal fatto che $\mathbf{X}(f)$ è lineare rispetto al campo vettoriale $\mathbf{X}$, una volta fissato il campo scalare $f$. Inoltre dalla regola di Leibnitz per la derivata rispetto ad un campo vettoriale segue la regola di Leibnitz per il differenziale:
\begin{align*}
    d(fg)=gdf+fdg
\end{align*}}{}{}
\paragraph{Differenziale di $\mathbf{(q^i)}$} Il nostro obiettivo adesso è quello di studiare il differenziale delle $(q^i)$, generiche coordinate su un aperto $U$.\\
Essendo delle funzioni reali su $U$ possiamo considerarne i differenziali $(dq^i)$. Queste, come già visto, fanno corrispondere ad un campo $\mathbf{X}$ le sue componenti $X^i$:
\begin{align*}
    \langle\mathbf{X},dq^i\rangle=X^i
\end{align*}
E quindi in particolare:
\begin{align*}
    \langle \mathbf{E}_k,dq^i\rangle=\delta_k^i
\end{align*}
D'altra parte, nel dominio $U$, ogni forma lineare è combinazione lineare dei differenziali delle coordinate, ovvero ammette una rappresentazione locale:
\begin{equation}
    \label{sus1}
    \mathbf{\varphi}=\varphi_idq^i
\end{equation}
Dove le $(\varphi_i)$ sono funzioni reali su $U$ dette \underline{componenti} di $\mathbf{\varphi}$ nelle coordinate $(q^i)$, definite da:
\begin{equation}
    \label{sus2}
    \varphi_i=\langle\mathbf{E}_i,\mathbf{\varphi}\rangle
\end{equation}
Si noti come \ref{sus1}$\leadsto$\ref{sus2}, infatti:
\begin{align*}
    \langle \mathbf{E}_k,\mathbf{\varphi}\rangle = \varphi_i\langle \mathbf{E}_k,dq^i\rangle = \varphi_i \delta_k^i=\varphi_k
\end{align*}
Viceversa, \ref{sus2}$\leadsto$\ref{sus1}:
\begin{align*}
    \langle \mathbf{X},\varphi_idq^i\rangle=\varphi_i \langle \mathbf{X},dq^i\rangle= \langle \mathbf{E}_i, \mathbf{\varphi}\rangle X^i=\langle X^i\mathbf{E}_i, \mathbf{\varphi}\rangle =\langle \mathbf{X}, \mathbf{\varphi}\rangle
\end{align*}
Dalle formule precedenti segue che la valutazione di una forma lineare sopra un campo vettoriale è data, \textit{qualunque siano le coordinate scelte}, dalla somma dei prodotti delle componenti omologhe:
\begin{align*}
    \boxed{\langle \mathbf{X},\mathbf{\varphi}\rangle = X^i\varphi_i}
\end{align*}
%%
%%
%%
%%
%% FORME DIFFERENZIALI
%%
%%
%%
%%
\definizione{ Una \underline{forma differenziale} o \underline{p-forma} su uno spazio affine $A$ è un'applicazione \textit{p-lineare antisimmetrica} dello spazio $\mathcal{X}(A)^p$ nello spazio $\mathcal{F}(A)$:
\begin{align*}
    \phi \colon \underbrace{\mathcal{X}(A)\times \mathcal{X}(A)\times \dots \times \mathcal{X}(A)}_{p \text{ volte}}\to \mathcal{F}(A)
\end{align*}}
\notazione{Indicheremo con:
\begin{align*}
    \Phi^p(A)=\text{ spazio delle $p$ forme sopra } A
\end{align*}
definendo:
\begin{align*}
    \Phi^0(A)\colon =\mathcal{F}(A)
\end{align*}}
\paragrafo{Derivazione esterna}{ Sulle $p$-forme differenziali è definita l'operazione fondamentale chiamata \underline{derivazione esterna}.
Questa è una generalizzazione dell'operatore di differenziale applicabile sulle $0$-forme alle $p$-forme e per questo è indicato con $d$.
\begin{align*}
    d\colon \Phi^p(A)\to \Phi ^{p+1}(A)
\end{align*}
La sua proprietà fondamentale è $d^2\equiv 0$.}{}{}
\osservazione[]{ Per l'antisimmetria, se $p>n=dim(A)$, allora $\Phi^p(A)\equiv 0$.}
\paragrafo{Invarianza della rappresentazione differenziale 1-forma}{
Sia $\varphi$ una 1-forma su $A$ scritta in rappresentazione locale come:
\begin{align*}
    \varphi = \varphi_idq^i
\end{align*}
Il contenuto di questa paragrafo sarà quello di dimostrare la rappresentazione del suo differenziale in coordinate locali:
\begin{align*}
    d\varphi=d\varphi _i\wedge dq^i
\end{align*}
e il fatto che questa non dipenda dalle coordinate scelte. Ovvero presa:
\begin{align*}
    \varphi = \varphi_{i'}dq^{i'}\longrightarrow d\varphi = \varphi_{i'}\wedge dq^{i'}
\end{align*}
Mostriamo questo secondo fatto.\\
Siano $(q^i)$ e $(q^{i'})$ due sistemi di coordinate generiche. Ricordando che la matrice Jacobiana della trasformazione di coordinate è:
\begin{align*}
    E^{i'}_i=\frac{\partial q^{i'}}{\partial q^i}
\end{align*}
Si noti come:
\begin{align*}
    dq^{i'}=\frac{\partial q^{i'}}{\partial q^i}dq^i
\end{align*}
E quindi:
\begin{align*}
    \varphi_{i'}=E^i_{i'}\varphi_i
\end{align*}
Iniziamo allora i calcoli. Tenendo a mente che:
\begin{align*}
    d\varphi_{i'}=d(E^i_{i'}\varphi_i)=dE^i_{i'}\varphi_i+E^i_{i'}d\varphi_i
\end{align*}
Studiamo il membro destro della tesi:
\begin{align*}
    d\varphi_{i'}\wedge dq^{i'}=\varphi_i\frac{\partial E^{i}_{i'}}{\partial q^k}dq^k\wedge dq^{i'}+E^i_{i'}d\varphi_i\wedge dq^{i'}=\\
    =\underbrace{\varphi_i\underbrace{\frac{\partial q^i}{\partial q^{i'} \partial q^{k'}}}_{\text{simmetrico in $i',k'$}}\overbrace{dq^{k'}\wedge dq^{i'}}^{\text{antisimmetrico in $i',k'$}}}_{\equiv 0}+d\varphi_i\wedge E^i_{i'}dq^{i'}=\\
    =d\varphi_i\wedge dq^i
\end{align*}}{}{}
%%
%%
%%
%%CHAPTER 2 CURVE, SUPERFICI ETC
%%
%%
%%
\chapter{Curve negli spazi affini, rappresentazione in coordinate non affini, e sistemi dinamici}
%%
%%
%%
%%
%%CURVA PARAMETRIZZATA
%%
%%
\definizione{Chiamiamo \underline{curva parametrizzata} in uno spazio affine $A$ un'applicazione $\gamma \colon I \to A$ da un intervallo aperto $I\subseteq \mathbb{R}$ nello spazio affine.}
\definizione{Considerata un'origine $O\in A$, per la curva $\gamma$ vi è una \underline{rappresentazione vettoriale}:
\begin{align*}
    \mathbf{x}=\gamma(t)&& \text{con }\mathbf{x}=OP
\end{align*}
che dunque identifica i punti $P\in A$ con il loro vettore posizione rispetto al punto $O$.}


Siano $(x^\alpha)$ delle coordinate cartesiane aventi origine in $O$. Si possono allora considerare le equazioni parametriche:
\begin{align*}
    x^\alpha= \gamma^\alpha(t) && \alpha=1,\dots, n
\end{align*}
\paragrafo{Interpretazione cinematica}{Una curva può essere interpretata come moto di un punto $P$ nello spazio affine, se il parametro $t$ viene inteso come tempo.\\
Nel caso in cui la curva rappresenti il moto di un punto nello spazio affine tridimensionale euclideo, il generico vettore $OP=\mathbf{x}$ è chiamato \underline{vettore posizione}.}{}{}
\definizione{L'immagine della curva, cioè l'insieme 
\begin{align*}
    \gamma(I)=\{P\in A | \exists\,t \in I: \gamma(t)=P\}
\end{align*}
 è detta \underline{traiettoria} o \underline{orbita}\footnote{In geometria è questa in realtà la vera e propria curva, ovvero il luogo dei punti definito da $n-1$ equazioni.}.}
\definizione{Il \underline{vettore tangente} alla curva $\gamma$ nel punto $\gamma(t)$ è il vettore denotato con $\dot{\gamma}(t)$ definito dal limite:
\begin{align*}
    \dot{\gamma}(t)=\lim_{h\to 0}\frac{\gamma(t+h)-\gamma(t)}{h}
\end{align*}
Questo nel contesto cinematico prende il nome di \underline{velocità istantanea} e lo si denota con $\mathbf{v}(t)$.}
\paragrafo{Campo tangente come curva}{ Conviene interpretare il vettore tangente $\dot{\gamma}(t)=\mathbf{v}(t)$ come vettore applicato nel punto $\gamma(t)$. Ovvero come un'applicazione:
\begin{align*}
    \hat{\gamma}(t)\colon I \to A\times E\\
    t\mapsto (\gamma(t),\dot{\gamma}(t))
\end{align*}
che viene detta \underline{curva tangente} della curva $\gamma\colon I \to A$.}{}{}
\definizione{ Siano $\gamma \colon I \to \mathbb{R}$ una curva e $F\colon A \to \mathbb{R}$ un campo scalare, entrambi almeno di classe $C^1$.
Definiamo la \ul{derivata del campo scalare $F$ lungo la curva $\gamma$} come:
\begin{align*}
    \frac{d}{dt}(F\circ \gamma)(t)=\langle \mathbf{v}(t),dF\rangle \quad \forall t \in I
\end{align*}
Questa è la definizione naturale, infatti:
\begin{align*}
    \frac{d}{dt}(F\circ \gamma)(t)=\frac{\partial F}{\partial x^\alpha}\frac{d\gamma^\alpha}{dt}(t)=\frac{\partial F}{\partial x^\alpha}v^\alpha(t)=\langle\mathbf{v}(t),dF\rangle
\end{align*}}
%%
%%
%%CURVA INTEGRALE
%%
%%
\definizione{Una \underline{curva integrale di un campo vettoriale $\mathbf{X}$} è una curva:
\begin{align*}
    \gamma \colon I \to A
\end{align*}
tale che:
\begin{itemize}
    \item $0\in I\subseteq \mathbb{R}$;
    \item $\forall \,\gamma(t)\in A$, il vettore tangente $\dot{\gamma}(t)$ coincide con il valore del campo $\mathbf{X}$ in quel punto ovvero:
    \begin{align*}
        \dot{\gamma}=\mathbf{X}\circ \gamma=\mathbf{X}(\gamma(t))\\
        I\xlongrightarrow{\gamma}A\xlongrightarrow[]{\mathbf{X}}A\times E\\
        \dot{\gamma}(t)\colon I \to A\times E
    \end{align*}
\end{itemize}}
\definizione{Diciamo inoltre che la curva integrale è \underline{basata nel punto} $P_0$ se $\gamma(0)=P_0$.}
\notazione{In rappresentazione vettoriale sarebbe:
\begin{align*}
    \mathbf{x}=\gamma(t) \text{ curva integrale} \iff \frac{d\mathbf{x}}{dt}=\mathbf{X}(\mathbf{x})
\end{align*}}


Le curve integrali rappresentano i moti delle particelle del fluido secondo l'interpretazione del campo $\mathbf{X}$ come campo di velocità.\\
\definizione{ Un campo vettoriale interpretato come campo di velocità viene detto \underline{sistema dinamico}.}
\notazione[]{ Un sistema dinamico si può rappresentare dunque come un'equazione differenziale:
\begin{itemize}
        \item vettoriale:
        \begin{align*}
            \frac{d\mathbf{x}}{dt}=\mathbf{X}(\mathbf{x})
        \end{align*}
        \item in coordinate affini:
        \begin{align*}
            \frac{dx^\alpha}{dt}=X^\alpha (x^\beta)
        \end{align*}
        \item in coordinate generiche:
        \begin{align*}
            \frac{dq^i}{dt}=X^i (x^j)
        \end{align*}
\end{itemize}
In particolare sono $n$ equazioni differenziali ordinarie in forma normale autonome.}
\section{Risoluzione sistemi dinamici} Integrare significa trovare tutte le soluzioni del sistema dinamico e queste costituiscono lo spazio delle soluzioni/spazio dei moti.
\definizione{Dato un sistema dinamico e un punto $P_0\in A$, parliamo di \ul{curva integrale massimale} $\gamma_{P_0}\colon I_{P_0}\to A$, quando presa un'altra curva integrale $\gamma\colon I\to A$ basata in $P_0$, allora:
\begin{align*}
    I \subseteq I_{P_0}&& \gamma_{P_0|_I}=\gamma
\end{align*}}
Una volta fissate le condizioni iniziali/dati iniziali riusciamo ad individuare un'unica soluzione del sistema dinamico grazie al
\teorema[Teorema di Cauchy]{fgdslkjlsdaljkfsd}{ Sia $\mathbf{X}$ un campo vettoriale di classe $C^k$($k\ge1$) su un dominio $M$.
Fissato un punto $P_0\in M$ esiste una e una sola curva integrale massimale basata in $P_0$
\begin{align*}
    \gamma_{P_0}\colon I_{P_0}\to M
\end{align*}
Se $I_{P_0}=\mathbb{R}$, $\mathbf{X}$ si dice \ul{completo}.}
\definizione{Il \underline{flusso del campo di vettori $\mathbf{X}$} descrive lo spazio delle soluzioni (o insieme di tutte le curve integrali del campo $\mathbf{X}$).
Si definisce come la funzione:
\begin{align*}
    \varphi\colon D\subseteq \mathbb{R}\times M \to M\\
    (t,P_0)\mapsto \gamma_{P_0}(t)
\end{align*}}
\teorema[]{sdljfkdslkgjldgdd}{Sia $\mathbf{X}$ campo vettoriale di classe $C^k$($k\ge1$) su un dominio $M$, allora:
\begin{enumerate}
    \item Il dominio $D$ del flusso $\varphi$ è un aperto di $\mathbb{R}\times M$ e $\varphi\in C^k(D)$
    \item Sia $V\subseteq M$ aperto, $\delta >0$ e si consideri $(-\delta,\delta)\times V\subseteq D$. Allora $\forall \, t \in (-\delta,\delta)$:
    \begin{align*}
       \varphi_t\colon V\to V_t\\
       P\mapsto \varphi(t,P) 
    \end{align*}
    è un omemomorfismo $C^k$ di $V$ su $V_t\subseteq M$ con $\varphi_t\colon P\mapsto\varphi(-t,P)$ omeomorfismo inverso
    \item Vale:
    \begin{align*}
      \varphi(t,\varphi(s,P))=\varphi(t+s,P)  
    \end{align*}
    $\forall\, t,s,P$ per cui i due membri hanno significato.
\end{enumerate}}
\osservazione[]{Se $\mathbf{X}$ è completo, allora:
\begin{itemize}
    \item $D=\mathbb{R}\times M$
    \item $\varphi_t\colon M\to M,P\mapsto\varphi_t(P)=\gamma_P(t)$ è una trasformazione $C^k$ di M.
\end{itemize}}
\definizione{ Al variare del parametro $t\in \mathbb{R}$ i diversi flussi $\varphi_t$ costituiscono un \underline{gruppo ad un parametro}, ovvero un'insieme $\{\varphi_t|t\in \mathbb{R}\}$ tale che valgono:
\begin{itemize}
    \item $\varphi_t\circ \varphi_s=\varphi_{t+s}$
    \item $\varphi_t\circ\varphi_s=\varphi_s\circ \varphi_t$
    \item $\varphi_0=id_M$
    \item $(\varphi_t)^{-1}=\varphi_{-t}$
\end{itemize}}
\osservazione[]{ Ad ogni gruppo ad un parametro si può associare il campo vettoriale $\mathbf{X}$ corrispondente e viceversa. Ovvero questo sono condizioni necessarie e sufficienti affinché un insieme di curve $\varphi(t,P)$ sia lo spazio delle soluzioni di un certo campo vettoriale.}
\definizione{ Sia $\mathbf{X}\in \mathcal{X}(A)$. Definiamo l'\underline{integrale primo di $\mathbf{X}$} come il campo scalare $F\colon A\to \mathbb{R}$ tale che $\forall\, \gamma \colon I \to A$ curva integrale, vale:


\parbox{10em}{\centering
   $F\circ \gamma (t_1)=F\circ \gamma (t_2)$
   $\forall\, t_1,t_2\in I$
    }$\iff$
\parbox{10em}{
    $\cfrac{d}{dt}(F\circ\gamma)(t)=0, \forall \, t\in I$
  }
    $\iff$
\parbox{10em}{
         $\langle \mathbf{X},dF\rangle =0$}
}
%%
%%
%%
%%
%%
%%%VELOCITA' E ACCELERAZIONE
%%
%
%
%%
\section{Velocità e accelerazione in coordinate non affini}
\paragrafo{Velocità in coordinate non affini}{ Consideriamo una curva $\gamma$ e delle coordinate non affini $(q^i)$. La curva è rappresentata da:
\begin{align*}
    q^i(t)=\gamma^i(t)
\end{align*}
Ricordando che $\mathbf{x}=x^i\mathbf{E}_i$, abbiamo che:
\begin{align*}
    \mathbf{v}=\frac{d\mathbf{x}(t)}{dt}=\underbrace{\frac{\partial \mathbf{x}}{\partial q^i}}_{\mathbf{E}_i}\cdot \frac{dq^i}{dt}=v^i\mathbf{E}_i
\end{align*}
Quindi la velocità in componenti rispetto alle cordinate $(q^i)$ è data da 
\begin{align*}
    v^i(t)=\frac{dq^i(t)}{dt}
\end{align*}}{}{}
\osservazione[]{ Come già visto l'espressione del campo di vettori $\mathbf{v}$ non cambia nei due sistemi $(\mathbf{c}_\alpha)$ e $(\mathbf{E}_i)$, ovvero:
\begin{align*}
    \mathbf{v}=v^\alpha \mathbf{c}_\alpha = v^i\mathbf{E}_i
\end{align*}}
\paragrafo{Accelerazione in coordinate non affini}{ Nel riferimento affine $(\mathbf{c}_\alpha)$, l'accelerazione è definita come:
\begin{align*}
    \mathbf{a}=\frac{d\mathbf{v}}{dt}=\frac{dv^\alpha}{dt}(t)\mathbf{c}_\alpha=a^\alpha(t)\mathbf{c}_\alpha
\end{align*}
Ora allora analizziamo $\cfrac{d\mathbf{v}}{dt}$ passando per le coordinate non affini:
\begin{align*}
    \frac{d\mathbf{v}}{dt}=\frac{dv^i}{dt}\mathbf{E}_i+v^i\frac{d}{dt}\mathbf{E}_i=\frac{dv^i}{dt}\mathbf{E}_i+v^i\frac{\partial}{\partial q^j}(\mathbf{E}_i)\cdot\overbrace{\frac{dq^j}{dt}}^{v^j}=\frac{dv^i}{dt}\mathbf{E}_i+\boxed{v^iv^j\Gamma^k_{ji}\mathbf{E}_k}
\end{align*}
Ora, scambiando $i\leftrightarrow k$ in $\square$ e raccogliendo, otteniamo:
\begin{align*}
    \left(\frac{dv^i}{dt}+v^kv^j\Gamma_{jk}^i\right)\mathbf{E}_i=a^i\mathbf{E}_i
\end{align*}
Quindi si noti che \underline{a meno che} $\Gamma_{jk}^i=0$, abbiamo:
\begin{align*}
    a^i\ne \frac{dv^i}{dt}
\end{align*}}{}{}
\paragrafo{Coordinate polari}{ Consideriamo le coordinate polari. Per quanto riguarda la velocità abbiamo:
\begin{align*}
    \mathbf{v}=\dot{r}\mathbf{E}_r+\dot{\theta}\mathbf{E}_\theta=v^r\mathbf{E}_r+v^\theta\mathbf{E}_\theta
\end{align*}
Ora analizziamo l'accelerazione sfruttando l'espressione ottenuta precedentemente:
\begin{align*}
    \mathbf{a}=\left(\frac{dv^r}{dt}+\cancel{v^rv^\theta\Gamma^r_{r\theta}}+v^\theta v^\theta\Gamma^r_{\theta \theta}+\cancel{v^\theta v^r\Gamma^r_{\theta r}}+\cancel{v^rv^r\Gamma^r_{rr}}\right)\mathbf{E}_r+\\
    +\left(\frac{dv^\theta}{dt}+v^rv^\theta\Gamma^\theta_{r\theta}+\cancel{v^r v^r\Gamma^\theta_{rr}}+v^\theta v^r\Gamma^\theta_{\theta r}+\cancel{v^\theta v^\theta \Gamma^\theta_{\theta \theta}}\right)\mathbf{E}_\theta
\end{align*}
Ricordando inoltre che $\Gamma^r_{\theta \theta}=-r$ e $\Gamma^\theta_{r \theta}=\Gamma^\theta_{\theta r}=\cfrac{1}{r}$:
\begin{align*}
    \left(\frac{dv^r}{dt}-rv^\theta v^\theta\right)\mathbf{E}_r+\left(\frac{dv^\theta}{dt}+\frac{2}{r}v^\theta v^r\right)\mathbf{E}_\theta=\\
    =(\ddot{r}-r\dot{\theta}^2)\mathbf{E}_r+(\ddot{\theta}+\frac{2}{r}\dot{r}\dot{\theta})\mathbf{E}_\theta=\\
    =a^r\mathbf{E}_r+a^\theta\mathbf{E}_\theta
\end{align*}
Si osservi che ponendo $\mathbf{E}_r=\mathbf{u}$ e $\mathbf{E}_\theta=r\mathbf{\tau}$ con $\mathbf{u}$ e $\mathbf{\tau}$ versori nella rappresentazione radiale del moto si ottiene la classica scomposizione:
\begin{align*}
    \mathbf{a}=\mathbf{a}_{\text{radiale}}+\mathbf{a}_{\text{trasversale}}
\end{align*}}{}{}
\definizione{Definiamo la \ul{velocità areolare} come:
\begin{align*}
    \bm{v}_{ar}=\frac{1}{2}\bm{r}\wedge \bm{v}=\frac{1}{2}r^2\dot{\theta}\bm{k}
\end{align*}}
\definizione{Definiamo \ul{moto centrale} un moto tale per cui:
\begin{align*}
    \exists \,O\in A \text{ origine }\colon \forall\, t\in I \quad \bm{a}(t)\wedge \bm{r}(t)=0
\end{align*}}
Studiando adesso che la derivata della velocità areolare è:
\begin{align*}
    \frac{d}{dt}(\bm{v}_{ar})=\frac{1}{2}\frac{d}{dt}(\bm{r} \wedge \bm{v})=\frac{1}{2}(\overbrace{\bm{v}\wedge \bm{v}}^{0}+\bm{r}\wedge \bm{a})=\frac{1}{2}\bm{r}\wedge \bm{a}
\end{align*}
e che quindi è uguale a $0$, ovvero la velocità areolare è costante$\iff$ il moto è centrale.
\paragrafo{Dalla traiettoria agli enti fondamentali della cinematica}{ Sia $\mathbf{r}=r(\theta)$ . Da esso si possono scrivere tutti gli enti fonamentali della cinematica.\\
Sia $\mathbf{r}=\mathbf{r}(\theta(t))\mathbf{u}$ la nostra traiettoria del moto. Studiamo la velocità:
\begin{align*}
    \mathbf{v}=\frac{d\mathbf{r}}{dt}=\frac{dr}{d\theta}\cdot \dot{\theta}\mathbf{u}+r\frac{d\mathbf{u}}{d\theta}\cdot\dot{\theta}=\dot{\theta}\left(\frac{dr}{d\theta}\mathbf{u}+r\frac{d\mathbf{u}}{d\theta}\right)
\end{align*}
Ora utilizzando la costante delle aree $c=r^2\dot{\theta}$ otteniamo:
\begin{align*}
    \frac{c}{r^2}\left(\frac{dr}{d\theta}\mathbf{u}+r\frac{d\mathbf{u}}{d\theta}\right)=c\left(\frac{1}{r^2}\frac{dr}{d\theta}\mathbf{u}+\frac{1}{r}\frac{d\mathbf{u}}{d\theta}\right)
\end{align*}
e riconoscendo in $\cfrac{d\mathbf{u}}{d\theta}=\tau$, abbiamo che:
\begin{align*}
    \mathbf{v}=c\left(-\frac{d}{d\theta}\left(\frac{1}{r}\right)\mathbf{u}+\frac{1}{r}\tau\right)
\end{align*}
Prendiamo adesso in esame l'accelerazione, sfruttando l'espressione appena ottenuta per la velocità:
\begin{align*}
    \frac{d\mathbf{v}}{dt}=c\left(\frac{d}{d\theta}\left(\frac{d}{d\theta}\frac{1}{r}\right)\right)\mathbf{u}
\end{align*}}{}{}
%%SPAZI AFFINI EUCLIDEI
%%
%
%%
%
%
\definizione{ Uno spazio affine euclideo è una quaterna $(A,E,\delta,\mathbf{g})$, dove $(A,E,\delta)$ è uno spazio affine e $\mathbf{g}$ è un tensore metrico su $E$ (considerato come spazio vettoriale euclideo). Ovvero $\mathbf{g}$ è una forma bilineare simmetrica:
\begin{align*}
    \mathbf{g}\colon \mathcal{X}(A)\times \mathcal{X}(A)\to \mathcal{F}(A)\\
    (\mathbf{X},\mathbf{Y})\mapsto \mathbf{X}\cdot\mathbf{Y}=\mathbf{g}(\mathbf{X},\mathbf{Y})
\end{align*}
dove, considerata $g$ come il prodotto scalare su $E$, $\forall\, P \in A$ l'immagine è definita come:
\begin{align*}
    (\mathbf{X}\cdot\mathbf{Y})(P)=g(\mathbf{X}(P),\mathbf{Y}(P))
\end{align*}}
\definizione{ Definiamo l'\underline{ascissa euclidea} (o \underline{ascissa curvilinea}) una funzione monotona crescente $g$ tale che presa:
\begin{align*}
    g\colon I \to \mathbb{R}\\
    t\mapsto s(t)
\end{align*}
allora 
\begin{align*}
    \frac{ds}{dt}=|\mathbf{v}|=\sqrt{g_{ij}\frac{dq^i}{dt}\cdot \frac{dq^j}{dt}}
\end{align*}}
%%
%%
%% da fare meglio, lei spiega di merda, se si fa geo3 si capisce davvero la teoria semplice delle curve differenziabili(triedro etc...)
%%
%%
%%
\section{Moti geodetici su una superficie}
Sia $Q$ una superficie regolare nello spazio tridimensionale con rappresentazione $OP=\mathbf{r}(q^1,q^2)$. Sia assegnata sulla superficie una curva $\gamma$ di equazioni parametriche $q^i(t)=\gamma^i(t)$.\\
Il vettore $\mathbf{v}(t)$ tangente alla curva è banalmente tangente anche alla superficie. Se consideriamo invece il suo vettore derivata:
\begin{align*}
    \frac{d\mathbf{v}}{dt}
\end{align*}
questo in generale non è tangente alla superficie. Infatti:
\begin{align*}
   \frac{d\mathbf{v}}{dt} =\frac{dv^i}{dt}\mathbf{E}_i+v^i\frac{d\mathbf{E}_i}{dt}=\frac{dv^i}{dt}\mathbf{E}_i+v^i\frac{\partial \mathbf{E}_i}{\partial q^j}\frac{dq^j}{dt}=\frac{dv^i}{dt}\mathbf{E}_i+v^i\frac{dq^j}{dt}(\Gamma_{ji}^k\mathbf{E}_k+B_{ji}\mathbf{N})
\end{align*}
con $\mathbf{N}$ il vettore normale alla superficie. Quindi:
\begin{align*}
    \frac{d\mathbf{v}}{dt}=\underbrace{\left(\frac{dv^i}{dt}+v^k\frac{dq^i}{dt}\Gamma^i_{jk}\right)\mathbf{E}_i}_{\text{tangente a }Q}+v^i\frac{dq^j}{dt}B_{ji}\mathbf{N}
\end{align*}
La componente tangente della derivata della velocità è chiamata \ul{derivata intrinseca della velocità/accelerazione intrinseca}. Se volessimo scriverla però rispetto alle coordinate non affini $(q^i)$, essendo $v^i=\cfrac{dq^i}{dt}$:
\begin{align*}
    \mathbf{a}^k_{\text{intrinseca}}=\frac{d^2q^k}{dt^2}+\Gamma^k_{ij}\frac{dq^i}{dt}\cdot \frac{dq^j}{dt}
\end{align*}
Abbiamo cosi scomposto l'accelerazione come:
\begin{align*}
    \mathbf{a}=\mathbf{a}^k_{\text{intrinseca}}+\mathbf{a}_{\mathbf{N}}
\end{align*}
Una volta scissa l'accelerazione possiamo finalmente definire cos'è 
\definizione{ Il \ul{moto geodetico} o \ul{moto inerziale} su una superficie è il moto in cui:
\begin{align*}
    \mathbf{a}_{\text{intrinseca}}=0, \quad\forall \, t && [\mathbf{a} = \mathbf{a}_{\mathbf{N}}]
\end{align*}}
Questa condizione può essere anche rappresentata sotto forma di sistema di equazioni differenziali infatti:\\
\begin{minipage}{4cm}
    \begin{align*}
        \frac{d^2q^k}{dt^2}+\Gamma^k_{ij}\frac{dq^i}{dt}\cdot \frac{dq^j}{dt}=0
    \end{align*}
\end{minipage}$\longleftrightarrow$
\begin{minipage}{4cm}
    \begin{align*}
        \begin{cases}
            \cfrac{dq^i}{dt}=v^i\\
            \cfrac{dv^k}{dt}=-\Gamma^k_{ij}v^iv^j
        \end{cases}
    \end{align*}
\end{minipage}
%da aggiustare freccia ahha
\\
Cosi facendo i moti geodetici diventano le curve integrali di questo sistema dinamico associato al campo vettoriale $\mathbf{X}$ definito sui vettori dello spazio tangente alla superficie $Q$. Questo scritto come derivazione sarebbe:
\begin{align*}
    \mathbf{X}=v^i\frac{\partial}{\partial q^i}-\Gamma^k_{ij}v^iv^j\frac{\partial}{\partial v^k}
\end{align*}
$\mathbf{X}$ con un abuso di notazione prende il nome di \underline{flusso geodetico}.
Cosi facendo assegnato un $P_0\in Q$ e $\mathbf{v}_0$ tangente a $Q$ in $P_0$, $\exists!$ una curva geodetica massimale basata in $P_0$ e avente come vettore tangente in $t=0$ il vettore $\mathbf{v}_0$.
\paragraph*{Energia cinetica come integrale primo delle geodetiche}
\begin{align*}
    \mathbf{F}=\mathbf{F}(\mathbf{r},\mathbf{v}) && \mathbf{r}=\mathbf{r}(t)=\gamma(t)
\end{align*}
\begin{align*}
    \frac{d}{dt}\mathbf{F}(\mathbf{r}(t),\mathbf{v}(t))=0
\end{align*}
\begin{align*}
    \frac{d}{dv}(\mathbf{v}\cdot \mathbf{v})=2\mathbf{v}\cdot\mathbf{a}=0
\end{align*}
Poichè $\mathbf{a}=\mathbf{a}_{\mathbf{N}}\perp \mathbf{v}$\\
Quindi l'energia cinetica è un integrale primo delle geodetiche. In particolare il moto è uniforme, ovvero il modulo della velocità è costante lungo le geodetiche.
\chapter{Il modello della visione}
\section{Introduzione} Il modello della visione di Jean Petitot è un tentativo di rappresentare tramite la modellizzazione matematica come gli oggetti ed enti del mondo esterno vengano recepiti, codificati  e rappresentati dalla corteccia visuale del nostro cervello, in particolare si sofferma su una modellizzazione del primo stadio di rappresentazione degli oggetti esterni, il cosiddetto $V1$.
Ci si chiede appunto come enti geometrici esterni semplici come punti o anche più complessi come linee e forme possano essere interpretate e codificate dal nostro apparato neuro-visivo.
\section{Il modello}
\paragraph*{Def}La struttura geometrica più importante definita sulla mappa delle fibre che modella il funzionamento ottico di $V1$ è chiamata \underline{struttura di contatto}, denotata con $\mathcal{C}$.\\
Il modello geometrico della visione di Petito rappresenta le connessioni neuronali retina-corteccia visiva nel seguente modo:
\begin{align*}
    I\xlongrightarrow[]{\gamma}A \xlongrightarrow[]{\mathbf{X}}A\times E\xlongrightarrow[]{\dot{\gamma}}A'\xlongrightarrow[]{\mathbf{X}}A'\times E'
\end{align*}
Dove:
\begin{align*}
    \gamma \colon I \to A\\
    t\mapsto (x,y)
\end{align*}
che poi viene inviato tramite $\mathbf{X}$:
\begin{align*}
    \mathbf{X}\colon A \to A\times E\\
    (x,y)\mapsto (x,y,\dot{x},\dot{y})
\end{align*}
Successivamente:
\begin{align*}
    \dot{\gamma}\colon A\times E\to A'\\
    (x,y,\dot{x},\dot{y})\mapsto (x,y,p=\dot{y})
\end{align*}
E infine tramite $\mathbf{X}\in ker\omega$, con $\omega=dy-pdx$:
\begin{align*}
    \mathbf{X}\colon A'\to A'\times E'\\
    (x,y,p=\dot{y})\mapsto (x,y,p=\dot{y},\dot{x}=1,\dot{y}=p,\dot{p}=\ddot{y})
\end{align*}
Dove $\Gamma$ è la curva geodetica per $g_{\mathcal{C}}$ ed è definita come:
\begin{align*}
    \Gamma= \dot{\gamma}_{|(\dot{x}=1,\dot{y}=p)} && \textit{lift di Legendre}
\end{align*}
e invece:
\begin{align*}
    g_{\mathcal{C}}(\mathbf{t}_i,\mathbf{t}_j)=\delta_{ij}&& i=1,2
\end{align*}
e $\{\mathbf{t}_1,\mathbf{t}_2\}$ che generano il $\ker\omega$.


\paragraph{andrea donati}
Sia $\bm{X}$ un campo vettoriale e $\Gamma$ una curva integrale di $\bm{X}$, dove:
\begin{align*}
    \bm{X}\colon A' \to TA'\\
    (x,y,p)\mapsto (\xi, \eta, \pi)
\end{align*}
dove con $TA'$ indichiamo il fibrato tangente di $A'$.
Defiamo inoltre:
\begin{align*}
    \dot{\Gamma}=(\bm{X}\circ \Gamma)=\dot{x}\partial_x+\dot{y}\partial_y+\dot{p}\partial_p
\end{align*}
Supponiamo che:
\begin{align*}
    \begin{cases*}
        \dot{x}=1\\
        \dot{y}=p
    \end{cases*}
\end{align*}
lungo $\Gamma$, si ha dunque che:
\begin{align*}
    \dot{\Gamma}=\partial_x+p\partial_y+\dot{p}\partial_p=\partial_x+\dot{y}\partial_y+\ddot{y}\partial_y
\end{align*}
Ora, considerata la forma $w=-pdx+dy$, si noti che lungo $\Gamma$ $w(\bm{x})=0$ e in generale il $\ker w$ definisce una struttura di contatto su $A'$:
\begin{align*}
    \bm{x}\in \ker w \iff \eta-p\xi=0
\end{align*}
dove il membro a destra definisce un piano di contatto.

Un tale campo vettoriale può essere scritto nel seguente modo:
\begin{align*}
    \xi (\partial_x +p\partial_y)+\pi\partial_p = 0 = \xi\partial_x+\overbrace{\xi p}^{\eta}\partial_y+\pi\partial_p
\end{align*}
Ora definiamo i vettori:
\begin{align*}
    \bm{t}_1=\partial_x+p\partial_y \quad \bm{t}_2=\partial_p \quad \bm{t}_1\perp\bm{t}_2
\end{align*}
Creando cosi una coppia $\left\{\bm{t}_1,\bm{t}_2\right\}$ di vettori ortonormali (ortogonali sicuramente, ci basta normalizzarli successivamente).

Lungo $\Gamma$ $\dot{y}=p\dot{x}$, ma $\dot{x}=1\leadsto x=s+c$.

Possiamo così riparametrizzare $\Gamma$ per il parametro $x$ e vedere $\Gamma$ come l'indicatrice delle tangenti di una curva $\gamma$, dove quindi:
\begin{align*}
    \gamma=(x,y(x)) \\
    \Gamma\colon A'\to A'\times E \\
     (x,y,p)\longleftrightarrow (x,y,\dot{x},\dot{y})
\end{align*}

Esistono solo curve integrali di dimensione 1 dal teorema di Frobenius che dice che $dw \wedge w=0\Rightarrow$ $\nexists$ superfici integrali della distribuzione di contatto.

Le proprietà di $w$ permettono di definire una metrica sulla struttura di contatto detta \ul{metrica di Carnot-Caratheodory} denotata con $g_\mathcal{C}$.

Le curve geodetiche rispetto a $g_\mathcal{C}$ sono curve $\Gamma$ tangenti a $\mathcal{C}$ (struttura di contatto), ovvero curve integrali del campo di partenza e tali che:
\begin{align*}
    \begin{cases*}
        \dot{x}=1\\
        \dot{y}=p
    \end{cases*}\text{sollevamento di Legendre (non lui eh, qualcosa qui vicino)}
\end{align*}
Definiamo con$A'=$corteccia visiva primaria ($V_1$).

Stiamo studiando il modello di Petitot. Riprendendo il discorso fatto, definiamo la metrica di Carnot-Caratheodory come:
\begin{align*}
    g_{\mathcal{C}}(\bm{t}_1,\bm{t}_j)=\delta_{ij}\quad i,j\in \left\{1,2\right\}
\end{align*}
$A'$ è un gruppo di Lie con le operazioni:
\begin{enumerate}
    \item $(x,y,p)\cdot (x',y',p')=(x+x',y+y',p+p')$
    \item $[(\xi,\eta,\pi),(\xi',\eta',\pi')]=(0,\xi'\pi-\xi\pi',0)$
\end{enumerate}
Tramite questa metrica definiamo la distanza:
\begin{align*}
    d_{\mathcal{C}}({P'}_1,{P'}_2)=inf\left\{\int_I \norma{\dot{\Gamma}(s)}ds\right\}
\end{align*}
Estendiamo la nostra base ortonormale $\left\{\bm{t}_1,\bm{t}_2\right\}$ a una base $\left\{\bm{t}_1,\bm{t}_2,\bm{t}_3\right\}$ con un opportuno $\bm{t}_3$ tale che si mantenga l'ortonormalità della base.
Inoltre $\left\{\bm{t}_1,\bm{t}_2\right\}$ è una metrica sub-Riemaniana.

Ci poniamo l'obiettivo di minimizzare:
\begin{align*}
    \int_{{P'}_A}^{{P'}_{B}} ds \quad \quad ds^2=dx^2+dy^2+dp^2
\end{align*}
con norma del vettore tangente lungo $\Gamma$ che è:
\begin{align*}
    \sqrt{\xi^2+\eta^2+\pi^2}= \sqrt{1+\dot{y}^2(x)+\ddot{y}^2(x)}
\end{align*}


Definiamo:
\begin{align*}
    L(x,y,p,\dot{x},\dot{y},\dot{p})=\sqrt{\xi^2+\eta^2+\pi^2}=\sqrt{1+\dot{y}^2(x)+\ddot{y}^2(x)}
\end{align*}
e usiamo le equazioni di Eulero-Lagrange:
\begin{align*}
    \begin{cases*}
        \cfrac{\partial}{\partial y}L-\cfrac{d}{dx}\left(\cfrac{\partial}{\partial \eta}\right)L=0 \quad \leadsto \cfrac{\partial}{\partial y}L=0\\
        \cfrac{\partial}{\partial p}L-\cfrac{d}{dx}\left(\cfrac{\partial}{\partial \pi}\right)L=0 \quad \leadsto \cfrac{\partial}{\partial y}L=0
    \end{cases*}
\end{align*}
I due annullamenti sono perchè $L$ non dipende da $y$ e da $p$.

Poniamo $\Sigma=\xi p-\eta$, ma $\xi=1$ lungo $\Gamma$ quindi $\Sigma=p-\eta$.
Questo nuovo oggetto ci serve per imporre il vincolo del piano di contatto:
\begin{align*}
    \begin{cases*}
        \left(\cfrac{\partial}{\partial y}-\cfrac{d}{dx}\cfrac{\partial}{\partial \eta}\right)(L+\overbrace{\lambda(x)}^{\text{moltiplicatore di Lagrange}}\Sigma)=0 \\
        \left(\cfrac{\partial}{\partial p}-\cfrac{d}{dx}\cfrac{\partial}{\partial \pi}\right)(L+\lambda(x)\Sigma)=0 
    \end{cases*}\\
    \begin{cases*}
        \cfrac{d}{dx}\left(\cfrac{\partial L}{\partial \eta}-\lambda(x)\right)=0\longrightarrow \cfrac{\partial L}{\partial \eta}=\lambda(x)+A \quad A\in \R\\
        \lambda(x)-\cfrac{d}{dx}\left(\cfrac{\partial L}{\partial \pi}\right)=0\longrightarrow\lambda(x)+A=\cfrac{d}{dx}\left(\cfrac{\partial L}{\partial \pi}\right)+A
    \end{cases*}
\end{align*}
Ottenendo cosi:
\begin{align*}
    \frac{\partial L}{\partial \eta}=\lambda(x)+A=\frac{d}{dx}\left(\frac{\partial L}{\partial \pi}\right)
\end{align*}
Ovvero:
\begin{align*}
    \frac{\dot{y}(x)}{\sqrt{1+\dot{y}^2(x)+\ddot{y}^2(x)}}=A+\frac{d}{dx}\frac{\ddot{y}}{\sqrt{1+\dot{y}^2(x)+\ddot{y}^2(x)}}
\end{align*}
Svolgendo calcoli otteniamo:
\begin{align*}
    \ddot{y}^2=\frac{(1+\dot{y}^2)^2-(1+\dot{y}^2)(A\dot{y}+B)^2}{(A\dot{y}+B)^2} \quad A,B\in \R \quad *
\end{align*}
Le soluzioni di questa complessa espressione si ottengono da integrali ellittici.

Se $y$ è pari, $\gamma$ è simmetrica ($\gamma(x)=\gamma(-x)$), avendo $A=0$ e $K=\cfrac{1}{B}$.

Sostituendo otteniamo:
\begin{align*}
    \ddot{y}^2=(1+\dot{y}^2)^2[k^2(1+\dot{y}^2)-1]
\end{align*}
Concludendo:
\begin{align*}
    x(y)=c \int^{\dot{y}}_0 \frac{1}{\sqrt{(1+t^2)(1+\frac{k^2}{k^2-1}t^2)}}
\end{align*}